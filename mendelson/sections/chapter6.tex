\begin{tcolorbox}[title=Problem 1, breakable]
    If $f(x) = x^2 - 4x + 6$. Find 

    (a) $f(0)$.

    (b) $f(3)$.

    (c) $f(-2)$.

    Show that $f(\frac{1}{2}) \ne f(\frac{7}{2})$ 
        and $f(2 - h) = f(2 + h)$.
\end{tcolorbox}

\textbf{Solution (a):}
\[f(0) = 0^2 - 4(0) + 6 = 6\]
\textbf{Solution (a):}
\[f(3) = 3^2 - 4(3) + 6 = 9 - 12 + 6 = 3\]
\textbf{Solution (a):}
\[f(-2) = (-2)^2 - 4(-2) + 6 = 18\]

\textbf{Solution:}
\[f(\frac{1}{2}) 
    = (\frac{1}{2})^2 + 4(\frac{1}{2}) + 6 
    = \frac{1}{4} + 2 + 6
    = \frac{1}{4} + 8
    = \frac{33}{4}\]
\[f(\frac{7}{2}) 
    = (\frac{7}{2})^2 + 4(\frac{7}{2}) + 6
    = \frac{49}{4} + (2 * 7) + 6
    = \frac{49}{4} + 14 + 6
    = \frac{49}{4} + 20
    = \frac{129}{4}\]
Clearly $f(\frac{1}{2}) \ne f(\frac{7}{2})$.

\textbf{Solution:}
\[f(2 - h) 
    = (2 - h)^2 - 4(2 - h) + 6
    = 4 - 4h + h^2 - 8 + 4h + 6
    = h^2 + 2\]
\[f(2 = h)
    = (2 + h)^2 - 4(2 + h) + 6
    = 4 + 4h + h^2 - 8 - 4h + 6
    = h^2 + 2\]
Thus $f(2 - h) = f(2 + h)$.

\begin{tcolorbox}[title=Problem 16, breakable]
    Determine the domain of each of the following functions:

    (a) $y = x^2 + 4$.

    (b) $y = \sqrt{x^2 + 4}$.

    (c) $y = \sqrt{x^2 - 4}$.

    (d) $y = \frac{x}{x + 3}$.

    (e) $y = \frac{2x}{(x - 2)(x + 1)}$.

    (f) $y = \frac{1}{\sqrt{9 - x^2}}$.

    (g) $y = \frac{x^2 - 1}{x^2 + 1}$.

    (h) $y = \sqrt{\frac{x}{2 - x}}$
\end{tcolorbox}

\textbf{Solution (a):}
\[x \in \mathbb{R}\]

\textbf{Solution (b):}
\[x \in \mathbb{R}\]

\textbf{Solution (c):}
\[x \in \mathbb{R} \setminus \{-2, 2\}\]

\textbf{Solution (d):}
\[x \in \mathbb{R} \setminus \{-3\}\]

\textbf{Solution (e):}
\[x \in \mathbb{R} \setminus \{2, -1\}\]

\textbf{Solution (f):}
\[x \in \mathbb{R} \setminus \{-3, 3\}\]

\textbf{Solution (g):}
\[x \in \mathbb{R}\]

\textbf{Solution (h):}
\[x \in \mathbb{R} \setminus \{x \mid x < 0 \text{ or } x \ge 2\}\]

\begin{tcolorbox}[title=Problem 17, breakable]
    Compute $\frac{f(a + h) - f(a)}{h}$
        in the following cases:

    (a) $f(x) = \frac{1}{x - 2}$.

    (b) $f(x) = \sqrt{x - 4}$ when $a \ne 2$ and $a + h \ge 4$.

    (c) $f(x) = \frac{x}{x + 1}$ when $a \ne -1$ and $a + h \ne -1$.
\end{tcolorbox}

\textbf{Solution (a):}
\begin{align*}
    \frac{f(a + h) - f(a)}{h} 
        &= \frac{\frac{1}{(a + h) - 2} - \frac{1}{a - 2}}{h} \\
        &= \frac{\frac{a - 2}{a + h - 2} - \frac{a + h - 2}{(a + h - 2)(a - 2)}}{h} \\
        &= \frac{\frac{-h}{(a + h - 2)(a - 2)}}{h} \\
        &= \frac{-h}{h(a + h - 2)(a - 2)} \\
        &= \frac{-1}{(a + h - 2)(a - 2)}
\end{align*}
\textbf{Solution (b):}
\begin{align*}
    \frac{f(a + h) - f(a)}{h} 
        &= \frac{\sqrt{a + h - 4} - \sqrt{a - 4}}{h} \\
        &= \frac{\sqrt{a + h - 4} - \sqrt{a - 4}}{h} \cdot \frac{\sqrt{a + h - 4} + \sqrt{a - 4}}{\sqrt{a + h - 4} + \sqrt{a - 4}} \\
        &= \frac{a + h - (a - 4)}{h(\sqrt{a + h - 4} + \sqrt{a - 4})} \\
        &= \frac{1}{\sqrt{a + h - 4} + \sqrt{a - 4}}
    \end{align*}
\textbf{Solution (c):}
\begin{align*}
    \frac{f(a + h) - f(a)}{h} 
        &= \frac{\frac{a + h}{a + h + 1} - \frac{a}{a + 1}}{h} \\
        &= \frac{\frac{(a + h)(a + 1)}{(a + h + 1)(a + 1)} - \frac{a(a + h + 1)}{(a + 1)(a + h + 1)}}{h} \\
        &= \frac{\frac{a^2 + a + ah + h - (a^2 + ah + a)}{(a + h + 1)(a + 1)}}{h} \\
        &= \frac{\frac{h}{(a + h + 1)(a + 1)}}{h} \\
        &= \frac{1}{(a + h + 1)(a + 1)}
\end{align*}

\begin{tcolorbox}[title=Problem 20, breakable]
    Evaluate the expression $\frac{f(x + h) - f(x)}{h}$
        for the following functions $f$:

    (a) $f(x) = 2x - x^2$.

    (b) $f(x) = \sqrt{x - 4}$ when $a \ne 2$ and $a + h \ge 4$.

    (c) $f(x) = \frac{x}{x + 1}$ when $a \ne -1$ and $a + h \ne -1$.

    (d) $f(x) = \frac{x}{x + 1}$ when $a \ne -1$ and $a + h \ne -1$.
\end{tcolorbox}

\textbf{Solution (a):}
\begin{align*}
    \frac{f(x + h) - f(x)}{h} 
        &= \frac{3(x + h) - (x + h)^2 - (3x - x^2)}{h} \\
        &= \frac{3x + 3h - (x^2 + 2xh + h^2) - (3x - x^2)}{h} \\
        &= \frac{3x  + 3h - x^2 - 2xh - h^2 - 3x  + x^2}{h} \\
        &= \frac{2h - 2xh - h^2}{h} \\
        &= \frac{h(3 - 2x - h)}{h} \\
        &= 3 - 2x - h
\end{align*}
\textbf{Solution (b):}
\begin{align*}
    \frac{f(x + h) - f(x)}{h} 
        &= \frac{\sqrt{2(x + h)} - \sqrt{2x}}{h} \\
        &= \frac{\sqrt{2(x + h)} - \sqrt{2x}}{h} \cdot \frac{\sqrt{2(x + h)} + \sqrt{2x}}{\sqrt{2(x + h)} + \sqrt{2x}} \\
        &= \frac{2(x + h) - 2x}{h(\sqrt{2(x + h)} + \sqrt{2x})} \\
        &= \frac{2x + 2h - 2x}{h(\sqrt{2(x + h)} + \sqrt{2x})} \\
        &= \frac{2h}{h(\sqrt{2(x + h)} + \sqrt{2x})} \\
        &= \frac{2}{(\sqrt{2(x + h)} + \sqrt{2x})} 
\end{align*}
\textbf{Solution (c):}
\begin{align*}
    \frac{f(x + h) - f(x)}{h} 
        &= \frac{(3(x + h) - 5) - (3x - 5)}{h} \\
        &= \frac{(3x + 3h - 5) - (3x - 5)}{h} \\
        &= \frac{3h}{h} \\
        &= 3
\end{align*}
\textbf{Solution (d):}
\begin{align*}
    \frac{f(x + h) - f(x)}{h} 
        &= \frac{((x + h)^3 - 2) - (x^3 - 2)}{h} \\
        &= \frac{(x^3 + 3x^2 h + 3x h^2 + h^3 - 2) - (x^3  - 2)}{h} \\
        &= \frac{h(3x^2 + 3xh + h^2)}{h} \\
        &= 3x^2 + 3xh + h^2
\end{align*}

\begin{tcolorbox}[title=Problem 21, breakable]
    Find a formula for the function $f$ whose graph consists of all points
        satisfying each of the following equations.
    (In plain language, solve each equation for $y$.)

    (a) $x^5 y + 4x - 2 = 0$.

    (b) $x = \frac{2 + y}{2 - y}$.

    (c) $4x^2 - 4xy + y^2 = 0$.
\end{tcolorbox}

\textbf{Solution (a):}
Suppose $x \ne = 0$ then
\[x^5 y + 4x - 2 = 0 \iff y = \frac{-4x + 2}{x^5}\]
Thus 
\[f(x) = \frac{-4x + 2}{x^5}\]
\textbf{Solution (b):}
Suppose $y \ne 2$ then 
\[x = \frac{2 + y}{2 - y} \iff x(2 - y) - 2 + y = 0 \iff 2x - xy - 2 - y = 0 \iff 2x -2 = xy + y \iff 2x - 2 = x(y + 1) \iff \frac{2x - 2}{x + 1} = y\]
Thus 
\[f(x) = \frac{2x - 2}{x + 1}\]
\textbf{Solution (c):}
\[4x^2 - 4xy + y^2 = 0 \iff y^2 - 4xy + 4x^2 = 0 \iff (y - 2x)(y - 2x) = 0\]
Thus
\[f(x) = 2x\]