\subsection{Definition}

\begin{tcolorbox}[title=Problem 1, breakable]
    A particle starts from the point $(0, 6)$ in the plant.
    It is attracted by a magnet below the x-axis, and repelled 
    by a magnet along the y-axis in such a way that its coordinates 
        are given as a function of $t$ by 
    \[x(t) = 2t, y(t) = 6 - 15t^3\]
    (a) Find the time at which it hits the x-axis.

    (b) Give a simple equation in terms of $x$ and $y$ such that the coordinates $(x(t), y(t))$ of the particle 
        satisfy this equation. Sketch the graph of this equation.

    (c) Find the distance of the point at which the particle hits the x-axis from teh origin.

    (d) Find the time at which the particle is at distance $2$ units from the x-axis, below the x-axis.

    (e) Find the time at which the particle is at distance $5$ units from the x-axis, below the x-axis.

    (f) Find the time at which the particle is at distance $7$ units from the x-axis, below the x-axis.
\end{tcolorbox}

\textbf{Solution (a):}
We required $0 = 6 - 15t^3$. Thus $t = \sqrt[3]{\frac{6}{15}}$.

\textbf{Solution (b):}
\[y = 6 - 15\left(\frac{x}{3}\right)^3\]
\begin{figure}[h!]
    \centering
    \includegraphics[width=0.5\textwidth]{images/particle.PNG}
\end{figure}

\textbf{Solution (c):}
The distance is $\sqrt{(3 \cdot \sqrt[3]{\frac{6}{15}})^2}$ units.

\textbf{Solution (d):}
We require $-2 = 6 - 15t^3$. Thus $t = \sqrt[3]{\frac{8}{15}}$

\textbf{Solution (e):}
We require $-5 = 6 - 15t^3$. Thus $t = \sqrt[3]{\frac{11}{15}}$

\textbf{Solution (f):}
We require $-7 = 6 - 15t^3$. Thus $t = \sqrt[3]{\frac{13}{15}}$

\subsection{Formalism of Mappings}

\begin{tcolorbox}[title=Problem 1, breakable]
    Let $f : S \longrightarrow T$ and $g : S \longrightarrow T$ be mappings. 
    Let 
    \[h = T \longrightarrow U\]
    be a mapping have an inverse mapping denoted by
    \[h^{-1} : U \longrightarrow T\]
    If 
    \[h \circ f = h \circ g\]
    Prove that $f = g$. This is the \textbf{cancellation law} for mappings.
\end{tcolorbox}

\begin{proof}
    Composing with $h^{-1}$ shows $f = g$.
\end{proof}

\begin{tcolorbox}[title=Problem 2, breakable]
    Let $f : S \longrightarrow T$ be a mapping having an inverse mapping.
    Prove the following statements.

    (a) If $x, y$ are elements of $S$ and $f(x) = f(y)$, then $x = y$.

    (c) If $z$ is an element of $T$, then there exists an element $x$ of $S$ such that $f(x) = z$.
\end{tcolorbox}

\begin{proof}
    Suppose $x, y$ are elements of $S$, and $f(x) = f(y)$. Composing with $f^{-1}$ shows $x = y$.
\end{proof}

\begin{proof}
    Suppose $z$ is an element of $T$.
    There exists $c \in S$ such that $f^{-1}(z) = c$. Composing with $f$ shows 
        $f(c) = z$.
\end{proof}

\begin{tcolorbox}[title=Problem 3, breakable]
    Let $a, b$ be non-zero numbers.
    Let $F_{a, b} = \mathbb{R}^2 \longrightarrow \mathbb{R}^2$ be the mapping such 
        that 
    \[F_{a, b}(x, y) = (ax, by)\]
    Show that $F_{a, b}$ has an inverse mapping.
\end{tcolorbox}

\begin{proof}
    Let $G_{a, b}(x, y) = \left(\frac{x}{a}, \frac{y}{b}\right)$.
    Consider 
    \[(F_{a, b} \circ G_{a, b})(x, y) = F_{a, b}(G_{a, b}(x, y)) = F_{a, b}\left(\frac{x}{a}, \frac{y}{b}\right) = (x, y)\]
    Also 
    \[(G_{a, b} \circ F_{a, b})(x, y) = G_{a, b}(F_{a, b}(x, y)) = G_{a, b}(ax, by) = (x, y)\]
    Therefore $G_{a, b} = F_{a, b}^{-1}$.
\end{proof}

\begin{tcolorbox}[title=Problem 6, breakable]
    Let $f : S \longrightarrow S$ be a mapping which has an inverse mapping.

    (a) If $f^3 = I$ and $f^5 = I$, show that $f = I$.

    (b) If $f^2 = I$ and $f^7 = I$, show that $f = I$.

    (c) If $f^4 = I$ and $f^11 = I$, show that $f = I$.
\end{tcolorbox}

\begin{proof}
    Since $f^3 = I$, $f^2 = f^{-1}$.
    Then 
    \[f = f^5 \circ f^{-4} = f^5 \circ f^{-1} \circ f^{-1} \circ f^{-1} \circ f^{-1}\]
    \[= f^5 \circ f^8 = f^{13} = f^3 \circ f^{10} = f^3 \circ f^5 \circ f^5 = I\]
\end{proof}

\begin{proof}
    Since $f^2 = I$, $f = f^3$ and $f = f^{-1}$.
    Then 
    \[f = f^2 \circ f^{-1} = f^2 \circ f = f^2 \circ f^3 = f^4 \circ f = f^4 \circ f^3 = f^7 = I\]
\end{proof}

\begin{proof}
    Since $f^4 = I$, $f = f^5$ and $f = f^{-3}$.  
    Then
    \[
    f = f^{11} \circ f^{-10} = f^{11} \circ (f^{-4})^2 \circ f^{-2} = f^{11} \circ I^2 \circ f^{-2} = f^{11} \circ f^{-2}.
    \]
    Then $f^{11} = I$, so
    \[
    f = I \circ f^{-2} = f^{-2}.
    \]
    Composing both sides with $f^2$ gives
    \[
    f \circ f^2 = f^{-2} \circ f^2 \implies f^3 = I.
    \]
    Now $f^3 = I$ and $f^4 = I$, so
    \[
    f = f^4 \circ f^{-3} = I \circ I = I.
    \]
\end{proof}

\begin{tcolorbox}[title=Problem 7, breakable]
    Let $f, g$ be mappings of a set $S$ into itself, and assume that they have 
    inverse mappings. Assume also that $f \circ g = g \circ f$.
    Express each one of the following in the form $f^m \circ g^n$ 
    where $m, n$ are integers.

    (a) $f^3 \circ g^2 \circ f^5 \circ g^{-5}$

    (b) $f^7 \circ g \circ g^4 \circ f^{-6} \circ g^3$

    (c) $f^4 \circ g^5 \circ f^{-5} \circ g^{-7} \circ g^2 \circ f^2$

    (d) $f^4 \circ f^{-8} \circ g^2 \circ f^3 \circ g^3 \circ f^{-2}$
\end{tcolorbox}

\textbf{Solution (a):}
\[
f^3 \circ g^2 \circ f^5 \circ g^{-5} = f^3 \circ f^5 \circ g^2 \circ g^{-5} = f^8 \circ g^{-3}
\]
\textbf{Solution (b):}
\[
f^7 \circ g \circ g^4 \circ f^{-6} \circ g^3 = f^7 \circ f^{-6} \circ g \circ g^4 \circ g^3 = f \circ g^8
\]
\textbf{Solution (c):}
\[
f^4 \circ g^5 \circ f^{-5} \circ g^{-7} \circ g^2 \circ f^2
= f^4 \circ f^{-5} \circ f^2 \circ g^5 \circ g^{-7} \circ g^2
= f \circ g^0 = f
\]
\textbf{Solution (d):}
\[
f^4 \circ f^{-8} \circ g^2 \circ f^3 \circ g^3 \circ f^{-2}
= f^4 \circ f^{-8} \circ f^3 \circ f^{-2} \circ g^2 \circ g^3
= f^{-3} \circ g^5
\]

\begin{tcolorbox}[title=Problem 8, breakable]
    (a) Let $f, g$ be mappings of a set $S$ into itself,
        and assume that they have inverse mappings. Prove that $f \circ g$
        has an inverse mapping, and express it in terms of $f^{-1} \circ g^{-1}$.

    (b) Let $f_1, \ldots, f_m$ be maps of $S$ into itself, and assume that each 
        $f_i$ has an inverse mapping. Show that $f_1 \circ f_2 \circ \ldots \circ f_m$
        has an inverse mapping, and express this mapping in terms of the maps $f_i^{-1}$.
\end{tcolorbox}

\begin{proof}
    Consider 
    \[
    (f \circ g) \circ (g^{-1} \circ f^{-1}) = f \circ (g \circ g^{-1}) \circ f^{-1} = f \circ I \circ f^{-1} = f \circ f^{-1} = I
    \]
    Also 
    \[
    (g^{-1} \circ f^{-1}) \circ (f \circ g) = g^{-1} \circ (f^{-1} \circ f) \circ g = g^{-1} \circ I \circ g = g^{-1} \circ g = I
    \]
    Therefore, \(g^{-1} \circ f^{-1}\) is the inverse of \(f \circ g\).
\end{proof}

\begin{proof}
    Consider 
    \[
    (f_1 \circ f_2 \circ \cdots \circ f_m) \circ (f_m^{-1} \circ f_{m-1}^{-1} \circ \cdots \circ f_1^{-1})
    \]
    Then
    \[
    f_1 \circ f_2 \circ \cdots \circ f_m \circ f_m^{-1} \circ \cdots \circ f_1^{-1} 
    = f_1 \circ f_2 \circ \cdots \circ f_{m-1} \circ I \circ f_{m-1}^{-1} \circ \cdots \circ f_1^{-1} 
    = \cdots = I.
    \]
    Similarly,
    \[
    (f_m^{-1} \circ \cdots \circ f_1^{-1}) \circ (f_1 \circ f_2 \circ \cdots \circ f_m) = I
    \]
    Therefore, 
    \[
    (f_1 \circ f_2 \circ \cdots \circ f_m)^{-1} = f_m^{-1} \circ f_{m-1}^{-1} \circ \cdots \circ f_1^{-1}
    \]
\end{proof}


\begin{tcolorbox}[title=Problem 9, breakable]
    Let $f$ be a mapping of a set $S$ into itself, and assume that $f$ has an inverse mapping.

    (a) If $f^5 = I$, express $f^{-1}$ as a positive power of $I$.

    (b) In general, if $f^n = I$, for some positive power of $f$,
        express $f^{-1}$ as a positive power of $f$.
\end{tcolorbox}

\begin{proof}
    \[
    f^{-1} = f^{-1} \circ I = f^{-1} \circ f^5 = f^4
    \]
\end{proof}

\begin{proof}
    \[
    f^{-1} = f^{-1} \circ I = f^{-1} \circ f^n = f^{n-1}
    \]
\end{proof}

\begin{tcolorbox}[title=Problem 10, breakable]
    Let $f, g$ be mappings of a set $S$ into itself. Assume that $f^2 = g^2 = I$
        and that $f \circ g = g \circ f$. Prove that $(f \circ g)^2 = I$.
    Prove that $(f \circ g)^2 = I$.
    Prove that $(f \circ g)^3 = I$.
    What about $(f \circ g)^n$ for any positive integer $n$?
    What about $(f \circ g)^n$ where $n$ is a negative integer?
\end{tcolorbox}

\begin{proof}
    Notice
    \[
    (f \circ g)^2 = (f \circ g) \circ (f \circ g) = f \circ (g \circ f) \circ g = f \circ (f \circ g) \circ g = (f \circ f) \circ (g \circ g) = I \circ I = I.
    \]
    Also
    \[
    (f \circ g)^3 = (f \circ g)^2 \circ (f \circ g) = I \circ (f \circ g) = f \circ g \neq I
    \]
    Let $n$ be an arbitrary positive integer. Then
    \[
    (f \circ g)^n = 
    \begin{cases} 
        I & \text{if } n \text{ is even} \\[2mm]
        f \circ g & \text{if } n \text{ is odd.}
    \end{cases}
    \]
    For negative integers the pattern is the same.
\end{proof}

\subsection{Permutations}

\begin{tcolorbox}[title=Problem 10, breakable]
    Prove that the number of odd permutations of $J_n$
    for $n \ge 2$ is equal to the number of even permutations.
\end{tcolorbox}

\begin{proof}
    Let $\sigma_1, \ldots, \sigma_m$ be all the distinct even permutations.
    Let $\gamma$ be a transposition.
    Suppose $\sigma_i = \gamma_1 \gamma_2 \ldots \gamma_{2k}$
        for some $k \in \mathbb{Z}$.
    Composing with $\gamma$ we see $\gamma \sigma_i = \gamma \gamma_1 \gamma_2 \ldots \gamma_{2k}$
        is a product of $2k + 1$ transpositions, thus an odd permuation.
    Suppose $i \ne j$, $\gamma \sigma_i = \gamma \sigma_j$.
    Composing with $\gamma$ shows $\sigma_i = \sigma_j$, which is a contradiction.
    Thus $\gamma \sigma_1, \ldots, \gamma \sigma_m$ are distinct.
    Now suppose there exists an odd permuation not generated through this process.
    Let $\pi$ be this permutation.
    Then, $\pi$ can be written as a product of $2k + 1$ transpositions.
    Thus $\gamma \pi$ is a product of $2k$ transpositions, thus an even permutation,
        which must be in out list $\sigma_1, \ldots, \sigma_m$.
    Thus $\pi$ is of the form $\gamma \sigma_i$ for some $i$,
        and no odd permutation is as required.
\end{proof}

\begin{tcolorbox}[title=Problem 11, breakable]
    Prove that the number of permuations of $J_n$ is equal to $n!$.
\end{tcolorbox}

\begin{proof}
    (\textbf{Base Case}) Let $n = 1$. There is $1$ way to permute $1$ element, thus $1 = 1!$ as required.

    (\textbf{Induction Step}) Suppose for some $n \in \mathbb{N}$
        the theorem holds.
    Thus there are $n!$ ways to permute $n$ elements.
    Consider a permuation of $n + 1$ elements.
    By the induction hypothesis there are $n!$ ways to permute the first $n$ elements.
    For each of these permutations, there are $n + 1$ positions to place the $(n + 1)$-th 
    element, including placing it in its original position.
    Thus we have a total of $n!(n + 1) = (n + 1)!$ permuations as required.
\end{proof}
