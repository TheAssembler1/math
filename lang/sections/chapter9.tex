\subsection{Dilations and Reflections}

\begin{tcolorbox}[title=Problem 2, breakable]
    Let $A$ be a point, $A \ne 0$. If $b, c$ are numbers 
    such that $b A = c A$, prove that $b = c$.
\end{tcolorbox}

\begin{proof}
    Let $A = (a_1, a_2)$. Then $b A = (b a_1, b a_2)$
        and $c A = (c a_1, c a_2)$.
    Suppose $b A = c A$.
    Then $b a_1 = c a_1$ and $b a_2 = c a_2$.
    Since $A \ne 0$ either $a_1 \ne 0$ or $a_2 \ne 0$.
    Suppose $a_1 \ne 0$ then since $b a_1 = c a_1$
        dividing by $a_1$ shows $b = c$.
    Suppose $a_2 \ne 0$ then since $b a_2 = c a_2$
        dividing by $a_2$ shows $b = c$.

    Therefore $b = c$.
\end{proof}

\begin{tcolorbox}[title=Problem 3, breakable]
    Prove that reflection through $O$ preserves distances.
    In other words, prove that 
    \[d(A, B) = d(-A, -B)\]
\end{tcolorbox}

\begin{proof}
    Let $A = (a_1, a_2)$ and $B = (b_1, b_2)$. Then
    \[
    d(-A, -B) = \sqrt{((-a_1)-(-b_1))^2 + ((-a_2)-(-b_2))^2}
    = \sqrt{(a_1 - b_1)^2 + (a_2 - b_2)^2} = d(A, B).
    \]
\end{proof}

\begin{tcolorbox}[title=Problem 4, breakable]
    \textbf{The $3-$ dimensional case}

    (a) Define the multiplication (dilation) of a point $A = (a_1, a_2, a_3)$
        by a number $c$. Write out interpretations for this similar to those 
        we did in the plane. Draw pictures.

    (b) Define reflection $A$ through $O = (0, 0, 0)$.

    (c) State and prove the analogs of Theorems $1$ and $2$.
\end{tcolorbox}
\begin{figure}[h]
    \centering
    \includegraphics[width=0.6\textwidth]{images/arb2_rect.png}
\end{figure}
\begin{definition}
    Let $r$ be a real number.
    Let the dilation of a point $A = (a_1, a_2, a_3)$ by $r$ be defined as follows
    \[r A = (r a_1, r a_2, r a_3)\]
\end{definition}
\begin{definition}
    Let the reflection $R$ of $A$ through $O = (0, 0, 0)$
    dilation with $r = -1$.
\end{definition}
\begin{theorem}
    Let $r$ be a postiive number. If $A, B$ are points, then 
    \[d(A, B) = r \cdot d(A, B)\]
\end{theorem}
\begin{proof}
    Let $A = (a_1, a_2, a_3)$ and $B = (b_1, b_2, b_3)$.
    Then $r A = (r a_1, r a_2, r a_3)$ and $r B = (r b_1, r b_2, r b_3)$.
    Hence 
    \begin{align*}
        d(r A, r B)^2 
            &= (r b_1 - r a_1)^2 + (r b_2 - r a_2)^2 + (r b_3 - r a_3)^2 \\
            &= (r (b_1 - a_1))^2 + (r (b_2 - a_2))^2 + (r (b_3 - a_3))^2 \\
            &= r^2(b_1 - a_1)^2 + r^2(b_2 - a_2)^2 + r^2(b_3 - a_3)^2 \\
            &= r^2 \cdot d(A, B)^2
    \end{align*}
    Taking the square root proves our theorem.
\end{proof}
\begin{theorem}
    Let $c$ be a number. Then 
    \[d(c A, c B) = |c| \cdot d(A, B)\]
\end{theorem}
\begin{proof}
    Let $A = (a_1, a_2, a_3)$ and $B = (b_1, b_2, b_3)$.
    Then $c A = (c a_1, c a_2, c a_3)$ and $c B = (c b_1, c b_2, c b_3)$.
    Hence 
    \begin{align*}
        d(c A, c B)^2 
            &= (c b_1 - c a_1)^2 + (c b_2 - c a_2)^2 + (c b_3 - c a_3)^2 \\
            &= (c (b_1 - a_1))^2 + (c (b_2 - a_2))^2 + (c (b_3 - a_3))^2 \\
            &= c^2(b_1 - a_1)^2 + c^2(b_2 - a_2)^2 + c^2(b_3 - a_3)^2 \\
            &= c^2 \cdot d(A, B)^2
    \end{align*}
    Taking the square root proves our theorem since $\sqrt{c^2} = |c|$.
\end{proof}

\subsection{Addition, Subtraction, and Parallelogram Law}

\begin{tcolorbox}[title=Problem 11, breakable]
    Let $T_A$ be a translation by $A$. Prove that it is an isometry,
    in other words, that for any pair of points $P, Q$ we have 
    \[d(P, Q) = d(T_A(P), T_A(Q))\]
\end{tcolorbox}

\begin{proof}
    Let $P = (p_1, p_2), Q = (q_1, q_2), A = (a_1, a_2)$.
    Now $d(P, Q) = \sqrt{(p_1 - q_1)^2 + (p_2 - q_2)^2}$.
    Then
    \begin{align*}
        d(T_A(P), T_A(Q))^2 
        &= ((p_1 + a_1) - (q_1 + a_1))^2 + ((p_2 + a_2) - (q_2 + a_2))^2 \\
        &= (p_1 - q_1)^2 + (p_2 - q_2)^2.
    \end{align*}
    Taking the square root of both sides gives
    \[
        d(T_A(P), T_A(Q)) = \sqrt{(p_1 - q_1)^2 + (p_2 - q_2)^2} = d(P, Q).
    \]
\end{proof}

\begin{tcolorbox}[title=Problem 12, breakable]
    Let $D(r, A)$ denote the disc of radius $r$ centered at $A$.
    Show that $D(r, A)$ is the translation of $A$ of the disc $D(r, O)$
    of radius $r$ centered at $O$.
\end{tcolorbox}

\begin{proof}
    Let $X$ be a point in the disc of radius $r$ centered at $O$.
    This means that $|X| < r$.
    The translation of $X$ by $A$ which is $X + A$, satisfies the condition 
        $|X + A - A| < r$. Thus we see that $X + A$ is at distance $<r$ from $A$,
        and hence lies in the disc of radius $r$ centered at $A$.
    Conversely, given a point $Y$ in this disc, so that $|Y - A| < r$,
        let $X = Y - A$. Then $Y = X + A$ is the translation of $X$ by $A$,
        and $|X| < r$. Therefore every point in the disc of radius $r$, centered at $A$,
        is the image $T_A$ of a point in the disc of radius $r$ centered at $O$.
    This proves our theorem
\end{proof}

\begin{tcolorbox}[title=Problem 13, breakable]
    Let $S(r, P)$ denote the circle of radius $r$ centered at $P$.

    (a) Show that the reflection of this circle through $O$ is a again a circle.
        What is the center of the reflected circle.

    (b) Show that the reflection of the disc $D(r, P)$ through $O$ is a disc.
        What is the center of this reflected disc?
\end{tcolorbox}

\begin{proof}
    Let $X$ be a point on the circle of radius $r$ centered at $P$.
    This means that $|X - P| = r$.
    The reflection of $X$ through the origin, which is $-X$, satisfies
        $|-X - (-P)| = |-(X - P)| = |X - P| = r$. 
    Thus we see that $-X$ is at distance $r$ from $-P$,
        and hence lies in the circle of radius $r$ centered at $-P$.
    Conversely, given a point $Y$ in this circle, so that $|Y - (-P)| = r$,
        let $X = -Y$. Then $Y = -X$ is the reflection of $X$ through the origin,
        and $|X - P| = |-Y - P| = |Y - (-P)| = r$. 
    Therefore every point on the circle of radius $r$, centered at $-P$,
        is the reflection of a point in the circle of radius $r$ centered at $P$.
    This proves our theorem.
\end{proof}

\textbf{Solution:}
Let $(x, y) = P$. The image of $P$ under a reflection through the origin is $(-x, -y) = -P$.

\begin{proof}
    Let $X$ be a point in the disc of radius $r$ centered at $P$.
    This means that $|X - P| < r$.
    The reflection of $X$ through the origin, which is $-X$, satisfies
        $|-X - (-P)| = |-(X - P)| = |X - P| < r$.
    Thus we see that $-X$ is at distance $<r$ from $-P$,
        and hence lies in the disc of radius $r$ centered at $-P$.
    Conversely, given a point $Y$ in this disc, so that $|Y - (-P)| < r$,
        let $X = -Y$. Then $Y = -X$ is the reflection of $X$ through the origin,
        and $|X - P| = |-Y - P| = |Y - (-P)| < r$.
    Therefore every point in the disc of radius $r$, centered at $-P$,
        is the reflection of a point in the disc of radius $r$ centered at $P$.
    This proves our theorem.
\end{proof}

\textbf{Solution:}
Same as the center of a circle reflected through the origin.

\begin{tcolorbox}[title=Problem 14, breakable]
    Let $P, Q$ be points. Write $P = Q + A$, where $A = P - Q$. Define 
    the \textbf{reflection of $P$ through $Q$} to be the point $Q - A$.
    If $R_Q$ denotes reflection through $Q$, then we have $R_Q(P) = 2 Q - P$.
    (Why?) Draw the picture, showing $P, Q, A$ and $Q - A$ to convince yourself
        that this definition corresponds to our geometric intuition.
\end{tcolorbox}

\begin{figure}[h]
    \centering
    \includegraphics[width=0.6\textwidth]{images/14.jpg}
\end{figure}

\textbf{Solution:}
From the figure we see $Q - A$ is the reflection of $P$ through $Q$.
Since $A = P - Q$, $R_Q(P) = Q - (P - Q) = Q - A$.

\begin{tcolorbox}[title=Problem 15, breakable]
    (a) Prove that reflection through a point $Q$ can be expressed in terms 
        of reflection through $O$, followed by a translation.

    (b) Let $T_A$ be translation by $A$, and $R_O$ reflection with respect 
        to the origin. Prove that the composite $T_A \circ R_O$ is equal 
        to $R_Q$ for some point $Q$. Which one?
\end{tcolorbox}

\begin{proof}
    Let $P$ be an arbitrary point.
    Let $P = (x, y)$. Reflecting through $O$ gives $R_O(P) = -P = (-x, -y)$.
    Then translating by $2Q$ gives $-P + 2Q = 2Q - P = R_Q(P)$.
\end{proof}

\begin{proof}
    Let $P$ be an arbitrary point.
    Then 
    \[(T_A \circ R_O)(P) = T_A(R_O(P)) = T_A(2O - P) = 2O - P + A = A - P\]
    Also $R_Q(P) = 2Q - P$.
    Then 
    \[R_Q(P) = (T_A \circ R_O)(P) \iff 2Q - P = A - P \iff 2Q = A \iff Q = A/2\]
\end{proof}

\begin{tcolorbox}[title=Problem 16, breakable]
    (a) Let $r$ be a positive number.
        Give an analytic definition of \textbf{dilation by $r$ with respect to a point $Q$},
        and denote this dilation by $F_{r, Q}$. To give this definition, look at 
        Exercise $14$. You may also want to look at the discussion about line segments
        in $4$. If $P$ is a point, draw the picture with $O, P, Q, P - Q$, and $F_{r, Q}(P)$.

    (b) From your definition, it should be clear that $F_{r, Q}$ can be obtained as a composite 
        of dilation with respect to $O$, and a translation. Translation by what point?
\end{tcolorbox}

\begin{definition}
    Define the \textbf{dilation by $r$ with respect to a point $Q$ of $P$} 
        be the point $Q + r(P - Q)$.
\end{definition}

\begin{figure}[h]
    \centering
    \includegraphics[width=0.6\textwidth]{images/15.png}
\end{figure}

\textbf{Solution:}
\[F_{r, Q}(P) = Q + r(P - Q) = rP - rQ + Q = rP - (r - 1)Q = F_{r, O}(P) - (r - 1)Q\]
So a translation by $- (r - 1)Q$.

\newpage
\begin{tcolorbox}[title=Problem 17, breakable]
    Let $S(r, A)$ be the circle of radius $r$ and center $A$.
    Show that the reflection of this circle through a point $Q$ is a circle.
    What is the center of this reflected circle?
    What is its radius?
    Draw a picture.
\end{tcolorbox}

\begin{proof}
    Let $X$ be a point on the circle of radius $r$ centered at $P$.
    This means that $|X - P| = r$.
    The reflection of $P$ through a point $Q$ is $2Q - P$,
        and the reflection of $X$ through $Q$ is $2Q - X$.
    Then
    \[
        |(2Q - X) - (2Q - P)| = |-(X - P)| = |X - P| = r.
    \] 
    Thus $2Q - X$ is at distance $r$ from $2Q - P$,
        and hence lies on the circle of radius $r$ centered at $2Q - P$.
    Conversely, given a point $Y$ on the circle of radius $r$ centered at $2Q - P$,
        let $X = 2Q - Y$.  Then 
    \[|X - P| = |(2Q - Y) - P| = |(2Q - P) - Y| = |Y - (2Q - P)| = r\]
    So $X$ is a point on the original circle centered at $P$,
        and $Y$ is its reflection through $Q$.
    Therefore, the reflection of the circle through $Q$ is again a circle
        with center $2Q - P$ and radius $r$.
\end{proof}

\textbf{Solution:} The center of the reflected circle is $2Q - P$, and its radius is $r$.

\begin{figure}[h]
    \centering
    \includegraphics[width=0.6\textwidth]{images/circle.png}
\end{figure}

\begin{tcolorbox}[title=Problem 18, breakable]
    The inverse of the translation $T_A$ is also a translation.
    By what? Prove your assertion.
\end{tcolorbox}

\begin{proof}
    The inverse of $T_A$ is $T_{(-A)}$.
    Let $P$ be an arbitrary point.
    Then $(T_A \circ T_{(-A)})(P) = T_A(P - A) = (P - A) + A = P$.
    Also $(T_{(-A)} \circ T_A)(P) = T_{(-A)}(P + A) = (P + A) - A = P$.
    Thus $T_A^{-1} = T_{(-A)}$.
\end{proof}

\begin{tcolorbox}[title=Problem 19, breakable]
    Let $F_r$ be dilation by a postive number $r$,
    with respect to $O$, and let $T_A$ be a translation by $A$.

    (a) Show that $F_r^{-1}$ is also a dilation. By what number?

    (c) Show that $F_r \circ T_A \circ F_r^{-1}$ is a translation.
\end{tcolorbox}

\begin{proof}
    Consider $F_{(-r)}$. Let $P$ be an arbitrary point.
    Then $(F_r \circ F_{1/r})(P) = F_r(P/r) = r \cdot P/r = P$.
    Also $(F_{1/r} \circ F_r)(P) = F_{1/r}(rP) = 1/r \cdot rP = P$.
    Thus $F_r^{-1} = F_{1/r}$.
\end{proof}

\begin{proof}
    Let $P$ be an arbitrary point.
    Then
    \[(F_r \circ T_A \circ F_r^{-1})(P) = (F_r \circ T_A)(P/r) = F_r(P/r + A) = r(P/r + A) = P + Ar = T_{rA}(P)\]
\end{proof}

\begin{tcolorbox}[title=Problem 20, breakable]
    Show that the composite of two translations is a translation.
    $T_A \circ T_B = T_C$, how would you express $C$ in terms 
    of $A$ and $B$?
\end{tcolorbox}

\begin{proof}
    Let $P$ be an arbitrary point. Then 
    \[(T_A \circ T_B)(P) = T_A(P + B) = P + B + A = T_{B + A}(P) = T_C(P)\]
    So $C = B + A$.
\end{proof}

\begin{tcolorbox}[title=Problem 21, breakable]
    Let $R$ be a reflection through the origin.

    (a) Show that $R^{-1}$ exists.

    (b) Show that $R \circ T_A \circ R^{-1}$ is a translation. By what?
\end{tcolorbox}

\begin{proof}
    Let $P$ be an arbitrary point.
    Then $(R \circ R)(P) = R(-P) = -(-P) = P$.
    Thus $R^{-1} = R$.
\end{proof}

\begin{proof}
    Let $P$ be an arbitrary point.
    Then
    \[(R \circ T_A \circ R^{-1})(P) = (R \circ T_A \circ R)(P) = (R \circ T_A)(-P) = R(-P + A) = P - A = T_{(-A)}(P)\]
\end{proof}

\begin{tcolorbox}[title=Problem 22, breakable]
    Let $A = (a_1, a_2)$ be a point. Define its
    \textbf{reflection through the x-axis} to be the point 
    $(a_1, -a_2)$. Draw $A$ and its reflection through the 
    x-axis in the following cases.

    (a) $A = (1, 2)$

    (b) $A = (-1, 3)$

    (c) $A = (-2, -4)$

    (d) $A = (5, -2)$
\end{tcolorbox}

\begin{figure}[h]
    \centering
    \includegraphics[width=0.6\textwidth]{images/graph.png}
\end{figure}

\begin{tcolorbox}[title=Problem 23, breakable]
    Prove that reflection through the x-axis is an isometry.
\end{tcolorbox}

\begin{proof}
    Let $P = (p_1, p_2)$ and $Q = (q_1, q_2)$ be arbitrary points.  
    Then 
    \[
    d(P, Q) 
        = \sqrt{(p_1 - q_1)^2 + (p_2 - q_2)^2} 
        = d_{PQ}.
    \]  
    Reflecting both points through the $x$-axis gives
    \[
    P' = (p_1, -p_2) \quad \text{and} \quad Q' = (q_1, -q_2).
    \]  
    Then 
    \[
    d(P', Q') 
        = \sqrt{(p_1 - q_1)^2 + ((-p_2) - (-q_2))^2} 
        = \sqrt{(p_1 - q_1)^2 + (p_2 - q_2)^2} 
        = d_{PQ}.
    \]  
\end{proof}

\begin{tcolorbox}[title=Problem 24, breakable]
    Define the reflection through the y-axis in a similar way,
        and prove that it is an isometry.
    Draw the points of Exercise 22, and their reflections 
        through the y-axis.
\end{tcolorbox}

\begin{proof}
    Let $P = (p_1, p_2)$ and $Q = (q_1, q_2)$ be arbitrary points.  
    Then 
    \[
    d(P, Q) 
        = \sqrt{(p_1 - q_1)^2 + (p_2 - q_2)^2} 
        = d_{PQ}.
    \]  
    Reflecting both points through the $y$-axis gives
    \[
    P' = (-p_1, p_2) \quad \text{and} \quad Q' = (-q_1, q_2).
    \]  
    Then 
    \[
    d(P', Q') 
        = \sqrt{((-p_1) - (-q_1))^2 + (p_2 - q_2)^2} 
        = \sqrt{(p_1 - q_1)^2 + (p_2 - q_2)^2} 
        = d_{PQ}.
    \]  
\end{proof}

\begin{figure}[h]
    \centering
    \includegraphics[width=0.6\textwidth]{images/graph1.png}
\end{figure}

\begin{tcolorbox}[title=Problem 25, breakable]
    Recall that a \textbf{fixed point} of a mapping $F$
    is a point $P$ such that $F(P) = P$. Using the coordinate
    definition, determine the fixed points of

    (a) translation 

    (b) reflection through $O$

    (c) reflection through an arbitrary point $P$

    (d) reflection through the x-axis and through the y-axis
\end{tcolorbox}

\begin{proof}
    Let $P$ be an arbitrary point. If $T_0$, then all points are fixed points.  
    Consider $T_0(P) = P + 0 = P$.  
    Suppose $T_A$ with $A \ne 0$.  
    Then $T_A(P) = P + A \ne P$.
\end{proof}

\begin{proof}
    The point $O$ is a fixed point.  
    Consider $F_O(O) = 2O - O = O$. 
\end{proof}

\begin{proof}
    The point $P$ is a fixed point.  
    Consider $F_P(P) = 2P - P = P$.
\end{proof}

\begin{proof}
    All points on the x-axis are fixed points with respect to x-axis reflection.  
    Let $X = (x_1, x_2)$ be an arbitrary point on the x-axis.  
    Then $X = (x_1, 0) = (x_1, -x_2)$.  

    All points on the y-axis are fixed points with respect to y-axis reflection.  
    Let $Y = (y_1, y_2)$ be an arbitrary point on the y-axis.  
    Then $Y = (0, y_2) = (-y_1, y_2)$.
\end{proof}

\newpage
\begin{tcolorbox}[title=Problem 26, breakable]
    (a) Let 
    \[E_1 = (1, 0) \text{ and } E_2 = (0, 1)\]
    We call $E_1$ and $E_2$ the \textbf{basic units points}
    of the plane. Plot these points. If $A = (a_1, a_2)$,
    prove that 
    \[A = a_1 E_1 + a_2 E_2\]

    (b) If $c$ is a number, what are the coordinates of $cE_1, cE_2$?
\end{tcolorbox}

\begin{figure}[h]
    \centering
    \includegraphics[width=0.6\textwidth]{images/basis.png}
\end{figure}

\begin{proof}
    Suppose $A = (a_1, a_2)$.
    Consider 
    \[a_1 E_1 + a_2 E_2 = (a_1 \cdot 1, a_1 \cdot 0) + (a_2 \cdot 0, a_2 \cdot 1) = (a_1, 0) + (0, a_2) = (a_1, a_2) = A\]
\end{proof}

\textbf{Solution (b):}
\[c E_1 = (c \cdot 1, c \cdot 0) = (c, 0) \text{ and } c E_2 = (c \cdot 0, c \cdot 1) = (0, c)\]

\begin{tcolorbox}[title=Problem 29, breakable]
    Given a number $r > 0$ and a point $A$, we can 
    define the corners of a square, having sides of 
    length $r$ parallel to the axes, and $A$ as 
    its lower left-hand corner, to be the points 
    \[A, A + rE_1, A + rE_2, A + rE_1 + rE_2\]
    Let $s$ be a postitive number.
    Show that if these four points are dilated
        by multiplication with $s$, they again form 
        the corners of a square. What are the corners 
        of this dilated square.
\end{tcolorbox}

\begin{proof}
    The dilation by multiplication with $s$ sends each point $X$ to $sX$.
    Applying this to the four corners of the square gives
    \[
    A \mapsto sA,\qquad
    A + rE_1 \mapsto sA + srE_1,\qquad
    A + rE_2 \mapsto sA + srE_2,\qquad
    A + rE_1 + rE_2 \mapsto sA + srE_1 + srE_2.
    \]
    These four points again form a square, since the sides
    remain parallel to the coordinate axes and each side has length $sr$.
    Thus the corners of the dilated square are
    \[sA,\; sA + srE_1,\; sA + srE_2,\; sA + srE_1 + srE_2\]
\end{proof}

\begin{tcolorbox}[title=Problem 30, breakable]
    Let the notation be as in Exercise 29. What is the 
    area of the dilated square? How does it compare 
    with the area of the original square?
\end{tcolorbox}

\textbf{Solution:}
The original square has side length $r$ and area $r^{2}$.
The dilated square has side length $sr$ and area $s^{2} r^{2}$.
Thus the area of the square is multiplied by $s^{2}$ under the dilation.

\begin{tcolorbox}[title=Problem 31, breakable]
    Let $A$ be a point and $r, s$ positive numbers.
    How would you define the corners of a rectangle whose sides 
        are parallel to the axes, with $A$ as the lower left-hand 
        corner, and such that the vertical side has length $r$
        and the horizontal side has length $s$.
\end{tcolorbox}

\textbf{Solution:}
A rectangle with lower left-hand corner $A$, vertical side of length $r$,
horizontal side of length $s$, and sides parallel to the coordinate axes
has corners
\[
A,\quad A + sE_1,\quad A + rE_2,\quad A + sE_1 + rE_2.
\]

\begin{tcolorbox}[title=Problem 32, breakable]
    Let $t$ be a positive number. What is the effect 
    of dilation by $t$ on the sides and area of the rectangle
    in Exercise 31.
\end{tcolorbox}

\textbf{Solution:}
Dilation by $t$ multiplies each side by $t$. Thus the horizontal
side of length $s$ becomes $ts$, and the vertical side of length $r$
becomes $tr$. The area, originally $rs$, becomes
\[
(tr)(ts) = t^{2}rs.
\]
Thus the area is multiplied by $t^{2}$.