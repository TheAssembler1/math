\subsection{Coordinate Systems}

\begin{tcolorbox}[title=Problem 3, breakable]
    Let $(x, y)$ be the coordinates of a point
    in the second quadrant.
    Is $x$ positive or negative?
    Is $y$ positive or negative?
\end{tcolorbox}

\textbf{Solution:}

The $x$ is negative.
The $y$ is positive.

\begin{tcolorbox}[title=Problem 4, breakable]
    Let $(x, y)$ be the coordinates of a point in the 
    third quadrant. Is $x$ positive or negative.
    Is $y$ positive or negative.
\end{tcolorbox}

The $x$ is negative.
The $y$ is negative.

\subsection{Distance Between Points}

\begin{tcolorbox}[title=Problem 11, breakable]
    Prove that if $d(P, Q) = 0$, then $P = Q$. Thus we have now 
    proved two of the basic properties of distance.
\end{tcolorbox}

\begin{proof}
    Suppose $P \ne Q$. Let $(x_1, x_2) = P$ and $(y_1, y_2) = Q$.
    Either $x_1 \ne y_1$ or $x_2 \ne y_2$.

    Suppose $x_1 \ne y_1$. It follows that $x_1 - y_1 \ne 0$
    and therefore, since $(x_2 - y_2)^2 \ge 0$, $\sqrt{(x_1 - y_1)^2 + (x_2 - y_2)^2} \ne 0$.
    Thus $d(P, Q) \ne 0$.

    Suppose $x_2 \ne y_2$. It follows similarly that $d(P, Q) \ne 0$.

    Therefore, if $d(P, Q) = 0$, then $P = Q$.
\end{proof}

\begin{tcolorbox}[title=Problem 12, breakable]
    Let $A = (a_1, a_2)$ and $B = (b_1, b_2)$.
    Let $r$ be a positive number.
    Write down the formula for $d(A, B)$.
    Define the \textbf{dilation} $r A$ be 
    \[r A = (r a_1, r a_2)\]
    For instance, if $A = (-3, 5)$ and $r = 7$,
    then $r A = (-21, 35)$. Prove in general that 
    \[d(r A, r B) = r \cdot d(A, B)\]
\end{tcolorbox}

\textbf{Solution:}

\[d(A, B) = \sqrt{(a_1 - b_1)^2 + (a_2 - b_2)^2}\]

\begin{proof}
    \begin{align*}
        d(rA, rB) &= \sqrt{(r a_1 - r b_1)^2 + (r a_2 - r b_2)^2} \\
                  &= \sqrt{(r^2 a_1^2 - r^2 a_1 b_1 - r^2 a_1 b_1 + r^2 b_1^2) + (r^2 a_2^2 - r^2 a_2 b_2 - r^2 a_2 b_2 + r^2 b_2^2)} \\
                  &= \sqrt{r^2 (a_1^2 - 2 a_1 b_1 + b_1^2 +  a_2^2 - 2 a_2 b_2 + b_2^2)} \\
                  &= r \sqrt{(a_1 - b_1)^2 + (a_2 - b_2)^2} \\
                  &= r d(A, B)
    \end{align*}
    Therefore, $d(rA, rB) = r d(A, B)$.
\end{proof}

\subsection{Equations of a Circle}

\begin{tcolorbox}[title=Problem 19, breakable]
    (a) Write down the equation for a sphere of radius $1$ centered
        at the origin in $3-$space, in terms of coordinates $(x, y, z)$.

    (b) Same question for a sphere of radius $3$.

    (c) Same question for a sphere of radius $r$.
\end{tcolorbox}

\textbf{Solution (a):}
\[x^2 + y^2 + z^2 = 1\]
\textbf{Solution (b):}
\[x^2 + y^2 + z^2 = 9\]
\textbf{Solution (c):}
\[x^2 + y^2 + z^2 = r^2\]

\subsection{Rational Points on a Circle}

\begin{tcolorbox}[title=Problem 2, breakable]
    Prove that if $s$, $t$ are real numbers such that $0 \le s < t$, then 
    \[\frac{1 - s^2}{1 + s^2} > \frac{1 - t^2}{1 + t^2}\]
    [Hint: Prove appropriate inequalities for the numerators and denominators,
    before taking the quotient.] This proves that different values for $t > 0$
    already give different values for $x$.
\end{tcolorbox}

\begin{proof}
    Suppose $s, t$ are real numbers such that $0 \le s < t$.

    Since $s < t$ it follows that $s^2 < t^2$ and $s^2 - t^2 < 0$.
    Then 
    \begin{align*}
        (1 - s^2) - (1 - t^2) 
            &= 1 - s^2 - 1 + t^2 \\
            &= -s^2 + t^2 \\
            &= -(s^2 - t^2)
    \end{align*}
    Since $s^2 - t^2 < 0$, it follows that $-(s^2 - t^2) > 0$.
    Thus $1 - s^2 > 1 - t^2$.

    Also
    \begin{align*}
        (1 + s^2) - (1 + t^2) 
            &= 1 + s^2 - 1 - t^2 \\
            &= s^2 - t^2 < 0
    \end{align*}
    Thus $1 + s^2 < 1 + t^2$.
    
    Since $1 - s^2 > 1 - t^2$ and $1 + s^2 < 1 + t^2$ 
        it follows that $(1 - s^2)(1 + t^2) > (1 - t^2)(1 + s^2)$.
    Therefore $\frac{1 - s^2}{1 + s^2} > \frac{1 - t^2}{1 + t^2}$
\end{proof}

\begin{tcolorbox}[title=Problem 4, breakable]
    When $t$ becomes very large positive, what happens to 
    \[\frac{1 - t^2}{1 + t^2}\]
    When $t$ becomes very large negative, what happens to 
    \[\frac{1 - t^2}{1 + t^2}\]
    Substitute large values of $t$, like $10,000$ or $-10,000$, to get 
    a feeling for what happens.
\end{tcolorbox}

\textbf{Solution:}

Let $t = 10000$ then
\[\frac{1 - t^2}{1 + t^2} = \frac{1 - (10000)^2}{1 + (10000)^2} = -0.9999998\]
As $t$ becomes very large positive $\frac{1 - t^2}{1 + t^2}$ approaches $-1$.

At $t = -10000$ it also approaches $-1$ since $t$ is squared it doesn't make a difference.

\begin{tcolorbox}[title=Problem 5, breakable]
    Analyze what happens to 
    \[\frac{2t}{1 + t^2}\]
    when $t \le 0$ and when $t$ becomes very large negative.
    Next analyze what happens when $t \ge 0$ and $t$ becomes 
    very large positive.
\end{tcolorbox}

Let $t = -10000$ then
\[\frac{2t}{1 + t^2} = \frac{2(-10000)}{1 + (-10000)^2} = −0.0002\]
As $t$ becomes very large negative $\frac{2t}{1 + t^2}$ approaches $0$.

Let $t = 10000$ then
\[\frac{2t}{1 + t^2} = \frac{2(10000)}{1 + (10000)^2} = 0.0002\]
As $t$ becomes very large positive $\frac{2t}{1 + t^2}$ approaches $0$.