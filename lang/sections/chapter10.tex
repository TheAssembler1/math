\subsection{Rays}

\begin{tcolorbox}[title=Problem 1, breakable]
     Let $P, Q$ be the indicated points.
     Give the coordinates of the point 

     (a) halfway

     (b) one-third of the way

     (c) two-thirds of the way 

     between $P$ and $Q$.

     $P = (1, 5)$, $Q = (3, -1)$
\end{tcolorbox}

\textbf{Solution:}
We first compute the segment (as a vector based at $O$) between $P$ and $Q$:
\[
Q - P = (3 - 1,\,-1 - 5) = (2,\,-6).
\]
Then we scale this vector by $1/2$, $1/3$, and $2/3$:
\[
\frac12(2,-6) = (1,-3),\qquad
\frac13(2,-6) = \left(\frac{2}{3}, -2\right),\qquad
\frac23(2,-6) = \left(\frac{4}{3}, -4\right).
\]
Finally, we add each of these to $P$:
\[
P + (1,-3) = (2,2),\qquad
P + \left(\tfrac{2}{3}, -2\right) = \left(\tfrac{5}{3}, 3\right),\qquad
P + \left(\tfrac{4}{3}, -4\right) = \left(\tfrac{7}{3}, 1\right).
\]

\begin{tcolorbox}[title=Problem 5, breakable]
    Prove that the image of a line segment $\overline{PQ}$ under 
    translation $T_A$ is also a line segment. What are the 
    end points of this image.
\end{tcolorbox}

\begin{proof}
    A line segment from $P$ to $Q$ can be written as
    \[
        \overline{PQ} = P + t(Q - P), \quad 0 \le t \le 1.
    \]
    Then
    \[
        T_A(P + t(Q - P)) = (P + t(Q - P)) + A = (P + A) + t((Q + A) - (P + A)).
    \]
    This shows that the image is the line segment $\overline{(P + A)(Q + A)}$
    with endpoints $P + A$ and $Q + A$.
\end{proof}

\begin{tcolorbox}[title=Problem 6, breakable]
    Let $P, Q, M$ be the indicated points.
    In Exercise $6$, find the point $N$
        such that $\overrightarrow{PQ}$ has the same direction
        as $\overrightarrow{MN}$ and such that the length of 
        $\overrightarrow{MN}$ is

    (a) $3$ times the length of $\overrightarrow{PQ}$

    (b) one-third the length of $\overrightarrow{PQ}$

    $P = (1, 4)$, $Q = (1, -5)$, $M = (-2, 3)$
\end{tcolorbox}

\textbf{Solution (a):}
We require $N - M = 3(P - Q)$.
So $N = 3(P - Q) + M
    \iff N = 3((1, 4) - (1, -5)) + (-2, 3)
    \iff N = 3(0, 9) + (-2, 3)
    \iff N = (0, 27) + (-2, 3)
    \iff N = (-2, 30)$.

\textbf{Solution (b):}
We require $N - M = (1/3)(P - Q)$.
So $N = (1/3)(P - Q) + M
    \iff N = (1/3)((1, 4) - (1, -5)) + (-2, 3)
    \iff N = (1/3)(0, 9) + (-2, 3)
    \iff N = (0, 3) + (-2, 3)
    \iff N = (-2, 6)$.

\begin{tcolorbox}[title=Problem 10, breakable]
    Let $F$ be 

    (a) translation $T_A$

    (b) reflection through $O$

    (c) reflection through the x-axis

    (d) reflection through the y-axis

    (e) dilation by a number $r > 0$

    In each one of these cases, prove that the image under $F$ of (i)
    is a segment, (ii) a ray, is again (i) a segment, (ii) a ray, 
    respectively. Thus you really have 10 cases to consider $10 = 5 \times 2$,
    but they are all easy.
\end{tcolorbox}

\begin{proof}
    We first show the image of a segment under $F = T_A$ is a segment.
    Let $\overline{PQ}$ be a segment.
    Consider 
    \[F(\overline{PQ}) = T_A(\overline{PQ}) 
            = T_A(P + t(Q - P))\]
    Where $0 \le t \le 1$. Then
    \[P + t(Q - P) + A 
            = (P + A) +  t((Q + A) - (P + A)) 
            = \overline{P + A, Q + A}\]

    We now show the image of a ray under $F = T_A$ is a ray.
    Let $\overrightarrow{PQ}$ be a ray.
    Consider 
    \[F(\overrightarrow{PQ}) = T_A(\overrightarrow{PQ}) 
            = T_A(P + t(Q - P))\]
    Where $t \ge 0$. Then
    \[P + t(Q - P) + A 
            = (P + A) +  t((Q + A) - (P + A)) 
            = \overrightarrow{P + A, Q + A}\]
\end{proof}

\begin{proof}
    We first show the image of a segment under reflection through $O$ is a segment.
    Let $\overline{PQ}$ be a segment.
    Consider
    \[
        F(\overline{PQ}) = -\overline{PQ} = -(P + t(Q - P))
    \]
    where $0 \le t \le 1$. Then
    \[
        -P - t(Q - P) = (-P) + t((-Q) - (-P)) = \overline{-P, -Q}.
    \]

    We now show the image of a ray under reflection through $O$ is a ray.
    Let $\overrightarrow{PQ}$ be a ray.
    Consider
    \[
        F(\overrightarrow{PQ}) = -(P + t(Q - P))
    \]
    where $t \ge 0$. Then
    \[
        -P - t(Q - P) = (-P) + t((-Q) - (-P)) = \overrightarrow{-P, -Q}.
    \]
\end{proof}

\begin{proof}
    We first show the image of a segment under reflection through the x-axis is a segment.
    Let $\overline{PQ}$ be a segment with $P = (p_x, p_y)$, $Q = (q_x, q_y)$.
    Consider
    \[
        F(\overline{PQ}) = (p_x, -p_y) + t((q_x, -q_y) - (p_x, -p_y)), \quad 0 \le t \le 1
    \]
    which equals
    \[
        \overline{(p_x, -p_y), (q_x, -q_y)}.
    \]

    We now show the image of a ray under reflection through the x-axis is a ray.
    Let $\overrightarrow{PQ}$ be a ray. Then
    \[
        F(\overrightarrow{PQ}) = (p_x, -p_y) + t((q_x, -q_y) - (p_x, -p_y)), \quad t \ge 0
    \]
    which equals
    \[
        \overrightarrow{(p_x, -p_y), (q_x, -q_y)}.
    \]
\end{proof}

\begin{proof}
    We first show the image of a segment under reflection through the y-axis is a segment.
    Let $\overline{PQ}$ be a segment with $P = (p_x, p_y)$, $Q = (q_x, q_y)$.
    Consider
    \[
        F(\overline{PQ}) = (-p_x, p_y) + t((-q_x, q_y) - (-p_x, p_y)), \quad 0 \le t \le 1
    \]
    which equals
    \[
        \overline{(-p_x, p_y), (-q_x, q_y)}.
    \]

    We now show the image of a ray under reflection through the y-axis is a ray.
    Let $\overrightarrow{PQ}$ be a ray. Then
    \[
        F(\overrightarrow{PQ}) = (-p_x, p_y) + t((-q_x, q_y) - (-p_x, p_y)), \quad t \ge 0
    \]
    which equals
    \[
        \overrightarrow{(-p_x, p_y), (-q_x, q_y)}.
    \]
\end{proof}

\begin{proof}
    We first show the image of a segment under dilation by a number $r > 0$ is a segment.
    Let $\overline{PQ}$ be a segment. Consider
    \[
        F(\overline{PQ}) = r(P + t(Q - P)), \quad 0 \le t \le 1.
    \]
    Then
    \[
        rP + t\, r(Q - P) = (rP) + t((rQ) - (rP)) = \overline{rP, rQ}.
    \]

    We now show the image of a ray under dilation by a number $r > 0$ is a ray.
    Let $\overrightarrow{PQ}$ be a ray. Consider
    \[
        F(\overrightarrow{PQ}) = r(P + t(Q - P)), \quad t \ge 0.
    \]
    Then
    \[
        rP + t\, r(Q - P) = (rP) + t((rQ) - (rP)) = \overrightarrow{rP, rQ}.
    \]
\end{proof}

\begin{tcolorbox}[title=Problem 11, breakable]
    After you have read the definition of a straight line 
    in the next section, prove that the image under $F$ of a 
    straight line is again a straight line.
    [Here $F$ is any one of the mappings of Exercise 10.]
\end{tcolorbox}

\begin{proof}
    We show the image of a line under $F = T_A$ is a line.
    Let $\overline{PQ}$ be a line.
    Consider 
    \[F(\overline{PQ}) = T_A(\overline{PQ}) 
            = T_A(P + t(Q - P))\]
    Where $t \in \mathbb{R}$. Then
    \[P + t(Q - P) + A 
            = (P + A) +  t((Q + A) - (P + A)) 
            = \overline{P + A, Q + A}\]
\end{proof}

\begin{tcolorbox}[title=Problem 12, breakable]
    Give a definition for two located vectors to have \textbf{opposite direction}.
    Similary if $A \ne O$ and $B \ne O$, give a definition for $A$ and $B$ to 
    have opposite direction. Draw the corresponding pictures.
\end{tcolorbox}

\begin{definition}
    Let $\overrightarrow{AB}$ and $\overrightarrow{CD}$ be two vectors.
    We say that $\overrightarrow{AB}$ has \textbf{opposite direction} to $\overrightarrow{CD}$
    if and only if there exists a positive scalar $r > 0$ such that
    \[
        \overrightarrow{AB} + r\,\overrightarrow{CD} = \overrightarrow{0}.
    \]
\end{definition}

\begin{definition}
    Let $A$ and $B$ be points such that $A \ne O$ and $B \ne O$.
    We say that $A$ and $B$ have \textbf{opposite direction} if and only if 
    $\overrightarrow{OA}$ has opposite direction to $\overrightarrow{OB}$ in the sense above.
\end{definition}

\begin{tikzpicture}[scale=1]
    % Axes
    \draw[->] (-1,0) -- (5,0) node[right] {$x$};
    \draw[->] (0,-1) -- (0,5) node[above] {$y$};

    % Points for located vectors
    \coordinate (A) at (1,1);
    \coordinate (B) at (3,3);
    \coordinate (C) at (4,1);
    \coordinate (D) at (2,-1);

    % Vectors AB and CD (opposite)
    \draw[->, thick, blue] (A) -- (B) node[midway, above left] {$\overrightarrow{AB}$};
    \draw[->, thick, red] (C) -- (D) node[midway, above right] {$\overrightarrow{CD}$};

    % Points from origin
    \coordinate (O) at (0,0);
    \coordinate (E) at (3,1);
    \coordinate (F) at (-3/2, -1/2);

    % Vectors OA and OB (opposite)
    \draw[->, thick, green] (O) -- (E) node[midway, above] {$\overrightarrow{OA}$};
    \draw[->, thick, orange] (O) -- (F) node[midway, below] {$\overrightarrow{OB}$};
\end{tikzpicture}

\subsection{Lines}

\begin{tcolorbox}[title=Problem 1, breakable]
    For Exercise 1: (a) write down the parametric representations
    of the lines passing through the indicated points $P$ and $Q$,
    (b) find the point of intersection of the line and the x-axis,
    (c) find the point of intersection of the line and the y-axis.

    \[P = (1, -1), Q = (3, 5)\]
\end{tcolorbox}

\textbf{Solution (a):}
\[P + t(Q - P) = (1, -1) + t((3, 5) - (1, -1)) = (1, -1) + t(2, 6)\]
Thus
\[x = 1 + 2t, \quad y = -1 + 6t\]

\textbf{Solution (b):}
We require $y = 0$, thus $0 = -1 + 6t \Rightarrow t = \frac{1}{6}$.
Then plugging in we see $(x, y) = (1 + 2(\frac{1}{6}), 0) = (\frac{4}{3}, 0)$

\textbf{Solution (c):}
We require $x = 0$, thus $0 = 1 + 2t \Rightarrow t = -\frac{1}{2}$.
Then plugging in we see $(x, y) = (0, -1 + 6(-\frac{1}{2})) = (0, -4)$

\begin{tcolorbox}[title=Problem 12, breakable]
    Let $A = (a_1, a_2)$ and $B = (b_1, b_2)$. Assume $A \ne O$ 
        and $B \ne O$.
    Prove that $A$ is parallel to $B$ if and only if 
    \[a_1 b_2 - a_2 b_1 = 0\]
\end{tcolorbox}

\begin{proof}
    ($\longrightarrow$) Suppose $A$ is parallel to $B$.
    Thus $tA = c(tB)$ thus $t(A - cB) = 0 \implies A - cB = 0$.
    Thus $(a_1, a_2) = c(b_1, b_2)$ and it follows that 
        $a_1 = c b_1$ and $a_2 = c b_2$.
    Then $a_1 b_2 - a_2 b_1 =  c b_1 b_2 - c b_2 b_1 = 0$.

    ($\longleftarrow$) Since $B \ne O$ either $b_1 \ne 0$
        or $b_2 \ne 0$.

    (\textbf{Case 1:} $b_1 \ne 0$) 
    Let $c = \frac{a_1}{b_1}$.
    Then $A = c(B) \iff (a_1, a_2) = \frac{a_1}{b_1} (b_1, b_2)$
    Thus 
    \[a_1 = \frac{a_1}{b_1} b_1 = a_1 \text{ and } a_2 = \frac{a_1}{b_1} b_2\]
    Now $a_2 = \frac{a_1}{b_1} b_2 \iff b_1 a_2 = a_1 b_2$ which holds
        since $a_1 b_2 - a_2 b_1 = 0$.
    
    (\textbf{Case 2:} $b_2 \ne 0$)  
    Let $c = \frac{a_2}{b_2}$.
    Then $A = cB \iff (a_1, a_2) = \frac{a_2}{b_2} (b_1, b_2)$.
    Thus 
    \[a_2 = \frac{a_2}{b_2} b_2 = a_2 \text{ and } a_1 = \frac{a_2}{b_2} b_1\]
    Now $a_1 = \frac{a_2}{b_2} b_1 \iff b_2 a_1 = a_2 b_1$, which holds
        since $a_1 b_2 - a_2 b_1 = 0$.
\end{proof}

\begin{tcolorbox}[title=Problem 13, breakable]
    Prove: If two lines are not parallel, then they have exactly 
    one point in common. [Hint: Let the two lines by represented 
    parametrically by 
    \[\{P + tA\}_{t \in R} = \{(p_1, p_2) + t(a_1, a_2)_{t \in R}\}\]
    \[\{Q + sB\}_{s \in R} = \{(q_1, q_2) + s(b_1, b_2)_{s \in R}\}\]
    Write down the general system of two equations for $s$ and $t$
    and show that it can be solved.]
\end{tcolorbox}

\begin{proof}
    Suppose the two lines are not parallel.
    There are two cases.
    Either $a_1 = 0$ or $a_1 \ne 0$.
    
        (\textbf{Case 1:} $a_1 = 0$)
    Since the lines are not parallel, we must have $b_1 \ne 0$.  
    Then the lines intersect if and only if
    \[
        \text{\textcircled{1}} \; p_1 = q_1 + s b_1
        \quad \text{and} \quad 
        \text{\textcircled{2}} \; p_2 + t a_2 = q_2 + s b_2
    \]
    Solving for $s$ in \textcircled{1} shows
        $s = \frac{p_1 - q_1}{b_1}$.  
    Substituting into \textcircled{2} shows
    \[
        p_2 + t a_2 = q_2 + \left(\frac{p_1 - q_1}{b_1}\right) b_2
    \]
    Solving for $t$ shows
    \[
        t = \frac{q_2 + \frac{b_2 (p_1 - q_1)}{b_1} - p_2}{a_2}
    \]
    Since $A \ne O$, $a_2 \ne 0$, therefore $t$ is defined.
    
    (\textbf{Case 2:} $a_1 \ne 0$)
    The lines intersect if and only if  
    \[
        \text{\textcircled{1}} \; p_1 + t a_1 = q_1 + s b_1
        \quad \text{and} \quad 
        \text{\textcircled{2}} \; p_2 + t a_2 = q_2 + s b_2
    \]
    Solving for $t$ in \textcircled{1} shows 
        $t = \frac{q_1  + s b_1 - p_1}{a_1}$.
    Substituting into \textcircled{2} shows 
    \[
        p_2 + \left(\frac{q_1  + s b_1 - p_1}{a_1}\right)a_2 = q_2 + s b_2
    \]
    Multiplying through by $a_1$ shows 
    \[
        a_1 p_2 + (q_1 + s b_1 - p_1)a_2 = a_1 q_2 + a_1 s b_2
    \]
    Solving for $s$ shows 
    \[
        s = \frac{a_1(q_2 - p_2) - a_2(q_1 - p_1)}{a_2 b_1 - a_1 b_2}
    \]
    Since the lines are not parallel, by Problem 13,
        $a_2 b_1 - a_1 b_2 \ne 0$.
    Thus $s$ is defined. Substituting into \textcircled{1} shows 
    \[
        p_1 + t a_1 = q_1 + \left(\frac{a_1(q_2 - p_2) - a_2(q_1 - p_1)}{a_2 b_1 - a_1 b_2}\right) b_1
    \]
    Then solving for $t$ shows 
    \[
        t = \frac{q_1 + \left(\frac{a_1(q_2 - p_2) - a_2(q_1 - p_1)}{a_2 b_1 - a_1 b_2}\right) b_1 - p_1}{a_1}
    \]
    Since $a_1 \ne 0$, $t$ is defined.

    Therefore, in either case, the lines intersect at a unique point, as required.
\end{proof}

\begin{tcolorbox}[title=Problem 18, breakable]
    (Slightly harder.) Let $S$ be the circle of radius $r > 0$
    centered at the origin. Let $P = (p, q)$ be a point such that 
    \[p^2 + q^2 = r\]
    In other words, $P$ is a point in the disc of radius $r$ centered 
    at $O$. Show that any line passing through $P$ must intersect the 
    circle, and find the points of intersection. [Hint: Write the line 
    in the form \[P + tA\] where $A = (a, b)$, substitute  in the equation 
    of the circle, and find the coordinates of the points of intersection 
    in terms of $p, q, a, b$. Show that the quantity you get under the 
    square root sign is $\ge 0$.]
\end{tcolorbox}

\begin{proof}
    Let $A$ be an arbitrary point such that $A \ne P$.
    Consider the line $P + tA$ for $t \in \mathbb{R}$
    and let $A = (a, b)$. Then the parametric form is 
    \[x = p + ta, \quad y = q + tb.\]
    These points must satisfy the circle equation $x^2 + y^2 = r^2$.
    Substituting shows 
    \begin{align*}
        &(p + ta)^2 + (q + tb)^2 = r^2 \\
        \iff &p^2 + 2 p t a + t^2 a^2 + q^2 + 2 q t b + t^2 b^2 = r^2 \\
        \iff &r^2 + 2 p t a + t^2 a^2 + 2 q t b + t^2 b^2 = r^2 &&\text{since } p^2 + q^2 = r^2 \\
        \iff &2 t(p a + q b) + t^2 (a^2 + b^2) = 0
    \end{align*}
    Solving the quadratic equation with $a = a^2 + b^2, b = 2(pa + qb), c = 0$ shows 
    \[t = \frac{-2(pa + qb) \pm \sqrt{(2(pa + qb))^2}}{2(a^2 + b^2)}.\]
    Expanding $(2(pa + qb))^2$ shows
    \[(2(pa + qb))^2 = 4(pa + qb)^2\]
    Clearly $(pa + qb)^2 \ge 0$.
\end{proof}


\subsection{Ordinary Equation for a Line}

\begin{tcolorbox}[title=Problem 1, breakable]
    Find the ordinary equation of the line $\{P + tA\}_{t \in \mathbb{R}}$
    of the following.
    \[P = (3, 1), A = (7, 2)\]
\end{tcolorbox}

\textbf{Solution:}
The parametric equations for $x,y$ are 
\[x = 3 + 7t \text{ and } y = 1 + 2t\]
Multiplying $x$ by $-2$ and $y$ by $7$ shows 
\[-2x + 7y = -2(3 + 7t) + 7(1 + 2t) = -6 - 14t + 7 + 14 t = 1\]
Thus $-2x + 7y = 1$.

\begin{tcolorbox}[title=Problem 7, breakable]
    Find the ordinary equation of the line $\{P + tA\}_{t \in \mathbb{R}}$
    of the following.
    \[P = (1, 1), A = (1, 1)\]
\end{tcolorbox}

The parametric equations for $x,y$ are 
\[x = 1 + t \text{ and } y = 1 + t\]
Multiplying $y$ by $-1$ shows 
\[x + -y = 1 + t -(1 + t) = 0\]
Thus $x - y = 0$.