\subsection{Rays}

\begin{tcolorbox}[title=Problem 1, breakable]
     Let $P, Q$ be the indicated points.
     Give the coordinates of the point 

     (a) halfway

     (b) one-third of the way

     (c) two-thirds of the way 

     between $P$ and $Q$.

     $P = (1, 5)$, $Q = (3, -1)$
\end{tcolorbox}

\textbf{Solution:}
We first compute the segment (as a vector based at $O$) between $P$ and $Q$:
\[
Q - P = (3 - 1,\,-1 - 5) = (2,\,-6).
\]
Then we scale this vector by $1/2$, $1/3$, and $2/3$:
\[
\frac12(2,-6) = (1,-3),\qquad
\frac13(2,-6) = \left(\frac{2}{3}, -2\right),\qquad
\frac23(2,-6) = \left(\frac{4}{3}, -4\right).
\]
Finally, we add each of these to $P$:
\[
P + (1,-3) = (2,2),\qquad
P + \left(\tfrac{2}{3}, -2\right) = \left(\tfrac{5}{3}, 3\right),\qquad
P + \left(\tfrac{4}{3}, -4\right) = \left(\tfrac{7}{3}, 1\right).
\]

\begin{tcolorbox}[title=Problem 5, breakable]
    Prove that the image of a line segment $\overline{PQ}$ under 
    translation $T_A$ is also a line segment. What are the 
    end points of this image.
\end{tcolorbox}

\begin{proof}
    A line segment from $P$ to $Q$ can be written as
    \[
        \overline{PQ} = P + t(Q - P), \quad 0 \le t \le 1.
    \]
    Then
    \[
        T_A(P + t(Q - P)) = (P + t(Q - P)) + A = (P + A) + t((Q + A) - (P + A)).
    \]
    This shows that the image is the line segment $\overline{(P + A)(Q + A)}$
    with endpoints $P + A$ and $Q + A$.
\end{proof}

\begin{tcolorbox}[title=Problem 6, breakable]
    Let $P, Q, M$ be the indicated points.
    In Exercise $6$, find the point $N$
        such that $\overrightarrow{PQ}$ has the same direction
        as $\overrightarrow{MN}$ and such that the length of 
        $\overrightarrow{MN}$ is

    (a) $3$ times the length of $\overrightarrow{PQ}$

    (b) one-third the length of $\overrightarrow{PQ}$

    $P = (1, 4)$, $Q = (1, -5)$, $M = (-2, 3)$
\end{tcolorbox}

\textbf{Solution (a):}
We require $N - M = 3(P - Q)$.
So $N = 3(P - Q) + M
    \iff N = 3((1, 4) - (1, -5)) + (-2, 3)
    \iff N = 3(0, 9) + (-2, 3)
    \iff N = (0, 27) + (-2, 3)
    \iff N = (-2, 30)$.

\textbf{Solution (b):}
We require $N - M = (1/3)(P - Q)$.
So $N = (1/3)(P - Q) + M
    \iff N = (1/3)((1, 4) - (1, -5)) + (-2, 3)
    \iff N = (1/3)(0, 9) + (-2, 3)
    \iff N = (0, 3) + (-2, 3)
    \iff N = (-2, 6)$.

\newpage
\begin{tcolorbox}[title=Problem 10, breakable]
    Let $F$ be 

    (a) translation $T_A$

    (b) reflection through $O$

    (c) reflection through the x-axis

    (d) reflection through the y-axis

    (e) dilation by a number $r > 0$

    In each one of these cases, prove that the image under $F$ of (i)
    is a segment, (ii) a ray, is again (i) a segment, (ii) a ray, 
    respectively. Thus you really have 10 cases to consider $10 = 5 \times 2$,
    but they are all easy.
\end{tcolorbox}

\begin{proof}
    We first show the image of a segment under $F = T_A$ is a segment.
    Let $\overline{PQ}$ be a segment.
    Consider 
    \[F(\overline{PQ}) = T_A(\overline{PQ}) 
            = T_A(P + t(Q - P))\]
    Where $0 \le t \le 1$. Then
    \[P + t(Q - P) + A 
            = (P + A) +  t((Q + A) - (P + A)) 
            = \overline{P + A, Q + A}\]

    We now show the image of a ray under $F = T_A$ is a ray.
    Let $\overrightarrow{PQ}$ be a ray.
    Consider 
    \[F(\overrightarrow{PQ}) = T_A(\overrightarrow{PQ}) 
            = T_A(P + t(Q - P))\]
    Where $t \ge 0$. Then
    \[P + t(Q - P) + A 
            = (P + A) +  t((Q + A) - (P + A)) 
            = \overrightarrow{P + A, Q + A}\]
\end{proof}

\begin{proof}
    We first show the image of a segment under reflection through $O$ is a segment.
    Let $\overline{PQ}$ be a segment.
    Consider
    \[
        F(\overline{PQ}) = -\overline{PQ} = -(P + t(Q - P))
    \]
    where $0 \le t \le 1$. Then
    \[
        -P - t(Q - P) = (-P) + t((-Q) - (-P)) = \overline{-P, -Q}.
    \]

    We now show the image of a ray under reflection through $O$ is a ray.
    Let $\overrightarrow{PQ}$ be a ray.
    Consider
    \[
        F(\overrightarrow{PQ}) = -(P + t(Q - P))
    \]
    where $t \ge 0$. Then
    \[
        -P - t(Q - P) = (-P) + t((-Q) - (-P)) = \overrightarrow{-P, -Q}.
    \]
\end{proof}

\begin{proof}
    We first show the image of a segment under reflection through the x-axis is a segment.
    Let $\overline{PQ}$ be a segment with $P = (p_x, p_y)$, $Q = (q_x, q_y)$.
    Consider
    \[
        F(\overline{PQ}) = (p_x, -p_y) + t((q_x, -q_y) - (p_x, -p_y)), \quad 0 \le t \le 1
    \]
    which equals
    \[
        \overline{(p_x, -p_y), (q_x, -q_y)}.
    \]

    We now show the image of a ray under reflection through the x-axis is a ray.
    Let $\overrightarrow{PQ}$ be a ray. Then
    \[
        F(\overrightarrow{PQ}) = (p_x, -p_y) + t((q_x, -q_y) - (p_x, -p_y)), \quad t \ge 0
    \]
    which equals
    \[
        \overrightarrow{(p_x, -p_y), (q_x, -q_y)}.
    \]
\end{proof}

\begin{proof}
    We first show the image of a segment under reflection through the y-axis is a segment.
    Let $\overline{PQ}$ be a segment with $P = (p_x, p_y)$, $Q = (q_x, q_y)$.
    Consider
    \[
        F(\overline{PQ}) = (-p_x, p_y) + t((-q_x, q_y) - (-p_x, p_y)), \quad 0 \le t \le 1
    \]
    which equals
    \[
        \overline{(-p_x, p_y), (-q_x, q_y)}.
    \]

    We now show the image of a ray under reflection through the y-axis is a ray.
    Let $\overrightarrow{PQ}$ be a ray. Then
    \[
        F(\overrightarrow{PQ}) = (-p_x, p_y) + t((-q_x, q_y) - (-p_x, p_y)), \quad t \ge 0
    \]
    which equals
    \[
        \overrightarrow{(-p_x, p_y), (-q_x, q_y)}.
    \]
\end{proof}

\begin{proof}
    We first show the image of a segment under dilation by a number $r > 0$ is a segment.
    Let $\overline{PQ}$ be a segment. Consider
    \[
        F(\overline{PQ}) = r(P + t(Q - P)), \quad 0 \le t \le 1.
    \]
    Then
    \[
        rP + t\, r(Q - P) = (rP) + t((rQ) - (rP)) = \overline{rP, rQ}.
    \]

    We now show the image of a ray under dilation by a number $r > 0$ is a ray.
    Let $\overrightarrow{PQ}$ be a ray. Consider
    \[
        F(\overrightarrow{PQ}) = r(P + t(Q - P)), \quad t \ge 0.
    \]
    Then
    \[
        rP + t\, r(Q - P) = (rP) + t((rQ) - (rP)) = \overrightarrow{rP, rQ}.
    \]
\end{proof}

\begin{tcolorbox}[title=Problem 11, breakable]
    After you have read the definition of a straight line 
    in the next section, prove that the image under $F$ of a 
    straight line is again a straight line.
    [Here $F$ is any one of the mappings of Exercise 10.]
\end{tcolorbox}

\begin{proof}
    We show the image of a line under $F = T_A$ is a line.
    Let $\overline{PQ}$ be a line.
    Consider 
    \[F(\overline{PQ}) = T_A(\overline{PQ}) 
            = T_A(P + t(Q - P))\]
    Where $t \in \mathbb{R}$. Then
    \[P + t(Q - P) + A 
            = (P + A) +  t((Q + A) - (P + A)) 
            = \overline{P + A, Q + A}\]
\end{proof}

\begin{tcolorbox}[title=Problem 12, breakable]
    Give a definition for two located vectors to have \textbf{opposite direction}.
    Similary if $A \ne O$ and $B \ne O$, give a definition for $A$ and $B$ to 
    have opposite direction. Draw the corresponding pictures.
\end{tcolorbox}

\begin{definition}
    Let $\overrightarrow{AB}$ and $\overrightarrow{CD}$ be two vectors.
    We say that $\overrightarrow{AB}$ has \textbf{opposite direction} to $\overrightarrow{CD}$
    if and only if there exists a positive scalar $r > 0$ such that
    \[
        \overrightarrow{AB} + r\,\overrightarrow{CD} = \overrightarrow{0}.
    \]
\end{definition}

\begin{definition}
    Let $A$ and $B$ be points such that $A \ne O$ and $B \ne O$.
    We say that $A$ and $B$ have \textbf{opposite direction} if and only if 
    $\overrightarrow{OA}$ has opposite direction to $\overrightarrow{OB}$ in the sense above.
\end{definition}

\begin{tikzpicture}[scale=1]
    % Axes
    \draw[->] (-1,0) -- (5,0) node[right] {$x$};
    \draw[->] (0,-1) -- (0,5) node[above] {$y$};

    % Points for located vectors
    \coordinate (A) at (1,1);
    \coordinate (B) at (3,3);
    \coordinate (C) at (4,1);
    \coordinate (D) at (2,-1);

    % Vectors AB and CD (opposite)
    \draw[->, thick, blue] (A) -- (B) node[midway, above left] {$\overrightarrow{AB}$};
    \draw[->, thick, red] (C) -- (D) node[midway, above right] {$\overrightarrow{CD}$};

    % Points from origin
    \coordinate (O) at (0,0);
    \coordinate (E) at (3,1);
    \coordinate (F) at (-3/2, -1/2);

    % Vectors OA and OB (opposite)
    \draw[->, thick, green] (O) -- (E) node[midway, above] {$\overrightarrow{OA}$};
    \draw[->, thick, orange] (O) -- (F) node[midway, below] {$\overrightarrow{OB}$};
\end{tikzpicture}

\subsection{Lines}

\begin{tcolorbox}[title=Problem 1, breakable]
\end{tcolorbox}

\begin{tcolorbox}[title=Problem 6, breakable]
\end{tcolorbox}

\begin{tcolorbox}[title=Problem 12, breakable]
\end{tcolorbox}

\begin{tcolorbox}[title=Problem 13, breakable]
\end{tcolorbox}

\begin{tcolorbox}[title=Problem 18, breakable]
\end{tcolorbox}