\subsection{Definition of a Function}

\begin{tcolorbox}[title=Problem 2, breakable]
    For what numbers could you define a function $f$
    by the formula
    \[f(x) = \frac{1}{x^2 - 2}\]
    What is the value of the function for $x = 5$.
\end{tcolorbox}

\textbf{Solution:}
$f$ is defined at all real numbers other than $\pm \sqrt{2}$.
\[f(5) = \frac{1}{5^2 - 2} = \frac{1}{23}\]

\begin{tcolorbox}[title=Problem 8, breakable]
    A function (defined for all numbers) is said to be 
    an \textbf{even} function if
    $f(x) = f(-x)$ for all numbers $x$. It is said to be 
    an \textbf{odd} function if $f(x) = -(f(-x))$ for all $x$.
    Determine which of the following functions are even or odd.

    (a) $f(x) = x$

    (b) $f(x) = x^2$

    (c) $f(x) = x^3$

    (d) $f(x) = \frac{1}{x} \text{ if } x \ne 0 \text{ and } f(0)=0$
\end{tcolorbox}

\textbf{Solution (a):} odd

\textbf{Solution (b):} even

\textbf{Solution (c):} odd

\textbf{Solution (d):} odd

\begin{tcolorbox}[title=Problem 9, breakable]
    Show that any function defined for all numbers 
    can be written as a sum of an even and an odd function.
    [Hint: The term 
    \[\frac{f(x) + f(-x)}{2}\]
    will be an even function.]
\end{tcolorbox}

\begin{proof}
    Let any function $f$ be defined for all numbers.
    \[
    g(x) = \frac{f(x) + f(-x)}{2}, \quad
    l(x) = \frac{f(x) - f(-x)}{2}.
    \]
    Then $g$ is even and $l$ is odd, and
    \[
    f(x) = g(x) + l(x).
    \]
\end{proof}

\begin{tcolorbox}[title=Problem 11, breakable]
    (a) Show that the sum of odd functions is odd.

    (b) Show that the sum of even functions is even.
\end{tcolorbox}

\begin{proof}
    Let $f, g$ be odd functions. Then
    \[
    (f + g)(x) = f(x) + g(x) = -f(-x) - g(-x) = -(f(-x) + g(-x)) = -(f + g)(-x),
    \]
    so the sum is odd.
    Let $f, g$ be even functions. Then
    \[
    (f + g)(x) = f(x) + g(x) = f(-x) + g(-x) = (f + g)(-x),
    \]
    so the sum is even.
\end{proof}

\begin{tcolorbox}[title=Problem 12, breakable]
    Determine whether the product of the following types of functions 
    is odd, even, or neither. Prove your assertions.

    (a) Product of odd function with odd function 

    (b) Product of even function with odd function 

    (c) Product of even function with even function
\end{tcolorbox}

\begin{proof} 
    If $f$ and $g$ are odd:
    \[
    (fg)(-x) = f(-x)g(-x) = (-f(x))(-g(x)) = f(x)g(x) = (fg)(x)
    \]
    The product is even.
    
    If $f$ is even and $g$ is odd:
    \[
    (fg)(-x) = f(-x)g(-x) = f(x)(-g(x)) = -f(x)g(x) = -(fg)(x)
    \]
    The product is odd.

    If $f$ and $g$ are even:
    \[
    (fg)(-x) = f(-x)g(-x) = f(x)g(x) = (fg)(x)
    \]
    The product is even.
\end{proof}

\subsection{Polynomial Functions}

\begin{tcolorbox}[title=Problem 1, breakable]
    What is the degree of the following polynomials.

    (a) $3x^2 - 4x + 5$

    (b) $-5x^5 + x$

    (c) $-38x^4 + x^3 - x - 1$

    (d) $(3x^2 - 4x + 5)(-5x^5 + x)$

    (e) $(-5x^5 + x)(-7x + 3)$

    (f) $(-4x^2 + 5x - 4)(3x^3 + x - 1)$

    (g) $(6x^7 - x^3 + 5)(7x^4 - 3x^2 + x - 1)$

    (h) Let $f, g$ be polynomials which are not the zero polynomials.
        Show that $deg(fg) = deg(f) + deg(g)$.
\end{tcolorbox}

\textbf{Solution (a):} $2$

\textbf{Solution (b):} $5$

\textbf{Solution (c):} $4$

\textbf{Solution (d):} $7$

\textbf{Solution (e):} $6$

\textbf{Solution (f):} $5$

\textbf{Solution (g):} $11$

\begin{proof}
    Let $f, g$ be arbitrary polynomials. Let 
    \[f = a_n x^n + (\text{other terms}), \quad g = b_k x^k + (\text{other terms})\]
    where $a_n, b_k$ are the coefficients of the largest-degree terms of $f$ and $g$ respectively, with $a_n \ne 0$
        and $b_k \ne 0$.
    The term of largest degree in the product of $fg$ is produced by multiplying the highest-degree terms 
    \[(a_n x^n)(b_k x^k) = (a_n b_k)x^{n + k}\]
    This term as the degree $n + k$, and since $a_n b_k \ne 0$,
        the degree of $fg$ is $n + k$.
\end{proof}

\begin{tcolorbox}[title=Problem 3, breakable]
    Let $f$ be a polynomial of degree $3$. If there exists
    polynomials $g, h$ of degree $\ge 1$ such that $f = gh$,
        show that $f$ has a root.
\end{tcolorbox}

\begin{proof}
    By Problem $1$ either $deg(g) = 1$ or $deg(h) = 1$.
    Suppose wlog $deg(g) = 1$. Let $g = ax + b$ where $a, b$
        are constants and $a \ne 0$.
    A root is found when $ax + b = 0$ thus $x = \frac{-b}{a}$.
    Then $g\left(\frac{-b}{a}\right)h\left(\frac{-b}{a}\right) = f\left(\frac{-b}{a}\right) = 0$ as required.
\end{proof}

\begin{tcolorbox}[title=Problem 4, breakable]
    Give an example of polynomials of degree $2, 4$ which have no roots and degree $3$ that has one root in the real numbers.
\end{tcolorbox}

\textbf{Solution:}
\[x^2 + 2 = 0, (x^2 + 2)(x - 4) = 0, x^4 + 2 = 0\]

\subsection{Graphs of Functions}

\begin{tcolorbox}[title=Problem 1, breakable]
    Sketch the graph of the following functions.
    Make a small table of values $f(x) = \frac{1}{x^3}$.
\end{tcolorbox}

\textbf{Solution:}
\[x = -2, -1, -0.5, 0, 0.5, 1, 2\]
\[f(x) = -0.125, -1, -8, \text{undefined}, 8, 1, 0.125\]
\begin{figure}[h!]
    \centering
    \includegraphics[width=0.5\textwidth]{images/again.PNG}
\end{figure}

\subsection{Exponential Function}

\begin{tcolorbox}[title=Problem 7, breakable]
    The function $f(t) = 4 \cdot 16^t$ describes the growth of bacteria.

    (a) How many bacteria are present at the beggining when $t = 0$?

    (b) After $\frac{1}{2}$ hr, how many bacteria are there?

    (c) After $\frac{1}{4}$ hr, how many bacteria are there?

    (d) After $1$ hr, how many bacteria are there?
\end{tcolorbox}

\textbf{Solution (a):} $4 \cdot 16^{0}$

\textbf{Solution (b):} $4 \cdot 16^{\frac{1}{2}}$

\textbf{Solution (c):} $4 \cdot 16^{\frac{1}{4}}$

\textbf{Solution (d):} $4 \cdot 16^{1}$

\subsection{Logarithms}

\begin{tcolorbox}[title=Problem 3, breakable]
    Let $e$ be a fixed number $> 1$ and abbreviate
        $\log_e$ by $\log$.
    If $a$ is $> 1$ and $x$ is an arbitrary number, prove that 
    \[\log (a^x) = x \log (a)\]
\end{tcolorbox}

\begin{proof}
    \begin{align*}
        & \log a = \log a \\
        &\iff \log(e^{\log a}) = \log a \\
        &\iff e^{\log a} = a  \\
        &\iff (e^{\log a})^x = a^x  \\
        &\iff e^{x \log a} = a^x \\
        &\iff \log a^x = \log (e^{x \log a}) = x \log a
    \end{align*}
\end{proof}

\begin{tcolorbox}[title=Problem 6, breakable]
    Bacteria increase according to the formula 
    \[B(t) = Ce^{kt}\]
    where $C$ and $k$ are constants, and $B(t)$ gives the number of bacteria 
        as a function of $t$ in min.
    At time $t = 0$, there are $10^6$ bacteria.
    How long will it take before they increase to $10^7$ if it takes $12$ min
        to increase by $2 \times 10^6$.
\end{tcolorbox}

\textbf{Solution:}
At $t = 0$ there are $10^6$ bacteria. Thus $C = 10^6$.
Then plugging in $12$ min we have the formula 
\[3 \cdot 10^6 = 10^6 \cdot e^{12k}\]
Solving for $k$ shows 
\[k = \frac{\ln 3}{12}\]
Now plugging in 
\[10^7 = 10^6 e^{kt} \implies 10 = e^{kt}\]
\[t = \frac{\ln 10}{k} = \frac{12 \ln 10}{\ln 3}\]