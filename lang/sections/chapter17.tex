\subsection{Matrices}


In each of the following cases, write down the second row 
and first column of the indicated matrix.
Also write down its transpose.

\begin{tcolorbox}[title=Problem 1, breakable]
    \[\begin{bmatrix}
        2 & -5 \\
        -3 & -7
    \end{bmatrix}\]
\end{tcolorbox}

\textbf{Solution:}
Second row:
\[(-3, -7)\] 
First column:
\[\begin{bmatrix}
    2 \\
    -3
\end{bmatrix}\]
Transpose:
\[\begin{bmatrix}
    2 & -3 \\
    -5 & -7
\end{bmatrix}\]

\begin{tcolorbox}[title=Problem 4, breakable]
    \[\begin{bmatrix}
        3 & 5 & 6 \\
        -1 & 2 & 3 \\
        7 & 3 & -2
    \end{bmatrix}\]
\end{tcolorbox}

\textbf{Solution:}
Second row:
\[(-1, 2, 3)\] 
First column:
\[\begin{bmatrix}
    3 \\
    -1 \\
    7
\end{bmatrix}\]
Transpose:
\[\begin{bmatrix}
    3 & -1 & 7 \\
    5 & 2 & 3 \\
    6 & 3 & -2
\end{bmatrix}\]

\begin{tcolorbox}[title=Problem 7, breakable]
    Find the sum of the first two columns in the matrix in Exercise $4$.
\end{tcolorbox}

\textbf{Solution:}
\[\begin{bmatrix}
    3 \\
    5 \\
    6
\end{bmatrix} + \begin{bmatrix}
    -1 \\
    2 \\
    3
\end{bmatrix} = \begin{bmatrix}
    2 \\
    7 \\
    9
\end{bmatrix}\]

\subsection{Determinants of Order $2$}

\begin{tcolorbox}[title=Problem 2, breakable]
    Compute the determinant of 
    \[\begin{bmatrix}
        \cos \theta & -\sin \theta \\
        \sin \theta & \cos \theta
    \end{bmatrix}\]
\end{tcolorbox}

\textbf{Solution:}
\[D\left(\begin{bmatrix}
    \cos \theta & -\sin \theta \\
    \sin \theta & \cos \theta
\end{bmatrix}\right) = \cos \theta \cos \theta - (-\sin \theta)\sin \theta = \cos^2 \theta + \sin^2 \theta = 1\]

\begin{tcolorbox}[title=Problem 3, breakable]
    Compute the determinant 
    \[\begin{bmatrix}
        \cos \theta & \sin \theta \\
        \sin \theta & \cos \theta 
    \end{bmatrix}\]
    (a) $\theta = \pi$,

    (b) $\theta  = \pi/2$

    (c) $\theta = \pi/3$

    (d) $\theta = \pi/4$
\end{tcolorbox}

\textbf{General solution:}
\[D\left(\begin{bmatrix}
        \cos \theta & \sin \theta \\
        \sin \theta & \cos \theta 
\end{bmatrix}\right) = \cos \theta \cos \theta - \sin \theta \sin \theta = cos^2 \theta - sin^2 \theta\]

\textbf{Solution (a):} $cos^2 \pi - sin^2 \pi$

\textbf{Solution (b):} $cos^2 \pi/2 - sin^2 \pi/2$

\textbf{Solution (c):} $cos^2 \pi/3 - sin^2 \pi/3$

\textbf{Solution (d):} $cos^2 \pi/4 - sin^2 \pi/4$

\subsection{Properties of $2 \times 2$ Determinants}

\begin{tcolorbox}[title=Problem 1, breakable]
    Prove the other half of \textbf{D1}, i.e.
    distributivity on the other side other than that given in the text.
\end{tcolorbox}

\begin{proof}
    
    \begin{align*}
        D(C, B' + B'') &= D\left(\begin{bmatrix}
            b'_1 + b''_1 & c_1 \\
            b'_2 + b''_2 & c_2
        \end{bmatrix}\right) \\
                             &= c_2(b'_1 + b''_1) - c_1(b'_2 + b''_2) \\
                             &= c_2 b'_1 + c_2 b'_1 - c_1 b'_2 - c_1 b''_2 \\
                             &= D(C, B') + D(C, B'')
    \end{align*}
\end{proof}

\begin{tcolorbox}[title=Problem 2, breakable]
    Prove the other half of \textbf{D2}.
\end{tcolorbox}

\begin{proof}
    We have 
    \[D(B, xC) = D\left(\begin{bmatrix}
        b_1 & x c_1 \\
        b_2 & x c_2
    \end{bmatrix}\right) = x b_1 c_2 - x b_2 c_1 = x(b_1 c_2 - b_2 c_1) = x D(B, C)\]
\end{proof}

\begin{tcolorbox}[title=Problem 3, breakable]
    Prove the other half of \textbf{D5}.
\end{tcolorbox}

\begin{proof}
    Using \textbf{D1, D2, D4} we find 
    \[D(B, C + xB) = D(B, C) + D(B, xB) = D(B, C) + xD(B, B) = D(B, C)\]
\end{proof}

\begin{tcolorbox}[title=Problem 4, breakable]
    Using the same method as at the end of the section,
        find the value for $y$.
\end{tcolorbox}

\begin{proof}
    Suppose $x, y$ are solutions to the system of equations.
    We have 
    \[D(C, A) = D(xA + yB, A) = D(xA, A) + D(yB, A) = xD(A, A) + yD(B, A) = yD(B, A)\]
    Thus $y = \frac{D(C, A)}{D(B, A)}$.
\end{proof}

\begin{tcolorbox}[title=Problem 6, breakable]
    Let $c$ be a number, and let $A$ be a $2 \times 2$ matrix.
    Define $cA$ to be the matrix obtained by multiplying all components
    of $A$ by $c$. How does $D(cA)$ differ from $D(A)$.
\end{tcolorbox}

\textbf{Solution:} By \textbf{D2} it scales the value of the determinant by $c^2$.

\subsection{Determinants of Order $3$}

\begin{tcolorbox}[title=Problem 2 (a), breakable]
    Compute the following determinants by expandingaccording to the 
    second row, and also according to the third column.

    (a) \[\begin{bmatrix}
        3 & 1 & 2 \\
        0 & 3 & -1 \\
        4 & 1 & 1
    \end{bmatrix}\]
\end{tcolorbox}

\textbf{Solution:} By second row 
\[3(3 - 8) - (-1)(3 - 4) = -15 - 1 = -16\]
By third column 
\[2(0 - 12) - (-1)(3 - 4) + 1(9 - 0) = -24 - 1 + 9 = -16\]

\begin{tcolorbox}[title=Problem 4, breakable]
    Let $a, b, c$ be numbers. In terms of $a, b, c$
        what is the value of the determinant.
    \[\begin{bmatrix}
        a & 0 & 0 \\
        0 & b & 0 \\
        0 & 0 & c
    \end{bmatrix}\]
\end{tcolorbox}

\textbf{Solution:}
\[a(bc) = abc\]

\begin{tcolorbox}[title=Problem 6, breakable]
    In terms of the components of the matrix,  what is the value 
        of the determinant:
    \[(a) \begin{bmatrix}
        a_11 & a_12 & a_13 \\
        0 & a_22 & a_23 \\
        0 & 0 & a_33
    \end{bmatrix}, (b) \begin{bmatrix}
        a_11 & 0 & 0 \\
        a_21 & a_22 & 0 \\
        a_31 & a_32 & a_33
    \end{bmatrix}\]
\end{tcolorbox}

\textbf{Solution (a):}
\[a_{11}(a_{22} a_{33}) = a_{11} a_{22} a_{33}\]
\textbf{Solution (b):}
\[a_{11} a_{22} a_{33}\]

\subsection{Properties of $3 \times 3$ Determinants}

\begin{tcolorbox}[title=Problem 1, breakable]
    Write out in full and prove property \textbf{D1}
    with respect to the second and third column.
\end{tcolorbox}

\begin{theorem}
    Suppose that the second column can be written as a sum,
    \[A^2 = B + C\]
    that is,
    \[\begin{bmatrix}
        a_12 \\
        a_22 \\
        a_32
    \end{bmatrix} = \begin{bmatrix}
        b_1 \\
        b_2 \\
        b_3
    \end{bmatrix} + \begin{bmatrix}
        c_1 \\
        c_2 \\
        c_3
    \end{bmatrix}\]
    Then 
    \[D(A^1, B + C, A^3) = D(A^1, B, A^3) + D(A^1, C, A^3)\]
\end{theorem}

\begin{proof}
    We use the definition of the determinant, namely the expansion
    according to the second column. Each term splits into a sum of two
    terms corresponding to $B$ and $C$. For instance,
    \[a_{21} \begin{bmatrix}
        a_{12} & a_{13} \\
        a_{32} & a_{33} 
    \end{bmatrix} = b_1 \begin{bmatrix}
        a_{12} & a_{13} \\
        a_{32} & a_{33}
    \end{bmatrix} + c_1 \begin{bmatrix}
        a_{12} & a_{13} \\
        a_{32} & a_{33}
    \end{bmatrix}\]
    \[a_{22} \begin{bmatrix}
        b_2 + c_2 & a_{13} \\
        b_3 + c_3 & a_{33} 
    \end{bmatrix} = a_{22} \begin{bmatrix}
        b_2 & a_{13} \\
        b_3 & a_{33}
    \end{bmatrix} + a_{22} \begin{bmatrix}
        c_2 & a_{13} \\
        c_3 & a_{33}
    \end{bmatrix}\]
    \[a_{23} \begin{bmatrix}
        b_2 + c_2 & a_{12} \\
        b_3 + c_3 & a_{32} 
    \end{bmatrix} = a_{23} \begin{bmatrix}
        b_2 & a_{12} \\
        b_3 & a_{32}
    \end{bmatrix} + a_{23} \begin{bmatrix}
        c_2 & a_{12} \\
        c_3 & a_{32}
    \end{bmatrix}\]
    Summing with the appropriate signs yields the desired relation.
\end{proof}

\begin{theorem}
    Suppose that the third column can be written as a sum,
    \[A^3 = B + C\]
    that is,
    \[\begin{bmatrix}
        a_13 \\
        a_23 \\
        a_33
    \end{bmatrix} = \begin{bmatrix}
        b_1 \\
        b_2 \\
        b_3
    \end{bmatrix} + \begin{bmatrix}
        c_1 \\
        c_2 \\
        c_3
    \end{bmatrix}\]
    Then 
    \[D(A^1, A^2, B + C) = D(A^1, A^2, B) + D(A^1, A^2, C)\]
\end{theorem}

\begin{proof}
    We use the definition of the determinant, namely the expansion
    according to the third column. Each term splits into a sum of two
    terms corresponding to $B$ and $C$. For instance,
    \[a_{31} \begin{bmatrix}
        a_{12} & a_{13} \\
        a_{22} & a_{23} 
    \end{bmatrix} = b_1 \begin{bmatrix}
        a_{12} & a_{13} \\
        a_{22} & a_{23}
    \end{bmatrix} + c_1 \begin{bmatrix}
        a_{12} & a_{13} \\
        a_{22} & a_{23}
    \end{bmatrix}\]
    \[a_{32} \begin{bmatrix}
        b_2 + c_2 & a_{13} \\
        b_3 + c_3 & a_{23} 
    \end{bmatrix} = a_{32} \begin{bmatrix}
        b_2 & a_{13} \\
        b_3 & a_{23}
    \end{bmatrix} + a_{32} \begin{bmatrix}
        c_2 & a_{13} \\
        c_3 & a_{23}
    \end{bmatrix}\]
    \[a_{33} \begin{bmatrix}
        b_2 + c_2 & a_{12} \\
        b_3 + c_3 & a_{22} 
    \end{bmatrix} = a_{33} \begin{bmatrix}
        b_2 & a_{12} \\
        b_3 & a_{22}
    \end{bmatrix} + a_{33} \begin{bmatrix}
        c_2 & a_{12} \\
        c_3 & a_{22}
    \end{bmatrix}\]
    Summing with the appropriate signs yields the desired relation.
\end{proof}

\begin{tcolorbox}[title=Problem 2, breakable]
    Same thing for property \textbf{D2}.
\end{tcolorbox}

\begin{theorem}
    If $x$ is a number, then 
    \[D(A^1, x A^2, A^3) = x \cdot D(A^1, A^2, A^3)\]
\end{theorem}

\begin{proof}
We have:
\[
D(A^1, x A^2, A^3) =
- a_{11} \begin{vmatrix} x a_{22} & a_{23} \\ x a_{32} & a_{33} \end{vmatrix}
+ a_{21} \begin{vmatrix} x a_{12} & a_{13} \\ x a_{32} & a_{33} \end{vmatrix}
- a_{31} \begin{vmatrix} x a_{12} & a_{13} \\ x a_{22} & a_{23} \end{vmatrix}
= x \cdot D(A^1, A^2, A^3).
\]
\end{proof}

\begin{theorem}
    If $x$ is a number, then 
    \[D(A^1, A^2, x A^3) = x \cdot D(A^1, A^2, A^3)\]
\end{theorem}

\begin{proof}
We have:
\[
D(A^1, A^2, x A^3) =
a_{11} \begin{vmatrix} a_{22} & x a_{23} \\ a_{32} & x a_{33} \end{vmatrix}
- a_{21} \begin{vmatrix} a_{12} & x a_{13} \\ a_{32} & x a_{33} \end{vmatrix}
+ a_{31} \begin{vmatrix} a_{12} & x a_{13} \\ a_{22} & x a_{23} \end{vmatrix}
= x \cdot D(A^1, A^2, A^3).
\]
\end{proof}

\begin{tcolorbox}[title=Problem 3, breakable]
    Prove the two cases not treated in the text for 
    property \textbf{D4}.
\end{tcolorbox}

\begin{proof}
Suppose that $A_2 = A_3$, and look at the expansion of the determinant according to the first row. 
Then $a_{13} = a_{12}$, and the first two terms cancel. The third term is equal to 0 because it involves a 
$2 \times 2$ determinant whose two columns are equal. 
\end{proof}

\begin{proof}
Suppose that $A_1 = A_3$, and look at the expansion of the determinant according to the first row. 
Then $a_{13} = a_{11}$, and the first two terms cancel. The third term is equal to 0 because it involves 
a $2 \times 2$ determinant whose two columns are equal. 
\end{proof}

\begin{tcolorbox}[title=Problem 4, breakable]
Prove \textbf{D5}

(a) you add a multiple of the third column to the first;

(b) you add a multiple of the second column to the first;

(c) you add a third column to the second.
\end{tcolorbox}

\begin{proof}
    We have
    \[
        D(A_1 + x A_3, A_2, A_3) = D(A_1, A_2, A_3) + D(x A_3, A_2, A_3) \text{ (by D1)}
    \]
    \[
        = D(A_1, A_2, A_3) + x \cdot D(A_3, A_2, A_3) \text{ (by D2)}
        = D(A_1, A_2, A_3) \text{ (by D4).}
    \]
    We have
    \[
        D(A_1 + x A_2, A_2, A_3) = D(A_1, A_2, A_3) + D(x A_2, A_2, A_3) \text{ (by D1)}
    \]
    \[
        = D(A_1, A_2, A_3) + x \cdot D(A_2, A_2, A_3) \text{ (by D2)}
        = D(A_1, A_2, A_3) \text{ (by D4).}
    \]
    We have
    \[
        D(A_1, A_2 + A_3, A_3) = D(A_1, A_2, A_3) + D(A_1, A_3, A_3) \text{ (by D1)}
        = D(A_1, A_2, A_3) \text{ (by D4).}
    \]
\end{proof}

\begin{tcolorbox}[title=Problem 5, breakable]
    Prove \textbf{D6} in the second case.
\end{tcolorbox}

\begin{proof}
    \[
    0 = D(A_1, A_2 + A_3, A_2 + A_3)
    = D(A_1, A_2, A_2 + A_3) + D(A_1, A_3, A_2 + A_3)
    \]  
    \[
    = D(A_1, A_2, A_2) + D(A_1, A_2, A_3) + D(A_1, A_3, A_2) + D(A_1, A_3, A_3)
    = D(A_1, A_2, A_3) + D(A_1, A_3, A_2)
    \]
    Thus $D(A_1, A_2, A_3) = -D(A_1, A_3, A_2)$.
\end{proof}

\begin{tcolorbox}[title=Problem 6, breakable]
    If you interchange the first and third columns 
    of the given matrix, how does its determinant change?
    What about interchanging the first and third row?
\end{tcolorbox}

\textbf{Solution:} The determinant changes by a sign.

\begin{tcolorbox}[title=Problem 7, breakable]
    State \textbf{D5} and \textbf{D6} for rows.
\end{tcolorbox}

\begin{theorem}[D5 for rows]
If we add a multiple of one row to another, 
then the value of the determinant does not change. In other words, let $x$ be a number. Then, for instance,
\[
D(R_1 + x R_2, R_2, R_3) = D(R_1, R_2, R_3),
\]
and similarly in all other cases.
\end{theorem}

\begin{theorem}[D6 for rows]
If two adjacent rows are interchanged, then the determinant changes by a sign. In other words, we have
\[
D(R_2, R_1, R_3) = -D(R_1, R_2, R_3),
\]
and similarly in the other cases.
\end{theorem}

\begin{tcolorbox}[title=Problem 9, breakable]
    Let $c$ be a number and multiply each componenet 
    $a_{ij}$ of a $3 \times 3$ matrix $A$ by $c$, thus 
    obtaining a new matrix which we denote by $cA$.
    How does $D(A)$ differ from $D(cA)$.
\end{tcolorbox}

\textbf{Solution:} The determinant is scaled by $c^3$.

\begin{tcolorbox}[title=Problem 10, breakable]
    Let $x_1, x_2, x_3$ be numbers. Show that 
    \[
    \begin{vmatrix}
    1 & x_1 & x_1^2 \\
    1 & x_2 & x_2^2 \\
    1 & x_3 & x_3^2
    \end{vmatrix} = (x_2 - x_1)(x_3 - x_2)(x_3 - x_1)
    \]
\end{tcolorbox}

\begin{proof}
    Then
    \[
    \begin{vmatrix}
    1 & x_1 & x_1^2 \\
    1 & x_2 & x_2^2 \\
    1 & x_3 & x_3^2
    \end{vmatrix}
    = 1 \cdot (x_2 x_3^2 - x_3 x_2^2) 
    - x_1 \cdot (1 \cdot x_3^2 - 1 \cdot x_2^2) 
    + x_1^2 \cdot (1 \cdot x_3 - 1 \cdot x_2)
    = (x_2 - x_1)(x_3 - x_2)(x_3 - x_1).
    \]
\end{proof}

\begin{tcolorbox}[title=Problem 13, breakable]
    State the analogous property to that of Exercise $12$
    with respect to the second column. Then with respect 
    to the third column.
\end{tcolorbox}

\textbf{Solution (a):}
Let 
\[
A^2 = \sum_{j=1}^{n} x_j C^j
\]
where $C^j$ are column vectors and $x_j$ are numbers. Then
\[
D(A^1, A^2, A^3, \dots, A^n) = \sum_{j=1}^{n} x_j D(A^1, C^j, A^3, \dots, A^n).
\]

\textbf{Solution (b):}
Let 
\[
A^3 = \sum_{j=1}^{n} x_j C^j
\]
where $C^j$ are column vectors and $x_j$ are numbers. Then
\[
D(A^1, A^2, A^3, \dots, A^n) = \sum_{j=1}^{n} x_j D(A^1, A^2, C^j, \dots, A^n).
\]

\subsection{Cramer's Rule}

\begin{tcolorbox}[title=Problem 1, breakable]
    Fill in the missing steps in the proof of Cramer's rule.
    Cf. Exercises $11$ and $12$ of the proceeding section.
\end{tcolorbox}

\begin{proof}
    \begin{align*}
        D(B, A^2, A^3) &= D(x_1 A^1 + x_2 A^2 + x_3 A^3, A^2, A^3) \\
        &= D(x_1 A^1 + x_2 A^2, A^2, A^3) + D(x_3 A^3, A^2, A^3) \\
        &= D(x_1 A^1, A^2, A^3) + D(x^2 A^2, A^2, A^3) + D(x_3 A^3, A^2, A^3) \\
        &= x_1 D(A^1, A^2, A^3) + x_2 D(A^2, A^2, A^3) + x_3 D(A^3, A^2, A^3) \\
        &= x_1 D(A^1, A^2, A^3)
    \end{align*}
    Thus $x_1 = \frac{D(B, A^2, A^3)}{D(A^1, A^2, A^3)}$.
\end{proof}

\begin{tcolorbox}[title=Problem 2, breakable]
    Write out in full the proof of Cramer's rule for $x_2$ and $x_3$.
    It is very similar to the proof for $x_1$ in the text.
\end{tcolorbox}

\begin{proof}
    \begin{align*}
        D(B, A^1, A^3) &= D(x_1 A^1 + x_2 A^2 + x_3 A^3, A^1, A^3) \\
        &= D(x_1 A^1, A^1, A^3) + D(x_2 A^2 + x_3 A^3, A^1, A^3) \\
        &= D(x_1 A^1, A^1, A^3) + D(x_2 A^2, A^1, A^3) + D(x_3 A^3, A^1, A^3) \\
        &= x_1 D(A^1, A^1, A^3) + x_2 D(A^2, A^1, A^3) + x_3 D(A^3, A^1, A^3) \\
        &= x_2 D(A^2, A^1, A^3)
    \end{align*}
    Thus $x_2 = \frac{D(B, A^1, A^3)}{D(A^2, A^1, A^3)}$.
\end{proof}

\begin{proof}
    \begin{align*}
        D(B, A^1, A^2) &= D(x_1 A^1 + x_2 A^2 + x_3 A^3, A^1, A^2) \\
        &= D(x_1 A^1 + x_2 A^2, A^1, A^2) + D(x_3 A^3, A^1, A^2) \\
        &= D(x_1 A^1, A^1, A^2) + D(x_2 A^2, A^1, A^2) + D(x_3 A^3, A^1, A^2) \\
        &= x_1 D(A^1, A^1, A^2) + x_2 D(A^2, A^1, A^2) + x_3 D(A^3, A^1, A^2) \\
        &= x_3 D(A^3, A^1, A^2)
    \end{align*}
    Thus $x_2 = \frac{D(B, A^1, A^2)}{D(A^3, A^1, A^2)}$.
\end{proof}

\begin{tcolorbox}[title=Problem 3, breakable]
    Let $A^1, A^2, A^3$ be columns of a $3 \times 3$ matrix $A$,
    and assume that there are numbers $x_1, x_2, x_3$
    not all $0$ such that 
    \[x_1 A^1 + x_2 A^2 + x_3 A^3 = 0\]
    Prove that $D(A) = 0$.
\end{tcolorbox}

\begin{proof}
    Suppose wlog that $x_1 \ne 0$. 
    Then
    \[
        A^1 = -\frac{x_2}{x_1} A^2 - \frac{x_3}{x_1} A^3
    \]
    Clearly, $A^1$ is a linear combination of $A^2$ and $A^3$, so
    $D(A) = 0$.
\end{proof}