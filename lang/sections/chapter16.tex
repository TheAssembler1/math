\subsection{Induction}

\begin{tcolorbox}[title=Problem 1, breakable]
    Prove that, for all integers $n \ge 1$, we have 
    \[1 + 3 + 5 + \ldots + (2n - 1) = n^2\]
\end{tcolorbox}

\begin{proof}
    (\textbf{Base Case}) Let $n = 1$. Then $1 = n^2 = 1^2 = 1$ as required.

    (\textbf{Induction Step}) Suppose the equation holds for some $n \in \mathbb{N}$.
    Thus $1 + 3 + 5 + \ldots + (2n - 1) = n^2$. Consider 
    $1 + 3 + 5 + \ldots + (2n - 1) + (2n + 1) = n^2 + (2n + 1)$ by our hypothesis.
    Then $n^2 + 2n + 1 = (n + 1)^2$ as required.
\end{proof}

\begin{tcolorbox}[title=Problem 4, breakable]
    Prove that $n(n^2 + 5)$ is divisible by $6$ for all integers $n \ge 1$.
\end{tcolorbox}

\begin{tcolorbox}[title=Problem 5, breakable]
    Prove that, for $x \ne 1$, we have 
    \[(1 + x)(1 + x^2)(1 + x^4) \ldots (1 + x^2)^n = \frac{1 - x^{2n + 1}}{1 - x}\]
\end{tcolorbox}

\begin{tcolorbox}[title=Problem 6, breakable]
    Let $f$ be a function defined for all real numbers
        such that $f(xy) = f(x) + f(y)$.
    for all real numbers $x, y$.
    Show that $f(x^n) = n f(x)$ for all $x$.
\end{tcolorbox}

\begin{tcolorbox}[title=Problem 7, breakable]
    Let $f$ be a function defined for all numbers such that $f(xy) = f(x)f(y)$
    for all real numbers $x, y$. Show that $f(x^n) = f(x)^n$ for all positive integers $n$ 
    and all real numbers $x$.
\end{tcolorbox}

\begin{tcolorbox}[title=Problem 9, breakable]
    \textbf{Binomial Coefficients}. Let 
    \[\binom{n}{k} = \frac{n!}{k!(n - k)!}\]
    where $n, k$ are integers $\ge 0$, $0 \le k \le n$, and $0!$ is defined to be $1$.
    Prove that 

    (a) $\binom{n}{k} = \binom{n}{n - k}$

    (b) $\binom{n}{k - 1} + \binom{n}{k} = \binom{n + 1}{k}$ (for $k > 0$)

    (c) Prove by induction that for all numbers $x, y$ we have 
    \[(x + y)^n = \sum_{k = 0}^n \binom{n}{k} x^k y^{n - k}\]
\end{tcolorbox}

\begin{tcolorbox}[title=Problem 12, breakable]
    \begin{theorem}
        All billiard balls have the same color.
    \end{theorem}
    \begin{proof}
        By induction, on the number of $n$ billiard balls.
        Our theorem is certainly true for $n = 1$, i.e. for one billiard ball.
        Assume it for n billiard balls. We prove it for $n + 1$.
        Look at the first $n$ billiard balls amont those $n + 1$.
        By induction, they have the same color. Now look at the last $n$ among
            those $n + 1$. They have the same color.
        Hence all $n + 1$ have the same color.
    \end{proof}
\end{tcolorbox}

\begin{tcolorbox}[title=Problem 13, breakable]
    Let $E$ be the set with $n$ elements, and let $F$ be a set 
    with $m$ elements Show that the total number of mappings from $E$
        to $F$ is $m^n$.
\end{tcolorbox}

\subsection{Summations}

\begin{tcolorbox}[title=Problem 1, breakable]
\end{tcolorbox}

\begin{tcolorbox}[title=Problem 2, breakable]
\end{tcolorbox}

\begin{tcolorbox}[title=Problem 3, breakable]
\end{tcolorbox}

\begin{tcolorbox}[title=Problem 4, breakable]
\end{tcolorbox}

\begin{tcolorbox}[title=Problem 5, breakable]
\end{tcolorbox}

\subsection{Geometric Series}

\begin{tcolorbox}[title=Problem 2, breakable]
\end{tcolorbox}

\begin{tcolorbox}[title=Problem 3, breakable]
\end{tcolorbox}

\begin{tcolorbox}[title=Problem 4, breakable]
\end{tcolorbox}

\begin{tcolorbox}[title=Problem 5, breakable]
\end{tcolorbox}

\begin{tcolorbox}[title=Problem 6, breakable]
\end{tcolorbox}

\begin{tcolorbox}[title=Problem 7, breakable]
\end{tcolorbox}