\subsection{Cubics in $\mathbb{C}^2$}

\begin{tcolorbox}[title=Problem 1, breakable]
    Sketch the following cubics in the real plane $\mathbb{R}^2$.
    \begin{enumerate}
        \item $y^2 = x^3$.
        \item $y^2 = x(x - 1)^2$.
        \item $y^2 = x(x - 1)(x - 2)$.
        \item $y^2 = x(x^2 + x + 1)$.
    \end{enumerate}
\end{tcolorbox}

\textbf{Solution (1):}
\begin{figure}[H]
    \centering
    \includegraphics[width=0.3\textwidth]{images/chapter2/1.png}
\end{figure}

\textbf{Solution (2):}
\begin{figure}[H]
    \centering
    \includegraphics[width=0.3\textwidth]{images/chapter2/2.png}
\end{figure}

\textbf{Solution (3):}
\begin{figure}[H]
    \centering
    \includegraphics[width=0.3\textwidth]{images/chapter2/3.png}
\end{figure}

\textbf{Solution (4):}
\begin{figure}[H]
    \centering
    \includegraphics[width=0.3\textwidth]{images/chapter2/4.png}
\end{figure}

\begin{tcolorbox}[title=Problem 2, breakable]
    Consider the cubics in the above exercise.
    \begin{enumerate}
        \item Give the homogeneous form for each cubic, which extends each of
            the above cubics to the complex projective plane.
        \item For each of the above cubics, dehomogenize by setting $x = 1$,
        and graph the resulting cubic in $\mathbb{R}^2$ with coordinates $y$ and $z$.
    \end{enumerate}
\end{tcolorbox}

\textbf{Solution: }
\begin{enumerate}
    \item $y^2 = x^3$ becomes $Y^2 Z = X^3$.
    \item $y^2 = x(x-1)^2$ becomes $Y^2 Z = X (X - Z)^2$.
    \item $y^2 = x(x-1)(x-2)$ becomes $Y^2 Z = X (X - Z)(X - 2Z)$.
    \item $y^2 = x(x^2 + x + 1)$ becomes $Y^2 Z = X (X^2 + X Z + Z^2)$.
\end{enumerate}

\textbf{Solution (1):}
\begin{figure}[H]
    \centering
    \includegraphics[width=0.3\textwidth]{images/chapter2/5.png}
\end{figure}

\textbf{Solution (2):}
\begin{figure}[H]
    \centering
    \includegraphics[width=0.3\textwidth]{images/chapter2/6.png}
\end{figure}

\textbf{Solution (3):}
\begin{figure}[H]
    \centering
    \includegraphics[width=0.3\textwidth]{images/chapter2/7.png}
\end{figure}


\textbf{Solution (4):}
\begin{figure}[H]
    \centering
    \includegraphics[width=0.3\textwidth]{images/chapter2/8.png}
\end{figure}

\begin{tcolorbox}[title=Problem 3, breakable]
    Show that the following cubics are singular.
    \begin{enumerate}
        \item $V(xyz)$.
        \item $V(x(x^2 + y^2 - z^2))$.
        \item $V(x^3)$.
    \end{enumerate}
\end{tcolorbox}

\begin{proof}
    Here we list each function and calculate its gradient.
\begin{enumerate}
    \item For $f(x, y, z) = xyz$
    \[
        \nabla f = (yz, xz, xy).
    \]
    $(1:0:0)$ is a singular point.
    
    \item For $f(x, y, z) = x(x^2 + y^2 - z^2) = x^3 + xy^2 - x z^2$
    \[
        \nabla f = \left( 3x^2 + y^2 - z^2, 2xy, -2xz \right).
    \]
    $(0:0:0)$ is a singular point.

    \item For $f(x, y, z) = x^3$
    \[
        \nabla f = (3x^2, 0, 0).
    \]
    Any point with $x = 0$ is a singular point.
\end{enumerate}
\end{proof}


\begin{tcolorbox}[title=Problem 4, breakable]
    Sketch the cubic $y^2 = x^3$ in the real plane $\mathbb{R}^2$.
    Show that the corresonding cubic $V(x^3 - y^2z)$ in $\mathbb{P}^2$
    has a singular point at $(0 : 0 : 1)$.
    Show that this is the only singular point on this cubic.
\end{tcolorbox}

\textbf{Solution}
\begin{figure}[H]
    \centering
    \includegraphics[width=0.3\textwidth]{images/chapter2/9.png}
\end{figure}

\begin{proof}
    We have 
    \[\frac{\partial f}{\partial x}(x^3 - y^2 z) = 3x^2, \quad
    \frac{\partial f}{\partial y}(x^3 - y^2 z) = -2yz, \quad 
    \frac{\partial f}{\partial z}(x^3 - y^2 z) = -y^2.\]
    Thus $x = 0$ and $y = 0$. Then $z$ is any element in  $\mathbb{C} - \{0\}$.
    Therefore $(0 : 0 : 1) \in \mathbb{C} - \{(0, 0, 0)\}$ is a singularity.
\end{proof}

\begin{tcolorbox}[title=Problem 5, breakable]
    Show that the polynomial $P(x, y, z) = x^3 - y^2z$ is 
    irreducible, i.e., cannot be factored into two polynomials.
    (This is a fairly brute force high school algebra problem).
\end{tcolorbox}

\begin{proof}
    The only possible factorization is 
    \[(x + A)(x^2 + Bx + C) = x^3 + Bx^2 + Cx + Ax^2 + ABx + AC = x^3 + (A + B)x^2 + (AB + C)x + AC,\]
    where $A, B, C \in P(y, z)$.
    Comparing to $P(x, y, z)$ we must have
    \[A + B = 0, \quad AB + C = 0, \quad \text{and} \quad AC = -y^2 z.\]
    Then $A = -B$ thus $AB + C = A(-A) + C = -A^2 + C = 0 \iff C = A^2$.
    Then 
    \[AC = A(A^2) = A^3 \text{ thus } A^3 = -y^2 z.\]
    This is impossible since the cube of a polynomial must have all exponents divisible by $3$, but in $y^2 z$ the exponents of $y$ and $z$ are not divisible by $3$. Therefore $P(x,y,z)$ is irreducible.
\end{proof}

\subsection{Inflection Points}

\begin{tcolorbox}[title=Problem 1, breakable]
    If $(x - a)$ divides $P(x)$, show that $a$ is a root of $P(x)$.
\end{tcolorbox}

\begin{proof}
    We have $P(x) = (x - a)g(x)$ for some $g(x)$.
    Then $P(a) = (a - a)g(a) = 0 \cdot g(a) = 0$.
\end{proof}

\begin{tcolorbox}[title=Problem 2, breakable]
    If $a$ is a root of $P(x)$, show that $(x - a)$ divides 
    $P(x)$.
\end{tcolorbox}

\begin{proof}
    Suppose $a$ is a root of $P(x)$.
    By the Division Algorithm we obtain $q(x)$ and $r(x)$
    such that $P(x) = q(x)(x - a) + r(x)$.
    Then $P(a) = 0 = (a - a)q(a) + r(a) = r(a)$.
    Thus $r(a) = 0$. Since $r(x)$ has degree $<1$ it is a constant so $r(x) = 0$.
\end{proof}

\begin{tcolorbox}[title=Problem 3, breakable]
    Suppose that $a$ is a root of multiplicity two for $P(x)$.
    Show that there is a polynomial $g(x)$, with $g(a) \ne 0$,
    such that 
    \[P(x) = (x - a)^2 g(x).\]
\end{tcolorbox}

\begin{proof}
    By definition of multiplicity two, we can write $P(x) = (x - a)^2g(x)$ for some $g(x)$.
    Now suppose $g(a) = 0$ then $g(x) = (x - a)g'(x)$ for some $g'(x)$.
    But then $P(x) = (x - a)^2(x - a)g'(x) = (x - a)^3g'(x)$ contradicting 
    that $a$ has multiplicity two for $P(x)$.
    Therefore $g(a) \ne 0$.
\end{proof}

\begin{tcolorbox}[title=Problem 4, breakable]
    Suppose $a$ is a root of multiplicity two for $P(x)$.
    Show that $P(a) = 0$ and $P'(a) = 0$ but $P''(a) \ne 0$.
\end{tcolorbox}

\begin{proof}
    By definition of multiplicity two, we can write $P(x) = (x - a)^2g(x)$ for some $g(x)$ with $g(a) \ne 0$.
    Clearly, $P(a) = (a - a)^2g(a) = 0 \cdot g(a) = 0$.
    Then 
    \[P'(x) = 2(x - a)g(x) + (x - a)^2g'(x) \text{ and } P''(x) = 2g(x) + 4(x - a)g'(x) + (x - a)^2 g''(x).\]
    Then $P'(a) = 0$ and $P''(a) = 2 g(a)$ where $g(a) \ne 0$ thus $P''(a) \ne 0$.
\end{proof}

\begin{tcolorbox}[title=Problem 5, breakable]
    Suppose that $a$ is a root of multiplicity $k$ for $P(x)$.
    Show there is a polynomial $g(x)$ such that
    \[P(x) = (x - a)^kg(x)\]
    with $g(a) \ne 0$.
\end{tcolorbox}

\begin{proof}
    By definition of multiplicity $k$, we can write $P(x) = (x - a)^k g(x)$ for some $g(x)$.
    Now suppose $g(a) = 0$ then $g(x) = (x - a)g'(x)$ for some $g'(x)$.
    But then $P(x) = (x - a)^k (x - a)g'(x) = (x - a)^{k + 1} g'(x)$ contradicting 
    that $a$ has multiplicity $k$ for $P(x)$.
    Therefore $g(a) \ne 0$.
\end{proof}


\begin{tcolorbox}[title=Problem 6, breakable]
    Suppose that $a$ is a root of multiplicity $k$ for $P(x)$.
    Show that $P(a) = P'(a) = \cdots = P^{(k - 1)}(a) = 0$ but $P^k(a) \ne 0$.
\end{tcolorbox}

\begin{proof}
    Suppose $a$ is a root of multiplicity $k$. Then
    \[
    P(x) = (x-a)^k g(x)
    \]
    for some polynomial $g(x)$ with $g(a) \ne 0$.
    Taking the $k$-th derivative and applying the product rule repeatedly gives
    \[
    P^{(k)}(x)
    =
    \underbrace{k! g(x)}_{\text{no factor of }(x-a)}
    +
    \underbrace{k! (x-a) g'(x)}_{\text{has factor }(x-a)}
    +
    \underbrace{\cdots}_{\text{has factor }(x-a)}
    +
    \underbrace{(x-a)^k g^{(k)}(x)}_{\text{has factor }(x-a)}.
    \]
    Every term except the first contains a factor of $(x-a)$. Therefore, evaluating at $x=a$ gives
    \[
    P^{(k)}(a)
    =
    \underbrace{k! g(a)}_{\ne 0}
    +
    \underbrace{0 + \cdots + 0}_{\text{all contain factor }(x-a)}
    \ne 0,
    \]
    since $g(a) \ne 0$.
    Similarly, for any derivative of order $m < k$, every term contains a factor of $(x-a)$, so
    \[
    P(a) = P'(a) = \cdots = P^{(k-1)}(a) = 0.
    \]
\end{proof}


\begin{tcolorbox}[title=Problem 7, breakable]
    Suppose that $(a : b)$ is a root of multiplicity two for $P(x, y)$.
    Show that 
    \[P(a, b) = \frac{\partial f}{\partial x}(a,b) = \frac{\partial f}{\partial y}(a, b) = 0,\]
    but at least one of the second partials does not vanish at $(a : b)$.
\end{tcolorbox}

\begin{proof}
    Suppose $(a:b)$ is a root of multiplicity two for $P(x,y)$.  
    Then
    \[
    P(x,y) = (bx - ay)^2 g(x,y)
    \]
    for some polynomial $g(x,y)$ such that $g(a,b) \ne 0$.
    Then
    \[
    \frac{\partial P}{\partial x} = 2(bx - ay) \cdot b \, g(x,y) + (bx - ay)^2 \frac{\partial g}{\partial x}, \quad
    \frac{\partial P}{\partial y} = 2(bx - ay)(-a) \, g(x,y) + (bx - ay)^2 \frac{\partial g}{\partial y}.
    \]
    Then
    \[
    \frac{\partial P}{\partial x}(a,b) = \frac{\partial P}{\partial y}(a,b) = 0.
    \]
    Then
    \begin{align*}
    \frac{\partial^2 P}{\partial x^2} &= 2 b^2 g(x,y) + 4 b (bx - ay) \frac{\partial g}{\partial x} + (bx - ay)^2 \frac{\partial^2 g}{\partial x^2}, \\
    \frac{\partial^2 P}{\partial x \partial y} &= -2 ab g(x,y) + 2 b (bx - ay) \frac{\partial g}{\partial y} - 2 a (bx - ay) \frac{\partial g}{\partial x} + (bx - ay)^2 \frac{\partial^2 g}{\partial x \partial y}, \\
    \frac{\partial^2 P}{\partial y^2} &= 2 a^2 g(x,y) - 4 a (bx - ay) \frac{\partial g}{\partial y} + (bx - ay)^2 \frac{\partial^2 g}{\partial y^2}.
    \end{align*}
    Plugging in $(x,y) = (a, b)$ shows
    \[
    \frac{\partial^2 P}{\partial x^2}(a,b) = 2 b^2 g(a,b), \quad
    \frac{\partial^2 P}{\partial x \partial y}(a,b) = -2 ab g(a,b), \quad
    \frac{\partial^2 P}{\partial y^2}(a,b) = 2 a^2 g(a,b).
    \]
    Since $g(a,b) \ne 0$ and $(a,b) \ne (0,0)$, some second partial derivative is nonzero.
\end{proof}


\begin{tcolorbox}[title=Problem 8, breakable]
    Suppose $(a : b)$ is a root of multiplicity $k$ 
    for $P(x, y)$. Show that 
    \[P(a, b) = \frac{\partial P}{\partial x}(a, b) = \frac{\partial P}{\partial y}(a, b) = \cdots = \frac{\partial^{k - 1} P}{\partial x^i \partial y^j}(a, b) = 0,\]
    where $i + j = k - 1$ but 
    \[\frac{\partial^k P}{\partial x^i \partial y^j}(a, b) \ne 0,\]
    for at least one pair $i + j = k$.
\end{tcolorbox}

\begin{proof}
Suppose $(a:b)$ is a root of multiplicity $k$. Then
\[
P(x,y) = (bx-ay)^k g(x,y)
\]
for some polynomial $g(x,y)$ with $g(a,b)\ne 0$.

Taking a partial derivative of total order $k$ and applying the product rule repeatedly gives
\[
\frac{\partial^k P}{\partial x^i \partial y^j}
=
\underbrace{
k! \, b^i (-a)^j \, g(x,y)
}_{\text{no factor of }(bx-ay)}
+
\underbrace{
(bx-ay)(\cdots)
}_{\text{has factor }(bx-ay)}
+
\underbrace{
\cdots
}_{\text{has factor }(bx-ay)}
+
\underbrace{
(bx-ay)^k \frac{\partial^k g}{\partial x^i \partial y^j}
}_{\text{has factor }(bx-ay)}.
\]

Every term except the first contains a factor of $(bx-ay)$. Therefore, evaluating at $(x,y)=(a,b)$ gives
\[
\frac{\partial^k P}{\partial x^i \partial y^j}(a,b)
=
\underbrace{
k! \, b^i (-a)^j \, g(a,b)
}_{\ne 0}
+
\underbrace{0 + \cdots + 0}_{\text{all contain factor }(bx-ay)}
\ne 0,
\]
since $g(a,b)\ne 0$.

Similarly, any derivative of total order $m<k$ still contains a factor of $(bx-ay)$, so
\[
P(a,b)
=
\frac{\partial P}{\partial x}(a,b)
=
\frac{\partial P}{\partial y}(a,b)
=
\cdots
=
\frac{\partial^m P}{\partial x^i \partial y^j}(a,b)
=
0.
\]
\end{proof}


\begin{tcolorbox}[title=Problem 9, breakable]
    Supppose 
    \[P(a, b) = \frac{\partial P}{\partial x}(a, b) = \frac{\partial P}{\partial y}(a, b) = \cdots = \frac{\partial^{k - 1} P}{\partial x^i \partial y^j}(a, b) = 0,\]
    where $i + j =  k - 1$ and 
    \[\frac{\partial^k P}{\partial x^i \partial y^j}(a, b) \ne 0,\]
    for at least one pair $i + j = k$. Show that $(a : b)$ is a root of multiplicity $k$ for $P(x, y)$.
\end{tcolorbox}

\begin{proof}
    Suppose for contradiction that $(a:b)$ is a root of multiplicity $m \ne k$.
    Now suppose $m < k$. Then
    \[
    P(x,y) = (bx-ay)^m g(x,y)
    \]
    where $g(a,b)\ne 0$.
    Taking a partial derivative of total order $m$ gives
    \[
    \frac{\partial^m P}{\partial x^i \partial y^j}(a,b)
    =
    \underbrace{m! b^i (-a)^j g(a,b)}_{\ne 0}
    +
    \underbrace{0 + \cdots + 0}_{\text{all contain factor }(bx-ay)}
    \ne 0,
    \]
    which contradicts the assumption that all derivatives of order less than $k$ vanish.
    Next suppose $m > k$. Then
    \[
    P(x,y) = (bx-ay)^m g(x,y).
    \]
    Taking any derivative of total order $k$ every term contains a factor of $(bx-ay)$ thus
    \[
    \frac{\partial^k P}{\partial x^i \partial y^j}(a,b) = 0,
    \]
    which contradicts the assumption that at least one derivative of order $k$ is nonzero.
    Therefore $m=k$, and $(a:b)$ is a root of multiplicity $k$.
\end{proof}

\begin{tcolorbox}[title=Problem 10, breakable]
    Let $(x_0 : y_0 : z_0) \in V(P) \cap V(l)$. Show that $(x_0 : z_0)$
        is a root of the homogeneous two-variable polynomial $P(x, ax + cz, z)$
        show that $y_0 = a x_0 + c z_0$.
\end{tcolorbox}

\begin{proof}
    Since $(x_0 : y_0 : z_0) \in V(l)$
    \[
    a x_0 - y_0 + c z_0 = 0.
    \]
    Solving for $y_0$ gives
    \[
    y_0 = a x_0 + c z_0.
    \]
    Then define
    \[
    Q(x,z) := P(x, a x + c z, z).
    \]
    Then
    \[
    Q(x_0, z_0) = P(x_0, a x_0 + c z_0, z_0) = P(x_0, y_0, z_0) = 0,
    \]
    since $(x_0:y_0:z_0) \in V(P)$.
\end{proof}


\begin{tcolorbox}[title=Problem 11, breakable]
    Let $P(x, y, z) = x^2 - yz$ and $l(x, y, z) = \lambda x - y$.
    Show that the intersection multiplicity of $V(P)$ and $V(l)$ at $(0 : 0 : 1)$
    is one when $\lambda \ne 0$ and two when $\lambda = 0$.
\end{tcolorbox}

\begin{proof}
    We have $y = \lambda x$.
    \[P(x, y, z) = P(x, \lambda x, 1) = x^2 - \lambda x \cdot 1 = x(x - \lambda)= 0.\]
    If $\lambda = 0$ then $x = 0$ is the only solution.
    If $\lambda \ne 0$ then $x = 0$ and $x = \lambda$ are solutions.
\end{proof}


\begin{tcolorbox}[title=Problem 12, breakable]
    Let $(x_0 : y_0 : z_0) \in V(P) \cap V(l)$. 
    Let $x = a_1 s + b_1 t, y = a_2 s + b_2 t, z = a_3 s + b_3 t$ and 
    $x = c_1 u + d_1 v, y = c_2 u + d_2 v, z = c_3 u + d_3 v$ be two 
    parametrizations of the line $V(l)$ such that 
    $(x_0 : y_0 : z_0)$ corresponds to $(s_0 : t_0)$ and $(u_0 : v_0)$,
    respectively. 
    Show that the multiplicity of the root $(s_0 : t_0)$ of
    $P(a_1 s + b_1 t, a_2 s + b_2 t, a_3 s + b_3 t)$ is equal 
    to the multiplicityof the root $(u_0, v_0)$ of 
    $P(c_1 u + d_1 v, c_2 u + d_2 v, c_3 u + d_3 v)$. Conclude
    that our definition of the intersection multiplicity of $V(P)$
    and $V(l)$ is independent of the paramtrization used for the line $V(l)$.
\end{tcolorbox}

\begin{proof}
    Let 
    \[
    f(s,t) = P(a_1 s + b_1 t, a_2 s + b_2 t, a_3 s + b_3 t), \quad
    g(u,v) = P(c_1 u + d_1 v, c_2 u + d_2 v, c_3 u + d_3 v).
    \]
    Since $(s_0:t_0)$ and $(u_0:v_0)$ correspond to the same point $(x_0:y_0:z_0)$, there exist constants $\alpha, \beta, \gamma, \delta$ such that
    \[
    s = \alpha u + \beta v, \quad t = \gamma u + \delta v
    \]
    and therefore $f(s,t) = g(u,v)$ for all points on the line.
    It follows that the parametrizations have the same multiplicity of roots.
\end{proof}

\begin{tcolorbox}[title=Problem 13, breakable]
    Let $P(x, y, z) = x^2 + 2xy - yz + z^2$.
    Show that the intersection and multiplicity of $V(P)$
    and any line $l$ at a point of intersection is at most two.
\end{tcolorbox}

\begin{proof}
    Let $l = a x_0 + b y_0 + c z_0$ be an arbitrary line with $a, b$, or $c$ being nonzero.
    Wlog suppose $b \ne 0$. We can let $b = -1$ and get 
    \[y_0 = a x_0 + c z_0.\]
    Then substituting into $P$ we have 
    \[
        P(x_0, a x_0 + c z_0, z_0) = x_0^2 + 2 x_0 (a x_0 + c z_0) - (a x_0 + c z_0) z_0 + z_0^2.
    \]
    Simplifying, we get
    \[
        P(x_0, a x_0 + c z_0, z_0) = (1 + 2a)x_0^2 + (2c - a)x_0 z_0 + (1 - c) z_0^2.
    \]
    Therefore any root has multiplicity at most $2$.
\end{proof}

\begin{tcolorbox}[title=Problem 14, breakable]
    Let $P(x, y, z)$ be an irreducible second degree homogeneous polynomial.
    Show that the intersection multiplicity of $V(P)$ and any line $l$ at a point 
    of intersection is at most two.
\end{tcolorbox}

\begin{proof}
    If we substitute the equation of the line into the second degree homogeneous polynomial, 
    we get another second degree polynomial in one or two variables. 
    But if a root had multiplicity greater than $2$, it would require a polynomial with total degree greater than $2$.
    Thus the intersection multiplicity is at most two.
\end{proof}

\begin{tcolorbox}[title=Problem 15, breakable]
    Let $P(x, y, z) = x^2 + y^2 + 2xz - yz$.
    \begin{enumerate}
        \item Find the tangent line $l = V(l)$ to $V(P)$ at $(-2 : 1 : 1)$.
        \item Show that the intersection multiplicity of $V(P)$ and $l$ at 
        $(-2 : 1 : 1)$ is two.
    \end{enumerate}
\end{tcolorbox}

\begin{proof}
    Notice
    \[
    \nabla P(x,y,z) = (2x + 2z,\, 2y - z,\, 2x - y).
    \]
    Then, at $(-2,1,1)$ we have
    \[
    \nabla P(-2,1,1) = (-2,\,1,\,-5).
    \]
    Thus the tangent plane is
    \[
    -2(x + 2) + 1(y - 1) - 5(z - 1) = 0 \quad \Longrightarrow \quad -2x + y - 5z = 0.
    \]
    Parametrizing
    \[
    (x,y,z) = (-2,1,1) + t(2,9,1) = (-2+2t,\, 1+9t,\, 1+t).
    \]
    Plugging this into $P(x,y,z)$ gives
    \[
    \begin{aligned}
    P(-2+2t,\,1+9t,\,1+t) &= (-2+2t)^2 + (1+9t)^2 + 2(-2+2t)(1+t) - (1+9t)(1+t) \\
    &= 4 - 8t + 4t^2 + 1 + 18t + 81t^2 + 2(-2+2t)(1+t) - (1+9t)(1+t) \\
    &= 85 t^2.
    \end{aligned}
    \]
    Thus the polynomial vanishes with multiplicity 2 at $t=0$.
\end{proof}

\begin{tcolorbox}[title=Problem 16, breakable]
    Let $P(x, y, z) = x^3 - y^2 z + z^3$.
    \begin{enumerate}
        \item Find the tangent line to $V(P)$ at $(2 : 3 : 1)$ and show directly that 
        the intersection multiplicity of $V(P)$ and its tangent at $(2 : 3 : 1)$ is two.
        \item Find the tangent line to $V(P)$ at $(0 : 1 : 1)$ and show directly that 
        the intersection multiplicity of $V(P)$ and its tangent at $(0 : 1 : 1)$ is three.
    \end{enumerate}
\end{tcolorbox}

\begin{proof}
    Notice
    \[
    \nabla P(x,y,z) = (3x^2,\, -2yz,\, -y^2 + 3z^2).
    \]
    Then, at $(2,3,1)$ we have
    \[
    \nabla P(2,3,1) = (12, -6, -6).
    \]
    Thus the tangent plane is
    \[
    12(x - 2) - 6(y - 3) - 6(z - 1) = 0 
    \quad \Longrightarrow \quad 2x - y - z = 0.
    \]
    Parametrizing 
    \[
    (x,y,z) = (2,3,1) + t(1,2,0) = (2+t,\, 3+2t,\, 1).
    \]
    Plugging this into $P(x,y,z)$ gives
    \[
    \begin{aligned}
    P(2+t,\, 3+2t,\, 1) 
    &= (2+t)^3 - (3+2t)^2 \cdot 1 + 1^3 \\
    &= 8 + 12t + 6t^2 + t^3 - (9 + 12t + 4t^2) + 1 \\
    &= 2t^2 + t^3.
    \end{aligned}
    \]
    Let
    \[
    f(t) = 2t^2 + t^3.
    \]
    Then
    \[
    f(0) = 0, \quad f'(t) = 4t + 3t^2, \quad f'(0) = 0,
    \]
    and
    \[
    f''(t) = 4 + 6t, \quad f''(0) \neq 0.
    \]
    Thus by Exercise 2.2.9, the root has multiplicity 2 at $t=0$.

    Now, at $(0,1,1)$ we have
    \[
    \nabla P(0,1,1) = (0, -2, 2).
    \]
    Thus the tangent plane is
    \[
    0(x - 0) -2(y - 1) + 2(z - 1) = 0 
    \quad \Longrightarrow \quad y = z.
    \]
    Parametrizing 
    \[
    (x,y,z) = (0,1,1) + t(1,1,1) = (t,\, 1+t,\, 1+t).
    \]
    Plugging this into $P(x,y,z)$ gives
    \[
    \begin{aligned}
    P(t,\, 1+t,\, 1+t) 
    &= t^3 - (1+t)^2(1+t) + (1+t)^3 \\
    &= t^3.
    \end{aligned}
    \]
    Let
    \[
    g(t) = t^3.
    \]
    Then
    \[
    g(0) = 0, \quad g'(0) = 0, \quad g''(0) = 0,
    \]
    but
    \[
    g'''(0) \neq 0.
    \]
    Thus by Exercise 2.2.9, the root has multiplicity 3 at $t=0$.
\end{proof}

\begin{tcolorbox}[title=Problem 17, breakable]
    Redo the previous two exercises using Exercise 2.2.9.
\end{tcolorbox}

\begin{proof}
    \[P(x,y,z) = x^2 + y^2 + 2xz - yz, P = (-2:1:1).\]
    The tangent line computed previously is
    \[
    2x - y + 2z - z = 0 \quad \Longrightarrow \quad y = 2x + z.
    \]
    Plugging into the curve
    \[
    \begin{aligned}
    P(x,2x+z,z) &= x^2 + (2x+z)^2 + 2 x z - (2x+z) z \\
    &= x^2 + 4x^2 + 4xz + z^2 + 2xz - 2xz - z^2 \\
    &= 5x^2 + 4xz.
    \end{aligned}
    \]
    Compute derivatives at $x=-2$ 
    \[
    f(x) = 5x^2 + 4x \quad \Rightarrow \quad f(-2) = 0, \quad f'(-2) = 0, \quad f''(-2) \neq 0.
    \]
    Thus by Exercise 2.2.9, the intersection multiplicity at $(-2:1:1)$ is 2.
\end{proof}

\begin{proof}
    \[P(x,y,z) = x^3 - y^2 z + z^3,  P=(2:3:1).\] 
    The tangent line computed previously is
    \[
    2x - y - z = 0 \quad \Longrightarrow \quad y = 2x - z.
    \]
    Plugging into the curve
    \[
    \begin{aligned}
    P(x,2x-z,z) &= x^3 - (2x - z)^2 z + z^3 \\
    &= x(x-2z)^2.
    \end{aligned}
    \]
    Compute derivatives $x=2$ 
    \[
    f(2) = 0, \quad f'(2) = 0, \quad f''(2) \neq 0.
    \]
    Thus by Exercise 2.2.9, the intersection multiplicity at $P$ is 2.
\end{proof}

\begin{proof}
    \[Q=(0:1:1)\]
    The tangent line computed previously is
    \[
    y - z = 0 \quad \Longrightarrow \quad y = z.
    \]
    Plugging into the curve
    \[
    P(x,z,z) = x^3.
    \]
    Compute derivatives at $x=0$
    \[
    f(0) = 0, \quad f'(0) = 0, \quad f''(0) = 0, \quad f'''(0) \neq 0.
    \]
    Thus by Exercise 2.2.9, the intersection multiplicity at $Q$ is 3.
\end{proof}

\begin{tcolorbox}[title=Problem 18, breakable]
    Show for any nonsingular curve $V(P) \subset \mathbb{P}^2$,
    the intersection of multiplicity of $V(P)$ and its tangent line $l$
    at the point of tangency is at least two.
\end{tcolorbox}

\begin{proof}
    Suppose $V(P) \subset \mathbb{P}^2$ is nonsingular at a point $(x_0:y_0:z_0)$.  
    Then 
    \[
    \nabla P(x_0,y_0,z_0) \neq 0.
    \]
    The tangent line $l$ at $(x_0:y_0:z_0)$ is
    \[
    P_x(x_0,y_0,z_0) (x - x_0) + P_y(x_0,y_0,z_0) (y - y_0) + P_z(x_0,y_0,z_0) (z - z_0) = 0.
    \]
    Parametrize this line by $l(t)$ such that $l(0) = (x_0,y_0,z_0)$ thus
    \[
    f(t) = P(l(t)).
    \]
    Then
    \[
    f(0) = P(x_0,y_0,z_0) = 0, \quad f'(0) = 0
    \]
    since $l$ is tangent at the point.  
    Since $V(P)$ is nonsingular we have
    \[
    f''(0) \neq 0.
    \]
    By Exercise 2.2.9 the intersection multiplicity of $V(P)$ and $l$ at $(x_0:y_0:z_0)$ is at least $2$.
\end{proof}

\begin{tcolorbox}[title=Problem 19, breakable]
    \begin{enumerate}
        \item Let $P(x, y, z)$ be an irreducible degree three homogeneous polynomial.
        Show that the intersection multiplicity of $V(P)$ and any line $l$ at a point 
        of intersection is at most three.
        \item Let $P(x, y, z)$ be an irreducible homogeneous polynomial of degree $n$.
        Show that the intersection multiplicity of $V(P)$ and any line $l$ at a point of intersection 
        is at most $n$.
    \end{enumerate}
\end{tcolorbox}

\begin{proof}
    Composing $P$ with $l$ gives
    \[
        f(t) = P(l(t)),
    \] 
    which has degree at most $3$. 
\end{proof}

\begin{proof}
    Composing a degree $n$ polynomial $P$ with a line $l$ gives 
    \[
        f(t) = P(l(t)),
    \] 
    which has degree at most $n$.
\end{proof}


\begin{tcolorbox}[title=Problem 20, breakable]
    Let $P(x, y, z) = x^3 + y z^2$. Show that $(0 : 0 : 1)$ is an inflection 
    point of $V(P)$.
\end{tcolorbox}

\begin{proof}
    We have
    \[
    \nabla P(x, y, z) = (3x^2, z^2, 2yz).
    \]
    At $(0 : 0 : 1)$ we have 
    \[
    \nabla P(0, 0, 1) = (0, 1, 0).
    \]
    The tangent line at $(0:0:1)$ is
    \[
    P_x(0,0,1) (x-0) + P_y(0,0,1) (y-0) + P_z(0,0,1) (z-1) = 0,
    \]
    which simplifies to
    \[
    0 \cdot x + 1 \cdot y + 0 \cdot (z-1) = y = 0.
    \]
    Thus $y = 0$.  
    Plugging into $P$ we have
    \[
    P(x,0,z) = x^3 + 0 \cdot z^2 = x^3.
    \]
    Thus the tangent line intersects the curve with multiplicity $3$ at this point.  
    Therefore, $(0:0:1)$ is an inflection point of $V(P)$.
\end{proof}

\begin{tcolorbox}[title=Problem 21, breakable]
    Let $P(x, y, z) = x^3 + y^3 + z^3$ (the Fermat curve).
    Show that $(1 : -2 : 0)$ is an inflection point of $V(P)$.
\end{tcolorbox}

\begin{proof}
    We have
    \[
    \nabla P(x, y, z) = (3x^2, 3y^2, 3z^2).
    \]
    At $(1 : -2 : 0)$ we have 
    \[
    \nabla P(1, -2, 0) = (3, 12, 0).
    \]
    The tangent line at $(1:-2:0)$ is
    \[
    P_x(1,-2,0) (x-1) + P_y(1,-2,0) (y+2) + P_z(1,-2,0) (z-0) = 0,
    \]
    which simplifies to
    \[
    3(x-1) + 12(y+2) + 0 \cdot z = 3(x-1) + 12(y+2) = 0.
    \]
    Dividing by $3$ we have
    \[
    (x-1) + 4(y+2) = 0 \quad \implies \quad x + 4y + 7 = 0.
    \]
    Thus the tangent line is $x + 4y + 7 = 0$.  
    Plugging into $P$ we have
    \[
    P(-4y-7, y, z) = (-4y-7)^3 + y^3 + z^3.
    \]
    Thus the tangent line intersects the curve with multiplicity $3$ at this point.  
    Therefore, $(1:-2:0)$ is an inflection point of $V(P)$.
\end{proof}

\begin{tcolorbox}[title=Problem 22, breakable]
    Compute $H(P)$ for the following cubic polynomials.
    \begin{enumerate}
        \item $P(x, y, z) = x^3 + y z^2$.
        \item $P(x, y, z) = y^3 + z^3 + x y^2 - 3 y z^2  + 3 z y^2$.
        \item $P(x, y, z) = x^3 + y^3 + z^3$.
    \end{enumerate}
\end{tcolorbox}

\textbf{Solution (1):}
\[
\det 
\begin{bmatrix}
    P_{xx} & P_{xy} & P_{xz} \\
    P_{yx} & P_{yy} & P_{yz} \\
    P_{zx} & P_{zy} & P_{zz} \\
\end{bmatrix} 
=
\det 
\begin{bmatrix}
6x & 0 & 0 \\
0 & 0 & 2z \\
0 & 2z & 2y \\
\end{bmatrix} 
= 6x \cdot (0 \cdot 2y - 2z \cdot 2z) = -24 x z^2.
\]
\textbf{Solution (2):}
\begin{align*}
\det 
\begin{bmatrix}
    P_{xx} & P_{xy} & P_{xz} \\
    P_{yx} & P_{yy} & P_{yz} \\
    P_{zx} & P_{zy} & P_{zz} \\
\end{bmatrix} 
&=
\det 
\begin{bmatrix}
0 & 2y & 0 \\
2y & 6y + 2x + 6z & -6z + 6y \\
0 & -6z + 6y & 6z - 6y \\
\end{bmatrix} \\[2mm]
&= 0 \cdot ((6y+2x+6z)(6z-6y) - (-6z+6y)(-6z+6y)) \\ 
&\quad - 2y \cdot (2y (6z-6y) - (-6z+6y)\cdot 0) \\
&\quad + 0 \cdot (2y(-6z+6y) - (6y+2x+6z)\cdot 0) \\[1mm]
&= -2y \cdot (12y(z-y)) \\[1mm]
&= -24 y^2 (z-y).
\end{align*}
\textbf{Solution (3):}
\[
\det 
\begin{bmatrix}
    P_{xx} & P_{xy} & P_{xz} \\
    P_{yx} & P_{yy} & P_{yz} \\
    P_{zx} & P_{zy} & P_{zz} \\
\end{bmatrix} 
=
\det 
\begin{bmatrix}
6x & 0 & 0 \\
0 & 6y & 0 \\
0 & 0 & 6z \\
\end{bmatrix} 
= 6x \cdot 6y \cdot 6z = 216 xyz.
\]

\begin{tcolorbox}[title=Problem 23, breakable]
    Let $P(x, y, z)$ be an irreducible homogeneous polynomial
    of degree three. Show that $H(P)$ is also a third degree 
    homogeneous polynomial.
\end{tcolorbox}

\begin{tcolorbox}[title=Problem 24, breakable]
    Let $P(x, y, z) = x^3 + y^3 + z^3$ (th Fermat curve).
    Show that $(1 : -1 : 0) \in V(P) \cap V(H(P))$.
\end{tcolorbox}

\begin{proof}
    Clearly 
    \[
    P(1, -1, 0) = 1^3 + (-1)^3 + 0^3 = 0.
    \]
    Then 
    \[
    H(P) = \det \begin{bmatrix}
        6x & 0 & 0 \\
        0 & 6y & 0 \\
        0 & 0 & 6z
    \end{bmatrix} = 6^3 xyz.
    \]
    Then
    \[
    H(P)(1, -1, 0) = 6^3 \cdot 1 \cdot (-1) \cdot 0 = 0.
    \]
    Therefore, $(1 : -1 : 0) \in V(P) \cap V(H(P))$.
\end{proof}


\begin{tcolorbox}[title=Problem 25, breakable]
    Let $P(x, y, z) = y^3 + z^3 + xy^2 - 3 y z^2 + 3 z y^2$.
    Show that $(-2 : 1 : 1) \in V(P) \cap V(H(P))$.
\end{tcolorbox}

\begin{proof}
    Clearly 
    \[
    P(-2, 1, 1) = 1^3 + 1^3 + (-2)(1^2) - 3 \cdot 1 \cdot 1^2 + 3 \cdot 1 \cdot 1^2 = 1 + 1 - 2 - 3 + 3 = 0.
    \]
    Then 
    \[
    H(P) = \det \begin{bmatrix}
        0 & 2y & 0 \\
        2y & 6y + 2x + 6z & -6z + 6y \\
        0 & -6z + 6y & 6z - 6y
    \end{bmatrix} = -24 y^2 (z - y).
    \]
    Then
    \[
    H(P)(-2, 1, 1) = -24 \cdot 1^2 \cdot (1 - 1) = 0.
    \]
    Therefore, $(-2 : 1 : 1) \in V(P) \cap V(H(P))$.
\end{proof}

\begin{tcolorbox}[title=Problem 26, breakable]
    Let $P(x, y, z) = x^3+ y z^2$. Show that 
    $(0 : 0 : 1) = V(P) \cap V(H(P))$.
\end{tcolorbox}

\begin{proof}
    Clearly 
    \[
    P(0, 0, 1) = 0^3 + 0 \cdot 1^2 = 0.
    \]
    Then 
    \[
    H(P) = \det \begin{bmatrix}
        6x & 0 & 0 \\
        0 & 0 & 2z \\
        0 & 2z & 2y
    \end{bmatrix} = -4 z^2 \cdot 6x = -24 x z^2.
    \]
    Then
    \[
    H(P)(0, 0, 1) = -24 \cdot 0 \cdot 1^2 = 0.
    \]
    Therefore, $(0 : 0 : 1) \in V(P) \cap V(H(P))$.
\end{proof}

\begin{tcolorbox}[title=Problem 27, breakable]
    Let $P(x, y, z) = x^3 + yz^2$.
    \begin{enumerate}
        \item Show that $(0 : 1 : 0) \in V(P) \cap V(H(P))$.
        \item Explain why $(0 : 1 : 0)$ is not an inflection point of $V(P)$.
    \end{enumerate}
\end{tcolorbox}

\begin{proof}
    Clearly 
    \[
    P(0, 1, 0) = 0^3 + 1 \cdot 0^2 = 0.
    \]
    Then 
    \[
    H(P) = \det \begin{bmatrix}
        6x & 0 & 0 \\
        0 & 0 & 2z \\
        0 & 2z & 2y
    \end{bmatrix} = -24 x z^2.
    \]
    Then
    \[
    H(P)(0, 1, 0) = -24 \cdot 0 \cdot 0^2 = 0.
    \]
    Therefore, $(0 : 1 : 0) \in V(P) \cap V(H(P))$.

    Now, note that $P_x = P_y = P_z = 0$.
    The tangent line of $P$ at $(0 : 1 : 0)$ is 
    \[P_x(x - 0) + P_y(y - 1) + P_z(z - 0) = 0.\]
    Thus $(0 : 1  : 0)$ is not an inflection point.
\end{proof}

\newpage
\begin{tcolorbox}[title=Problem 29, breakable]
    Consider the following projective change of coordinates 
    \[
    \begin{bmatrix}
        u \\ v \\ w
    \end{bmatrix}
    = A 
    \begin{bmatrix}
        x \\ y \\ z
    \end{bmatrix},
    \]
    where 
    \[A = 
    \begin{bmatrix}
        a_{11} & a_{12} & a_{13} \\
        a_{21} & a_{22} & a_{23} \\
        a_{31} & a_{32} & a_{33}
    \end{bmatrix}.\]
    Suppose that under the projective transformation $A$ the polynomial
        $P(x, y, z)$ becomes the polynomial $P(u, v, w)$.
    \begin{enumerate}
        \item Show that the Hessian matrix of $P$ and $Q$ are related by 
        \[
        \begin{bmatrix}
            P_{xx} & P_{xy} & P_{xz} \\
            P_{yx} & P_{yy} & P_{yz} \\
            P_{zx} & P_{zy} & P_{zz}
        \end{bmatrix}
        = A^T 
        \begin{bmatrix}
            Q_{uu} & Q_{uv} & Q_{uw} \\
            Q_{vu} & Q_{vv} & Q_{vw} \\
            Q_{wu} & Q_{wv} & Q_{ww}
        \end{bmatrix}
        A
        \]
        \item Conclude that $H(P)(x, y, z) = 0$ if and only if $H(Q)(u, v, w) = 0$.
    \end{enumerate}
\end{tcolorbox}

\begin{proof}
    First notice 
    \[
    P(x, y, z) = Q(u(x, y, z), v(x, y, z), w(x, y, z)).
    \]
    Then using the chain rule
    \[
    \frac{\partial P}{\partial x} = \frac{\partial Q}{\partial u} \frac{\partial u}{\partial x} 
        + \frac{\partial Q}{\partial v} \frac{\partial v}{\partial x} 
        + \frac{\partial Q}{\partial w} \frac{\partial w}{\partial x}
        = Q_u \frac{\partial u}{\partial x} + Q_v \frac{\partial v}{\partial x} + Q_w \frac{\partial w}{\partial x}.
    \] 
    Now 
    \[
    \frac{\partial u}{\partial x} = a_{11}, \quad \frac{\partial v}{\partial x} = a_{21}, \quad \frac{\partial w}{\partial x} = a_{31},
    \]
    so
    \[
    \frac{\partial P}{\partial x} = Q_u a_{11} + Q_v a_{21} + Q_w a_{31}.
    \]
    Then
    \[
    \frac{\partial^2 P}{\partial x^2} = a_{11} \frac{\partial Q_u}{\partial x} + a_{21} \frac{\partial Q_v}{\partial x} + a_{31} \frac{\partial Q_w}{\partial x}.
    \]
    Applying the chain rule again 
    \[
    \frac{\partial Q_u}{\partial x} = Q_{uu} a_{11} + Q_{uv} a_{21} + Q_{uw} a_{31}, \quad
    \frac{\partial Q_v}{\partial x} = Q_{vu} a_{11} + Q_{vv} a_{21} + Q_{vw} a_{31}, \quad
    \frac{\partial Q_w}{\partial x} = Q_{wu} a_{11} + Q_{wv} a_{21} + Q_{ww} a_{31}.
    \]
    Plugging these in gives
    \[
    \frac{\partial^2 P}{\partial x^2} = \sum_{k,l=1}^3 a_{k1} H(Q)_{kl} a_{l1},
    \]
    which is the \((1,1)\) entry of \(A^T H(Q) A\).
    Similarly every other entry of \(H(P)\) can be computed this way thus
    \[
    H(P) = A^T H(Q) A.
    \]
\end{proof}

\begin{proof}
Notice 
\begin{align*}
    H(P)(x, y, z) = 0 
    &\iff A^T H(Q)(u, v, w) A = 0 \\
    &\iff (A^T)^{-1} \, 0 \, A^{-1} = H(Q)(u, v, w) \\
    &\iff H(Q)(u, v, w) = 0.
\end{align*}
\end{proof}

\begin{tcolorbox}[title=Problem 30, breakable]
    Suppose $p$ is a point of inflection $V(P)$, and that 
    under a projective change of cooridnates the polynomial $P$
    becomes the polynomial $Q$ and $p \mapsto q$.
    Show that $q$ is a point of inflection of $Q$.
\end{tcolorbox}

\begin{proof}
    Let $l$ be the tangent line to $V(P)$ at $p$, and let $p, p', p''$ be the three points of intersection of $l$ with $V(P)$.
    Under a linear projective change of coordinates, $l \mapsto l'$, $V(P) \mapsto V(Q)$, $p \mapsto q$, $p' \mapsto q'$, and $p'' \mapsto q''$.  
    Since the change of coordinates preserve intersection multiplicities, the tangent line $l'$ intersects $V(Q)$ at $q$ with multiplicity three.  
    Thus $q$ is a point of inflection of $V(Q)$.
\end{proof}

\begin{tcolorbox}[title=Problem 31, breakable]
    Use Exercises 2.2.29 and 2.2.30 to explain why,
    in proving Theorem 2.2.28, it is enough to show that $p$ is a point 
    of inflection if and only if $H(P) = 0$ in the case where 
    $p = (0 : 0 : 1) \in V(P)$ and the tangent line $l$ 
    to $V(P)$ at $p$ is $y = 0$, i.e. $l = V(y)$.
\end{tcolorbox}

\begin{proof}
    Suppose $p$ is a point 
    of inflection if and only if $H(P) = 0$ in the case where 
    $p = (0 : 0 : 1) \in V(P)$ and the tangent line $l$ 
    to $V(P)$ at $p$ is $y = 0$, i.e., $l = V(y)$.  

    For a general point and tangent line we can perform a linear change of coordinates to move $p$ to $(0:0:1)$ and $l$ to $V(y)$.
    By Exercise 2.2.29 part 2, the Hessian vanishes at the transformed point if and only if it vanishes at the original point under the change of coordinates.  
    By Exercise 2.2.30 part 1 the vanishing of the Hessian in the transformed coordinates implies the vanishing of the Hessian for the original curve.  
    Thus it is enough to check the Hessian at the point $(0:0:1)$ with tangent $y = 0$.
\end{proof}

\begin{tcolorbox}[title=Problem 32, breakable]
    Explain why in the affine patch $z = 1$ the dehomogenized curve is 
    \[\lambda y + (a x^2 + bxy + c y^2) + \text{ higher order terms,}\]
    where $\lambda \ne 0$. [Hint: We know that $p \in V(P)$ and $p$ is nonsingular.]
\end{tcolorbox}

\textbf{Solution: } We can perform a change of coordinates to map $p$ to $(0 : 0 : 1)$
with tangent $y = 0$. Then, since $P$ is nonsingular, either $\frac{\partial P}{\partial x}(p) \ne 0$
or $\frac{\partial P}{\partial y}(p) \ne 0$. If $\frac{\partial P}{\partial y}(p) = 0$, we can perform
another change of coordinates so that $\frac{\partial P}{\partial y}(p) \ne 0$ and $\frac{\partial P}{\partial x}(p) = 0$.
Thus $\lambda$ is the slope of the tangent with respect to this final curve, and we have the curve in the form 
\[\lambda y + (a x^2 + bxy + c y^2) + \text{ higher order terms,}\]
with $\lambda \ne 0$.

\newpage
\begin{tcolorbox}[title=Problem 33, breakable]
    Explain why the intersection of $V(P)$ with the tangent $V(y)$
        at $p$ corresponds to the root $(0 : 1)$ of the equation 
    \[P(x, 0, z) = a x^2 z^{d - 2} + \text{ higher order terms } = 0,\]
    where $z = deg(P)$.
\end{tcolorbox}

\textbf{Solution: }
We write the point as $(x/z, 0, 1)$ which lies on the tangent line.  
Thus, plugging into $P$ gives
\[
P(x/z, 0, 1) = P(x, 0, z) = 0,
\]
so the intersection corresponds to the root $(0 : 1)$.

\begin{tcolorbox}[title=Problem 34, breakable]
    Show that $p$ is a point of inflection of $V(P)$ if and only if $a = 0$.
    [Hint: For $p$ to be an inflection point, what must the multiplicity $(0 : 1)$
    be in the equation in Exercise 2.2.33?]
\end{tcolorbox}

\begin{proof}
    Suppose $p$ is a point of inflection of $V(P)$.
    The multiplicity of the tangent line substituted into the curve must be $\ge 3$ since 
    it is an inflection point, the root $(0 : 1)$ of
    \[
    P(x,0,z) = a x^2 z^{d-2} + \text{ higher order terms}
    \]
    must have multiplicity at least $3$.
    Thus the $x^2$ term must vanish, so $a = 0$.

    Conversely, suppose $a = 0$.
    Then the lowest degree term in $P(x,0,z)$ is degree at least $3$ in $x$,
    so the root $(0 : 1)$ has multiplicity at least $3$.
    Thus the tangent line intersects the curve with multiplicity at least $3$,
    so $p$ is a point of inflection of $V(P)$.
\end{proof}

\begin{tcolorbox}[title=Problem 35, breakable]
    \begin{enumerate}
        \item Show that 
        \[H(P)(p) = \det 
        \begin{bmatrix}
            2a & b & 0 \\
            b & 2c & \lambda (d - 1) \\
            0 & \lambda (d - 1) & 0
        \end{bmatrix}.
        \]
        \item Conclude that $p \in V(H(P))$ if and only if $a = 0$.
    \end{enumerate}
\end{tcolorbox}

\begin{tcolorbox}[title=Problem 36, breakable]
    Let $P(x, y, z)$ be an irreducible second degree homogeneous polynomial.
    Using the Hessian curve, show that $V(P)$ has no points of inflection.
\end{tcolorbox}

\begin{tcolorbox}[title=Problem 38, breakable]
    Use Exercise 2.2.38 and Theorems 2.2.28 and 2.2.37 
    to show that if $V(P)$ is a smooth curve, then $V(P)$
    has exactly nine inflection points.
\end{tcolorbox}

\begin{tcolorbox}[title=Problem 39, breakable]
    Find all nine points of inflection of the Fermat curve,
    \[P(x, y, z) = x^3 + y^3 + z^3.\]
\end{tcolorbox}

\subsection{Group Law}

\begin{tcolorbox}[title=Problem 1, breakable]
    Explain why the chord-tangent composition law is commutative, i.e.
    $PQ = QP$ for all points $P, Q$ on $C$.
\end{tcolorbox}

\textbf{Solution: } The third point of intersection of $V(p)$ and $V(l)$ is the same since the line between $P, Q$ 
is equivalent to the line between $Q, P$. Therefore, the resulting point is the same, and thus $PQ = QP$.


\begin{tcolorbox}[title=Problem 2, breakable]
    Consider the cubic curve  
    \[C = \{(x, y) \in \mathbb{C}^2 \mid y^2 = x^3 - x\},\]
    and the points $P, Q, R$ on $C$, as shown below. (Note that only the 
    real part of $C$ is shown.)
    Using a straightedge, locate $PQ$ and then $(PQ)R$ on the curve $C$.
    Now locat ethe point $QR$ and the point $P(QR)$ on the curve $C$.
    Is it true that $P(QR) = (PQ)R$? That is, is the chord-tangent composition
    law associative for these points on $C$?
\end{tcolorbox}

\textbf{Solution: } Clearly not from the figure below.

\begin{figure}[H]
    \centering
    \includegraphics[width=\textwidth]{images/chapter2/10.png}
\end{figure}


\begin{tcolorbox}[title=Problem 3, breakable]
    Consider the cubic curve 
    \[C = \{(x, y) \in \mathbb{C}^2 \mid x^3 + y^3 = 1\},\]
    and the points $P = (0, 1)$ and $Q = (1, 0)$ on $C$.
    \begin{enumerate}
        \item Using the equation of the cubic curve $C$ and its Hessian,
        verify that $P$ and $Q$ are inflection points of $C$.
        \item Verify that $PP = P$. Conclude that if $C$ has an idententity element $e$, then $e = P$.
        \item Verify that $QQ = Q$. Conclude that if $C$ has an idententity element $e$, then $e = Q$.
        \item Conclude that $C$ does not have an idententity element for the chord-tangent composition law.
    \end{enumerate}
\end{tcolorbox}

\begin{proof}
    Clearly
    \[0^3 + 1^3  - 1 = 0 \text{ and } 1^3 + 0^3 - 1 = 0.\]
    Thus $P, Q \in V(C)$.
    Now
    \[
    F_x = 3x^2, \quad F_y = 3y^2,
    \]
    and
    \[
    F_{xx} = 6x, \quad F_{yy} = 6y, \quad F_{xy} = 0, \quad F_{yx} = 0.
    \]
    Finally
    \[
    H(F) =
    \begin{bmatrix}
    6x & 0 \\
    0 & 6y
    \end{bmatrix}.
    \]
    Computing the determinant we have
    \[
    \det(H(F)) = (6x)(6y) - 0 = 36xy.
    \]
    Then 
    \[36(0)(1) = 0 \text{ and } 36(1)(0) = 0,\]
    thus $P, Q \in V(H(C))$.
    It follows that $P, Q$ are inflection points.
    Since $PP = P$ it follows that if $C$ has an idententity element $e$ then $e = P$.
    Similarly, since $QQ = Q$ it follows that if $C$ has an idenity element $e$ then $e = Q$.
    The idenity is unique and since $P \ne Q$ the chord-tangent is not a group.
\end{proof}


\begin{tcolorbox}[title=Problem 4, breakable]
    Let $P_1 = (2 : 3 : 1), P_2 = (0 : 1 : 1), P_3 = (-1 : 0 : 1), P_4 = (0 : -1 : 1), P_5 = (2 : -3 : 1)$ and 
    \[C = V(x^3 - y^2 z + z^3) \subset \mathbb{P}^2.\]
    Use the equations of the cubic curve $C$ and its Hessian to 
    verify that $P_2$ and $P_4$ are inflection points of $C$.
\end{tcolorbox}

\begin{proof}
    Clearly
    \[
    0^3 - 1^2\cdot 1 + 1^3 = 0 \text{ and } 0^3 - (-1)^2\cdot 1 + 1^3 = 0.
    \]
    Thus $P_2, P_4 \in V(C)$.
    Now
    \[
    F_x = 3x^2, \quad F_y = -2yz, \quad F_z = -y^2 + 3z^2,
    \]
    and
    \[
    F_{xx} = 6x, \quad F_{yy} = -2z, \quad F_{zz} = 6z, \quad F_{xy} = F_{yx} = 0, \quad F_{xz} = F_{zx} = 0, \quad F_{yz} = F_{zy} = -2y.
    \]
    Finally
    \[
    H(F) =
    \begin{bmatrix}
    6x & 0 & 0 \\
    0 & -2z & -2y \\
    0 & -2y & 6z
    \end{bmatrix}.
    \]
    Computing the determinant we have
    \[
    \det(H(F)) = 6x((-2z)(6z) - (-2y)(-2y)) - 0 + 0 = 6x(-12z^2 - 4y^2) = -72 x z^2 - 24 x y^2.
    \]
    Then 
    \[
    \det(H(F))(0,1,1) = -72\cdot 0 \cdot 1^2 - 24\cdot 0 \cdot 1^2 = 0 \quad \text{and} \quad
    \det(H(F))(0,-1,1) = -72\cdot 0 \cdot 1^2 - 24\cdot 0 \cdot (-1)^2 = 0,
    \]
    thus $P_2, P_4 \in V(H(C))$.
\end{proof}


\begin{tcolorbox}[title=Problem 5, breakable]
    Let $O = P_2$ be the specified inflection point so that $+$
    is defined relative to $P_2$, i.e., $Q + R = P_2(QR)$ for points $Q, R$
    on $C$.
    \begin{enumerate}
        \item Compute $P_1 + P_2, P_2 + P_2, P_3 + P_2, P_4 + P_2$, and $P_5 + P_2$.
        \item Explain why $P_2$ is the idententity element for $C$.
        \item Find the inverses of $P_1, P_2, P_3, P_4$ and $P_5$ on $C$.
        \item Verify that $P_1 + (P_3 + P_4) = (P_1 + P_3) + P_4$. In general,
        addition of points on $C$ is associative.
    \end{enumerate}
\end{tcolorbox}

\begin{proof}
    We have 
    \[P_1 + P_2 = P_2(P_1 P_2) = P_2 P_3 = P_1,\]
    \[P_2 + P_2 = P_2(P_2 P_2) = P_2 P_2 = P_2,\]
    \[P_4 + P_2 = P_2(P_4 P_2) = \]
\end{proof}