\documentclass[6pt]{article}

\usepackage[utf8]{inputenc}
\usepackage{needspace}
\usepackage{geometry}
\geometry{a4paper}
\usepackage{graphicx}
\usepackage[parfill]{parskip}
\usepackage{booktabs} 
\usepackage{array} 
\usepackage{paralist}
\usepackage{verbatim}
\usepackage{subfig} 
\usepackage{fancyhdr}
\pagestyle{fancy} 
\renewcommand{\headrulewidth}{0pt} 
\lhead{}\chead{}\rhead{}
\lfoot{}\cfoot{\thepage}\rfoot{}

\usepackage{sectsty}
\allsectionsfont{\sffamily\mdseries\upshape} 


\usepackage[nottoc,notlof,notlot]{tocbibind}
\usepackage[titles,subfigure]{tocloft}
\renewcommand{\cftsecfont}{\rmfamily\mdseries\upshape}
\renewcommand{\cftsecpagefont}{\rmfamily\mdseries\upshape} 

\usepackage{amsmath, amssymb, amsthm}
\usepackage{tcolorbox}

\tcbuselibrary{breakable}

\title{Chapter 3 Real Numbers}
\author{Noah Lewis}
\begin{document}
\maketitle

\section{Addition and Multiplication}

\begin{tcolorbox}[title=Problem 1, breakable]
    Let $E$ be an abbreviation for even, and let $I$ be an abbreviation for odd.
    We know that: \\
    $E + E = E$, \\
    $E + I = I + E = I$, \\
    $I + I = E$, \\
    $EE = E$, \\
    $II = I$ \\
    $IE = EI = E$. \\
    (a) Show that addition for $E$ and $I$ is associative and commutative.
    Show that $E$ plays the role of a zero element for addition. What is
    the additive inverse of $E$? What is the additive inverse of $I$? \\
    (b) Show that multiplication for $E$ and $I$ is commutative and associative.
    Which of $E$ or $I$ behaves like $1$? Which behaves like $0$ for multiplication? 
    Show that multiplication is distributive with respect to addition.
\end{tcolorbox}

\textbf{Solution 1 (a)}

\textbf{Associative over Addition:} We check that $(A + B) + C = A + (B + C)$ 
    for all $A, B, C \in \{E, I\}$ by verifying all 8 cases:

- $(E + E) + E = E + E = E$, and $E + (E + E) = E + E = E$

- $(E + E) + I = E + I = I$, and $E + (E + I) = E + I = I$

- $(E + I) + E = I + E = I$, and $E + (I + E) = E + I = I$

- $(E + I) + I = I + I = E$, and $E + (I + I) = E + E = E$

- $(I + E) + E = I + E = I$, and $I + (E + E) = I + E = I$

- $(I + E) + I = I + I = E$, and $I + (E + I) = I + I = E$

- $(I + I) + E = E + E = E$, and $I + (I + E) = I + I = E$

- $(I + I) + I = E + I = I$, and $I + (I + I) = I + E = I$

\textbf{Commutative over Addition:} We check that $A + B = B + A$
    for all $A, B \in \{E, I\}$.

- $E + E = E = E + E$

- $E + I = I = I + E$

- $I + I = E = I + I$

\textbf{Zero Element:} $E$ plays the role of additive identity (zero element), since:

- $E + E = E$

- $I + E = I$

- $E + I = I$

\textbf{Additive Inverse of $E$:} $E$, since $E + E = E$.

\textbf{Additive Inverse of $I$:} $I$, since $I + I = E$.

\textbf{Solution 1 (b)}

\textbf{Associative over Multiplication:} We check that $(A \cdot B) \cdot C = A \cdot (B \cdot C)$
for all $A, B, C \in \{E, I\}$:

\begin{align*}
( E \cdot E ) \cdot E &= E \cdot E = E, &\quad E \cdot ( E \cdot E ) &= E \cdot E = E \\
( E \cdot E ) \cdot I &= E \cdot I = E, &\quad E \cdot ( E \cdot I ) &= E \cdot E = E \\
( E \cdot I ) \cdot E &= E \cdot E = E, &\quad E \cdot ( I \cdot E ) &= E \cdot E = E \\
( E \cdot I ) \cdot I &= E \cdot I = E, &\quad E \cdot ( I \cdot I ) &= E \cdot I = E \\
( I \cdot E ) \cdot E &= E \cdot E = E, &\quad I \cdot ( E \cdot E ) &= I \cdot E = E \\
( I \cdot E ) \cdot I &= E \cdot I = E, &\quad I \cdot ( E \cdot I ) &= I \cdot E = E \\
( I \cdot I ) \cdot E &= I \cdot E = E, &\quad I \cdot ( I \cdot E ) &= I \cdot E = E \\
( I \cdot I ) \cdot I &= I \cdot I = I, &\quad I \cdot ( I \cdot I ) &= I \cdot I = I
\end{align*}

\textbf{Commutative over Multiplication:} We check that $AB = BA$ 
    for all $A, B \in \{E, I\}$.

\begin{align*}
E \cdot I &= I \cdot E = E \\
I \cdot I &= I \cdot I = I \\
E \cdot E &= E \cdot E = E
\end{align*}

\textbf{Multiplicative Identity:} $I$ behaves like $1$ over multiplication.

- $II = I$

- $EI = E$

\textbf{Multiplicative Zero:} $E$ behaves like $0$ over multiplication.

- $IE = E$

- $EE = E$

\textbf{Distributive Over Addition:} We check that $A \cdot (B + C) = A \cdot B + A \cdot C$ for all $A, B, C \in \{E, I\}$. For example:

- $E(I + E) = E(I) = E = EI + EE = E + E = E$

- $I(I + E) = I(E) = E = II + IE = E + E = E$

- $E(E + I) = E(I) = E = EE + EI = E + E = E$

- $I(E + I) = I(I) = I = IE + II = E + I = I$

- $E(E + E) = E(E) = E = EE + EE = E + E = E$

- $I(E + E) = I(E) = E = IE + IE = E + E = E$

- $E(I + I) = E(E) = E = EI + EI = E + E = E$

- $I(I + I) = I(E) = E = II + II = I + I = E$

\begin{tcolorbox}[title=Problem 1, breakable]
    Prove:

    (a) If $a$ is a real number, then $a^2$ is positive.

    (b) If $a$ is positive and $b$ is negative, then $ab$ is negative.

    (c) If $a$ is negative and $b$ is negative, then $ab$ is positive.
\end{tcolorbox}

\begin{proof}
    By POS $2$ either $a = 0$, $a > 0$, or $a < 0$.

    \textbf{Case 1 ($a = 0$)}

    If $a = 0$ then $a^2 = a \cdot a = 0 \cdot 0  = 0 \ge 0$.

    \textbf{Case 2 ($a > 0$)}

    If $a > 0$, then by POS $1$, $a \cdot a = a^2 \ge 0$.

    \textbf{Case 3 ($a < 0$)}

    Since $a < 0$, by POS $2$, $-a > 0$.
    Then by POS $1$, $(-a) \cdot (-a) = a^2 > 0$.

    Therefore, $a^2 \ge 0$.
\end{proof}

\begin{proof}
    Assume for contradiction, $ab > 0$. By POS $2$, $-ab < 0$. Since $b < 0$ then, by POS $2$, $-b > 0$.
    Then by POS $1$, $a \cdot -b > 0$ so $-ab > 0$
    which is a contradiction. Therefore, if $a$ is positive and $b$ is negative, then $ab$ is negative.
\end{proof}

\begin{proof}
    Assume for contradiction, $ab < 0$. By POS $2$, $-ab > 0$. Since $b < 0$, $a < 0$ then, by POS $2$, $-b > 0$, $-a > 0$.
    Then by POS $1$, $-a \cdot -b > 0$ so $ab > 0$
    which is a contradiction. Therefore, if $a$ is negative and $b$ is negative, then $ab$ is positive.
\end{proof}

\begin{tcolorbox}[title=Problem 2, breakable]
    Prove: If $a$ is positive, then $a^{-1}$ is positive.
\end{tcolorbox}

\begin{proof}
    Suppose $a > 0$ and assume for contradiction $a^{-1} = \frac{1}{a} < 0$.  
    By Excersize $1$ part $c$, $a \cdot \frac{1}{a} < 0$.
    But $a \cdot \frac{1}{a} = \frac{a}{a} = 1 > 0$.
    Therefore, if $a$ is positive, then $a^{-1}$ is positive.
\end{proof}

\begin{tcolorbox}[title=Problem 3, breakable]
    Prove: If $a$ is negative, then $a^{-1}$ is negative.
\end{tcolorbox}

\begin{proof}
    Suppose $a < 0$ and assume for contradiction $\frac{1}{a} > 0$.  
    Since $a < 0$, by POS $2$, $0 < -a$. 
    Then by POS $1$, $-a \cdot \frac{1}{a} > 0$.
    But $-a \cdot \frac{1}{a} = \frac{-a}{a} = -1 < 0$ which is a contradiction.
    Therefore, if $a$ is negative, then $a^{-1}$ is negative.
\end{proof}

\begin{tcolorbox}[title=Problem 4, breakable]
    Prove: If $a$, $b$ are positive numbers, then 

    \[\sqrt{\frac{a}{b}} = \frac{\sqrt{a}}{\sqrt{b}}\]
\end{tcolorbox}

\begin{proof}
    \begin{align*}
        \sqrt{\frac{a}{b}} = \frac{\sqrt{a}}{\sqrt{b}} 
        \iff {\sqrt{\frac{a}{b}}}^2 = \left(\frac{\sqrt{a}}{\sqrt{b}}\right)^2
        \iff {\sqrt{\frac{a}{b}}}^2 = {\frac{\sqrt{a}^2}{\sqrt{b}^2}}
        \iff {\frac{a}{b} = \frac{a}{b}}
    \end{align*}
\end{proof}

\begin{tcolorbox}[title=Problem 5, breakable]
    Prove that 
    \[\frac{1}{1 - \sqrt{2}} = -(1 + \sqrt{2})\]
\end{tcolorbox}

\begin{proof}
    \begin{align*}
        &\frac{1}{1 - \sqrt{2}}
        = \frac{1 + \sqrt{2}}{1 + \sqrt{2}} \cdot \frac{1}{1 - \sqrt{2}}
        = \frac{1 + \sqrt{2}}{1 - 2}
        = \frac{1 + \sqrt{2}}{-1} \\
        &= \frac{-1}{-1} \cdot \frac{1 + \sqrt{2}}{-1} 
        = \frac{-(1 + \sqrt{2})}{1}
        = -(1 + \sqrt{2}) 
    \end{align*}
\end{proof}

\begin{tcolorbox}[title=Problem 8, breakable]
    Let $a$, $b$ be rational numbers. Prove that the multiplicative inverse of 
    $a + b\sqrt{2}$ can be expressed in the form $c + d\sqrt{2}$, where $c$, $d$
    are rational numbers.
\end{tcolorbox}

\begin{proof}
    First note since $a \in \mathbb{Q}$ and $b \in \mathbb{Q}$ therefore $a^2 - 2b^2 \in \mathbb{Q}$.
    In addition $a + b\sqrt{2} \not = 0$ (otherwise the inverse operation is undefined).
    If $b = 0$ then $a^2 \not = 0$ so $a^2 - 2b^2 \in \mathbb{Q}$ is defined.
    Now suppose $b \not = 0$.
    \begin{align*}
        a^2 = 2b^2 \iff \frac{a^2}{b^2} = 2 \iff \frac{a}{b} = \pm \sqrt{2}
    \end{align*}
    But $a \in \mathbb{Q}$ and $b \in \mathbb{Q}$ so their quotient is rational.
    This is impossible since $\sqrt{2}$ is irrational, so $a^2 - 2b^2 \neq 0$.
    Futhermore since $a^2 - 2b^2 \in \mathbb{Q}$ and $a^2 - 2b^2 \not = 0$,
        $\frac{a}{a^2 - 2b^2} \in \mathbb{Q}$ and $\frac{-b}{a^2 - 2b^2} \in \mathbb{Q}$.
    Now, let $c = \frac{a}{a^2 - 2b^2}$ and $d = \frac{-b}{a^2 - 2b^2}$.
    Then
    \begin{align*}
        & (a + b\sqrt{2}) \cdot (c + d\sqrt{2}) && \\
        = &(a + b\sqrt{2}) \cdot \left(\frac{a}{a^2 - 2b^2} + \frac{-b}{a^2 - 2b^2} \cdot \sqrt{2}\right) && \\
        = &(a + b\sqrt{2}) \cdot \left(\frac{a}{a^2 - 2b^2} + \frac{-b\sqrt{2}}{a^2 - 2b^2}\right) && \\
        = &(a + b\sqrt{2}) \cdot \left(\frac{a}{a^2 - 2b^2} - \frac{b\sqrt{2}}{a^2 - 2b^2}\right) && \\
        = &\left(\frac{a(a + b\sqrt{2})}{a^2 - 2b^2} - \frac{b\sqrt{2}(a + b\sqrt{2})}{a^2 - 2b^2}\right) && \\
        = &\frac{(a^2 + ab\sqrt{2}) - (ab\sqrt{2} + 2b^2)}{a^2 - 2b^2} && \\
        = &\frac{a^2 + ab\sqrt{2} - ab\sqrt{2} - 2b^2}{a^2 - 2b^2} && \\
        = &\frac{a^2 - 2b^2}{a^2 - 2b^2} && \\
        = &1
    \end{align*}
\end{proof}

\begin{tcolorbox}[title=Problem 11, breakable]
    Generalize Excersize $10$, replacing $\sqrt{5}$ by $\sqrt{a}$ for any positive integer $a$.
\end{tcolorbox}

\begin{proof}
    First note since $d \in \mathbb{Q}$ and $b \in \mathbb{Q}$ therefore $d^2 - ab^2 \in \mathbb{Q}$.
    In addition $d + b\sqrt{a} \not = 0$ (otherwise the inverse operation is undefined).

    If $b = 0$ then $d^2 \not = 0$ so $d^2 - ab^2 \in \mathbb{Q}$ is defined.

    Now suppose $b \not = 0$ and $\sqrt{a} \not \in \mathbb{Q}$.
    \begin{align*}
        d^2 = ab^2 \iff \frac{d^2}{b^2} = a \iff \frac{d}{b} = \pm \sqrt{a}
    \end{align*}
    But $d \in \mathbb{Q}$ and $b \in \mathbb{Q}$ so their quotient is rational.
    This is impossible if $\sqrt{a} \notin \mathbb{Q}$, so $d^2 - ab^2 \neq 0$.

    Now suppose $b \not = 0$ and $\sqrt{a} \in \mathbb{Q}$.
    \begin{align*}
        d = b\sqrt{a} \iff d^2 = b^2 a \iff d^2 - ab^2 = 0
    \end{align*}
    Since, $d \not = b\sqrt{a}$, $d^2 - ab^2 \not = 0$.

    Furthermore since $d^2 - ab^2 \in \mathbb{Q}$ and $d^2 - ab^2 \not = 0$,
        $\frac{d}{d^2 - ab^2} \in \mathbb{Q}$ and $\frac{-b}{d^2 - ab^2} \in \mathbb{Q}$.
    Now let $c = \frac{d}{d^2 - ab^2}$ and $e = \frac{-b}{d^2 - ab^2}$.
    Then
    \begin{align*}
        & (d + b\sqrt{a}) \cdot (c + e\sqrt{a}) && \\
        = &(d + b\sqrt{a}) \cdot \left(\frac{d}{d^2 - ab^2} + \frac{-b}{d^2 - ab^2} \cdot \sqrt{a}\right) && \\
        = &(d + b\sqrt{a}) \cdot \left(\frac{d}{d^2 - ab^2} + \frac{-b\sqrt{a}}{d^2 - ab^2}\right) && \\
        = &(d + b\sqrt{a}) \cdot \left(\frac{d}{d^2 - ab^2} - \frac{b\sqrt{a}}{d^2 - ab^2}\right) && \\
        = &\left(\frac{d(d + b\sqrt{a})}{d^2 - ab^2} - \frac{b\sqrt{a}(d + b\sqrt{a})}{d^2 - ab^2}\right) && \\
        = &\frac{(d^2 + db\sqrt{a}) - (db\sqrt{a} + ab^2)}{d^2 - ab^2} && \\
        = &\frac{d^2 + db\sqrt{a} - db\sqrt{a} - ab^2}{d^2 - ab^2} && \\
        = &\frac{d^2 - ab^2}{d^2 - ab^2} && \\
        = &1
    \end{align*}
\end{proof}

\begin{tcolorbox}[title=Problem 14, breakable]
    Find all possible numbers $x$ such that 

    (a) $|2x - 1| = 3$

    (b) $|3x + 1| = 2$

    (c) $|2x + 1| = 4$

    (d) $|3x - 1| = 1$

    (e) $|4x - 5| = 6$
\end{tcolorbox}

\textbf{Solution 14 (a)}

$x = 2$ or $x = -1$

\textbf{Solution 14 (b)}

$x = \frac{1}{3}$ or $x = -1$

\textbf{Solution 14 (c)}

$x = \frac{3}{2}$ or $x = \frac{-5}{2}$

\textbf{Solution 14 (d)}

$x = \frac{2}{3}$ or $x = 0$

\textbf{Solution 14 (e)}

$x = \frac{11}{4}$ or $x = \frac{-1}{4}$

\begin{tcolorbox}[title=Problem 15, breakable]
    Rationalize the numerator in the following expressions.

    (a) $\frac{\sqrt{x} + \sqrt{y}}{\sqrt{x} - \sqrt{y}}$

    (b) $\frac{\sqrt{x + y} - \sqrt{y}}{\sqrt{x + y} + \sqrt{y}}$

    (c) $\frac{\sqrt{x + 1} + \sqrt{x - 1}}{\sqrt{x + 1} - \sqrt{x - 1}}$

    (d) $\frac{\sqrt{x - 3} + \sqrt{x}}{\sqrt{x - 3} - \sqrt{x}}$

    (e) $\frac{\sqrt{x + y} - 1}{3 + \sqrt{x + y}}$

    (f) $\frac{\sqrt{x + y} + x}{\sqrt{x + y}}$
\end{tcolorbox}

\textbf{Solution 15 (a)}

\begin{align*}
    \frac{\sqrt{x} + \sqrt{y}}{\sqrt{x} - \sqrt{y}} \cdot \frac{\sqrt{x} - \sqrt{y}}{\sqrt{x} - \sqrt{y}}
        &= \frac{x - y}{x - 2\sqrt{xy} + y}
\end{align*}

\textbf{Solution 15 (b)}

\begin{align*}
    \frac{\sqrt{x + y} - \sqrt{y}}{\sqrt{x} + \sqrt{y}} \cdot \frac{\sqrt{x + y} + \sqrt{y}}{\sqrt{x + y} + \sqrt{y}}
        &= \frac{x}{\sqrt{x(x + y)} + \sqrt{xy} + \sqrt{y(x + y)} + y}
\end{align*}

\textbf{Solution 15 (c)}

\begin{align*}
    \frac{\sqrt{x + 1} + \sqrt{x - 1}}{\sqrt{x + 1} - \sqrt{x - 1}} 
        \cdot \frac{\sqrt{x + 1} - \sqrt{x - 1}}{\sqrt{x + 1} - \sqrt{x - 1}}
    &= \frac{2}{(\sqrt{x + 1} - \sqrt{x - 1})(\sqrt{x + 1} - \sqrt{x - 1})} && \\
    &= \frac{2}{{(\sqrt{x + 1} - \sqrt{x - 1})}^2}
\end{align*}

\textbf{Solution 15 (d)}

\begin{align*}
    \frac{\sqrt{x - 3} + \sqrt{x}}{\sqrt{x - 3} - \sqrt{x}}
        \cdot \frac{\sqrt{x - 3} - \sqrt{x}}{\sqrt{x - 3} - \sqrt{x}} 
        &= \frac{(x - 3) + x}{{(\sqrt{x - 3} - \sqrt{x})}^2} && \\
        &= \frac{-3}{{(\sqrt{x - 3} - \sqrt{x})}^2}
\end{align*}

\textbf{Solution 15 (e)}

\begin{align*}
    \frac{\sqrt{x + y} - 1}{3 + \sqrt{x + y}}
        \cdot \frac{\sqrt{x + y} + 1}{\sqrt{x + y} + 1}
        &= \frac{x + y - 1}{(3 + \sqrt{x + y})(\sqrt{x + y} + 1)}
\end{align*}

\textbf{Solution 15 (f)}

\begin{align*}
    \frac{\sqrt{x + y} + x}{\sqrt{x + y}}
        \cdot \frac{\sqrt{x + y} - x}{\sqrt{x + y} - x}
        &= \frac{x + y - x^2}{\sqrt{x + y}(\sqrt{x + y} - x)}
\end{align*}

\begin{tcolorbox}[title=Problem 17, breakable]
    Prove that there is no real number $x$ such that 

    \[\sqrt{x - 1} = 3 + \sqrt{x}\]

    [\emph{Hint}: Start by squaring both sides.]
\end{tcolorbox}

\begin{proof}
    Assume for contradiction there does exist a real number $x$ such that $\sqrt{x - 1} = 3  + \sqrt{x}$.
    Then 
        \begin{align*}
            & \sqrt{x - 1} = 3 + \sqrt{x} && \\
            &\leftrightarrow x - 1 = 9 + 6\sqrt{x} + x && \\
            &\leftrightarrow -1 = 9 + 6\sqrt{x} && \\
            &\leftrightarrow -10 = 6\sqrt{x} && \\
            &\leftrightarrow \frac{-10}{6} = \sqrt{x}
        \end{align*}
    Which is a contradiction. Therefore, there is no real number $x$
    such that $\sqrt{x - 1} = 3 + \sqrt{x}$.
\end{proof}

\begin{tcolorbox}[title=Problem 20, breakable]
    If $a$, $b$ are two numbers, prove that $|a - b| = |b - a|$.
\end{tcolorbox}

\begin{proof}
    Let $c = b - a$. By POS $2$ there are three cases.
    
    \textbf{Case 1 ($c = 0$)}
    If $b - a = 0$ then $b = a$ therefore $a - b = 0$.
    \begin{align*}
        |b - a| &= |a - b| && \\
        \leftrightarrow |0| &= |0| && \\
        \leftrightarrow 0 &= 0 
    \end{align*}
    \textbf{Case 2 ($c > 0$)}
    If $c > 0$ then $|c| = c$. Also $-c < 0$ so $|-c| = -(-c) = c$. Then
    \begin{align*}
        |b - a| &= |a - b| && \\
        \leftrightarrow |c| &= |-c| && \\
        \leftrightarrow c &= c
    \end{align*}
    \textbf{Case 3 ($c < 0$)}
    If $c < 0$ then $|c| = -c$. Also $-c > 0$ so $|-c| = -c$. Then
    \begin{align*}
        |b - a| &= |a - b| && \\
        \leftrightarrow |c| &= |-c| && \\
        \leftrightarrow -c &= -c
    \end{align*}

    Therefore $|a - b| = |b - a|$.
\end{proof}

\section{Powers and Roots}

\begin{tcolorbox}[title=Extra Problem, breakable]
    Suppose $a$ is a nonzero rational number and $b$ is an irrational real number.
    Show that $ab$ is irrational.
\end{tcolorbox}

\begin{proof}
    A number is rational if it can be written as $\frac{x}{y}$ with $x, y \in \mathbb{Z}$ and $y \neq 0$.  
    Assume for contradiction that $a \cdot b$ is rational, where $a \neq 0$ is rational and $b$ is irrational.  
    Since $a \neq 0$, we can divide both sides by $a$:  
    \[
        b = \frac{a \cdot b}{a}.
    \]  
    But the right-hand side is rational (a rational divided by a nonzero rational is rational), so $b$ would be rational.  
    This contradicts the assumption that $b$ is irrational.  
    Therefore, $a \cdot b$ must be irrational.
\end{proof}

\begin{tcolorbox}[title=Problem 1, breakable]
    Express each of the following in the form 
    $2^k 2^m a^r b^s$ where $k$, $m$, $r$,$s$
    are integers.

    (a) $\frac{1}{8} a^3 b^{-4} 2^5 a^{-2}$

    (b) $3^{-4} 2^5 a^3 b^6 \cdot \frac{1}{2^3} \cdot \frac{1}{a^4} \cdot b^{-1} \cdot \frac{1}{9}$

    (c) $\frac{3a^3b^4}{2a^5b^6}$

    (d) $\frac{16a^{-3}b^{-5}}{9b^4 a^7 2^{-3}}$
\end{tcolorbox}

\textbf{Solution (a):}
\[\frac{1}{8} a^3 b^{-4} 2^5 a^{-2} 
    = \frac{2^5}{8} a^3 a^{-2} b^{-4}
    = \frac{2^5}{8} a^1 b^{-4}
    = \frac{2^5}{2^3} a^1 b^{-4}
    = 2^2 3^0 a^1 b^{-4}\]
\textbf{Solution (b):}
\[3^{-4} 2^5 a^3 b^6 \cdot \frac{1}{2^3} \cdot \frac{1}{a^4} \cdot b^{-1} \cdot \frac{1}{9}
    = \frac{2^5}{2^3} \frac{3^{-4}}{9} \frac{a^3}{a^4} \frac{b^6}{b}
    = 2^2 \frac{3^{-4}}{3^2} \frac{a^3}{a^4} \frac{b^6}{b}
    = 2^2 3^{-6} a^{-1} b^5\]
\textbf{Solution (c):}
\[\frac{3 a^3 b^4}{2 a^5 b^6}
    = 2^{-1} 3^1 a^{-2} b^{-2}\]
\textbf{Solution (d):}
\[\frac{16 a^{-3} b^{-5}}{9 b^4 a^7 2^{-3}}
    = \frac{2^4 a^{-10} b^{-5}}{3^2 2^{-3}}
    = 2^7 3^{-2} a^{-10} b^{-9}\]

\begin{tcolorbox}[title=Problem 2, breakable]
    What integer is $81^{\frac{1}{4}}$ equal to?
\end{tcolorbox}

\textbf{Solution:}
\[81^{\frac{1}{4}} = (81^{\frac{1}{2}})^{\frac{1}{2}} = 9^{\frac{1}{2}} = 3\]

\begin{tcolorbox}[title=Problem 3, breakable]
    What integer is $(\sqrt{2})^6$ equal to?
\end{tcolorbox}

\textbf{Solution:}
\[(\sqrt{2})^6 = (\sqrt{2})^2 (\sqrt{2})^2 (\sqrt{2})^2 = 2 \cdot 2 \cdot 2 = 8\]

\begin{tcolorbox}[title=Problem 4, breakable]
    Is $(\sqrt{2})^5$ an integer?
\end{tcolorbox}

\textbf{Solution:}
\[(\sqrt{2})^5 = (\sqrt{2})^2 (\sqrt{2})^2 (\sqrt{2}) = 2 \cdot 2 \cdot \sqrt{2} = 4 \sqrt{2}\]

It is not an integer see extra problem proof.

\begin{tcolorbox}[title=Problem 5, breakable]
    Is $(\sqrt{2})^{-5}$ a rational number? Is $(\sqrt{2})^5$ a rational number?
\end{tcolorbox}

\textbf{Solution part 1:}
\[(\sqrt{2})^{-5} 
    = \frac{1}{(\sqrt{2})^5} 
    = \frac{1}{4 \sqrt{2}} 
    = \frac{1}{4 \sqrt{2}} \frac{4 \sqrt{2}}{4 \sqrt{2}} 
    = \frac{4 \sqrt{2}}{16 \cdot 2}
    = \frac{4 \sqrt{2}}{32}
    = \frac{4}{32} \sqrt{2} \]

By the extra problem this is not a rational number.

\textbf{Solution part 2:}
Same reason as problem $4$.

\begin{tcolorbox}[title=Problem 6, breakable]
    In each case, the expression is equal to an integer. Which one?

    (a) $16^{\frac{1}{4}}$

    (b) $8^{\frac{1}{3}}$

    (c) $9^{\frac{3}{2}}$

    (d) $1^{\frac{5}{4}}$

    (e) $8^{\frac{4}{3}}$

    (f) $64^{\frac{2}{4}}$

    (g) $25^{\frac{3}{2}}$
\end{tcolorbox}

\textbf{Solution:}

\begin{align*}
\text{(a)}\quad & 16^{\frac{1}{4}} = (16^{\frac{1}{2}})^{\frac{1}{2}} = 4^{\frac{1}{2}} = 2 \\
\text{(b)}\quad & 8^{\frac{1}{3}} = (2^3)^{\frac{1}{3}} = 2 \\
\text{(c)}\quad & 9^{\frac{3}{2}} = (9^{\frac{1}{2}})^3 = 3^3 = 27 \\
\text{(d)}\quad & 1^{\frac{5}{4}} = 1 \\
\text{(e)}\quad & 8^{\frac{4}{3}} = (8^{\frac{1}{3}})^4 = 2^4 = 16 \\
\text{(f)}\quad & 64^{\frac{2}{4}} = 64^{\frac{1}{2}} = 8 \\
\text{(g)}\quad & 25^{\frac{3}{2}} = (25^{\frac{1}{2}})^3 = 5^3 = 125
\end{align*}

\begin{tcolorbox}[title=Problem 7, breakable]
    Express each of the following expressions as a simple decimal.

    (a) $(0.09)^\frac{1}{2}$

    (b) $(0.027)^\frac{1}{3}$

    (c) $(0.125)^\frac{2}{3}$

    (d) $(1.21)^\frac{1}{2}$
\end{tcolorbox}

\textbf{Solution:}
\begin{align*}
\text{(a)}\quad & (0.9)^{\frac{1}{2}} \approx 0.3 \\
\text{(b)}\quad & (0.027)^{\frac{1}{3}} = 0.3 \\
\text{(c)}\quad & (0.125)^{\frac{2}{3}} = ((0.125)^{\frac{1}{3}})^2 = 0.5^2 = 0.25 \\
\text{(d)}\quad & (1.21)^{\frac{1}{2}} = 1.1
\end{align*}

\begin{tcolorbox}[title=Problem 8, breakable]
    Express each of the following expressions as a quotient $\frac{m}{n}$, where $m$,
    $n$ are integers $> 0$.

    (a) $\left(\frac{8}{27}\right)^{\frac{2}{3}}$

    (b) $\left(\frac{4}{9}\right)^{\frac{1}{2}}$

    (c) $\left(\frac{25}{16}\right)^{\frac{3}{2}}$

    (d) $\left(\frac{49}{4}\right)^{\frac{3}{2}}$
\end{tcolorbox}

\textbf{Solution:}
\begin{align*}
\text{(a)}\quad & \left(\frac{8}{27}\right)^{\frac{2}{3}} = \frac{8^{2/3}}{27^{2/3}} = \frac{4}{9} \\
\text{(b)}\quad & \left(\frac{4}{9}\right)^{\frac{1}{2}} = \frac{2}{3} \\
\text{(c)}\quad & \left(\frac{25}{16}\right)^{\frac{3}{2}} = \frac{(25^{1/2})^3}{(16^{1/2})^3} = \frac{125}{64} \\
\text{(d)}\quad & \left(\frac{49}{4}\right)^{\frac{3}{2}} = \frac{(49^{1/2})^3}{(4^{1/2})^3} = \frac{343}{8}
\end{align*}

\begin{tcolorbox}[title=Problem 9, breakable]
    Solve each of the following equations for $x$.

    (a) $(x - 2)^{3} = 5$

    (b) $(x + 3)^{2} = 4$

    (c) $(x - 5)^{-2} = 9$

    (d) $(x + 3)^{3} = 27$

    (e) $(2x - 1)^{-3} = 27$

    (f) $(3x + 5)^{-4} = 64$
\end{tcolorbox}

\textbf{Solution:}
\begin{align*}
\text{(a)}\quad & x - 2 = \sqrt[3]{5} \iff x = 2 + \sqrt[3]{5} \\
\text{(b)}\quad & x + 3 = \pm 2 \iff x = -1 \text{ or } x = -5 \\
\text{(c)}\quad & \frac{1}{(x-5)^2} = 9 \iff (x-5)^2 = \frac{1}{9} \iff x = 5 \pm \frac{1}{3} \\
\text{(d)}\quad & x + 3 = 3 \iff x = 0 \\
\text{(e)}\quad & \frac{1}{(2x-1)^3} = 27 \iff (2x-1)^3 = \frac{1}{27} \iff 2x-1 = \frac{1}{3} \iff x = \frac{2}{3} \\
\text{(f)}\quad & \frac{1}{(3x+5)^4} = 64 \iff (3x+5)^4 = \frac{1}{64} \iff 3x+5 = \frac{1}{2} \iff x = -\frac{3}{2}
\end{align*}

\section{Inequalities}

\begin{tcolorbox}[title=Problem 1, breakable]
    Prove \textbf{IN 3}.
\end{tcolorbox}

\textbf{IN 3} If $a > b$ and $b > c$ then $a > c$.

\begin{proof}
    Suppose $a > b$ and $b > c$.
    Since $a > b$, $a - b > 0$.
    Also, since $b > c$, $b - c > 0$.
    So $(a - b) + (b - c) > 0 \iff a - c > 0$.
    Therefore $a > c$.
\end{proof}

\begin{tcolorbox}[title=Problem 2, breakable]
    Prove: If $0 < a < b$, if $c < d$, and $c > 0$ then
    \[ac < bd\]
\end{tcolorbox}

\begin{proof}
    Suppose $0 < a < b$, $c < d$, and $c > 0$.
    Since $a < b$ and $c > 0$ it follows that $ac < bc$ (\textbf{IN 2}).
    Since $c < d$ and  $b > 0$ it follows that $bc < bd$ (\textbf{IN 2}).
    Since $ac < bc < bd$ it follows that $ac < bd$ (Problem $1$).
\end{proof}

\begin{tcolorbox}[title=Problem 3, breakable]
    Prove: If $a < b < 0$, if $c < d < 0$ then 
    \[ac > bd\]
\end{tcolorbox}

\begin{proof}
    Suppose $a < b < 0$ and $c < d < 0$.
    Since $a < b$ it follows that $b - a > 0$.
    Since $b - a > 0$ and $c < 0$ it follows that $bc - ac < 0$ so $bc < ac$ (\textbf{IN 3}).
    Since $c < d$ it follows that $d - c > 0$.
    Since $d - c > 0$ and $b < 0$ it follows that $bd - bc < 0$ so $bd < bc$ (\textbf{IN 3}).
    So $bd < bc < ac$ and therefore $bd < ac$ (Problem $1$).
\end{proof}

\begin{tcolorbox}[title=Problem 4, breakable]
    (a) If $x < y$ and $x > 0$, prove that $\frac{1}{y} < \frac{1}{x}$.

    (b) Prove a rule of cross-multiplication of inequalities: If $a$, $b$, $c$, $d$
    are numbers and $b > 0$, $d > 0$, and if 
    \[\frac{a}{b} < \frac{c}{d}\]
    prove that 
    \[ad < bc\]
    Also prove the converse, that if $ad < bc$, then $\frac{a}{b} < \frac{c}{d}$.
\end{tcolorbox}

\begin{proof}
    Obviously we can assume $x \not = 0$ and $y \not = 0$.
    Suppose $x < y$ and $x > 0$.
    Since $x < y$ it follows that $y - x > 0$.
    Since $y > x > 0$ it follows that $\frac{1}{xy} > 0$.
    Then $\frac{1}{xy}(y - x) > 0 \iff \frac{1}{x} - \frac{1}{y} > 0$
        therefore $\frac{1}{x} > \frac{1}{y}$.
\end{proof}

\begin{proof}
    Suppose $a$, $b$, $c$, and $d$ are numbers such that $b > 0$ and $d > 0$.
    Suppose $\frac{a}{b} < \frac{c}{d}$.
    It follows that $\frac{c}{d} - \frac{a}{b} > 0$.
    Since $b > 0$ and $d > 0$ it follows that $bd > 0$.
    Then $bd(\frac{c}{d} - \frac{a}{b}) > 0 \iff cb - ad > 0 \iff ad < bc$.
\end{proof}

\begin{proof}
    Suppose $a$, $b$, $c$, and $d$ are numbers such that $b > 0$ and $d > 0$.
    Suppose $\frac{a}{b} > \frac{c}{d}$.
    So $\frac{a}{b} > \frac{c}{d} \iff \frac{c}{d} < \frac{a}{b}$.
    Since $\frac{c}{d} < \frac{a}{b}$ then $bc < ad$ (Previous Proof).
\end{proof}

\begin{tcolorbox}[title=Problem 5, breakable]
    Prove: If $a < b$ and $c$ is any real number, then 
    \[a + c < b + c\]
    Also,
    \[a - c < b - c\]
    Thus a number may be subtracted from each side of an inequality
    without changing the validity of the inequality.
\end{tcolorbox}

\begin{proof}
    Suppose $a < b$ and $c$ is a 
    real number.
    Since $a < b$ it follows that $b - a > 0$.
    Then $b - a > 0 
            \iff b - a + c -c > 0 
            \iff b + c - a - c > 0 
            \iff b + c - (a + c) > 0 \iff b + c > a + c$.
\end{proof}

\begin{proof}
    Suppose $a < b$ and $t$ is a real number.
    Apply previous proof with $-t$ in place of $c$.
    Therefore $a + (-t) < b + (-t) \iff a - t < b - t$
\end{proof}

\begin{tcolorbox}[title=Problem 6, breakable]
    Prove: If $a < b$ and $a > 0$ that 
    \[a^2 < b^2\]
    More generally, prove successively that
    \[a^3 < b^3\]
    \[a^4 < b^4\]
    \[a^5 < b^5\]
    Proceeding stepwise, we conclude that 
    \[a^n < b^n\]
    for every positive integer $n$. To make this stepwise argument formal,
    one must state explicitly a property of integers which is called induction,
    and is discussed later in the book.
\end{tcolorbox}

\begin{proof}
    Suppose $a < b$ and $a > 0$.
    It follows that $b - a > 0$.
    Since $b > 0$ and $b - a > 0$ it follows that $b^2 - ab > 0$ so $b^2 > ab$.
    Also, since $a > 0$ and $b - a > 0$ it follows that $ab - a^2 > 0$ so $ab > a^2$.
    Since $a^2 < ab < b^2$ it follows that $a^2 < b^2$
\end{proof}

\begin{proof}
    Suppose $a < b$ and $a > 0$. It follows that $a^2 < b^2$ (Previous Proof).
    Therefore $b^2 - a^2 > 0$.
    Since $b > 0$ and $b^2 - a^2 > 0$ it follows that $b^3 - a^2b > 0$.
    Also, since $a > 0$ and $b^2 - a^2 > 0$ it follows that $ab^2 - a^3 > 0$.
    Since $b^3 - a^2b > 0$ and $ab^2 - a^3 > 0$ it follows that 
        $(b^3 - a^2b) + (ab^2 - a^3) > 0 
        \iff (b^3 - a^3) + (ab^2 - a^2b) > 0
        \iff (b^3 - a^3) + (ab(b - a)) > 0$.
    Since $a, b > 0$ it follows that $ab > 0$.
    Since $ab > 0$ and $(b - a) > 0$ it follows that $ab(b - a) > 0$.
    Therefore $(b^3 - a^3) + (ab(b - a)) \ge b^3 - a^3 > 0$.
    It then follows that $b^3 > a^3$.
\end{proof}

\begin{proof}
    Suppose $a < b$ and $a > 0$. It follows that $a^2 < b^2$ (Previous Proof).
    Therefore $b^2 - a^2 > 0$.
    Since $b > 0$ and $b^2 - a^2 > 0$ it follows that $b^3 - a^2b > 0$.
    Also, since $a > 0$ and $b^2 - a^2 > 0$ it follows that $ab^2 - a^3 > 0$.
    Since $b^3 - a^2b > 0$ and $ab^2 - a^3 > 0$ it follows that 
        $(b^3 - a^2b) + (ab^2 - a^3) > 0 
        \iff (b^3 - a^3) + (ab^2 - a^2b) > 0
        \iff (b^3 - a^3) + (ab(b - a)) > 0$.
    Since $a, b > 0$ it follows that $ab > 0$.
    Since $ab > 0$ and $(b - a) > 0$ it follows that $ab(b - a) > 0$.
    Therefore $(b^3 - a^3) + (ab(b - a)) \ge b^3 - a^3 > 0$.
    It then follows that $b^3 > a^3$.
\end{proof}

\begin{proof}
    Suppose $a < b$ and $a > 0$. It follows that $a^3 < b^3$ (Previous Proof).  
    Therefore $b^3 - a^3 > 0$.  
    Since $b > 0$ and $b^3 - a^3 > 0$ it follows that $b^4 - a^3b > 0$.  
    Also, since $a > 0$ and $b^3 - a^3 > 0$ it follows that $ab^3 - a^4 > 0$.  
    Since $b^4 - a^2b^2 > 0$ and $ab^3 - a^4 > 0$ it follows that  
        $(b^4 - a^2b^2) + (ab^3 - a^4) > 0  
        \iff (b^4 - a^4) + (ab^3 - a^2b^2) > 0  
        \iff (b^4 - a^4) + (ab^2(b - a)) > 0$.  
    Since $a, b > 0$ it follows that $ab^2 > 0$.  
    Since $ab^2 > 0$ and $(b - a) > 0$ it follows that $ab^2(b - a) > 0$.  
    Therefore $(b^4 - a^4) + (ab^2(b - a)) \ge b^4 - a^4 > 0$.  
    It then follows that $b^4 > a^4$.
\end{proof}

\begin{proof}
    Suppose $a < b$ and $a > 0$. It follows that $a^4 < b^4$ (Previous Proof).  
    Therefore $b^4 - a^4 > 0$ (Previous Proof).  
    Since $b > 0$ and $b^4 - a^4 > 0$ it follows that $b^5 - a^4b > 0$.  
    Also, since $a > 0$ and $b^4 - a^4 > 0$ it follows that $ab^4 - a^5 > 0$.  
    Since $b^5 - a^3b^2 > 0$ and $ab^4 - a^5 > 0$ it follows that  
        $(b^5 - a^3b^2) + (ab^4 - a^5) > 0  
        \iff (b^5 - a^5) + (ab^4 - a^3b^2) > 0  
        \iff (b^5 - a^5) + (ab^2(b^2 - a^2)) > 0$.  
    Since $a, b > 0$ it follows that $ab^2 > 0$.  
    Since $ab^2 > 0$ and $(b^2 - a^2) > 0$ it follows that $ab^2(b^2 - a^2) > 0$.  
    Therefore $(b^5 - a^5) + (ab^2(b^2 - a^2)) \ge b^5 - a^5 > 0$.  
    It then follows that $b^5 > a^5$.
\end{proof}

\begin{tcolorbox}[title=Problem 7, breakable]
    Prove: If $0 < a < b$, then $a^{\frac{1}{n}} < b^{\frac{1}{n}}$. [Hint: Use Excersize $6$.]
\end{tcolorbox}

\begin{tcolorbox}[title=Problem 8, breakable]
    Let $a$, $b$, $c$, $d$ be numbers and assume $b > 0$ and $d > 0$. Assume that 
    \[\frac{a}{b} < \frac{c}{d}\]
    
    (a) Prove that 
    \[\frac{a}{b} < \frac{a + c}{b + d} < \frac{c}{d}\]
    (There are two inequalities to be proved here, the one on the left 
    and the one on the right.)

    (b) Let $r$ be a number $> 0$. Prove that \
    \[\frac{a}{b} < \frac{a + rc}{b + rd} < \frac{c}{d}\]
    
    (c) If $0 < r < s$, prove that 
    \[\frac{a + rc}{b + rd} = \frac{a + sc}{b + sd}\]
\end{tcolorbox}

\end{document}