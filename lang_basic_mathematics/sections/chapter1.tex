\subsection{The Integers}
\subsection{Rules For Addition}

\textbf{Leadup Instructions}

Justify each step, using commutativity and associativity in proving the
following identities.

\begin{tcolorbox}[title=Problem 1, breakable]
$(a + b) + (c + d) = (a + d) + (b + c)$
\end{tcolorbox}

\textbf{Solution:}
\begin{align*}
(a + b) + (c + d) & \quad \text{} \\
=((a + b) + c) + d & \quad \text{associative} \\
=(a + (b + c)) + d & \quad \text{associative} \\
=d + (a + (b + c)) & \quad \text{commutative} \\
=(d + a) + (b + c) & \quad \text{associative} \\
=(a + d) + (b + c) & \quad \text{commutative}
\end{align*}


\begin{tcolorbox}[title=Problem 2, breakable]
$(a + b) + (c + d) = (a + c) + (b + d)$
\end{tcolorbox}

\textbf{Solution:}
\begin{align*}
(a + b) + (c + d) = ((a + b) + c) + d & \quad \text{associative} \\
=(c + (a + b)) + d & \quad \text{associative} \\
=((c + a) + b) + d & \quad \text{commutative} \\
=(c + a) + (b + d) & \quad \text{associative} \\
=(a + c) + (b + d) & \quad \text{commutative} \\
\end{align*}

\begin{tcolorbox}[title=Problem 3, breakable]
$(a - b) + (c - d) = (a + c) + (-b - d)$
\end{tcolorbox}

\textbf{Solution:}
\begin{align*}
(a - b) + (c - d) = ((a - b) + c) - d & \quad \text{associative} \\
=(c + (a - b)) - d & \quad \text{commutative} \\
=((c + a) - b) - d & \quad \text{associative} \\
=((a + c) - b) - d & \quad \text{commutative} \\
=((a + c) + (-b)) + (-d) & \quad \text{} \\
=(a + c) + (-b - d) & \quad \text{associative} \\
\end{align*}

\begin{tcolorbox}[title=Problem 4, breakable]
$(a - b) + (c - d) = (a + c) - (b + d)$
\end{tcolorbox}

\textbf{Solution:}
\begin{align*}
(a - b) + (c - d) = ((a - b) + c) - d  & \quad \text{associative} \\
=(c + (a - b)) - d  & \quad \text{commutative} \\
=((c + a) + (-b)) + (-d) & \quad \text{} \\
=(c + a) + ((-b) + (-d))  & \quad \text{associative} \\
=(c + a) - (b + d) & \quad \text{} \\
=(a + c) - (b + d) & \quad \text{commutative} \\
\end{align*}

\begin{tcolorbox}[title=Problem 5, breakable]
$(a - b) + (c - d) = (a - d) + (c - b)$
\end{tcolorbox}

\textbf{Solution:}
\begin{align*}
(a - b) + (c - d) = ((a - b) + c) - d & \quad \text{associative} \\
=((a + (-b)) + c) - d  & \quad \text{} \\
=(a + ((-b) + c)) - d & \quad \text{associative} \\
=(((-b) + c) + a) - d & \quad \text{commutative} \\
=((-b) + c) + (a - d) & \quad \text{associative} \\
=(c + (-b)) + (a - d) & \quad \text{commutative} \\
=(c - b) + (a - d) & \quad \text{} \\
=(a - d) + (c - b) & \quad \text{commutative} \\
\end{align*}

\begin{tcolorbox}[title=Problem 6, breakable]
$(a - b) + (c - d) = -(b + d) + (a + c)$
\end{tcolorbox}

\textbf{Solution:}
\begin{align*}
(a - b) + (c - d) = ((a - b) + c) - d \quad \text{associative} \\
=((a + (-b)) + c) + (-d)  & \quad \text{} \\
=(c + (a + (-b))) + (-d) \quad \text{commutative} \\
=((c + a) + (-b)) + (-d) \quad \text{associative} \\
=((a + c) + (-b)) + (-d) \quad \text{commutative} \\
=(a + c) + ((-b) + (-d)) \quad \text{associative} \\
=((-b) + (-d)) + (a + c) \quad \text{commutative} \\
=(-b - d) + (a + c)  & \quad \text{} \\
=-(b + d) + (a + c)  & \quad \text{distributive property} \\
\end{align*}


\begin{tcolorbox}[title=Problem 7, breakable]
$(a - b) + (c - d) = -(b + d) - (-a - c)$
\end{tcolorbox}

\textbf{Solution:}
\begin{align*}
(a - b) + (c - d) = (a + (-b)) + (c + (-d)) & \quad \text{} \\
=((a + (-b)) + c) + (-d) & \quad \text{associative} \\
=(c + (a + (-b))) + (-d) & \quad \text{commutative} \\
=((c + a) + (-b)) + (-d) & \quad \text{associative} \\
=(c + a) + ((-b) + (-d)) & \quad \text{associative} \\
=((-b) + (-d)) + (c + a) & \quad \text{commutative} \\
=-(b + d) + (c + a) & \quad \text{} \\
=-(b + d) + (-(-c) + -(-a)) & \quad \text{} \\
=-(b + d) - (-c - a) & \quad \text{} \\
=-(b + d) - (-a - c) & \quad \text{commutative} \\
\end{align*}

\begin{tcolorbox}[title=Problem 8, breakable]
$((x + y) + z) + w = (x + z) + (y + w)$
\end{tcolorbox}

\textbf{Solution:}
\begin{align*}
((x + y) + z) + w  = (z + (x + y)) + w & \quad \text{commutative} \\
=((z + x) + y) + w & \quad \text{associative} \\
=(z + x) + (y + w) & \quad \text{associative} \\
=(x + z) + (y + w) & \quad \text{commutative} \\
\end{align*}


\begin{tcolorbox}[title=Problem 9, breakable]
$(x - y) - (z - w) = (x + w) - y - z$
\end{tcolorbox}

\textbf{Solution:}
\begin{align*}
(x - y) - (z - w)  = (x + (-y)) + ((-z) + w)  & \quad \text{} \\
=((x + (-y)) + (-z)) + w & \quad \text{associative} \\
=(x + ((-y) + (-z))) + w & \quad \text{associative} \\
=(((-y) + (-z)) + x) + w & \quad \text{commutative} \\
=((-y) + (-z)) + (x + w) & \quad \text{associative} \\
=(x + w) + ((-y) + (-z)) & \quad \text{commutative} \\
=(x + w) - y - z & \quad \text{} \\
\end{align*}

\begin{tcolorbox}[title=Problem 10, breakable]
$(x - y) - (z - w) = (x - z) + (w - y)$
\end{tcolorbox}

\textbf{Solution:}
\begin{align*}
(x - y) - (z - w)  = (x + (-y)) + ((-z) + w) & \quad \text{distributive} \\
=((x + (-y)) + (-z)) + w & \quad \text{associative} \\
=(x + ((-y) + (-z))) + w  & \quad \text{commutative} \\
=(((-y) + (-z)) + x) + w & \quad \text{associative} \\
=((-y) + ((-z) + x)) + w & \quad \text{associative} \\
=w + ((-y) + ((-z) + x)) & \quad \text{commutative} \\
=(w + (-y)) + ((-z) + x) & \quad \text{associative} \\
=(w + (-y)) + (x + (-z)) & \quad \text{commutative} \\
=(w - y) + (x - z) & \quad \text{} \\
\end{align*}


\begin{tcolorbox}[title=Problem 11, breakable]
Show that $-(a + b + c) = -a + (-b) + (-c)$.
\end{tcolorbox}

\textbf{Solution:}
\begin{align*}
-(a + b + c) &= -(a + (b + c)) & \quad \text{} \\
&= (-a + -(b + c)) & \quad \text{distributive} \\
&= (-a + (-b + (-c))) & \quad \text{distributive} \\
&= -a + (-b) + (-c) & \quad \text{} \\
\end{align*}

\begin{tcolorbox}[title=Problem 12, breakable]
Show that $-(a - b - c) = -a + b + c$.
\end{tcolorbox}

\textbf{Solution:}
\begin{align*}
-(a - b - c) &= -(a + (-b) + (-c)) & \quad \text{} \\
&= (-a - (-b) - (-c)) & \quad \text{distributive} \\
&= (-a + b + c) & \quad \text{double negation} \\
&= -a + b + c & \quad \text{} \\
\end{align*}

\begin{tcolorbox}[title=Problem 13, breakable]
Show that $-(a - b) = b - a$.
\end{tcolorbox}

\textbf{Solution:}
\begin{align*}
-(a - b) &= (-a) - (-b)  & \quad \text{distributive} \\
&= -a + b & \quad \text{double negation} \\
&= b + (-a) & \quad \text{commutative} \\
&= b - a & \quad \text{} \\
\end{align*}


Solve for x in the following equations.
\begin{tcolorbox}[title=Problem 14, breakable]
$-2 + x = 4$
\end{tcolorbox}

\textbf{Solution:}
\begin{align*}
-2 + x = 4  & \\
=-2 + 2 + x = 4 + 2 & \\
=x = 6 & \\
\end{align*}

\begin{tcolorbox}[title=Problem 19, breakable]
$-5 - x = -2$
\end{tcolorbox}

\textbf{Solution:}
\begin{align*}
-5 - x &= -2 \Leftrightarrow & \quad \text{} \\ 
(-5 - x) + x &= -2 + x \Leftrightarrow  & \quad \text{}  \\
-5 + ((-x) + x) &= -2 + x \Leftrightarrow  & \quad \text{associative}  \\
-5 + 0 &= -2 + x \Leftrightarrow  & \quad \text{N2}  \\
-5 &= -2 + x \Leftrightarrow & \quad \text{N1} \\
-5 + 2 &= (-2 + x) + 2 \Leftrightarrow  & \quad \text{}  \\ 
-5 + 2 &= 2 + (-2 + x) \Leftrightarrow  & \quad \text{commutative}  \\ 
-5 + 2 &= (2 + (-2)) + x \Leftrightarrow  & \quad \text{associative}  \\ 
-5 + 2 &= 0 + x \Leftrightarrow  & \quad \text{N2}  \\ 
-3 &= x  & \quad \text{N1}  \\
\end{align*}


\begin{tcolorbox}[title=Problem 20, breakable]
$-7 + x = -10$
\end{tcolorbox}

\textbf{Solution:}
\begin{align*}
-7 + x &= -10 \Leftrightarrow & \quad \text{} \\
(-7 + x) + 7 &= -10 + 7 \Leftrightarrow & \quad \text{} \\ 
7 + (-7 + x) &= -3 \Leftrightarrow & \quad \text{commutative}  \\ 
(7 + (-7)) + x &= -3 \Leftrightarrow & \quad \text{associative}  \\ 
0 + x &= -3 \Leftrightarrow & \quad \text{N2}  \\
x &= -3 & \quad \text{}  \\
\end{align*}


\begin{tcolorbox}[title=Problem 21, breakable]
$-3 + x = 4$
\end{tcolorbox}

\textbf{Solution:}
\begin{align*}
-3 + x &= 4 \Leftrightarrow & \quad \text{} \\
(-3 + x) + 3 &= 4 + 3 \Leftrightarrow & \quad \text{} \\
3 + (-3 + x) &= 7 \Leftrightarrow & \quad \text{commutative}  \\ 
(3 + (-3)) + x &= 7 \Leftrightarrow & \quad \text{associative}  \\ 
0 + x &= 7  \Leftrightarrow & \quad \text{N2}  \\
x &= 7 & \quad \text{}  \\
\end{align*}

\begin{tcolorbox}[title=22 Prove the cancellation law for addition, breakable]
If $a + b = a + c$ then $b = c$.
\end{tcolorbox}

\textbf{Solution:}
\begin{align*}
a + b &= a + c \Leftrightarrow & \quad \text{} \\
(a + b) + (-a) &= (a + c) + (-a) \Leftrightarrow & \quad \text{} \\
-a + (a + b) &= -a + (a + c) \Leftrightarrow & \quad \text{commutative}  \\ 
(-a + a) + b &= (-a + a) + c \Leftrightarrow & \quad \text{associative}  \\ 
0 + b &= 0 + c \Leftrightarrow & \quad \text{N2}  \\
b &= c & \quad \text{}  \\
\end{align*}


\begin{tcolorbox}[title=23 Prove, breakable]
If $a + b = a$, then $b = 0$.
\end{tcolorbox}

\textbf{Solution:}
\begin{align*}
a + b &= a \Leftrightarrow & \quad \text{} \\
(a + b) + (-a) &= a + (-a) \Leftrightarrow & \quad \text{} \\
(-a) + (a + b) &= a - a & \quad \text{commutative} \\ 
(-a) + (a + b) &= 0 & \quad \text{N2}  \\
((-a) + a) + b &= 0 \Leftrightarrow & \quad \text{associative}  \\ 
0 + b &= 0  & \quad \text{N2}  \\
b &= 0 & \\
\end{align*}

\subsection{Rules For Multiplication}

Express each of the following expressions in the form $2^m 3^n a^r b^s$, where $m$,
$n$, $r$, $s$ are positive integers.
\begin{tcolorbox}[title=Problem 1, breakable]
(a) $8 a^2 b^3 (27 a^4) (2^5 a b)$ \\
(b) $16 b^3 a^2 (6 a b^4) {(a b)}^3$ \\
(c) $3^2 {(2 a b)}^3 (16 a^2 b^5) (24 b^2 a)$ \\
(d) $24 a^3 {(2 a b^2)}^3 {(3 a b)}^2$ \\
(e) ${(3 a b)}^2 (27 a^3 b) (16 a b^5)$ \\
(f) $32 a^4 b^5 a^3 b^2 {(6 a b^3)}^4$ \\
\end{tcolorbox}

\textbf{Solution: 1 (a)}
\begin{align*}
8 a^2 b^3 (27 a^4) (2^5 a b) &= 8 (27 a^4) a^2 b^3 (2^5 a b) & \quad \text{commutative} \\
&= (8 \cdot 27 ) a^4 a^2 b^3 (2^5 a b) & \quad \text{associative} \\
&= (8 \cdot 27 ) (2^5 a b) a^4 a^2 b^3 & \quad \text{commutative} \\
&= (8 \cdot 27 \cdot 2^5) a b a^4 a^2 b^3 & \quad \text{associative} \\
&= (8 \cdot 27 \cdot 2^5) a a^4 a^2 b b^3 & \quad \text{commutative} \\
&= (2^3 3^3 2^5) a^7 b^4 & \quad \text{N11} \\
&= (2^3 2^5 3^3) a^7 b^4 & \quad \text{commutative} \\
&= (2^8 3^3) a^7 b^4 & \quad \text{N11} \\
&= 2^8 3^3  a^7 b^4 & \\
\end{align*}
\textbf{Solution: 1 (b)}
\begin{align*}
16 b^3 a^2 (6 a b^4) {(a b)}^3 &= b^3 a^2 {(a b)}^3 (6 a b^4) 16 & \quad \text{commutative} \\
&= b^3 a^2 {(a b)}^3  6 (a b^4) 16 & \quad \text{associative} \\
&= b^3 a^2 {(a b)}^3 (a b^4) 16 \cdot 6 & \quad \text{commutative} \\
&= b^3 a^2 a^3 b^3 a b^4 16 \cdot 6 & \quad \text{N12} \\
&= a^2 a^3 a b^3 b^3 b^4 16 \cdot 6 & \quad \text{commutative} \\
&= a^2 a^3 a b^3 b^3 b^4 2^4 2 \cdot 3 & \quad \text{} \\
&= a^6 b^{10} 2^5 3  & \quad \text{N11} \\
&= 2^5 3 a^6 b^{10} & \quad \text{commutative} \\
\end{align*}
\textbf{Solution: 1 (c)}
\begin{align*}
3^2 {(2 a b)}^3 (16 a^2 b^5) (24 b^2 a) &= 3^2 2^3 a^3 b^3 (16 a^2 b^5) (24 b^2 a) & \quad \text{N12} \\
&= 2^3 24 \cdot 3^2 16 a^3 a^2 a b^3 b^5 b^2 & \quad \text{commutative} \\
&= 2^3 3 \cdot 2^3 3^2 2^4 a^3 a^2 a b^3 b^5 b^2 & \quad \text{} \\
&= 2^3 2^3 2^4 3^2 3 a^3 a^2 a b^3 b^5 b^2 & \quad \text{associative} \\
&= 2^{10} 3^3 a^6 b^{10} & \quad \text{N11} \\
\end{align*}
\textbf{Solution: 1 (d)}
\begin{align*}
24 a^3 {(2 a b^2)}^3 {(3 a b)}^2 &= 24 a^3 2^3 a^3 {(b^2)}^3 3^2 a^2 b^2 & \quad \text{N12} \\
&= 24 a^3 2^3 a^3 b^6 3^2 a^2 b^2 & \quad \text{N12} \\
&= 24 \cdot 2^3 3^2 a^3 a^3 a^2 b^6 b^2 & \quad \text{commutative} \\
&= 2^3 3 \cdot 2^3 3^2 a^3 a^3 a^2 b^6 b^2 & \quad \text{} \\
&= 2^3 2^3 3^2 3 a^3 a^3 a^2 b^6 b^2 & \quad \text{commutative} \\
&= 2^6 3^3 a^8 b^8 & \quad \text{N11} \\
\end{align*}
\textbf{Solution: 1 (e)}
\begin{align*}
{(3 a b)}^2 (27 a^3 b) (16 a b^5) &= 3^2 a^2 b^2 27 a^3 b 16 a b^5 & \quad \text{N12} \\
&= 27 \cdot 16 \cdot 3^2 a^2 a^3 a b^2 b b^5 & \quad \text{commutative} \\
&= 3^3 2^4 3^2 a^2 a^3 a b^2 b b^5 & \quad \text{} \\
&=  2^4 3^3 3^2 a^2 a^3 a b^2 b b^5 & \quad \text{commutative} \\
&=  2^4 3^5 a^6 b^8 & \quad \text{N11} \\
\end{align*}
\textbf{Solution: 1 (f)}
\begin{align*}
32 a^4 b^5 a^3 b^2 {(6 a b^3)}^4 &= 32 a^4 b^5 a^3 b^2 6^4 a^4 {(b^3)}^4 & \quad \text{N12} \\
&= 32 a^4 b^5 a^3 b^2 6^4 a^4 b^{12} & \quad \text{N12} \\
&= 6^4 32 a^3 a^4 a^4 b^5 b^2 b^{12} & \quad \text{commutative} \\
&= {(2 \cdot 3)}^4 2^5 a^3 a^4 a^4 b^5 b^2 b^{12} & \quad \text{} \\
&= 2^4 3^4 2^5 a^3 a^4 a^4 b^5 b^2 b^{12} & \quad \text{N12} \\
&= 2^4 2^5 3^4 a^3 a^4 a^4 b^5 b^2 b^{12} & \quad \text{commutative} \\
&= 2^9 3^4 a^{11} b^{19} & \quad \text{N11} \\
\end{align*}


\begin{tcolorbox}[title=Problem 2, breakable]
Prove \\
(a) ${(a + b)}^3 = a^3 + 3 a^2 b + 3 a b^2 + b^3$ \\
(b) ${(a - b)}^3 = a^3 - 3 a^2 b + 3 a b^2 - b^3$
\end{tcolorbox}

\text{Solution: 2 (a)}
\begin{align*}
{(a + b)}^3 &= (a + b)(a + b)(a + b) & \\
&= ((a + b)(a + b))(a + b) & \quad \text{associative} \\
&= (a(a + b) + b(a + b))(a + b) & \quad \text{distributive} \\
&= (a^2 + ab + ba + b^2)(a + b) & \quad \text{distributive} \\
&= (a^2 + 2ab + b^2)(a + b) & \quad \text{} \\
&= a^2(a + b) + 2ab(a + b) + b^2(a + b) & \quad \text{distributive} \\
&= a^2 a + a^2 b + 2ab a + 2ab b + b^2 a + b^2 b & \quad \text{distributive} \\
&= a^3 + a^2 b + 2 a^2 b + 2ab^2 + b^2 a + b^3 & \quad \text{N11} \\
&= a^3 + 3 a^2 b + 3ab^2 + b^3 & \quad \text{} \\
\end{align*}

\text{Solution: 2 (b)}
\begin{align*}
{(a - b)}^3 &= (a - b)(a - b)(a - b) & \\
&= ((a - b)(a - b))(a - b) & \quad \text{associative} \\
&= (a(a - b) - b(a - b))(a - b) & \quad \text{distributive} \\
&= (a^2 - ab - ba + b^2)(a - b) & \quad \text{distributive} \\
&= (a^2 - 2ab + b^2)(a - b) & \quad \text{} \\
&= a^2(a - b) - 2ab(a - b) + b^2(a - b) & \quad \text{distributive} \\
&= a^2 a - a^2 b - 2ab a + 2ab b + b^2 a - b^2 b & \quad \text{distributive} \\
&= a^3 - a^2 b - 2 a^2 b + 2ab^2 + b^2 a - b^3 & \quad \text{N11} \\
&= a^3 - 3 a^2 b + 3ab^2 - b^3 & \quad \text{} \\
\end{align*}


\begin{tcolorbox}[title=Problem 3, breakable]
Obtain expansions for ${(a+b)}^4$ and ${(a-b)}^4$.
\end{tcolorbox}

\text{Solution: 3}

From 2: ${(a + b)}^3 = a^3 + 3 a^2 b + 3ab^2 + b^3$.
\begin{align*}
{(a + b)}^3 * (a + b) &= {(a + b)}^4 && \quad \text{N11} \\
&= ((a^3 + 3 a^2 b) + (3ab^2 + b^3)) * (a + b) && \quad \text{} \\
&= (a^3 + 3 a^2 b)(a + b) + (3ab^2 + b^3)(a + b) && \quad \text{distributive} \\
&= (a^3(a + b) + 3 a^2 b(a + b)) + (3ab^2(a + b) + b^3(a + b)) && \quad \text{distributive} \\
&= (a a^3 + b a^3) + (a 3 a^2 b + b 3 a^2 b) + (a 3ab^2 + b 3ab^2) + (a b^3 + b b^3) && \quad \text{distributive} \\
&= (a^4 + a^3 b) + (3 a^3 b + 3 a^2 b^2) + (3 a^2 b^2 + 3 a b^3) + (a b^3 + b^4) && \quad \text{N11} \\
&= a^4 + 4 a^3 b + 6 a^2 b^2 + 4 a b^3 + b^4 && \quad \text{} \\
\end{align*}

From prev: ${(a + b)}^4 = a^4 + 4 a^3 b + 6 a^2 b^2 + 4 a b^3 + b^4$.
\begin{align*}
{(a-b)}^4 &= {(a + (-b))}^4 \\
&= a^4 + 4 a^3 (-b) + 6 a^2 {(-b)}^2 + 4 a {(-b)}^3 + {(-b)}^4 \\
&= a^4 - 4 a^3 b + 6 a^2 b^2 - 4 a b^3 + b^4 \\
\end{align*}

\begin{tcolorbox}[title=Problem 5, breakable]
${(1 - 2 x)}^2$
\end{tcolorbox}

\textbf{Solution}
\begin{align*}
{(1 - 2 x)}^2 &= (1 - 2 x) * (1 - 2 x) && \\
&= (1(1 - 2 x) - 2 x(1 - 2 x)) && \quad \text{distributive} \\
&= ((1 - 2 x) - (2x - 2 x 2 x)) && \quad \text{distributive} \\
&= (1 - 2 x) - (2x - 4 x^2) && \quad \text{N11} \\
&= ((1 - 2 x) - 2x) - 4 x^2 && \quad \text{associative} \\
&= (1 + ((- 2 x) - 2x)) - 4 x^2 && \quad \text{associative} \\
&= (1 + (-4x)) - 4 x^2 && \quad \text{} \\
&= 1 - 4x - 4 x^2 && \quad \text{} \\
\end{align*}

\begin{tcolorbox}[title=Problem 7, breakable]
${(x - 1)}^2$
\end{tcolorbox}

\textbf{Solution}
\begin{align*}
{(x - 1)}^2 &= (x - 1) \cdot (x - 1) \\
&= x^2 - 2x + 1 \quad \text{perfect square} \\
\end{align*}

\begin{tcolorbox}[title=Problem 11, breakable]
$(1 + x^3)(1 - x^3)$
\end{tcolorbox}

\textbf{Solution}
\begin{align*}
(1 + x^3)(1 - x^3) &= (1 - x^6) \quad \text{difference of squares} \\
\end{align*}

\begin{tcolorbox}[title=Problem 13, breakable]
${(x^2 - 1)}^2$
\end{tcolorbox}

\textbf{Solution}
\begin{align*}
{(x^2 - 1)}^2 = x^4 - 2x^2 + 1 \quad \text{perfect square} \\
\end{align*}

\begin{tcolorbox}[title=Problem 17, breakable]
$(x^3 - 4)(x^3 + 4)$
\end{tcolorbox}

\textbf{Solution}
\begin{align*}
(x^3 - 4)(x^3 + 4) = x^6 - 16 \quad \text{difference of squares} \\
\end{align*}

\begin{tcolorbox}[title=Problem 19, breakable]
$(-2 + 3x)(-2 - 3x)$
\end{tcolorbox}

\textbf{Solution}
\begin{align*}
(-2 + 3x)(-2 - 3x) = 4 - 9x^2 \quad \text{difference of squares} \\
\end{align*}


\begin{tcolorbox}[title=Problem 23, breakable]
$(-1 - x)(-2 + x)(1 - 2x)$
\end{tcolorbox}

\textbf{Solution}
\begin{align*}
(-1 - x)(-2 + x)(1 - 2x) &= (2 + x - x^2)(1 - 2x) && \quad \text{distributive} \\
&= (2(1 - 2x) + x(1 - 2x) - x^2(1 - 2x)) && \quad \text{distributive} \\
&= 2 - 4x + x - 2x^2 - x^2 + 2x^3 && \quad \text{distributive} \\
&= 2 - 3x - 3x^2 + 2x^3 && \quad \text{} \\
\end{align*}

\begin{tcolorbox}[title=Problem 29, breakable]
${(2x + 1)}^2(2 - 3x)$
\end{tcolorbox}

\textbf{Solution}
\begin{align*}
{(2x + 1)}^2(2 - 3x) &= (4x^2 +4x + 1)(2 - 3x) && \quad \text{perfect square} \\
&= (4x^2(2 - 3x) + 4x(2 - 3x) + 1(2 - 3x)) && \quad \text{distributive} \\
&= (8x^2 -12x^3 + 8x - 12x^2 + 2 - 3x) && \quad \text{distributive} \\
&= (-12x^3 - 4x^2 + 5x + 2) && \quad \text{} \\
\end{align*}

\begin{tcolorbox}[title=Problem 30, breakable]
The population of a city in $1910$ was $50,000$, and it doubles every $10$
years. What will it be (a) in $1970$ (b) in $1990$ (c) in $2,000$?
\end{tcolorbox}

\textbf{Solution}
\begin{align*}
\text{(a) } 50000 * 2^{((1970-1910) / 10)} &= 3200000 \\
\text{(b) } 50000 * 2^{((1990-1910) / 10)} &= 12800000 \\
\text{(c) } 50000 * 2^{((2000-1910) / 10)} &= 25600000 \\
\end{align*}


\begin{tcolorbox}[title=Problem 31, breakable]
The population of a city in $1905$ was $100,000$, and it doubles every $25$
years. What will it be after (a) $50$ years (b) $100$ years (c) $150$ years?
\end{tcolorbox}

\textbf{Solution}
\begin{align*}
\text{(a) } 100000 * 2^{(50 / 25)} &= 400000 \\
\text{(b) } 100000 * 2^{(100 / 25)} &= 1600000 \\
\text{(c) } 100000 * 2^{(150 / 25)} &= 6400000 \\
\end{align*}

\begin{tcolorbox}[title=Problem 32, breakable]
The population of a city was $200$ thousand in $1915$, and it triples every
$50$ years. What will be the population What will be the population \\
(a) in the year $2215$? \\
(b) in the year $2165$? \\
\end{tcolorbox}

\textbf{Solution}
\begin{align*}
\text{(a) } 200000 * 3^{((2215 - 1915) / 50)} &= 145800000 \\
\text{(b) } 200000 * 3^{((2165 - 1915) / 50)} &= 48600000 \\
\end{align*}

\begin{tcolorbox}[title=Problem 33, breakable]
The population of a city was $25,000$ in $1870$, and it triples every $40$ years.
What will it be. \\
(a) in $1990$? \\
(b) in $2030$? \\
\end{tcolorbox}

\textbf{Solution}
\begin{align*}
\text{(a) } 25000 * 3^{((1990 - 1870) / 40)} &= 675000 \\
\text{(b) } 25000 * 3^{((2030 - 1870) / 40)} &= 2025000 \\
\end{align*}

\subsection{Even and Odd Integers; Divisibility}

\begin{tcolorbox}[title=Problem 1, breakable]
Give the proofs for the cases of theorem $1$ which were not proved in the text. \\
(a) If $a$ is even and $b$ is even, then $a + b$ is even. \\
(b) If $a$ is odd and $b$ is even, then $a + b$ is odd. \\
(c) If $a$ is odd and $b$ is odd, then $a + b$ is even. 
\end{tcolorbox}

\textbf{Solution (a)}

Since $a$ and $b$ are even they can be written as $2n_1$ and $2n_2$ respectively,
where $n_1$ and $n_2$ are integers. \\
Let $x$ = $n_1 + n_2$. Note $x$ is an integer because the sum of two integers is an integer. \\
\begin{align*}
a + b &= 2n_1 + 2n_2 & \\
&= 2(n_1 + n_2) & \\
&= 2x 
\end{align*}
Since $a + b$ can be written as $2x$ where $x$ is an integer; $a + b$ is even.

\textbf{Solution (b)}
\begin{align*}
a + b &= 2n_1 + 1 + 2n_2 & \\
&= 2n_1 + 2n_2 + 1 & \\
&= 2(n_1 + n_2) + 1 & \text{let } x = n_1 + n_2 & \\
&= 2x + 1
\end{align*}

\textbf{Solution (c)}
\begin{align*}
a + b &= 2n_1 + 1 + 2n_2 + 1 & \\
&= 2n_1 + 2n_2 + 2 & \\
&= 2(n_1 + n_2 + 1) & \text{let } x = n_1 + n_2 + 1 & \\
&= 2x
\end{align*}


\begin{tcolorbox}[title=Problem 2, breakable]
If $a$ is even and $b$ is any positive integer then $ab$ is even.
\end{tcolorbox}

\begin{proof}
By def. of an even number $a$ can be written as $2n$ where $n$ is an integer. \\
Let $x = n \cdot b$. Note the product of two integers is an integer ig. \\ 
Something about multiplication being repeated addition and the sum of two integers being an integer.
\begin{align*}
    a \cdot b &= 2n \cdot b & \\
    &= 2x & 
\end{align*}
Since $ab$ can be written as $2x$ where x is an integer $ab$ is even.
\end{proof}

\begin{tcolorbox}[title=Problem 3, breakable]
If $a$ is even, then $a^3$ is even.
\end{tcolorbox}

\begin{proof}
By def. of an even number $a$ can be written as $2n$ where $n$ is an integer. \\
Let $x = 2^2 n^3$. Note $x$ is an integer. 
\begin{align*}
a^3 &= {(2n)}^3 & \quad \text{} && \\
&= 2^3 n^3 & \quad \text{N12} && \\
&= 2 \cdot 2^2 n^3 & \quad \text{N11} && \\
&= 2 x & \quad \text{}
\end{align*}
Since $a^3$ can be written as $2x$ where x is an integer $a^3$ is even.
\end{proof}

\begin{tcolorbox}[title=Problem 4, breakable]
If $a$ is odd, then $a^3$ is odd.
\end{tcolorbox}

\begin{proof}
By def. of an odd number $a$ can be written as $2n + 1$ where $n$ is an integer. \\
Let $x = 4 n^3 + 6 n^2 + 3n$. Note $x$ is an integer. 
\begin{align*}
a^3 &= {(2n + 1)}^3 & \quad \text{} && \\
&= 8 n^3 + 12 n^2 + 6n + 1 & \quad \text{distributive} && \\
&= 2 (4 n^3 + 6 n^2 + 3n) + 1 & \quad \text{distributive} && \\
&= 2 x + 1 & \quad \text{}
\end{align*}
Since $a^3$ can be written as $2x + 1$ where $x$ is an integer $a^3$ is odd.
\end{proof}

\begin{tcolorbox}[title=Problem 5, breakable]
If $n$ is even, then ${(-1)}^n = 1$.
\end{tcolorbox}

\begin{proof}
By def. of an even number $n$ can be written as $2a$ where $a$ is an integer.
\begin{align*}
{(-1)}^n &= {(-1)}^{2a} && \\
&= {((-1)^2)}^{a} && \quad \text{N12} \\
&= {1}^{a} && \quad \text{} \\
&= 1
\end{align*}
\end{proof}

\begin{tcolorbox}[title=Problem 6, breakable]
If $n$ is odd, then ${(-1)}^n = -1$.
\end{tcolorbox}

\begin{proof}
By def. of an odd number $n$ can be written as $2a + 1$ where $a$ is an integer.
\begin{align*}
{(-1)}^n &= {(-1)}^{2a + 1} && \\
&= {(-1)}^{2a} \cdot {(-1)}^1 && \quad \text{N11} \\
&= 1 \cdot (-1) && \quad \text{2a is even so by prob. $5$} \\
&= -1 && \quad \text{N7}
\end{align*}
\end{proof}

\begin{tcolorbox}[title=Problem 7, breakable]
If $m$, $n$ are odd, then the product $mn$ is odd.
\end{tcolorbox}

\begin{proof}
By def. of an odd number $m$ and $n$ can be written as $2n_1 + 1$ and $2n_2 + 1$ 
where $n_1$ and $n_2$ are integers. \\
Let $x = 2n_1 n_2 + n_1 + n_2$. Note $x$ is an integer.
\begin{align*}
mn &= (2n_1 + 1)(2n_2 + 1) && \\
&= 4 n_1 n_2 + 2 n_1 + 2 n_2 + 1 && \quad \text{distributive} \\
&= 2 (2n_1 n_2 + n_1 + n_2) + 1 && \quad \text{distributive} \\
&= 2 x + 1 && \quad \text{} \\
\end{align*}
Since $mn$ can be written as $2x + 1$ where x is an integer, therefore $mn$ is odd.
\end{proof}

\begin{tcolorbox}[title=Problem 24, breakable]
Let $a$, $b$ be integers, Define $a \equiv b\ (mod\ 5)$, which we read "$a$ is congruent
to $b$ modulo $5$, to mean that $a - b$ is divisible by $5$. \\
Prove if $a \equiv b\ (mod\ 5)$ and $x \equiv y\ (mod\ 5)$ then  $a + x \equiv b + y\ (mod\ 5)$
and $ax = by\ (mod\ 5)$.
\end{tcolorbox}

\begin{proof} 
Need to show $(a + x) - (b + y) = 5n$ where n is an integer. \\
From $a \equiv b\ (mod\ 5)$, $a - b = 5n_1$ where $n_1$ is an integer. \\
From $x \equiv y\ (mod\ 5)$, $x - y = 5n_2$ where $n_2$ is an integer. \\
Let $t = n_1 + n_2$.
\begin{align*}
(a + x) - (b + y) &= (a - b) + (x - y) \\
&= 5n_1 + 5n_2 \\
&= 5(n_1 + n_2) \\
&= 5t
\end{align*}
Since $(a + x) - (b + y) = 5t$ where $t$ is an integer, $a + x \equiv b + y\ (mod\ 5)$.
\end{proof}

\begin{proof} 
Need to show $ax - by = 5n$ where n is an integer. \\
From $a \equiv b\ (mod\ 5)$, $a - b = 5n_1$ where $n_1$ is an integer. \\
From $x \equiv y\ (mod\ 5)$, $x - y = 5n_2$ where $n_2$ is an integer. \\
Let $t = bn_2 - yn_1 -5n_1n_2$.
\begin{align*}
ax - by &= (b - 5n_1)(y + 5n_2) - by && \\
&= by + 5bn_2 - 5yn_1 -25n_2n_1 - by && \\
&= 5bn_2 - 5yn_1 -25n_2n_1 && \\
&= 5(bn_2 - yn_1 -5n_2n_1) && \\
&= 5t && \\
\end{align*}
Since $ax - by = 5t$ where $t$ is an integer, $ax = by\ (mod\ 5)$.
\end{proof}


\begin{tcolorbox}[title=Problem 25, breakable]
Let $d$ be a positive integer. Let $a$, $b$ be integers. \\
Define $a \equiv b\ (mod\ d)$ to mean that $a - b$ is divisible by $d$. \\
Prove that if $a \equiv b\ (mod\ d)$ and $x \equiv y\ (mod\ d)$, then 
$a + x \equiv b + y\ (mod\ d)$ and $ax = by\ (mod\ d)$.
\end{tcolorbox}

\begin{proof}
Need to show $(a + x) - (b + y) = dn$ where $n$ is an integer. \\
From $a \equiv b\ (mod\ d)$, $a - b = dn_1$ where $n_1$ is an integer. \\
From $x \equiv y\ (mod\ d)$, $x - y = dn_2$ where $n_2$ is an integer.
\begin{align*}
(a + x) - (b + y) &= (a - b) + (x - y) && \\
&= dn_1 + dn_2 && \\
&= d(n_1 + n_2) && \quad \text{let $t = n_1 + n_2$} \\
&= dt &&
\end{align*}
Since $(a + x) - (b + y)$ can be written as $dt$ where $t$ is an integer, $a + x \equiv b + y\ (mod\ d)$. 
\end{proof}

\begin{proof}
Need to show $ax - by = dn$. \\
From $a \equiv b\ (mod\ d)$, $a - b = dn_1$ where $n_1$ is an integer. \\
From $x \equiv y\ (mod\ d)$, $x - y = dn_2$ where $n_2$ is an integer.
\begin{align*}
ax - by &= (b + dn_1)(y + dn_2) - by && \\
&= by + bdn_2 + ydn_1 + dbn_1n_2 - by && \\
&= bdn_2 + ydn_1 + dbn_1n_2 && \\
&= d(bn_2 + yn_1 + bn_1n_2) & \quad \text{Let $t = bn_2 + yn_1 + bn_1n_2$} \\
&= dt && \\
\end{align*}
Since $ax - by$ can be written as $dt$ where $t$ is an integer, $ax = by\ (mod\ d)$.
\end{proof}


\begin{tcolorbox}[title=Problem 26, breakable]
Assume that every positive integer can be written in one of the forms
$3k$, $3k + 1$, or $3k + 2$ for some integer $k$.\\ 
Show that if the square of a positive integer is divisible by $3$,
then so is the integer $x$.
\end{tcolorbox}

\begin{proof}
From the assumptions $x$ can either be 
written $3k$, $3k + 1$, or $3k + 2$. \\
Need to show that if $x^2 = 3n_1$, $x = 3n_2$ for 
some integers $n_1$ and $n_2$. \\
\textbf{Case 1} ($x = 3k$): \\
Let $t_1 = 3k^2$
\begin{align*}
x^2 &= {(3k)}^2 && \\
&= 3 \cdot 3k^2 && \\
&= 3 t_1
\end{align*}
Therefore in this case $x$ is divisible by $3$. \\
%==================================================
\textbf{Case 2} ($x = 3k + 1$): \\
Let $t_2 = 2k^2 + 2k$
\begin{align*}
x^2 &= {(3k + 1)}^2 && \\
&= 6k^2 + 3k + 3k + 1 && \\
&= 6k^2 + 6k + 1&& \\
&= 3(2k^2 + 2k) + 1 && \\
&= 3t_2 + 1
\end{align*}
In this case $x^2$ is not divisible by $3$ which contradicts
our assumption, therefore $x \not = 3k + 1$. \\
%==================================================
\textbf{Case 3} ($x = 3k + 2$): \\
\text{Let $t_3 = 3k^2 + 4k$}
\begin{align*}
x^2 &= {(3k + 2)}^2 && \\
&= 9k^2 + 6k + 6k + 4 && \\
&= 9k^2 + 12k + 4 && \\
&= 3(3k^2 + 4k) + 4 && \\
&= 3t_3+ 4 
\end{align*}
In this case $x^2$ is not divisible by $3$ which contradicts
our assumption, therefore $x \not = 3k + 2$. \\ \\

Note there is no solution for $1 = 3m_1$ or $2 = 3m_2$ 
where $m_1$ and $m_2$ are integers. \\
Assume $3k_1 + 1 = 3m_1$ where $k_1$ and $m_1$ are integers. \\
\begin{align*}
3k_1 + 1 &= 3m_1 & \\
1 &= 3m_1 - 3k_1 & \\
1 &= 3(m_1 - k_1) 
\end{align*}
Therefore, $3k_1 + 1$ is not divisible by 3. \\ \\
Assume $3k_2 + 2 = 3m_2$ where $k_2$ and $m_2$ are integers. \\
\begin{align*}
3k_2 + 1 &= 3m_2 & \\
2 &= 3m_2 - 3k_2 & \\
2 &= 3(m_2 - k_2)
\end{align*}
Therefore, $3k_2 + 2$ is not divisible by 3. \\ \\
\end{proof}

\subsection{Rational Numbers}

\begin{tcolorbox}[title=Problem 4, breakable]
Let $a = \frac{m}{n}$ be a rational number expressed as a quotient of integers $m$, $n$
with $m \not = 0$ and $n \not = 0$. \\
Show that there is a rational number $b$ such that $ab = ba = 1$.
\end{tcolorbox}

\begin{proof}
Let $b = \frac{n}{m}$. Since $n$ and $m$ are integers and $m \not = 0$, 
$b$ is the ratio of two integers where the denomitor is not $0$ 
making it a rational number by definition.
\begin{align*}
ab &= \frac{m}{n} \cdot \frac{n}{m} && \\
&= \frac{mn}{nm} && \\
&= \frac{nm}{nm} && \quad \text{commutative} \\
&= 1 && \quad \text{cancellation rule for fractions}
\end{align*}
\begin{align*}
ba &= \frac{n}{m} \cdot \frac{m}{n} && \\
&= \frac{nm}{mn} && \\
&= \frac{nm}{nm} && \quad \text{commutative} \\
&= 1 && \quad \text{cancellation rule for fractions}
\end{align*}
Therefore $ab = ba = 1$.
\end{proof}

\begin{tcolorbox}[title=Problem 6, breakable]
Solve for $x$ in the following equations.
\end{tcolorbox}

\paragraph{Solution (d)}
\begin{align*}
\frac{4x}{3} + \frac{3}{4} &= 2x - 5 \\
12\left(\frac{4x}{3} + \frac{3}{4}\right) &= 12(2x - 5) \\
16x + 9 &= 24x - 60 \\
9 + 60 &= 24x - 16x \\
69 &= 8x \\
x &= \frac{69}{8}
\end{align*}

\paragraph{Solution (e)}
\begin{align*}
\frac{4(1-3x)}{7} &= 2x - 1 \\
4(1 - 3x) &= 7(2x - 1) \\
4 - 12x &= 14x - 7 \\
4 + 7 &= 14x + 12x \\
11 &= 26x \\
x &= \frac{11}{26}
\end{align*}

\paragraph{Solution (f)}
\begin{align*}
\frac{2 - x}{3} &= \frac{7}{8}x \\
8(2 - x) &= 3 \cdot 7x \\
16 - 8x &= 21x \\
16 &= 29x \\
x &= \frac{16}{29}
\end{align*}


\begin{tcolorbox}[title=Problem 6, breakable]
Let $n$ be a positive integer. By $n$ factorial, written $n!$, we mean the product: \\
$1 \cdot 2 \cdot 3 \ldots n$ \\
of the first $n$ positive integers. For Instance \\
$2! = 2$ \\
$3! = 2 \cdot 3 = 6 $ \\
$4! = 2 \cdot 3 \cdot 4 = 24$ \\
(a) Find the value $5!$, $6!$, $7!$, and $8!$. \\
(b) Define $0! = 1$. Define the binomial coefficient \\
$\binom{n}{m} = \frac{m!}{n!(m-n)!}$ \\
for any natural numbers $m$, $n$ such that $n$ lies between $0$ and $m$.
Compute tons of binomial coefficients. \\
(c) Show that $\binom{m}{n} = \binom{m}{m - n}$. \\
(d) Show that if $n$ is a positive integer at most equal to $m$, then \\
$\binom{m}{n} + \binom{m}{n - 1} = \binom{m + 1}{n}$.
\end{tcolorbox}

\textbf{Solution (a)}
\begin{align*}
5! &= 2 \cdot 3 \cdot 4 \cdot 5 &&= 120 &&& \\
6! &= 2 \cdot 3 \cdot 4 \cdot 5 \cdot 6 &&= 720 &&& \\
7! &= 2 \cdot 3 \cdot 4 \cdot 5 \cdot 6 \cdot 7 &&= 5040 &&& \\
8! &= 2 \cdot 3 \cdot 4 \cdot 5 \cdot 6 \cdot 7 \cdot 8 &&= 40320 &&& \\
\end{align*}
\textbf{Solution (b)}
\begin{align*}
\binom{3}{0} &= \frac{3!}{0!(3 - 0)!} = 1 \\
\binom{3}{1} &= \frac{3!}{1!(3 - 1)!} = 3 \\
\binom{3}{2} &= \frac{3!}{2!(3 - 2)!} = 3 \\
\binom{3}{3} &= \frac{3!}{3!(3 - 3)!} = 1 \\
\binom{4}{0} &= \frac{4!}{0!(4 - 0)!} = 1 \\
\binom{4}{1} &= \frac{4!}{1!(4 - 1)!} = 4 \\
\binom{4}{2} &= \frac{4!}{2!(4 - 2)!} = 6 \\
\binom{4}{3} &= \frac{4!}{3!(4 - 3)!} = 4 \\
\binom{4}{4} &= \frac{4!}{4!(4 - 4)!} = 1 \\
\binom{5}{0} &= \frac{5!}{0!(5 - 0)!} = 1 \\
\binom{5}{1} &= \frac{5!}{1!(5 - 1)!} = 5 \\
\binom{5}{2} &= \frac{5!}{2!(5 - 2)!} = 10 \\
\binom{5}{3} &= \frac{5!}{3!(5 - 3)!} = 10 \\
\binom{5}{4} &= \frac{5!}{4!(5 - 4)!} = 5 \\
\binom{5}{5} &= \frac{5!}{5!(5 - 5)!} = 1 \\
\end{align*}


\textbf{Solution (c)}
\begin{align*}
\binom{m}{n} &= \binom{m}{m - n} \\
\frac{m!}{n!(m - n)!} &= \frac{m!}{(m-n)!(m - (m - n))!} \\
\frac{m!}{n!(m - n)!} &= \frac{m!}{(m-n)!(n)!} \\
\frac{m!}{n!(m - n)!} &= \frac{m!}{(n)!(m-n)!} \\
\end{align*}


\textbf{Solution (d)} \\
Need to show:
\begin{align*}
\binom{m}{n} + \binom{m}{n - 1} = \binom{m + 1}{n}
\end{align*}
First note: \\
\begin{align*}
\binom{m + 1}{n} = \frac{(m + 1)!}{n!((m + 1) - n)!}
\end{align*}
Then:
\begin{align*}
\binom{m}{n} + \binom{m}{n - 1}
&= \frac{m!}{n!(m - n)!} + \frac{m!}{(n - 1)!(m - n + 1)!} \\
&= \frac{m!(m - n + 1) + m!n}{n!(m - n + 1)!} \quad \text{(common denominator)} \\
&= \frac{m!(m - n + 1 + n)}{n!(m - n + 1)!} \\
&= \frac{m!(m + 1)}{n!(m - n + 1)!} \\
&= \frac{(m + 1)!}{n!((m + 1) - n)!} \\
&= \binom{m + 1}{n}
\end{align*}


\begin{tcolorbox}[title=Problem 8, breakable]
Prove that there is no positive rational number $a$ such that $a^3 = 2$.
\end{tcolorbox}

\begin{proof} Let $a = \frac{m}{n}$ where $m$, $n$ are integers, $n \not = 0$, and $\frac{m}{n}$
is in its lowest form. If ${(\frac{m}{n})}^3 = 2$.
\begin{align*}
\frac{m^3}{n^3} &= 2 \\
m^3 &= 2n^3 
\end{align*}
If $m^3 = 2k$ for some integer $k$, then $m = 2a$ for some integer $a$ (shown in a previous problem). 
But:
\begin{align*}
{(2k)}^3 &= 2n^3 \\
2^3k^3 &= 2n^3 \\
2^2k^3 &= n^3 \\
2 \cdot (2k^3) &= n^3 \\
\end{align*}
If $n^3 = 2k$ for some integer $k$, then $n = 2a$ for some integer $a$ (shown in a previous problem). 
This contradicts our assumption that $\frac{m}{n}$ is in its lowest form, therefore
there is no positive rational number $a$ such that $a^3 = 2$.
\end{proof}


\begin{tcolorbox}[title=Problem 9, breakable]
Prove that there is no positive rational number $a$ such that $a^4 = 2$.
\end{tcolorbox}

\begin{proof} Suppose for contradiction $a^4 = 2$ where $a$ is a rational number.
Since $a$ is rational, it can be expressed as $\frac{m}{n}$ where $m$, $n$ are integers, 
$n \not = 0$, and $\frac{m}{n}$
is in its lowest form.
\begin{align*}
\frac{m^4}{n^4} &= 2 \\
m^4 &= 2n^4
\end{align*}
If $m^4 = 2k$ for some integer $k$, then $m = 2a$ for some integer $a$ (shown in a previous problem). 
But:
\begin{align*}
{(2k)}^4 &= 2n^4 \\
2^4k^4 &= 2n^4 \\
2^3k^4 &= n^4 \\
2 \cdot (2^2k^3) &= n^4 \\
\end{align*}
If $n^4 = 2k$ for some integer $k$, then $n = 2a$ for some integer $a$ (shown in a previous problem). 
This contradicts our assumption that $\frac{m}{n}$ is in its lowest form, therefore
there is no positive rational number $a$ such that $a^4 = 2$.
\end{proof}


\begin{tcolorbox}[title=Problem 10, breakable]
Prove that there is no positive rational number $a$ such that $a^2 = 3$. You 
may assume that a positive integer can be written in one of the forms
$3k$, $3k + 1$, $3k + 2$ for some integer $k$. Prove that if the square of a
positive integer is divisible by $3$ so is the integer. Then use a similar proof
for $\sqrt{2}$.
\end{tcolorbox}

\begin{proof}
Since $a$ is rational, it can be expressed as $\frac{m}{n}$ where $m$, $n$ are integers, 
$n \not = 0$, and $\frac{m}{n}$
is in its lowest form.
\begin{align*}
\frac{m^2}{n^2} &= 3 \\
m^2 &= 3n^2
\end{align*}
If $m^2 = 3k$ for some integer $k$, then $m = 3a$ for some integer $a$ (shown in a previous problem). 
But:
\begin{align*}
{(3a)}^2 &= 3n^2 \\
3^2a^2 &= 3n^2 \\
3^3a^2 &= n^2 \\
3 \cdot (3^2a^2) &= n^2
\end{align*}
If $n^2 = 3k$ for some integer $k$, then $n = 3a$ for some integer $a$. 
This contradicts our assumption that $\frac{m}{n}$ is in its lowest form, therefore
there is no positive rational number $a$ such that $a^2 = 3$.
\end{proof}

\begin{proof}
Need to show that $a^2 = 2$ has no rational solution $a$.
Since $a$ is rational, it can be expressed as $\frac{m}{n}$ where $m$, $n$ are integers, 
$n \not = 0$, and $\frac{m}{n}$
is in its lowest form.
\begin{align*}
\frac{m^2}{n^2} &= 2 \\
m^2 &= 2n^2
\end{align*}
If $m^2 = 2k$ for some integer $k$, then $m = 2a$ for some integer $a$ (shown in a previous problem). 
But:
\begin{align*}
{(2a)}^2 &= 2n^2 \\
2^2a^2 &= 2n^2 \\
2 a^2 &= n^2 \\
2 \cdot a^2 &= n^2
\end{align*}
If $n^2 = 2k$ for some integer $k$, then $n = 2a$ for some integer $a$. 
This contradicts our assumption that $\frac{m}{n}$ is in its lowest form, therefore
there is no positive rational number $a$ such that $a^2 = 2$.
\end{proof}


\begin{tcolorbox}[title=Problem 16, breakable]
A chemical substance decomposes in such a way that it halves every $3$ min. If 
there are $6$ grams (g) of the substance at present at the beginning, how much will 
will be left \\
(a) after $3$ min? \\
(b) after $27$ min? \\
(c) after $36$ min?
\end{tcolorbox}

\textbf{Solution} \\
(a) after $3$ min? $6{(\frac{1}{2})}^{(\frac{3}{3})} = 3$g \\
(b) after $27$ min? $6{(\frac{1}{2})}^{(\frac{27}{3})} = 0.01171875$g \\
(c) after $36$ min? $6{(\frac{1}{2})}^{(\frac{36}{3})} = 0.001464843$g

\begin{tcolorbox}[title=Problem 18, breakable]
A substance reacts in water in such a way that one-fourth of the undissolved parts
disolves every $10$ minutes. If you put $25$g of a substance in water at a given
time, how much will be left after: \\
(a) $10$ min? \\
(b) $30$ min? \\
(c) $50$ min?
\end{tcolorbox}

\textbf{Solution} \\
(a) after $10$ min? $25{(\frac{3}{4})}^{(\frac{10}{10})} = 18.75$g \\
(b) after $30$ min? $25{(\frac{3}{4})}^{(\frac{30}{10})} = 10.546875$g \\
(c) after $50$ min? $25{(\frac{3}{4})}^{(\frac{50}{10})} = 5.933$g

\begin{tcolorbox}[title=Problem 20, breakable]
A chemical pollutant is being emptied in a  lake with $50,000$ fishes. Every month, 
one-thrid of the fish still alive die from this pollutant. How many fish will be 
alive after: \\
(a) $1$ month? \\
(b) $2$ month? \\
(c) $4$ month?
\end{tcolorbox}

\textbf{Solution} \\
(a) after $1$ month? $50000{(\frac{2}{3})}^{1} = 33333.33$fishes \\
(b) after $2$ month? $50000{(\frac{2}{3})}^{2} = 22222.22$fishes \\
(c) after $4$ month? $50000{(\frac{2}{3})}^{4} = 9876.54$fishes


\begin{tcolorbox}[title=Problem 21, breakable]
Every $10$ years the population of a city is five-fourths of what it was the $10$
years before. How many years does it take \\
(a) before the population doubles \\
(b) before it triples
\end{tcolorbox}

Formula: $p \cdot \frac{5}{4}^{\frac{y}{10}} = 2p$ \\
\textbf{Solution (a)}
\begin{align*}
p \cdot \frac{5}{4}^{\frac{y}{10}} = 2p && \\
= y \approx 31\text{ years}
\end{align*}
\textbf{Solution (b)}
\begin{align*}
p \cdot \frac{5}{4}^{\frac{y}{10}} = 3p && \\
= y \approx 49\text{ years}
\end{align*}

\subsection{Multiplicative Inverse}

\begin{tcolorbox}[title=Problem 2, breakable]
Prove the following relations. It is assumed that all values of $x$ and $y$
which occur are such that the denominators in the indicated fractions are not $0$. \\
(a) $\frac{1}{x + y} - \frac{1}{x - y} = \frac{-2y}{x^2 - y^2}$ \\
(b) $\frac{x^3 - 1}{x - 1} = 1 + x + x^2$ \\
(c) $\frac{x^4 - 1}{x - 1} = 1 + x + x^2 + x^3$
\end{tcolorbox}

\begin{proof}
\begin{align*}
\frac{1}{x + y} - \frac{1}{x - y} &= \frac{-2y}{x^2 - y^2} \\
(x + y)[\frac{1}{x + y} - \frac{1}{x - y}] &= \frac{-2y}{x^2 - y^2}(x + y) \\
1 - \frac{x + y}{x - y} &= \frac{-2y}{x - y}(x + y) \\
(x - y)(1 - \frac{x + y}{x - y}) &= \frac{-2y}{x - y}(x - y) \quad \text{} \\
(x - y) - (x + y) &= -2y\\
-2y &= -2y
\end{align*}
\end{proof}
\begin{proof}
\begin{align*}
\frac{x^3 - 1}{x - 1} &= 1 + x + x^2 && \\
x^3 - 1 &= (x-1)(1 + x + x^2) && \\
x^3 - 1 &= x + x^2 + x^3 - 1 - x - x^2  && \\
x^3 - 1 &= x^3 - 1
\end{align*}
\end{proof}
\begin{proof}
\begin{align*}
\frac{x^4 - 1}{x - 1} &= 1 + x + x^2 + x^3 && \\
x^4 - 1 &= (x - 1)(1 + x + x^2 + x^3) && \\
x^4 - 1 &= x + x^2 + x^3 + x^4 - 1 - x - x^2 - x^3 && \\
x^4 - 1 &= x + x^2 + x^3 + x^4 - 1 - x - x^2 - x^3 && \\
x^4 - 1 &= x^4 - 1
\end{align*}
\end{proof}
\begin{proof}
\begin{align*}
\frac{x^n - 1}{x - 1} &= x^{n - 1} + x^{n - 2} + \cdots + x + 1 && \\
x^n - 1 &= (x - 1)(x^{n - 1} + x^{n - 2} + \cdots + x + 1) && \\
x^n - 1 &= x^{n} + x^{n - 1} + \cdots + x^2 + x - x^{n - 1} - x^{n - 2} - \cdots -x - 1  && \\
x^n - 1 &= x^n - 1
\end{align*}
\end{proof}

\begin{tcolorbox}[title=Problem 3, breakable]
Prove the following relations. \\
(a) $\frac{1}{2x + y} + \frac{1}{2x - y} = \frac{4x}{4x^2 - y^2}$ \\
(b) $\frac{2x}{x + 5} + \frac{3x + 1}{2x + 1} = \frac{x^2 - 14x - 5}{2x^2 + 11x + 5}$ \\
(c) $\frac{1}{x + 3y} + \frac{1}{x - 3y} = \frac{2x}{x^2 - 9y^2}$ \\
(c) $\frac{1}{3x - 2y} + \frac{x}{x + y} = \frac{x + y + 3x^2 - 2xy}{3x^2 + xy - 2y^2}$
\end{tcolorbox}

\begin{proof}
\begin{align*}
\frac{1}{2x + y} + \frac{1}{2x - y} &= \frac{4x}{4x^2 - y^2} && \\
\leftrightarrow \left((2x + y)(2x - y)\right)\left(\frac{1}{2x + y} + \frac{1}{2x - y}\right) 
    &= \left((2x + y)(2x - y)\right)\left(\frac{4x}{4x^2 - y^2}\right) && \\
\leftrightarrow (2x - y) + (2x + y) &= 4x && \\
\leftrightarrow 4x &= 4x 
\end{align*}
\end{proof}
\begin{proof}
\begin{align*}
\frac{2x}{x + 5} + \frac{3x + 1}{2x + 1} &= \frac{x^2 - 14x - 5}{2x^2 + 11x + 5} && \\
\leftrightarrow \left((x + 5)(2x + 1)\right)\left(\frac{2x}{x + 5} + \frac{3x + 1}{2x + 1}\right)
    &= \left((x + 5)(2x + 1)\right)\left(\frac{x^2 - 14x - 5}{2x^2 + 11x + 5}\right) && \\
\leftrightarrow 2x(2x + 1) - (3x + 1)(x + 5) &= x^2 - 14x - 5 && \\
\leftrightarrow (4x^2 + 2x) - (3x^2 + 15x + 5) &= x^2 - 14x - 5 && \\
\leftrightarrow x^2 - 14x - 5 &= x^2 - 14x - 5 && \\
\end{align*}
\end{proof}

\begin{proof}
\begin{align*}
\frac{1}{x + 3y} + \frac{1}{x - 3y} &= \frac{2x}{x^2 - 9y^2} && \\
\leftrightarrow \left((x + 3y)(x - 3y)\right)\left(\frac{1}{x + 3y} + \frac{1}{x - 3y}\right)
    &= \left((x + 3y)(x - 3y)\right)\left(\frac{2x}{x^2 - 9y^2}\right) && \\
\leftrightarrow (x - 3y) + (x + 3y) &= 2x && \\
\leftrightarrow 2x &= 2x 
\end{align*}
\end{proof}
\begin{proof}
\begin{align*}
\frac{1}{3x - 2y} + \frac{x}{x + y} &= \frac{x + y + 3x^2 - 2xy}{3x^2 + xy - 2y^2} && \\
\leftrightarrow \left((3x - 2y)(x + y)\right)\left(\frac{1}{3x - 2y} + \frac{x}{x + y}\right)
    &= \left((3x - 2y)(x + y)\right)\left(\frac{x + y + 3x^2 - 2xy}{3x^2 + xy - 2y^2}\right) && \\
\leftrightarrow (x + y) + x(3x-2y) &= x + y + 3x^2 - 2xy && \\
\leftrightarrow (x + y) + (3x^2 - 2xy) &= x + y + 3x^2 - 2xy && \\
\leftrightarrow x + y + 3x^2 - 2xy &= x + y + 3x^2 - 2xy
\end{align*}
\end{proof}


\begin{tcolorbox}[title=Problem 4, breakable]
Prove the following relations. \\
(a) $\frac{x^3 - y^3}{x - y} = x^2 + xy + y^2$ \\
(b) $\frac{x^4 - y^4}{x - y} = x^3 + x^2y + xy^2 + y^3$ \\
Let \\
$x = \frac{1 - t^2}{1 + t^2}$ and $y = \frac{2t}{1 + t^2}$ \\
Show that $x^2 + y^2 = 1$
\end{tcolorbox}

\begin{proof}
\begin{align*}
\frac{x^3 - y^3}{x - y} = x^2 + xy + y^2 && \\
\leftrightarrow x^3 - y^3 &= (x - y)(x^2 + xy + y^2) && \\
\leftrightarrow x^3 - y^3 &= x^3 + x^2y + xy^2 - x^2y - xy^2 - y^3 && \\
\leftrightarrow x^3 - y^3 &= x^3 - y^3
\end{align*}
\end{proof}
\begin{proof}
\begin{align*}
\frac{x^4 - y^4}{x - y} &= x^3 + x^2y + xy^2 + y^3 && \\
\leftrightarrow x^4 - y^4 &= (x - y)(x^3 + x^2y + xy^2 + y^3) && \\
\leftrightarrow x^4 - y^4 &= x^4 + x^3y + x^2y^2 + xy^3 - x^3y - x^3y - x^2y^2 - xy^3 - y^4 && \\
\leftrightarrow x^4 - y^4 &= x^4 - y^4
\end{align*}
\end{proof}
\begin{proof}
\begin{align*}
x^2 + y^2 &= 1 && \\
\leftrightarrow {\left(\frac{1 - t^2}{1 + t^2}\right)}^2 
    + {\left(\frac{2t}{1 + t^2}\right)}^2 &= 1 && \\
    \leftrightarrow \frac{{(1 - t^2)}^2 + {(2t)}^2}{{(1 + t^2)}^2} &= 1 && \\
    \leftrightarrow \frac{t^4 - 2t^2 + 1 + 4t^2}{{(1 + t^2)}^2} &= 1 && \\
    \leftrightarrow \frac{t^4 + 2t^2 + 1}{{(1 + t^2)}^2} &= 1 && \\
    \leftrightarrow \frac{{(1 + t^2)}^2}{{(1 + t^2)}^2} &= 1 && \\
    \leftrightarrow 1 &= 1 && \\
\end{align*}
\end{proof}


\begin{tcolorbox}[title=Problem 5, breakable]
Prove the following relations. \\
(a) $\frac{x^3 + 1}{x + 1} = x^2 - x + 1$ \\
(b) $\frac{x^5 + 1}{x + 1} = x^4 - x^3 + x^2 - x + 1$ \\
(c) If $n$ is an odd integerl, prove that \\
$\frac{x^n + 1}{x + 1} = x^{(n - 1)} - x^{(n - 2)} + x^{(n - 3)} - \cdots -x + 1$
\end{tcolorbox}

\begin{proof}
\begin{align*}
\frac{x^3 + 1}{x + 1} &= x^2 - x + 1 && \\
\leftrightarrow x^3 + 1 &= (x + 1)(x^2 - x + 1) && \\
\leftrightarrow x^3 + 1 &= x^3 - x^2 + x + x^2 - x + 1 && \\
\leftrightarrow x^3 + 1 &= x^3 + 1 && \\
\end{align*}
\end{proof}
\begin{proof}
\begin{align*}
\frac{x^5 + 1}{x + 1} &= x^4 - x^3 + x^2 - x + 1 && \\
\leftrightarrow x^5 + 1 &= (x + 1)(x^4 - x^3 + x^2 - x + 1) && \\
\leftrightarrow x^5 + 1 &= x^5 - x^4 + x^3 - x^2 + x +x^4 - x^3 + x^2 - x + 1 && \\
\leftrightarrow x^5 + 1 &= x^5 + 1 && \\
\end{align*}
\end{proof}
\begin{proof}
\begin{align*}
\frac{x^n + 1}{x + 1} &= x^{(n - 1)} - x^{(n - 2)} + x^{(n - 3)} - \cdots -x + 1 && \\
\leftrightarrow x^n + 1 &= (x + 1)(x^{(n - 1)} - x^{(n - 2)} + x^{(n - 3)} - \cdots -x + 1) && \\
\leftrightarrow x^n + 1 &= x^n - x^{(n - 1)} + x^{(n - 2)} - \cdots - x^2  + x
    + x^{(n - 1)} - x^{(n - 2)} + x^{(n - 3)} - \cdots - x + 1 && \\
    \leftrightarrow x^n + 1 &= x^n + 1&& \\
\end{align*}
\end{proof}


\begin{tcolorbox}[title=Problem 7, breakable]
If a solid has a uniform density $d$, occpies a volume $v$, and has a mass $m$,
then we have the formula \\
$m = vd$ \\
Find the density if: \\
(a) $m = \frac{3}{10}$lb and $v = \frac{2}{3}$in$^3$ \\
(a) $m = 6$lb and $v = \frac{4}{3}$in$^3$ \\
Find the volume if the mass is $15\,\text{lb}$ and the density 
is $\frac{2}{3}\,\text{lb/in}^3$.
\end{tcolorbox}

\textbf{Solution (a)}
\begin{align*}
\frac{3}{10} &= \frac{2}{3} d && \\
d &= \frac{9}{20}
\end{align*}
\textbf{Solution (b)}
\begin{align*}
6 &= \frac{4}{3} d && \\
d &= \frac{18}{4}
\end{align*}
\textbf{Solution (c)}
\begin{align*}
15 &= v \frac{2}{3} && \\
v &= \frac{45}{2} &&
\end{align*}

\begin{tcolorbox}[title=Problem 13, breakable]
Tickets for a performance sell \$$5.00$ and \$$2.00$. The total amount collected was
\$$4,100$, and there are $1,300$ tickets in all. How many tickets of each price 
were sold.
\end{tcolorbox}

\textbf{Solution}
Let $x =$ be the number of tickets sold at \$$2.00$.
\begin{align*}
2x + 5(1300 - x) &= 4100 && \\
2x + 6500 - 5x = 4100 && \\
-3x = -2400 && \\
x = 800
\end{align*}
$800$ tickets sold at \$$2.00$ and $500$ sold at \$$5.00$.

\begin{tcolorbox}[title=Problem 16, breakable]
A boat travels a distance of $500$mi, along two rivers, for $50$hr. The current goes in 
the same direction as the boat along one river, and then the boat averages $20$mph. The 
current goes in the opposite dirction along the other river, and then the boat averages 
$8$mph. During how many hours was the boat on the first river.
\end{tcolorbox}

\textbf{Solution}
Let $x$ be the time spent on the first river.
\begin{align*}
20x + 8(50 - x) &= 500 && \\
20x + 400 - 8x &= 500 && \\
12x &= 100 && \\
x &= \frac{100}{12} 
\end{align*}
$x = \frac{100}{12}$ hours on first river.

\begin{tcolorbox}[title=Problem 18, breakable]
The radiator of a car can contain $10$kg of liquid. If it is half full with a mixture 
having $60$\%  antifreeze and $40$\% water, how much more water must be added so that the
resulting mixture has only.
\end{tcolorbox}

\textbf{Solution (a) $40$\% antifreeze} \\
$40$\% antifreeze means $60$\% water. \\
\begin{align*}
\frac{4 + x}{10 + x} &= 0.6 && \\
4 + x &= 0.6(10 + x) && \\
4 + x &= 6 + 0.6x && \\
x = 5
\end{align*}
$15$ kg will fit in the radiator. \\
\textbf{Solution (b) $10$\% antifreeze} \\
$10$\% antifreeze means $90$\% water. \\
\begin{align*}
\frac{4 + x}{10 + x} &= 0.9&& \\
4 + x &= 0.9(10 + x) && \\
4 + x &= 9 + 0.9x && \\
x = 50
\end{align*}
$55$ kg will not fit in the radiator. \\