\begin{tcolorbox}[title=Problem 1, breakable]
    \begin{enumerate}
        \item We showed that in any primitive Pythagorean triple 
                $(a, b, c)$, either $a$ or $b$ is even.
                Use the same sort of argument to show that either $a$ or $b$
                must be a multiple of $3$.
        \item By examining the above list of primitive triples, make 
                 a guess about when $a, b,$ or $c$ is a multiple of $5$.
                 Try to show that your guess is correct.
    \end{enumerate}
\end{tcolorbox}

\begin{proof}
    Notice for $k \in \mathbb{Z}$:
    \begin{enumerate}
        \item $(3k)^2 = 9k^2 \equiv 0 \pmod{3}$.
        \item $(3k + 1)^2 = 9k^2 + 6k + 1 \equiv 1 \pmod{3}$.
        \item $(3k + 2)^2 = 9k^2 + 12k + 4 \equiv 1 \pmod{3}$.
    \end{enumerate}
    Then if $3 \nmid a, b$
    \[
        (1 \pmod{3}) + (1 \pmod{3}) \equiv 2 \pmod{3} = c^2.
    \]
    But no integer squared has remainder $2$ modulo $3$, which is a contradiction.
\end{proof}

\begin{proof}
    Notice for $k \in \mathbb{Z}$:
    \begin{enumerate}
        \item $(5k)^2 \equiv 0 \pmod{5}$,
        \item $(5k \pm 1)^2 = 25k^2 \pm 10k + 1 \equiv 1 \pmod{5}$,
        \item $(5k \pm 2)^2 = 25k^2 \pm 20k + 4 \equiv 4 \pmod{5}$.
    \end{enumerate}
    Then if $5 \nmid a, b, c$
    \[
        a^2, b^2, c^2 \equiv 1 \text{ or } 4 \pmod{5}.
    \]
    Checking all possibilities,
    \[
        1+1 \equiv 2,\quad 1+4 \equiv 0,\quad 4+4 \equiv 3 \pmod{5}.
    \]
    Since $a^2 + b^2 = c^2$, the only possible case is
    \[
        a^2 + b^2 \equiv 0 \pmod{5},
    \]
    thus $5 \mid c^2$ and therefore $5 \mid c$, which is a contradiction.
\end{proof}

\begin{tcolorbox}[title=Problem 2, breakable]
    A nonzero integer $d$ is said to \emph{divide}
    an integer $m$ if $m = dk$ for some number $k$.
    Show that if $d$ divides both $m$ and $n$,
    then $d$ also divides $m - n$ and $m + n$.
\end{tcolorbox}

\begin{proof}
    Suppose $d$ divides both $m$ and $n$.
    Thus there exists $k_1, k_2$ such that $m = k_1 d$ and $n = k_2 d$.
    Then $m - n = k_1 d - k_2 d = (k_1 - k_2)d$.
    Thus $d \mid m - n$.
    Similarly, $m + n = k_1 d + k_2 d = (k_1 + k_2)d$.
    Thus $d \mid m + n$.
\end{proof}

\begin{tcolorbox}[title=Problem 6, breakable]
    If you look at the table of Pythagorean triples in this chapter,
        you will see many triples in which $c$ is $2$ greater than $a$.
    For example, the triples $(3, 4 ,5)$, $(15, 8, 17)$, $(35, 12, 37)$, and $(63, 16, 65)$ 
    all have this property.
    \begin{enumerate}
        \item Find two more primitive Pythagorean triples $(a, b, c)$ have $c = a + 2$.
        \item Find a primitive Pythagorean triple $(a, b, c)$ having $c = a + 2$ and $c > 100$.
        \item Try to find a formula that describes all primitive Pythagorean triples $(a, b, c)$
                having $c = a + 2$.
    \end{enumerate}
\end{tcolorbox}

\textbf{Solution (a):}
\begin{enumerate}
    \item $(99, 20, 101)$
    \item $(143, 24, 145)$
    \item $(195, 28, 197)$
    \item $(255, 32, 257)$
    \item $(323, 36, 325)$
    \item $(399, 40, 401)$
    \item $(483, 44, 485)$
    \item $(575, 48, 577)$
    \item $(675, 52, 677)$
    \item $(783, 56, 785)$
    \item $(899, 60, 901)$
\end{enumerate}

\textbf{Solution (b):} All previous solutions have $c > 100$.

\begin{proof}
    We require $a^2 + b^2 = c^2$ with $c = a + 2$.
    Thus
    \[
        a^2 + b^2 = (a + 2)^2
        \iff b^2 = (a + 2)^2 - a^2
        \iff b^2 = a^2 + 4a + 4 - a^2
        \iff b^2 = 4a + 4 = 4(a + 1).
    \]
    Thus $4 \mid b^2$ and therefore $2 \mid b$. Then $b = 2k$ for some integer $k$.
    Then
    \[
        4k^2 = 4(a + 1) \iff k^2 = a + 1.
    \]
    Thus $a = k^2 - 1$ and $c = a + 2 = k^2 + 1$.
    Therefore $(a,b,c) = (k^2 - 1,\, 2k,\, k^2 + 1)$ satisfies $a^2 + b^2 = c^2$.
\end{proof}

\begin{tcolorbox}[title=Problem 7, breakable]
    For each primitive Pythagorean triple $(a, b, c)$
    in the table in this chapter, compute the quantity 
    $2c - 2a$. Do these values seem to have some special form?
    Try to prove that your observation is true for all primitive Pythagorean triples.
\end{tcolorbox}

\textbf{Solution: }
\begin{enumerate}
    \item $(3, 4, 5)$ then
    $2c - 2a = 2(5) - 2(3) = 10 - 6 = 4$.
    \item $(5, 12, 13)$ then
    $2c - 2a = 2(13) - 2(5) = 26 - 10 = 16$.
    \item $(7, 24, 25)$ then
    $2c - 2a = 2(25) - 2(7) = 50 - 14 = 36$.
    \item $(9, 40, 41)$ then
    $2c - 2a = 2(41) - 2(9) = 82 - 18 = 64$.
    \item $(15, 8, 17)$ then
    $2c - 2a = 2(17) - 2(15) = 34 - 30 = 4$.
    \item $(21, 20, 29)$ then
    $2c - 2a = 2(29) - 2(21) = 58 - 42 = 16$.
    \item $(35, 12, 37)$ then
    $2c - 2a = 2(37) - 2(35) = 74 - 70 = 4$.
    \item $(45, 28, 53)$ then
    $2c - 2a = 2(53) - 2(45) = 106 - 90 = 16$.
    \item $(63, 16, 65)$ then
    $2c - 2a = 2(65) - 2(63) = 130 - 126 = 4$.
\end{enumerate}

Those are all perfect squares.

\begin{proof}
    We know that for $s > t > 1$ we have 
    \[
        a = st
        \quad \text{and} \quad
        c = \frac{s^2 + t^2}{2}.
    \]
    Then
    \[
        2c - 2a
        = (s^2 + t^2) - 2st
        = (s - t)^2.
    \]
    Thus $2c - 2a$ is a perfect square.
\end{proof}

\begin{tcolorbox}[title=Problem 9, breakable]
    \begin{enumerate}
        \item Read about the Babylonian number system and write a short description,
        including the symbols from $1$ to $10$ and the multiples of $10$ from $20$ to $50$.
        \item Read about the Babylonian tablet called `Plimpton 322' and write a brief report,
        including its approximate date of origin.
        \item The second and third columns of `Plimpton 322' give pairs of integers $(a, c)$
        having the property that $c^2 - a^2$ is a perfect square. Convert some of these 
        pairs from Babylonian numbers to decimal numbers and compute the value $b$ so that 
        $(a, b, c)$ is a Pythagorean triple.
    \end{enumerate}
\end{tcolorbox}

The Babylonians used a sexagesimal (base-60) positional 
numeral system. This system was inherited from either 
the Sumerian or Akkadian civilizations. However, neither 
of these predecessors used a positional system. In other words, 
they did not have a convention for which ``end'' of the numeral 
represented the units. The system first appeared around 2000~BC. 
It is credited as being the first known positional numeral system 
in which the digit itself and its position within the number 
reflect its value.

\begin{figure}[h!]
    \centering
    \includegraphics[width=0.5\textwidth]{images/chapter2/numerals.png}
    \caption{Babylonian numerals from $1$ to $50$}
    \label{numerals}
\end{figure}

Figure~\ref{numerals} shows the numbers $1$--$50$. 
The Babylonians did not have a digit for the number zero.

\begin{figure}[h!]
    \centering
    \includegraphics[width=0.5\textwidth]{images/chapter2/tablet.png}
    \caption{The Babylonian tablet \emph{Plimpton 322}}
    \label{tablet}
\end{figure}

Figure~\ref{tablet} shows a tablet named \emph{Plimpton 322}. 
It was made of clay and is believed to have been written around 
1800~BC. Each row of the table corresponds to a Pythagorean triple. 
In other words, each row corresponds to three integers, 
say $(a,b,c)$, such that $a^2 + b^2 = c^2$. The tablet was written 
approximately 13--15 centuries prior to the Greek discoveries 
in geometry. The table exclusively lists numbers in which the 
longer leg has prime factors only $2$, $3$, or $5$. Thus, the other 
two sides have exact terminating representations in the 
Mesopotamian sexagesimal number system. The purpose of the table 
is unknown.

On the table, the second and third columns of \emph{Plimpton 322} 
give pairs $(a,c)$ such that $c^2 - a^2$ is a perfect square. 
For example, one row corresponds to 
$a = 119$ and $c = 169$. Then
\[
b^2 = c^2 - a^2 = 169^2 - 119^2 = 28561 - 14161 = 14400 = 120^2,
\]
thus $b = 120$ and it follows that $(119,120,169)$ is a Pythagorean triple. 
Another row shows $a = 65$ and $c = 97$. Then
\[
b^2 = 97^2 - 65^2 = 5184 = 72^2,
\]
so $(65,72,97)$ is also a Pythagorean triple. These examples 
show that the pairs listed on \emph{Plimpton 322}  
generate valid Pythagorean triples.

\textbf{References}
\begin{enumerate}
    \item Wikipedia, \emph{Babylonian cuneiform numerals}.  
    \url{https://en.wikipedia.org/wiki/Babylonian_cuneiform_numerals}
    \item Wikipedia, \emph{Plimpton 322}.  
    \url{https://en.wikipedia.org/wiki/Plimpton_322}
\end{enumerate}