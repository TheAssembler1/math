\begin{tcolorbox}[title=Problem 1, breakable]
    As we have just seen, we get every Pythagorean triple $(a, b, c)$ with $b$ even from the formula 
    \[(a, b, c) = (u^2 - v^2, 2uv, u^2 + v^2).\]
    by substituting in different integers $u$ and $v$. For example $(u, v) = (2, 1)$ gives the smallest 
    triple $(3, 4, 5)$.
    \begin{enumerate}
        \item If $u$ and $v$ have a common factor, explain why $(a, b, c)$ will not be a primitive Pythagorean triple.
        \item Find an example of integers $u > v > 0$ that do not have a common factor, yet the Pythagorean triple $(u^2 - v^2, 2uv, u^2 + v^2)$ is not primitive.
        \item Make a table of the Pythagorean triples that arise when you substitute in all values of $u$ and $v$ with $1 \le v \le 10$.
        \item Using your table from (c), find some simple conditions on $u$ and $v$ that ensures the Pythagorean triple $(u^2 - v^2, 2uv, u^2 + v^2)$ is primitive.
        \item Prove that your conditions in (d) really work.
    \end{enumerate}
\end{tcolorbox}

\begin{proof}
    Suppose $u$ and $v$ have a common factor $a > 1$.
    Thus we can write $u = k_1 a$ and $v = k_2 a$ where $k_1, k_2 \in \mathbb{Z}$.
    Then 
    \[(a, b, c) = (u^2 - v^2, 2uv, u^2 + v^2) = ((k_1 a)^2 - (k_2 a)^2), 2(k_1 a)(k_2 a), (k_1 a)^2 + (k_2 a)^2 = (a(k_1 a - k_2 a), a(2 k_1 k_2), a(k_1 a + k_2 a)).\]
    Thus $(a, b, c)$ is not a primitive.
\end{proof}

\textbf{Solution (2): } Take $u = 3, v = 1$.
\[(a, b, c) = (u^2 - v^2, 2uv, u^2 + v^2) = (8, 6, 10).\]

\textbf{Solution (3): }
\begin{table}[h!]
\centering
\begin{tabular}{|c|c|c|c|c|c|}
\hline
$u$ & $v$ & $a = u^2 - v^2$ & $b = 2uv$ & $c = u^2 + v^2$ & Primitive? \\
\hline
2 & 1 & 3 & 4 & 5 & Yes \\
3 & 1 & 8 & 6 & 10 & No \\
3 & 2 & 5 & 12 & 13 & Yes \\
4 & 1 & 15 & 8 & 17 & Yes \\
4 & 2 & 12 & 16 & 20 & No \\
4 & 3 & 7 & 24 & 25 & Yes \\
5 & 1 & 24 & 10 & 26 & No \\
5 & 2 & 21 & 20 & 29 & Yes \\
5 & 3 & 16 & 30 & 34 & No \\
5 & 4 & 9 & 40 & 41 & Yes \\
6 & 1 & 35 & 12 & 37 & Yes \\
6 & 2 & 32 & 24 & 40 & No \\
6 & 3 & 27 & 36 & 45 & No \\
6 & 4 & 20 & 48 & 52 & No \\
6 & 5 & 11 & 60 & 61 & Yes \\
7 & 1 & 48 & 14 & 50 & No \\
7 & 2 & 45 & 28 & 53 & Yes \\
7 & 3 & 40 & 42 & 58 & No \\
7 & 4 & 33 & 56 & 65 & Yes \\
7 & 5 & 24 & 70 & 74 & No \\
7 & 6 & 13 & 84 & 85 & Yes \\
8 & 1 & 63 & 16 & 65 & Yes \\
8 & 2 & 60 & 32 & 68 & No \\
8 & 3 & 55 & 48 & 73 & Yes \\
8 & 4 & 48 & 64 & 80 & No \\
8 & 5 & 39 & 80 & 89 & Yes \\
8 & 6 & 28 & 96 & 100 & No \\
8 & 7 & 15 & 112 & 113 & Yes \\
9 & 1 & 80 & 18 & 82 & No \\
9 & 2 & 77 & 36 & 85 & Yes \\
9 & 3 & 72 & 54 & 90 & No \\
9 & 4 & 65 & 72 & 97 & Yes \\
9 & 5 & 56 & 90 & 106 & No \\
9 & 6 & 45 & 108 & 117 & No \\
9 & 7 & 32 & 126 & 145 & Yes \\
9 & 8 & 17 & 144 & 145 & Yes \\
10 & 1 & 99 & 20 & 101 & Yes \\
10 & 2 & 96 & 40 & 104 & No \\
10 & 3 & 91 & 60 & 109 & Yes \\
10 & 4 & 84 & 80 & 116 & No \\
10 & 5 & 75 & 100 & 125 & No \\
10 & 6 & 64 & 120 & 136 & No \\
10 & 7 & 51 & 140 & 149 & Yes \\
10 & 8 & 36 & 160 & 164 & No \\
10 & 9 & 19 & 180 & 181 & Yes \\
\hline
\end{tabular}
\caption{Pythagorean triples generated by $1 \le v \le 10$ and $u>v$.}
\end{table}


\begin{tcolorbox}[title=Problem 2, breakable]
    \begin{enumerate}
        \item Use the lines though the point $(1, 1)$ to describe all points on the circle 
        \[x^2 + y^2 = 2.\]
        whose coordinates are rational numbers.
        \item What goes wrong if you try to apply the same procedure to find all points on the circle $x^2 + y^2 = 3$ with rational coordinates.
    \end{enumerate}
\end{tcolorbox}

\begin{tcolorbox}[title=Problem 3, breakable]
    Find a formula for all the points on the hyperbola
    \[x^2 - y^2 = 1.\]
    whose coordinates are rational numbers. 
    [Hint: Take the line through the point $(-1, 0)$ having rational slope $m$ and find a formula in terms of $m$ for the second point where the line 
    intersects the hyperbola.]
\end{tcolorbox}

\begin{tcolorbox}[title=Problem 4, breakable]
    The curve 
    \[y^2 = x^3 + 8.\]
    contains the points $(-1, 3)$ and $(-7/4, 13/8)$. The line through these two points intersect 
    the curve in exactly one other point. Find this third point.
    Can you explain why the coordinates of this number are rational numbers.
\end{tcolorbox}