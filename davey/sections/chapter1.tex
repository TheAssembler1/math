\begin{tcolorbox}[title=Problem 5, breakable]
    Prove that the ordered set $\sum^{**}$ of all binary strings 
    is a \textbf{tree} (that is, an ordered set $P$ with $\perp$
    such that $\downarrow x$ is a chain for each $x \in P$).
    For each $u \in \sum^{**}$ describe the set of elements covering $u$.
\end{tcolorbox}

\begin{proof}
    Consider the empty string $e$.
    Now, $e \in \sum^{**}$, and for all $t \in \sum^{**}$,
        $e \le t$.
    Thus $e = \perp$.

    Let $x$ be an arbitrary element in $\sum^{**}$. 
    Note that $\downarrow x \neq \emptyset$ since $e \le x$.
    Now, let $s_1, s_2$ be arbitrary elements in $\downarrow x$.
    Clearly, either $s_1 \le s_2$ or $s_2 \le s_1$,
    since both are prefixes of the same string $x$.
    Thus $\downarrow x$ is a chain as required.

    Let $s$ be an arbitrary element in $\sum^{**}$.
    Now, if $s$ is infinite then there is no element covering $s$.
    Therefore, suppose $s$ is finite.
    There are two elements $s', s'' \in \sum^{**}$ covering $s$.
    These are $s' = s0$ and $s'' = s1$, where concatenation
    appends the symbol to the end of the string.
\end{proof}

\begin{tcolorbox}[title=Problem 7, breakable]
    Let $P$ and $Q$ be ordered sets. 
    Prove that $(a_1, b_1) \prec (a_2, b_2)$ in $P \times Q$
        if and only if $(a_1 = a_2 \text{ \& } b_1 \prec b_2)$ or $(a_1 \prec a_2 \text{ \& } b_1 = b_2)$.
\end{tcolorbox}
\begin{proof}

    ($\longrightarrow$) 
    Suppose $(a_1, b_1) \prec (a_2, b_2)$ in $P \times Q$.
    Furthermore, suppose $a_1 < a_2$ and $b_1 < b_2$.
    Considering the pairs $(a_2, b_1)$ and $(a_1, b_2)$,
        we find the following two relations: 
        $(a_1, b_1) < (a_2, b_1) < (a_2, b_2)$ and
        $(a_1, b_1) < (a_1, b_2) < (a_2, b_2)$.
    Both of which contradict the $\prec$ relation.
    Therefore, either $a_1 = a_2$ or $b_1 = b_2$.
    Suppose $a_1 = a_2$. Then $b_1 < b_2$. If $b_1 \not\prec b_2$ then 
        there exists $b'$ such that $b_1 < b' < b_2$.
    But then $(a_1, b_1) < (a_1, b') < (a_2, b_2)$.
    Thus $b_1 \prec b_2$.
    A similar argument shows $a_1 \prec a_2$ if $b_1 = b_2$.

    ($\longleftarrow$) 
    Suppose $(a_1 = a_2 \text{ \& } b_1 \prec b_2)$ or
    $(a_1 \prec a_2 \text{ \& } b_1 = b_2)$.
    Now, suppose $(a_1 = a_2 \text{ \& } b_1 \prec b_2)$.
    Futhermore, suppose there exists $(a', b')$ such that
        $(a_1, b_1) < (a', b') < (a_2, b_2)$.
    Then $a_1 \le a' \le a_2$ and $b_1 \le b' \le b_2$.
    But $a_1 = a_2$ and $b_1 \prec b_2$, thus $a' = a_1$ and $b' = b_1$.
    It follows that $(a_1, b_1) = (a', b')$.
    Thus $(a_1, b_1) \prec (a_2, b_2)$.
    A similar argument shows $(a_1, b_1) \prec (a_2, b_2)$ if $b_1 = b_2$.
\end{proof}

\begin{tcolorbox}[title=Problem 8, breakable]
    Draw the diagrams of the products shown in Figure 1.12.
\end{tcolorbox}
\begin{figure}[h!]
    \centering
    \includegraphics[scale=0.4]{images/chapter1/1.8.1.PNG}
\end{figure}
\begin{figure}[h!]
    \centering
    \includegraphics[scale=0.5]{images/chapter1/1.8.2.PNG}
\end{figure}

\begin{tcolorbox}[title=Problem 13, breakable]
    Draw and label a diagram for $\mathcal{O}(P)$ for each of the ordered sets $P$
        of Figure 1.13.
\end{tcolorbox}

\begin{tcolorbox}[title=Problem 14, breakable]
    Let $P$ be a finite ordered set.

    (i) Show that $Q = \downarrow Max Q$ for all $Q \in \mathcal{O}(P)$.

    (ii) Establish a one-to-one correspondence between the elements of $\mathcal{O}(P)$
        and antichains in $P$.

    (iii) Hence show that, for all $x \in P$, $|\mathcal{O}(P)| = |\mathcal{O}(P \setminus \{x\}) + |\mathcal{O}(P \setminus (\downarrow x \cup \uparrow x))|$.
\end{tcolorbox}

\begin{tcolorbox}[title=Problem 17, breakable]
\end{tcolorbox}

\begin{tcolorbox}[title=Problem 24, breakable]
\end{tcolorbox}

\begin{tcolorbox}[title=Problem 25, breakable]
\end{tcolorbox}

\begin{tcolorbox}[title=Problem 26, breakable]
\end{tcolorbox}

\begin{tcolorbox}[title=Problem 27, breakable]
\end{tcolorbox}