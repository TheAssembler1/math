\subsection{Axioms for the Natural Numbers}

\begin{tcolorbox}[title=Problem 1, breakable]
    Fill in the missing details in the proof of Theorem $1.2.6$.
\end{tcolorbox}

\begin{proof}
    We must show the uniquness of the binary operation $\cdot : \mathbb{N} \times \mathbb{N} \rightarrow \mathbb{N}$
    that satisfies the following two properties for all $n, m \in \mathbb{N}$.
    \begin{enumerate}[label=\textbf{\alph*.}]
        \item $n \cdot 1 = n$.
        \item $n \cdot s(m) = (n \cdot m) + n$.
    \end{enumerate}
    Suppose there are two binary operations $\cdot$ and $\times$
        on $\mathbb{N}$ that satisfy the two properties for all $n, m \in \mathbb{N}$.
    Let 
    \[G = \{x \in \mathbb{N} \mid n \cdot x = n \times x \text{ for all } n \in \mathbb{N}\}\]
    We will prove that $G = \mathbb{N}$, which will imply that $\cdot$ and $\times$ are 
        the same binary operation.
    It is clear that $G \subseteq \mathbb{N}$.
    By Part (a) applied to each of $\cdot$ and $\times$ we see that 
        $n \cdot 1 = n = n \times 1$ for all $n \in \mathbb{N}$
        and hence $1 \in G$.
    Now let $q \in G$. Let $n \in \mathbb{N}$.
    Then $n \cdot q = n \times q$ by hypothesis on $q$.
    It then follows from Part (b) that $n \cdot s(q) = (n \cdot q) + n = (n \times q) + n = n \times s(q)$.
    Hence $s(q) \in G$.
    By Part (c) of the Peano Postulates we conclude that $G = \mathbb{N}$.
\end{proof}

\begin{proof}
    We must show the two properties hold.
    Now, $n \cdot 1 = g_n(1) = n$, which is Part (a),
    and $n \cdot s(m) 
        = g_n(s(m)) 
        = (g_n \circ s)(m) 
        = (h_n \circ g_n)(m)
        = g_n(m) + n
        = (n \cdot m) + n$, which is Part (b).
\end{proof}

\begin{tcolorbox}[title=Problem 2, breakable]
    Prove Theorem $1.2.7$ ($2$) ($3$) ($4$) ($7$) ($8$) ($9$) ($10$) ($11$) ($13$).
\end{tcolorbox}

\begin{proof}
    Let $a, b, c \in \mathbb{N}$.
    We must show $(a + b) + c = a + (b + c)$.
\end{proof}

\begin{proof}
    Let $a, b, c \in \mathbb{N}$.
    We must show $1 + a = s(a) = a + 1$.
\end{proof}

\begin{proof}
    Let $a, b, c \in \mathbb{N}$.
    We must show $a + b = b + a$.
\end{proof}

\begin{proof}
    Let $a, b, c \in \mathbb{N}$.
    We must show $a \cdot 1 = a = 1 \cdot a$.
\end{proof}

\begin{proof}
    Let $a, b, c \in \mathbb{N}$.
    We must show $(a + b)c = ac + bc$.
\end{proof}

\begin{proof}
    Let $a, b, c \in \mathbb{N}$.
    We must show $ab = ba$.
\end{proof}

\begin{proof}
    Let $a, b, c \in \mathbb{N}$.
    We must show $c(a + b) = ca + cb$.
\end{proof}

\begin{proof}
    Let $a, b, c \in \mathbb{N}$.
    We must show $(ab)c = a(bc)$.
\end{proof}

\begin{proof}
    Let $a, b, c \in \mathbb{N}$.
    We must show $ab = 1$ if and only if $a = 1 = b$.
\end{proof}

\begin{tcolorbox}[title=Problem 3, breakable]
\end{tcolorbox}

\begin{tcolorbox}[title=Problem 4, breakable]
\end{tcolorbox}

\begin{tcolorbox}[title=Problem 5, breakable]
\end{tcolorbox}

\begin{tcolorbox}[title=Problem 6, breakable]
\end{tcolorbox}

\begin{tcolorbox}[title=Problem 7, breakable]
\end{tcolorbox}

\begin{tcolorbox}[title=Problem 8, breakable]
\end{tcolorbox}