\subsection{Axioms for the Natural Numbers}

\begin{tcolorbox}[title=Problem 1, breakable]
    Fill in the missing details in the proof of Theorem $1.2.6$.
\end{tcolorbox}

\begin{proof}
    We must show the uniquness of the binary operation $\cdot : \mathbb{N} \times \mathbb{N} \rightarrow \mathbb{N}$
    that satisfies the following two properties for all $n, m \in \mathbb{N}$.
    \begin{enumerate}[label=\textbf{\alph*.}]
        \item $n \cdot 1 = n$.
        \item $n \cdot s(m) = (n \cdot m) + n$.
    \end{enumerate}
    Suppose there are two binary operations $\cdot$ and $\times$
        on $\mathbb{N}$ that satisfy the two properties for all $n, m \in \mathbb{N}$.
    Let 
    \[G = \{x \in \mathbb{N} \mid n \cdot x = n \times x \text{ for all } n \in \mathbb{N}\}\]
    We will prove that $G = \mathbb{N}$, which will imply that $\cdot$ and $\times$ are 
        the same binary operation.
    It is clear that $G \subseteq \mathbb{N}$.
    By part (a) applied to each of $\cdot$ and $\times$ we see that 
        $n \cdot 1 = n = n \times 1$ for all $n \in \mathbb{N}$
        and hence $1 \in G$.
    Now let $q \in G$. Let $n \in \mathbb{N}$.
    Then $n \cdot q = n \times q$ by hypothesis on $q$.
    It then follows from part (b) that $n \cdot s(q) = (n \cdot q) + n = (n \times q) + n = n \times s(q)$.
    Hence $s(q) \in G$.
    By part (c) of the Peano Postulates we conclude that $G = \mathbb{N}$.
\end{proof}

\begin{proof}
    We must show the two properties hold.
    Now, $n \cdot 1 = g_n(1) = n$, which is part (a),
    and $n \cdot s(m) 
        = g_n(s(m)) 
        = (g_n \circ s)(m) 
        = (h_n \circ g_n)(m)
        = g_n(m) + n
        = (n \cdot m) + n$, which is part (b).
\end{proof}

\begin{tcolorbox}[title=Problem 2, breakable]
    Prove Theorem $1.2.7$ ($2$) ($3$) ($4$) ($7$) ($8$) ($9$) ($10$) ($11$) ($13$).
\end{tcolorbox}

\begin{proof}
    Let $a, b, c \in \mathbb{N}$.
    We must show $(a + b) + c = a + (b + c)$.
    Consider the set 
    \[G = \{z \in \mathbb{N} \mid \text{if } x,y \in \mathbb{N} \text{ then } (x + y) + z = x + (y + z)\}\]
    We will show $G = \mathbb{N}$.
    Clearly $G \subseteq \mathbb{N}$.
    We first show $1 \in G$.
    Suppose $z \in G$.
    Consider 
    \[(x + y) + 1 = s(x + y) = x + s(y) = x + (y + 1)\]
    Thus $1 \in G$.
    Futher let $x, y, z \in \mathbb{N}$, and consider 
    \[(x + y) + s(z) = s((x + y) + z)\]
    By our hypothesis on $z$, $(x + y) + z = x + (y + z)$ so
    \[s((x + y) + z) = s(x + (y + z)) = x + s(y + z) = x + (y + s(z))\]
    So $s(z) \in G$. Thus $G = \mathbb{N}$ by part (c) of the Peano Postulates.
\end{proof}

\begin{proof}
    Let $a \in \mathbb{N}$.
    We must show $1 + a = s(a) = a + 1$.
    Let $x \in \mathbb{N}$.
    Suppose $x = 1$. Then 
    \[1 + x = 1 + 1 = s(1) = x + 1\]
    Suppose $x > 1$.
    By Lemma $1.2.3$ there exists $y \in \mathbb{N}$ such that $s(y) = x$.
    First note $y + (1 + 1) = (y + 1) + 1$ by Theorem $1.2.7$ part ($2$).
    Then
    \[1 + s(y) = 1 + (y + 1) = s(y + 1) = y + s(1) = y + (1 + 1) = (y + 1) + 1 = s(y) + 1\]
    Thus $1 + x = x + 1$.
\end{proof}

\begin{proof}
    Let $a, b \in \mathbb{N}$.
    We must show $a + b = b + a$.
    Consider the set 
    \[G = \{x \in \mathbb{N} \mid \text{if } y \in \mathbb{N} \text{ then } x + y = y + x\}\]
    We will show $G = \mathbb{N}$.
    Clearly $G \subseteq \mathbb{N}$.
    We first show $1 \in G$.
    Let $x \in \mathbb{N}$.
    By Theorem $1.2.7$ part ($3$), $1 + x = x + 1$.
    Thus $1 \in G$.
    Now suppose $x \in G$.
    Let $y \in \mathbb{N}$.
    First note by Theorem $1.2.7$ part ($2$), $1 + (x + y) = (1 + x) + y$.
    Consider
    \[y + s(x) = s(y + x) = s(x + y) \text{ hypothesis on $x$} = 1 + (x + y) = (1 + x) + y = s(x) + y\]
    So $s(x) \in G$. Thus $G = \mathbb{N}$ by part (c) of the Peano Postulates.
\end{proof}

\begin{proof}
    Let $x, y \in \mathbb{N}$.
    We must show $a \cdot 1  = 1 \cdot a$.
\end{proof}

\begin{proof}
    STILL NEED TO DO THIS!!!
    Let $a \in \mathbb{N}$.
    We must show $a \cdot 1 = a = 1 \cdot a$.
    Consider the set 
    \[G = \{x \in \mathbb{N} \mid x \cdot 1 = x = 1 \cdot x\}\]
    We will show $G = \mathbb{N}$.
    Clearly $G \subseteq \mathbb{N}$.
    We first show $1 \in G$.
    Consider $x \cdot 1 = 1 \cdot 1 = x \cdot 1 = x$ by Theorem $1.2.6$ part (a).
    Suppose $x \in \mathbb{N}$ and assume $x \in G$.
\end{proof}

\begin{proof}
    Let $a, b, c \in \mathbb{N}$.
    We must show $(a + b)c = ac + bc$.
\end{proof}

\begin{proof}
    Let $a, b, c \in \mathbb{N}$.
    We must show $ab = ba$.
\end{proof}

\begin{proof}
    Let $a, b, c \in \mathbb{N}$.
    We must show $c(a + b) = ca + cb$.
\end{proof}

\begin{proof}
    Let $a, b, c \in \mathbb{N}$.
    We must show $(ab)c = a(bc)$.
\end{proof}

\begin{proof}
    Let $a, b, c \in \mathbb{N}$.
    We must show $ab = 1$ if and only if $a = 1 = b$.
\end{proof}

\begin{tcolorbox}[title=Problem 3, breakable]
    Let $a, b \in \mathbb{N}$.
    Suppose $a < b$. 
    Prove that there is a unique $p \in \mathbb{N}$
        such that $a + p = b$
\end{tcolorbox}

\begin{proof}
    We first prove uniqueness.
    Let $a, b \in \mathbb{N}$ such that $a < b$.
    Suppose $x, y \in \mathbb{N}$
        such that $a + x = b$ and $a + y = b$.
    Then $a + x = a + y$.
    By Theorem $1.2.7$ part ($4$),
        $x + a = y + a$.
    Then by Theorem $1.2.7$ part ($1$), $x = y$.

    We now prove existence.
    Since $a < b$ it follows that $a + 1 < b$ by Theorem $1.2.9$ part ($11$).
\end{proof}

\begin{tcolorbox}[title=Problem 4, breakable]
    Prove Theorem $1.2.9$ ($1$) ($3$) ($4$) ($5$) ($11$).
\end{tcolorbox}

\begin{proof}
    Let $a \in \mathbb{N}$.
    We must show $a \le a$, and $a \not < a$, and $a < a + 1$.

    To show $a \le a$ consider $a = a$ thus $a \le a$.
    To show $a \not < a$, first, suppose $a < a$.
    By definition of $<$, there exists $p \in \mathbb{N}$ such that $a + p = a$
        contradicting Theorem $1.2.7$ part ($6$).
    To show $a < a + 1$ consider $s(a) = a + 1 = a + 1$ thus $a < a + 1$.
\end{proof}

\begin{proof}
    Let $a, b, c \in \mathbb{N}$.
    We must show if $a < b$ and $b < c$, then $a < c$;
        if $a \le b$ and $b < c$, then $a < c$;
        if $a < b$ and $b \le c$, then $a < c$;
        if $a \le b$ and $b \le c$, then $a \le c$.

    \textcircled{1} 
    Suppose $a < b$ and $b < c$.
    By definition of $<$, there exists $p_1, p_2 \in \mathbb{N}$
        such that $a + p_1 = b$ and $b + p_2 = c$.
    Then $b + p_2 = (a + p_1) + p_2 = c$.
    By definition of $<$, $a < c$.

    \textcircled{2}
    Suppose $a \le b$ and $b < c$.
    By definition of $\le$,
        either $a = b$ or $a < b$.
    Suppose $a < b$. By \textcircled{1}, $a < c$.
    Suppose $a = b$.
    By definition of $<$, there exists $p \in \mathbb{N}$
        such that $b + p = c$.
    Then $b + p = a + p = c$.
    By definition of $<$, $a < c$.

    \textcircled{3}
    Suppose $a < b$ and $b \le c$.
    By definition of $\le$,
        either $b = c$ or $b < c$.
    Suppose $b < c$. By \textcircled{1}, $a < c$.
    Suppose $b = c$.
    By definition of $<$, there exists $p \in \mathbb{N}$
        such that $a + p = b$.
    Then $b = a + p = c$ thus, by definition of $<$, $a < c$.

    Suppose $a \le b$ and $b \le c$.
    There are four cases:
    \begin{enumerate}
        \item Suppose $a < b$ and $b < c$. By \textcircled{1}, $a < c$.
        \item Suppose $a \le b$ and $b < c$. By \textcircled{2}, $a < c$.
        \item Suppose $a < b$ and $b \le c$. By \textcircled{3}, $a < c$.
        \item Suppose $a \le b$ and $b \le c$. 
              There are four cases:
              \begin{enumerate}
                \item Suppose $a = b$ and $b < c$.
                      By definition of $<$, there exists $p \in \mathbb{N}$
                      such that $b + p = c$.
                      Then $b + p = a + p = c$ so $a < c$.
                \item Suppose $a < b$ and $b < c$. 
                      By \textcircled{1}, $a < c$.
                \item Suppose $a = b$ and $b = c$. 
                      Clearly $a = b = c$ thus $a = c$.
                \item Suppose $a < b$ and $b = c$. 
                    By definition of $<$, there exists $p \in \mathbb{N}$
                        such that $a + p = b$.
                      Then $a + p = b = c$ so $a < c$.
              \end{enumerate}
    \end{enumerate}
    Thus either $a < c$ or $a = c$ thus, by definition of $\le$, $a \le c$.
\end{proof}

\begin{proof}
    Let $a, b, c \in \mathbb{N}$.
    We must show if $a < b$ if and only if $a + c < b + c$.
    
    Suppose $a < b$. By definition of $<$, 
        there exists $p \in \mathbb{N}$ such that $a + p = b$.
    By Theorem $1.2.7$ part ($1$), $(a + p) + c = b + c$.
    By Theorem $1.2.7$ part ($2$), $a + (p + c) = b + c$.
    By Theorem $1.2.7$ part ($4$), $a + (c + p) = b + c$.
    By Theorem $1.2.7$ part ($2$), $(a + c) + p = b + c$.
    Thus by definition of $<$, $a + c < b + c$.

    Suppose $a + c < b + c$. There exists $p \in \mathbb{N}$
        such that $(a + c) + p = b + c$.
    By Theorem $1.2.7$ part ($4$), $p + (a + c) = b + c$.
    By Theorem $1.2.7$ part ($2$), $(p + a) + c = b + c$.
    By Theorem $1.2.7$ part ($1$), $p + a = b$ so,
        by Theorem $1.2.7$ part ($4$),  $a + p = b$.
    Thus by definition of $<$, $a < b$.
\end{proof}

\begin{proof}
    Let $a, b, c \in \mathbb{N}$.
    We must show $a < b$ if and only if $ac < bc$.
    
    Suppose $a < b$.
    Now suppose $ac > bc$.
    By definition of $<$, there exists $p_1, p_2 \in \mathbb{N}$
        such that $a + p_1 = b$ and $bc + p_2 = ac$.
    Then, by Theorem $1.2.8$ part ($8$), $(a + p_1)c + p_2 = ac + p_1 c  + p_2 = ac$.
    By definition of $<$, $ac < ac$ contradicting
        Theorem $1.2.9$ part ($1$).

    Suppose $ac < bc$.
    Now suppose $a > b$. 
    By definition of $<$, there exists $p$ such that $b + p = a$.
    Then, by Theorem $1.2.8$ part ($8$), $(b + p)c = bc$.
    So $bc + pc = bc$.
    From this we deduce that $bc < bc$ contradicting 
        Theorem $1.2.9$ part ($1$).
\end{proof}

\begin{proof}
    Let $a, b \in \mathbb{N}$.
    We must show $a < b$ if and only if $a + 1 \le b$.

    Suppose $a < b$.
    Now suppose $a + 1 > b$.
    By definition of $<$, there exists $p_1, p_2 \in \mathbb{N}$ 
        such that $b + p_1 = a + 1$ and $a + p_2 = b$.
    Then $b + p_1 = (a + p_2) + p_1 = a + 1$.
    By Theorem $1.2.7$ part ($4$), $p_1 + (a + p_2) = 1 + a$.
    By Theorem $1.2.7$ part ($4$), $p_1 + (p_2 + a) = 1 + a$.
    By Theorem $1.2.7$ part ($2$), $(p_1 + p_2) + a = 1 + a$.
    By Theorem $1.2.7$ part ($1$), $p_1 + p_2 = 1$
        contradicting Theorem $1.2.7$ part ($5$).

    Suppose $a + 1 \le b$.
    By definition of $\le$, either $a + 1 = b$ or $a + 1 < b$.

    Suppose $a + 1 < b$.
    Now suppose $a > b$.
    By definition of $<$, there exists $p_1, p_2 \in \mathbb{N}$ 
        such that $(a + 1) + p_1 = b$ and $b + p_2 = a$.
    Then $(a + 1) + p_1 = ((b + p_2) + 1) + p_1 = b$.
    By definition of $<$, $b < b$ contradicting
        Theorem $1.2.9$ part ($1$).

    Suppose $a + 1 = b$.
    Now suppose $a > b$.
    By definition of $<$, there exists $p \in \mathbb{N}$ 
        such that $b + p = a$.
    Then $a + 1 = b + p + 1 = b$.
    By definition of $<$, $b < b$ contradicting
        Theorem $1.2.8$ part ($6$).
\end{proof}

\begin{tcolorbox}[title=Problem 5, breakable]
    Let $a, b \in \mathbb{N}$. 
    Prove that if $a + a = b + b$, then $a = b$.
\end{tcolorbox}

\begin{proof}
    Suppose $a + a = b + b$.
    First, by Theorem $1.2.6$ part (a),
         $a + a = a \cdot 1 + a \cdot 1$.
    Then, by Theorem $1.2.7$ part ($10$),
        $a \cdot 1 + a \cdot 1 = a(1 + 1) = a\cdot 2$.
    Similarly $b + b = b\cdot 2$.
    Then, by Theorem $1.2.7$ part ($12$),
        since $a\cdot 2 = b\cdot 2$, $a = b$.
\end{proof}

\begin{tcolorbox}[title=Problem 6, breakable]
    Let $b \in \mathbb{N}$. Prove that 
    \[\{n \in \mathbb{N} \mid 1 \le n \le b\} \cup \{n \in \mathbb{N} \mid b + 1 \le n\} = \mathbb{N}\]
    \[\{n \in \mathbb{N} \mid 1 \le n \le b\} \cap \{n \in \mathbb{N} \mid b + 1 \le n\} = \emptyset\]
\end{tcolorbox}

\begin{proof}
    Let $A = \{n \in \mathbb{N} \mid 1 \le n \le b\}$ and $B = \{n \in \mathbb{N} \mid b + 1 \le n\}$.
    It is clear that $A \subseteq \mathbb{N}$ and $B \subseteq \mathbb{N}$.
    Thus $A \cup B \subseteq \mathbb{N}$.
    Now let $x$ be an arbitrary element in $\mathbb{N}$.
    By Theorem $1.2.9$ part ($6$), either $x < b$, $x = b$, or $x > b$.
    Suppose $x < b$. Then $x \in A$, so $x \in A \cup B$.
    Suppose $x = b$. Then $x \in A$, so $x \in A \cup B$.
    Suppose $x > b$. Then $x \in B$, so $x \in A \cup B$.
    Therefore $\mathbb{N} \subseteq A \cup B$.
    It follows that $A \cup B = \mathbb{N}$.
\end{proof}

\begin{proof}
    Suppose $\{n \in \mathbb{N} \mid 1 \le n \le b\} \cap \{n \in \mathbb{N} \mid b + 1 \le n\} \ne \emptyset$.
    Let $x \in \{n \in \mathbb{N} \mid 1 \le n \le b\} \cap \{n \in \mathbb{N} \mid b + 1 \le n\}$.
    Then $1 \le x \le b$ and $b + 1 \le x$.
    By Theorem $1.2.9$ part ($3$), $b + 1 \le x \le b$
        contradicting Theorem $1.2.9$ part ($9$).
\end{proof}

\begin{tcolorbox}[title=Problem 7, breakable]
    Let $A \subseteq N$ be a set.
    The set $A$ is \textbf{closed} if $a \in A$ implies $a + 1 \in A$.
    Suppose $A$ is closed.
    \begin{enumerate}
        \item Prove that if $a \in A$ and $n \in \mathbb{N}$, then $a + n \in A$.
        \item Prove that if $a \in A$, then $\{x \in \mathbb{N} \mid x \ge a\} \subseteq A$.
    \end{enumerate}
\end{tcolorbox}

\begin{proof}
    If $A = \emptyset$ then clearly the implication vacuously holds.
    Suppose $A \ne \emptyset$.
    Consider the set
    \[G = \{x \in \mathbb{N} \mid a + x \in A\}.\]
    We will show $G = \mathbb{N}$, proving our implication.
    Now, since $a \in A$ and $A$ is closed, $a + 1 \in A$, thus $1 \in G$.
    Suppose $x \in \mathbb{N}$ and $x \in G$.
    Then consider $a + s(x) = a + (x + 1)$.
    By Theorem $1.2.7$ part ($2$), $a + (x + 1) = (a + x) + 1$.
    By our hypothesis, $a + x \in A$.
    But since $A$ is closed, $(a + x) + 1 \in A$.
    Thus $s(x) \in G$.
    By the part (c) of the Peano Postulates, we conclude that $G = \mathbb{N}$.
\end{proof}

\begin{proof}
    Suppose $a \in A$. Let $x \in \mathbb{N}$ such that $x \ge a$.
    Either $x = a$ or $a < x$.
    Suppose $x = a$, then trivially $x = a \in A$.
    Suppose $a < x$.
    By definition of $<$, there exists $p \in \mathbb{N}$ 
        such that $a + p = x$.
    By the previous proof, $a + p = x \in A$.
\end{proof}

\begin{tcolorbox}[title=Problem 8, breakable]
    Suppose that the set $\mathbb{N}$ together with the element 
        $1 \in mathbb{N}$ and the function $s : \mathbb{N} \rightarrow \mathbb{N}$,
        and the set $\mathbb{N}'$ together with the element $1' \in \mathbb{N}$
        and the function $s' : \mathbb{N}' \rightarrow \mathbb{N}'$, both satisfy 
        the Peano Postulates. Prove that there is a bijective function 
        $f : \mathbb{N} \rightarrow \mathbb{N}'$ such that $f(1) = 1'$
        and $f \circ s = s' \circ f$. The existence of such a bijective function 
\end{tcolorbox}

\begin{tcolorbox}[title=Extra Problem, breakable]
    Show the Peano axioms are independent. 
    That is, for any two Peano axioms, 
        find a structure that satisfies them but not the third. 
    You may assume the regular math of $\mathbb{Z}$, $\mathbb{Q}$, $\mathbb{R}$.
\end{tcolorbox}