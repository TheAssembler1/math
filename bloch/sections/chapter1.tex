\subsection{Axioms for the Natural Numbers}

\begin{tcolorbox}[title=Problem 1, breakable]
    Fill in the missing details in the proof of Theorem $1.2.6$.
\end{tcolorbox}

\begin{proof}
    We must show the uniquness of the binary operation $\cdot : \mathbb{N} \times \mathbb{N} \rightarrow \mathbb{N}$
    that satisfies the following two properties for all $n, m \in \mathbb{N}$.
    \begin{enumerate}[label=\textbf{\alph*.}]
        \item $n \cdot 1 = n$.
        \item $n \cdot s(m) = (n \cdot m) + n$.
    \end{enumerate}
    Suppose there are two binary operations $\cdot$ and $\times$
        on $\mathbb{N}$ that satisfy the two properties for all $n, m \in \mathbb{N}$.
    Let 
    \[G = \{x \in \mathbb{N} \mid n \cdot x = n \times x \text{ for all } n \in \mathbb{N}\}\]
    We will prove that $G = \mathbb{N}$, which will imply that $\cdot$ and $\times$ are 
        the same binary operation.
    It is clear that $G \subseteq \mathbb{N}$.
    By part (a) applied to each of $\cdot$ and $\times$ we see that 
        $n \cdot 1 = n = n \times 1$ for all $n \in \mathbb{N}$
        and thus $1 \in G$.
    Now let $q \in G$. Let $n \in \mathbb{N}$.
    Then $n \cdot q = n \times q$ by hypothesis on $q$.
    It then follows from part (b) that $n \cdot s(q) = (n \cdot q) + n = (n \times q) + n = n \times s(q)$.
    thus $s(q) \in G$.
    By part (c) of the Peano Postulates we conclude that $G = \mathbb{N}$.
\end{proof}

\begin{proof}
    We must show the two properties hold.
    Now, $n \cdot 1 = g_n(1) = n$, which is part (a),
    and $n \cdot s(m) 
        = g_n(s(m)) 
        = (g_n \circ s)(m) 
        = (h_n \circ g_n)(m)
        = g_n(m) + n
        = (n \cdot m) + n$, which is part (b).
\end{proof}

\begin{tcolorbox}[title=Problem 2, breakable]
    Prove Theorem $1.2.7$ ($2$) ($3$) ($4$) ($7$) ($8$) ($9$) ($10$) ($11$) ($13$).
\end{tcolorbox}

\begin{proof}
    Let $a, b, c \in \mathbb{N}$.
    We must show $(a + b) + c = a + (b + c)$.
    Consider the set 
    \[G = \{z \in \mathbb{N} \mid \text{if } x,y \in \mathbb{N} \text{ then } (x + y) + z = x + (y + z)\}\]
    We will show $G = \mathbb{N}$.
    Clearly $G \subseteq \mathbb{N}$.
    We first show $1 \in G$.
    Suppose $z \in G$.
    Consider 
    \[(x + y) + 1 = s(x + y) = x + s(y) = x + (y + 1)\]
    Thus $1 \in G$.
    Futher let $x, y, z \in \mathbb{N}$, and consider 
    \[(x + y) + s(z) = s((x + y) + z)\]
    By our hypothesis on $z$, $(x + y) + z = x + (y + z)$ so
    \[s((x + y) + z) = s(x + (y + z)) = x + s(y + z) = x + (y + s(z))\]
    So $s(z) \in G$. Thus $G = \mathbb{N}$ by part (c) of the Peano Postulates.
\end{proof}

\begin{proof}
    Let $a \in \mathbb{N}$.
    We must show $1 + a = s(a) = a + 1$.
    Consider the set 
    \[G = \{a \in \mathbb{N} \mid 1 + a = s(a) = a + 1\}\]
    We will show $G = \mathbb{N}$.
    Clearly $G \subseteq \mathbb{N}$.
    We first show $1 \in G$.
    Let $a \in \mathbb{N}$ such that $a = 1$.
    \[1 + a = s(a) = s(1) = 1 + 1 = a + 1\]
    Thus $1 \in G$.
    Suppose $x \in \mathbb{N}$ and $x \in G$.
    By our hypothesis, $1 + x = x + 1$.
    Then 
    \[1 + s(x) = s(1 + x) = s(x + 1) = s(x) + 1\]
    So $s(x) \in G$. Thus $G = \mathbb{N}$ by part (c) of the Peano Postulates.
\end{proof}

\begin{proof}
    Let $a, b \in \mathbb{N}$.
    We must show $a + b = b + a$.
    Consider the set 
    \[G = \{x \in \mathbb{N} \mid \text{if } y \in \mathbb{N} \text{ then } x + y = y + x\}\]
    We will show $G = \mathbb{N}$.
    Clearly $G \subseteq \mathbb{N}$.
    We first show $1 \in G$.
    Let $x \in \mathbb{N}$.
    By Theorem $1.2.7$ part ($3$), $1 + x = x + 1$.
    Thus $1 \in G$.
    Now suppose $x \in G$.
    Let $y \in \mathbb{N}$.
    First note by Theorem $1.2.7$ part ($2$), $1 + (x + y) = (1 + x) + y$.
    Consider
    \[y + s(x) = s(y + x) = s(x + y) \text{ hypothesis on $x$} = 1 + (x + y) = (1 + x) + y = s(x) + y\]
    So $s(x) \in G$. Thus $G = \mathbb{N}$ by part (c) of the Peano Postulates.
\end{proof}

\begin{proof}
    Let $a \in \mathbb{N}$.
    We must show $a \cdot 1 = a = 1 \cdot a$.
    Consider the set 
    \[G = \{x \in \mathbb{N} \mid x \cdot 1 = x = 1 \cdot x\}\]
    We will show $G = \mathbb{N}$.
    Clearly $G \subseteq \mathbb{N}$.
    We first show $1 \in G$. Consider
    \[
    \begin{aligned}
    x \cdot 1 
        &= x && \text{Theorem 1.2.6 part (a)} \\
        &= 1 \\
        &= 1 \cdot 1 \\
        &= x \cdot 1
    \end{aligned}
    \]
    Thus $1 \in G$.
    Consider 
    \[
    \begin{aligned}
    s(x) \cdot 1 
        &= s(x) && \text{Theorem 1.2.6 part (a)} \\
        &= x + 1 && \text{Theorem 1.2.5 part (a)} \\
        &= x \cdot 1 + 1 && \text{Theorem 1.2.6 part (a)} \\
        &= 1 \cdot x + 1 && \text{Induction hypothesis} \\
        &= 1 \cdot s(x) && \text{Theorem 1.2.6 part (b)}
    \end{aligned}
    \]
    So $s(x) \in G$. Thus $G = \mathbb{N}$ by part (c) of the Peano Postulates.
\end{proof}

\begin{proof}
    Let $a, b, c \in \mathbb{N}$.
    We must show $(a + b)c = ac + bc$.
    Consider the set 
    \[G = \{c \in \mathbb{N} \mid \text{if } a, b \in \mathbb{N} \text{ then } (a + b)c = ac + bc\}\]
    We will show $G = \mathbb{N}$.
    Clearly $G \subseteq \mathbb{N}$.
    We first show $1 \in G$. Let $a, b \in \mathbb{N}$. Then
    \begin{align*}
        (a + b)1 
            &= a + b \quad &&\text{(Theorem $1.2.6$ part (a))} \\
            &= a \cdot 1 + b \cdot 1 \quad &&\text{(Theorem $1.2.6$ part (a))}
    \end{align*}
    Suppose $a, b, c \in \mathbb{N}$ and $c \in G$.
    Then
    \begin{align*}
        (a + b) \cdot s(c) 
            &= ((a + b) c) + (a + b) \quad &&\text{(Theorem 1.2.6 part (a))} \\
            &= (ac + b c + a + b) \quad &&\text{(Induction Hypothesis)} \\
            &= (ac + a + bc + b) \quad &&\text{(Theorem 1.2.7 part (4))} \\
            &= a \cdot s(c) + b \cdot s(c) \quad &&\text{(Theorem 1.2.5 part (a))}
    \end{align*}
    So $s(c) \in G$. Thus $G = \mathbb{N}$ by part (c) of the Peano Postulates.
\end{proof}

\begin{proof}
    Let $a, b \in \mathbb{N}$.
    We must show $ab = ba$.
        Consider the set 
    \[G = \{a \in \mathbb{N} \mid \text{if } b \in \mathbb{N} \text{ then } ab = ba\}\]
    We will show $G = \mathbb{N}$.
    Clearly $G \subseteq \mathbb{N}$.
    We first show $1 \in G$. By Theorem $1.2.7$ part ($7$), $a \cdot 1 = 1 \cdot a$.
    Thus $1 \in G$.
    Suppose $a, b \in \mathbb{N}$ and $a \in G$.
    \begin{align*}
        s(a) \cdot b 
            &= (a + 1)b \quad &&\text{(Theorem $1.2.5$ part (a))} \\
            &= ab + 1 b \quad &&\text{(Theorem $1.2.7$ part ($8$))} \\
            &= ab + b 1 \quad &&\text{(Theorem $1.2.7$ part ($7$))} \\
            &= ab + b \quad &&\text{(Theorem $1.2.6$ part ($7$))} \\
            &= ba + b \quad &&\text{(Induction Hypothesis)} \\
            &= b \cdot s(a) \quad &&\text{(Theorem $1.2.6$ part (b))} 
    \end{align*}
    So $s(a) \in G$. Thus $G = \mathbb{N}$ by part (c) of the Peano Postulates.
\end{proof}

\begin{proof}
    Let $a, b \in \mathbb{N}$.
    We must show $c(a + b) = ca + cb$.
    By Theorem $1.2.7$ part ($9$), $c(a + b) = (a + b)c$.
    By Theorem $1.2.7$ part ($8$), $(a + b)c = ac + bc$.
     By Theorem $1.2.7$ part ($9$), $ac + bc = ca + cb$.
\end{proof}

\begin{proof}
    Let $a, b, c \in \mathbb{N}$.
    We must show $(ab)c = a(bc)$.
\end{proof}

\begin{proof}
    Let $a, b, c \in \mathbb{N}$.
    We must show $(ab)c = a(bc)$.
    Consider the set 
    \[G = \{c \in \mathbb{N} \mid \text{if } a, b \in \mathbb{N} \text{ then } (ab)c = a(bc)\}\]
    We will show $G = \mathbb{N}$.
    Clearly $G \subseteq \mathbb{N}$.
    We first show $1 \in G$. Let $a, b \in \mathbb{N}$. Then
    \[(ab)1 = ab \text{ (Theorem $1.2.7$ part ($7$)) } = a (b \cdot 1) \text{ (Theorem $1.2.6$ part (a)) }\]
    Thus $1 \in G$.
    Suppose $a, b, c \in \mathbb{N}$ and $c \in G$.
    Then
    \begin{align*}
        (ab) \cdot s(c)
            &= (ab)(c + 1) \quad &&\text{(Theorem 1.2.5 part (a))} \\
            &= (ab)c + (ab) 1 \quad &&\text{(Theorem 1.2.7 part ($10$))} \\
            &= a(bc) + (ab) 1 \quad &&\text{(Induction Hypothesis)} \\
            &= a(bc) + ab \quad &&\text{(Theorem 1.2.7 part ($7$))} \\
            &= a(bc + b) \quad &&\text{(Theorem 1.2.7 part ($8$))} \\
            &= a(bc + b \cdot 1) \quad &&\text{(Theorem 1.2.7 part ($7$))} \\
            &= a(b(c + 1)) \quad &&\text{(Theorem 1.2.7 part ($8$))} \\
            &= a(b \cdot s(c)) \quad &&\text{(Theorem 1.2.5 part (a))}
    \end{align*}
    So $s(c) \in G$. Thus $G = \mathbb{N}$ by part (c) of the Peano Postulates.
\end{proof}

\begin{proof}
    Let $a, b \in \mathbb{N}$.
    We must show $ab = 1$ if and only if $a = 1 = b$.

    Suppose $ab = 1$. For contradiction, suppose $a \ne 1 $ or $b \ne 1$.
    Suppose $a \ne 1$.
    By Lemma $1.2.3$ there exists $c \in \mathbb{N}$ such that $s(c) = a$.
    Then
    \[ab = s(c) b = (c + 1)b \text{ (Theorem $1.2.5$ part (a)) } = c b + b \text{ (Theorem $1.2.7$ part ($8$)) } = 1\]
    Contradicting Theorem $1.2.7$ part ($5$).
    Suppose $b \ne 1$.
    By Lemma $1.2.3$ there exists $c \in \mathbb{N}$ such that $s(c) = b$.
    Then
    \[ab = a \cdot s(c) = a(c + 1) \text{ (Theorem $1.2.5$ part (a)) } = ac + a \text{ (Theorem $1.2.7$ part ($10$)) } = 1\]
    Contradicting Theorem $1.2.7$ part ($5$).

    Suppose $a = 1 = b$.
    Then $a b = a \cdot 1 = a = 1$ by Theorem $1.2.6$ part (a).
\end{proof}

\begin{tcolorbox}[title=Problem 3, breakable]
    Let $a, b \in \mathbb{N}$.
    Suppose $a < b$. 
    Prove that there is a unique $p \in \mathbb{N}$
        such that $a + p = b$
\end{tcolorbox}

\begin{proof}
    We first prove uniqueness.
    Let $a, b \in \mathbb{N}$ such that $a < b$.
    Suppose $x, y \in \mathbb{N}$
        such that $a + x = b$ and $a + y = b$.
    Then $a + x = a + y$.
    By Theorem $1.2.7$ part ($4$),
        $x + a = y + a$.
    Then by Theorem $1.2.7$ part ($1$), $x = y$.

    We now prove existence.
    Since $a < b$, by definition of $<$ 
        there exists $p \in \mathbb{N}$
        such that $a + p = b$.
\end{proof}

\begin{tcolorbox}[title=Problem 4, breakable]
    Prove Theorem $1.2.9$ ($1$) ($3$) ($4$) ($5$) ($11$).
\end{tcolorbox}

\begin{proof}
    Let $a \in \mathbb{N}$.
    We must show $a \le a$, and $a \not < a$, and $a < a + 1$.

    To show $a \le a$, suppose for contradiction $a = a$. Thus $a \le a$.
    To show $a \not < a$, first, suppose $a < a$.
    By definition of $<$, there exists $p \in \mathbb{N}$ such that $a + p = a$
        contradicting Theorem $1.2.7$ part ($6$).
    To show $a < a + 1$ consider $s(a) = a + 1 = a + 1$ thus $a < a + 1$.
\end{proof}

\begin{proof}
    Let $a, b, c \in \mathbb{N}$.
    We must show if $a < b$ and $b < c$, then $a < c$;
        if $a \le b$ and $b < c$, then $a < c$;
        if $a < b$ and $b \le c$, then $a < c$;
        if $a \le b$ and $b \le c$, then $a \le c$.

    \textcircled{1} 
    Suppose $a < b$ and $b < c$.
    By definition of $<$, there exists $p_1, p_2 \in \mathbb{N}$
        such that $a + p_1 = b$ and $b + p_2 = c$.
    Then $b + p_2 = (a + p_1) + p_2 = c$.
    By definition of $<$, $a < c$.

    \textcircled{2}
    Suppose $a \le b$ and $b < c$.
    By definition of $\le$,
        either $a = b$ or $a < b$.
    Suppose $a < b$. By \textcircled{1}, $a < c$.
    Suppose $a = b$.
    By definition of $<$, there exists $p \in \mathbb{N}$
        such that $b + p = c$.
    Then $b + p = a + p = c$.
    By definition of $<$, $a < c$.

    \textcircled{3}
    Suppose $a < b$ and $b \le c$.
    By definition of $\le$,
        either $b = c$ or $b < c$.
    Suppose $b < c$. By \textcircled{1}, $a < c$.
    Suppose $b = c$.
    By definition of $<$, there exists $p \in \mathbb{N}$
        such that $a + p = b$.
    Then $b = a + p = c$ thus, by definition of $<$, $a < c$.

    Suppose $a \le b$ and $b \le c$.
    There are four cases:
    \begin{enumerate}
        \item Suppose $a < b$ and $b < c$. By \textcircled{1}, $a < c$.
        \item Suppose $a \le b$ and $b < c$. By \textcircled{2}, $a < c$.
        \item Suppose $a < b$ and $b \le c$. By \textcircled{3}, $a < c$.
        \item Suppose $a \le b$ and $b \le c$. 
              There are four cases:
              \begin{enumerate}
                \item Suppose $a = b$ and $b < c$.
                      By definition of $<$, there exists $p \in \mathbb{N}$
                      such that $b + p = c$.
                      Then $b + p = a + p = c$ so $a < c$.
                \item Suppose $a < b$ and $b < c$. 
                      By \textcircled{1}, $a < c$.
                \item Suppose $a = b$ and $b = c$. 
                      Clearly $a = b = c$ thus $a = c$.
                \item Suppose $a < b$ and $b = c$. 
                    By definition of $<$, there exists $p \in \mathbb{N}$
                        such that $a + p = b$.
                      Then $a + p = b = c$ so $a < c$.
              \end{enumerate}
    \end{enumerate}
    Thus either $a < c$ or $a = c$ thus, by definition of $\le$, $a \le c$.
\end{proof}

\begin{proof}
    Let $a, b, c \in \mathbb{N}$.
    We must show if $a < b$ if and only if $a + c < b + c$.
    
    Suppose $a < b$. By definition of $<$, 
        there exists $p \in \mathbb{N}$ such that $a + p = b$.
    By Theorem $1.2.7$ part ($1$), $(a + p) + c = b + c$.
    By Theorem $1.2.7$ part ($2$), $a + (p + c) = b + c$.
    By Theorem $1.2.7$ part ($4$), $a + (c + p) = b + c$.
    By Theorem $1.2.7$ part ($2$), $(a + c) + p = b + c$.
    Thus by definition of $<$, $a + c < b + c$.

    Suppose $a + c < b + c$. There exists $p \in \mathbb{N}$
        such that $(a + c) + p = b + c$.
    By Theorem $1.2.7$ part ($4$), $p + (a + c) = b + c$.
    By Theorem $1.2.7$ part ($2$), $(p + a) + c = b + c$.
    By Theorem $1.2.7$ part ($1$), $p + a = b$ so,
        by Theorem $1.2.7$ part ($4$),  $a + p = b$.
    Thus by definition of $<$, $a < b$.
\end{proof}
\begin{proof}
    Let $a, b, c \in \mathbb{N}$.
    We must show $a < b$ if and only if $ac < bc$.
    
    Suppose $a < b$.
    For contradiction, suppose $ac \ge bc$.
    By definition of $\ge$, either $ac = bc$ or $ac > bc$.

    Suppose $ac = bc$.
    By Theorem $1.2.7$ part ($12$), $a = b$.
    But $a = b < b$ contradicting
        Theorem $1.2.9$ part ($1$).

    Suppose $ac > bc$.
    By definition of $<$, there exists $p_1, p_2 \in \mathbb{N}$
        such that $a + p_1 = b$ and $bc + p_2 = ac$.
    Then $bc + p_2 = (a + p_1)c + p_2 = ac + p_1 c + p_2 \;(\text{by Theorem $1.2.8$ part (8) for distributivity}) = ac$.
    By definition of $<$, $ac < ac$ contradicting
        Theorem $1.2.9$ part ($1$).

    Suppose $ac < bc$.
    For contradiction, suppose $a \ge b$. 
    By definition of $\ge$, either $a = b$ or $a > b$ 

    Suppose $a = b$.
    Then $ac = bc < bc$ which contradicts Theorem $1.2.9$ part ($1$).

    Suppose $a > b$.
    By definition of $<$, there exists $p \in \mathbb{N}$ such that $b + p = a$.
    Then, by Theorem $1.2.8$ part ($8$), $ac = (b + p)c = bc + pc$.
    By definition of $<$, $bc < ac$.
\end{proof}

\begin{proof}
    Let $a, b \in \mathbb{N}$.
    We must show $a < b$ if and only if $a + 1 \le b$.

    Suppose $a < b$.
    For contradiction, suppose $a + 1 > b$.
    By definition of $<$, there exists $p_2 \in \mathbb{N}$ such that $a + p_2 = b$.
    Since $a + 1 > b$, there exists $p_1 \in \mathbb{N}$ such that $b + p_1 = a + 1$.
    Then $b + p_1 = (a + p_2) + p_1 = a + 1$.
    By Theorem $1.2.7$ part ($4$), $p_1 + (a + p_2) = 1 + a$.
    By Theorem $1.2.7$ part ($4$), $p_1 + (p_2 + a) = 1 + a$.
    By Theorem $1.2.7$ part ($2$), $(p_1 + p_2) + a = 1 + a$.
    By Theorem $1.2.7$ part ($1$), $p_1 + p_2 = 1$
        contradicting Theorem $1.2.7$ part ($5$).

    Suppose $a + 1 \le b$.
    By definition of $\le$, either $a + 1 = b$ or $a + 1 < b$.
    
    Suppose $a + 1 = b$.
    By definition of $<$, $a < b$.

    Suppose $a + 1 < b$.
    For contradiction, suppose $a \ge b$.
    By definition of $\ge$, either $a = b$ or $a > b$.
    Suppose $a = b$, then $a + 1 = b + 1 > b$
        contradicting Theorem $1.2.7$ part ($6$).
    Suppose $a > b$.
    By definition of $<$, there exists $p_1, p_2 \in \mathbb{N}$ 
        such that $(a + 1) + p_1 = b$ and $b + p_2 = a$.
    Then $(a + 1) + p_1 = ((b + p_2) + 1) + p_1 = b$.
    By definition of $<$, $b < b$ contradicting
        Theorem $1.2.9$ part ($1$).
\end{proof}

\begin{tcolorbox}[title=Problem 5, breakable]
    Let $a, b \in \mathbb{N}$. 
    Prove that if $a + a = b + b$, then $a = b$.
\end{tcolorbox}

\begin{proof}
    Suppose $a + a = b + b$.
    First, by Theorem $1.2.6$ part (a),
         $a + a = a \cdot 1 + a \cdot 1$.
    Then, by Theorem $1.2.7$ part ($10$),
        $a \cdot 1 + a \cdot 1 = a(1 + 1) = a\cdot 2$.
    Similarly $b + b = b\cdot 2$.
    Then, by Theorem $1.2.7$ part ($12$),
        since $a\cdot 2 = b\cdot 2$, $a = b$.
\end{proof}

\begin{tcolorbox}[title=Problem 6, breakable]
    Let $b \in \mathbb{N}$. Prove that 
    \[\{n \in \mathbb{N} \mid 1 \le n \le b\} \cup \{n \in \mathbb{N} \mid b + 1 \le n\} = \mathbb{N}\]
    \[\{n \in \mathbb{N} \mid 1 \le n \le b\} \cap \{n \in \mathbb{N} \mid b + 1 \le n\} = \emptyset\]
\end{tcolorbox}

\begin{proof}
    Let $A = \{n \in \mathbb{N} \mid 1 \le n \le b\}$ and $B = \{n \in \mathbb{N} \mid b + 1 \le n\}$.
    It is clear that $A \subseteq \mathbb{N}$ and $B \subseteq \mathbb{N}$.
    Thus $A \cup B \subseteq \mathbb{N}$.
    Now let $x$ be an arbitrary element in $\mathbb{N}$.
    By Theorem $1.2.9$ part ($6$), either $x < b$, $x = b$, or $x > b$.
    Suppose $x < b$. Then $x \in A$, so $x \in A \cup B$.
    Suppose $x = b$. Then $x \in A$, so $x \in A \cup B$.
    Suppose $x > b$. Then $x \in B$, so $x \in A \cup B$.
    Therefore $\mathbb{N} \subseteq A \cup B$.
    It follows that $A \cup B = \mathbb{N}$.

    Suppose $A \cap B \ne \emptyset$.
    Let $x \in A \cap B$.
    Then $1 \le x \le b$ and $b + 1 \le x$.
    By Theorem $1.2.9$ part ($3$), $b + 1 \le x \le b$
        contradicting Theorem $1.2.9$ part ($9$).
\end{proof}

\begin{tcolorbox}[title=Problem 7, breakable]
    Let $A \subseteq N$ be a set.
    The set $A$ is \textbf{closed} if $a \in A$ implies $a + 1 \in A$.
    Suppose $A$ is closed.
    \begin{enumerate}
        \item Prove that if $a \in A$ and $n \in \mathbb{N}$, then $a + n \in A$.
        \item Prove that if $a \in A$, then $\{x \in \mathbb{N} \mid x \ge a\} \subseteq A$.
    \end{enumerate}
\end{tcolorbox}

\begin{proof}
    If $A = \emptyset$ then clearly the implication vacuously holds.
    Suppose $A \ne \emptyset$.
    Consider the set
    \[G = \{x \in \mathbb{N} \mid a + x \in A\}.\]
    We will show $G = \mathbb{N}$, proving our implication.
    Now, since $a \in A$ and $A$ is closed, $a + 1 \in A$, thus $1 \in G$.
    Suppose $x \in \mathbb{N}$ and $x \in G$.
    Then consider $a + s(x) = a + (x + 1)$.
    By Theorem $1.2.7$ part ($2$), $a + (x + 1) = (a + x) + 1$.
    By our hypothesis, $a + x \in A$.
    But since $A$ is closed, $(a + x) + 1 \in A$.
    Thus $s(x) \in G$.
    By the part (c) of the Peano Postulates, we conclude that $G = \mathbb{N}$.
\end{proof}

\begin{proof}
    Suppose $a \in A$. Let $x \in \mathbb{N}$ such that $x \ge a$.
    Either $x = a$ or $a < x$.
    Suppose $x = a$, then trivially $x = a \in A$.
    Suppose $a < x$.
    By definition of $<$, there exists $p \in \mathbb{N}$ 
        such that $a + p = x$.
    By the previous proof, $a + p = x \in A$.
\end{proof}

\begin{tcolorbox}[title=Problem 8, breakable]
    Suppose that the set $\mathbb{N}$ together with the element 
        $1 \in \mathbb{N}$ and the function $s : \mathbb{N} \rightarrow \mathbb{N}$,
        and the set $\mathbb{N}'$ together with the element $1' \in \mathbb{N}$
        and the function $s' : \mathbb{N}' \rightarrow \mathbb{N}'$, both satisfy 
        the Peano Postulates. Prove that there is a bijective function 
        $f : \mathbb{N} \rightarrow \mathbb{N}'$ such that $f(1) = 1'$
        and $f \circ s = s' \circ f$. The existence of such a bijective function.
\end{tcolorbox}

\begin{proof}
    We can apply Theorem $1.2.4$
        to the set $\mathbb{N}'$, the element $1'$
        and the function  $s' : \mathbb{N}' \rightarrow \mathbb{N}'$,
        to deduce that there is a unique function $f : \mathbb{N} \rightarrow \mathbb{N}'$ such that 
        $f \circ s = s' \circ f$ and $f(1) = 1'$.

    We can apply Theorem $1.2.4$ again,
        to the set $\mathbb{N}$, the element $1$
        and the function  $s : \mathbb{N} \rightarrow \mathbb{N}$,
        to deduce that there is a unique function $f' : \mathbb{N}' \rightarrow \mathbb{N}$ such that 
        $f' \circ s' = s \circ f'$ and $f'(1') = 1$.

    Now we must show $f'$ is the inverse of $f$.

    Consider $f' \circ f$.  
    Let $x \in \mathbb{N}$.  

    \textbf{Base case:} $x = 1$.  
    \[(f' \circ f)(x) = f'(f(1)) = f'(1') = 1 = x\]  

    \textbf{Inductive step:} Suppose $x > 1$. By Lemma $1.2.3$ there exists $y \in \mathbb{N}$ such that $s(y) = x$.  
    Suppose for $y \in \mathbb{N}$ such that $y < x$, $(f' \circ f)(y) = y$.
    Then 
    \begin{align*}
    (f' \circ f)(x) 
    &= f'(f(s(y))) \\
    &= f'(s'(f(y))) && \text{(by $f \circ s = s' \circ f$)} \\
    &= s(f'(f(y))) && \text{(by $f' \circ s' = s \circ f'$)} \\
    &= s(y) && \text{$y < x$} \\
    &= x
    \end{align*}

    Consider $f \circ f'$.  
    Let $x' \in \mathbb{N}'$.  

    \textbf{Base case:} $x' = 1'$.  
    \[(f \circ f')(x') = f(f'(1')) = f(1) = 1' = x'\]  

    \textbf{Inductive step:} Suppose $x' > 1'$. By Lemma $1.2.3$ there exists $y' \in \mathbb{N}'$ such that $s'(y') = x'$.  
    Suppose for $y' \in \mathbb{N}'$ such that $y' < x'$, $(f \circ f')(y') = y'$.
    Then
    \begin{align*}
    (f \circ f')(x') 
    &= f(f'(s'(y'))) \\
    &= f(s(f'(y'))) && \text{(by $f' \circ s' = s \circ f'$)} \\
    &= s'(f(f'(y'))) && \text{(by $f \circ s = s' \circ f$)} \\
    &= s'(y') && \text{(induction hypothesis)} \\
    &= x'
    \end{align*}

    Since $(f' \circ f)(x) = x$ and $(f \circ f')(x') = x'$, we conclude that $f'$ is the inverse of $f$.  
    Thus $f$ is bijective.
\end{proof}

\begin{tcolorbox}[title=Extra Problem, breakable]
    Show the Peano axioms are independent. 
    That is, for any two Peano axioms, 
        find a structure that satisfies them but not the third. 
    You may assume the regular math of $\mathbb{Z}$, $\mathbb{Q}$, $\mathbb{R}$.
\end{tcolorbox}

\begin{axiom}[Peano Postulates]
    There exists a set $\mathbb{N}$ with an element $1 \in \mathbb{N}$
        and a function $s : \mathbb{N} \rightarrow \mathbb{N}$
        that satisfy the following three properties.
    \begin{enumerate}[label=\textbf{\alph*.}]
        \item There is no $n \in \mathbb{N}$ such that $s(n) = 1$.
        \item The function $s$ is injective.
        \item Let $G \subseteq \mathbb{N}$. Suppose that $1 \in G$, and
              that if $g \in G$ then $s(g) \in G$. Then $G = \mathbb{N}$.
    \end{enumerate}
\end{axiom}

\begin{proof}
    (\textbf{a., b.})
    Let $s : \mathbb{N} \rightarrow \mathbb{N}$ be defined by $s(x) = x + 2$.
    Let $G = \{x \mid \exists k \in \mathbb{Z}, x = 2k + 1\}$.
    Clearly $s$ is injective, $1 \in G$, and $G \subseteq \mathbb{N}$.
    But $G \ne \mathbb{N}$, and if $g \in G$ then $s(g) = g + 2 \in G$.
    Clearly \textbf{a., b.} 
        hold while \textbf{c.} does not hold.

    (\textbf{a., c.}) 
    Let $M = \{1, p\}$ and let $s : M \rightarrow M$ be defined
        by $s(1) = p$ and $s(p) = p$. Clearly \textbf{a., c.} 
        hold while \textbf{b.} does not hold.

    (\textbf{b., c.}) Let $M = \{1, p\}$ and let $s : M \rightarrow M$ be defined
        by $s(1) = p$ and $s(p) = 1$. Clearly \textbf{b., c} 
        hold while \textbf{a.} does not hold.
\end{proof}

\subsection{Constructing the Integers}

\begin{tcolorbox}[title=Problem 2, breakable]
    Complete the proof of Lemma $1.3.2$.
    That is, prove that the relation $\sim$ is transitive.
\end{tcolorbox}

\begin{proof}
    Let $(a, b), (c, d), (e, f) \in \mathbb{N} \times \mathbb{N}$.
    Assume $(a, b) \sim (c, d)$ and $(c, d) \sim (e, f)$.
    By definition of $\sim$,
        $a + d = b + c$ and $c + f = d + e$.
    Then taking sums shows $a + d + c + f = b + c + d + e$.
    Cancelling terms $a + f = b + e$.
    Thus, by definition of $\sim$, $(a, b) \sim (f, e)$.
    Since $\sim$ is symmetric, $(a, b) \sim (e, f)$.
\end{proof}

\begin{tcolorbox}[title=Problem 3, breakable]
    Test Complete the proof of Lemma $1.3.4$.
    That is, prove that $\cdot$ and $-$ for $\mathbb{Z}$
    are well-defined. The proof for $\cdot$ is a bit more complicated
    than might be expected. [Use Exercise $1.2.5$.]
\end{tcolorbox}

\begin{proof}
    Let $(a, b), (c, d), (x, y), (z, w) \in \mathbb{N} \times \mathbb{N}$.
    Suppose $(a, b) \sim (c, d)$ and $(x, y) \sim (z, w)$.
    So $a + d = b + c$ and $x + w = y + z$.
    We compute the following equations.
    \begin{enumerate}
        \item $ax + aw = ay + az$. Multiply $x + w = y + z$ by $a$.
        \item $by + bz = bx + bw$. Multiply $y + z = x + w$ by $b$.
        \item $cx + cw = cy + cz$. Multiply $x + w = y + z$ by $c$.
        \item $dy + dz = dx + dw$. Multiply $y + z = x + w$ by $d$.
    \end{enumerate}
    Then taking sums.
    \[ax + aw + by + bz + cx + cw + dy + dz = ay + az + bx + bw + cy + cz + dx + dw\]
    Grouping terms.
    \[ax + by + cw + dz + (aw + bz + cx + dy) = ay + bx + cz + dw + (az + bw + cy + dx)\]
    We can complete the proof by ignoring bloch's hint because it doesn't help 
        dumasses like me and cheating by showing $aw + bz + cx + dy = az + bw + cy + dx$.
    \begin{align*}
        &\big([(aw+1,aw)] + [(bz+1,bz)] + [(cx+1,cx)] + [(dy+1,dy)]\big)  \\ 
        &\quad - \big([(az+1,az)] + [(bw+1,bw)] + [(cy+1,cy)] + [(dx+1,dx)]\big) \\
        &= [(aw+1,aw)] + [(bz+1,bz)] + [(cx+1,cx)] + [(dy+1,dy)]  \\ 
        &\quad + [(az,az+1)] + [(bw,bw+1)] + [(cy,cy+1)] + [(dx,dx+1)] \\
        &= [(aw+bz+cx+dy + az+bw+cy+dx + 4,\ aw+bz+cx+dy + az+bw+cy+dx + 4)] \\
        &= [(1,1)] = 0
    \end{align*}
    Thus $ax + by + cw + dz = ay + bx  + cz + dw$.
    Then it follows that 
    \[(ax + by, ay + bx) \sim (cz + dw, cw + dz)\]
    Then from the definition of $\cdot$
    \[(a, b) \cdot (x, y) \sim (c, d) \cdot (z, w)\]
\end{proof}

\begin{proof}
    Let $(a, b), (c, d), (x, y), (z, w) \in \mathbb{N} \times \mathbb{N}$.
    Suppose $(a, b) \sim (c, d)$ and $(x, y) \sim (z, w)$.
    So $a + d = b + c$ and $x + w = y + z$.
    Summing shows $a + y + d + z = b + x + c + w$.
    Which is to say $(a + y, b + x) \sim (c + w, d + z)$.
    Therefore  $(a, b) + (y, x) \sim (c, d) + (w, z)$.
    It then follows that $(a, b) - (x, y) \sim (c, d) - (z, w)$.
    Thus $-$ is well defined.
\end{proof}

\begin{tcolorbox}[title=Problem 4, breakable]
    Let $a, b \in \mathbb{N}$.
    \begin{enumerate}
        \item Prove that $[(a,b)] = \hat{0}$ if and only if $a = b$.
        \item Prove that $[(a, b)] = \hat{1}$ if and only if $a = b + 1$.
        \item Prove that \textcircled{$1$} $[(a, b)] = [(n, 1)]$ for some $n \in \mathbb{N}$
              such that $n \ne 1$ if and only if \textcircled{$2$} $a > b$
              if and only if \textcircled{$3$} $[(a, b)] > \hat{0}$.
        \item Prove that \textcircled{$1$} $[(a, b)] = [(1, m)]$ for some $m \in \mathbb{N}$
              such that $m \ne 1$ if and only if \textcircled{$2$} $a < b$ if and only if 
              \textcircled{$3$} $[(a, b)] < \hat{0}$.
    \end{enumerate}
\end{tcolorbox}

\begin{proof}
    Suppose $[(a,b)] = \hat{0}$.
    Thus $(a, b) \sim (1, 1)$.
    Therefore $a + 1 = b + 1$.
    It follows that $a = b$.

    Suppose $a = b$.
    Then $a + 1 = b + 1$
    Therefore $(a, b) \sim (1, 1)$.
    It follows that $[(a,b)] = \hat{0}$.
\end{proof}

\begin{proof}
    Suppose $[(a, b)] = \hat{1}$.
    Thus $(a, b) \sim (1 + 1, 1)$.
    Therefore $a + 1 = b + (1 + 1)$.
    It follows that $a = b + 1$.

    Suppose $a = b + 1$.
    Thus $a + 1 = b + (1 + 1)$.
    Thus $(a, b) \sim (1 + 1, 1)$.
    It follows that $[(a,b)] = \hat{1}$.
\end{proof}

\begin{proof}
    (\textcircled{$1$} \rightarrow \textcircled{$2$})  
    Suppose $[(a, b)] = [(n, 1)]$ for some $n \in \mathbb{N}$ such that $n \ne 1$.  
    Thus $a + 1 = b + n$.  
    Since $n \ne 1$, $n > 1$.  
    There exists $p \in \mathbb{N}$ such that $s(p) = n$.  
    Then $a + 1 = b + s(p) = b + p + 1$.  
    It follows that $a = b + p$.  
    Thus $b < a$.  

    (\textcircled{$2$} \rightarrow \textcircled{$1$})  
    Suppose $a > b$.  
    There exists $p \in \mathbb{N}$ such that $a = b + p$.  
    Then $a + 1 = b + p + 1$.  
    It follows that $a + 1 = b + s(p)$.  
    Let $n = s(p)$.  
    Therefore $[(a, b)] = [(n, 1)]$ for some $n \in \mathbb{N}$ such that $n \ne 1$.  

    (\textcircled{$2$} \rightarrow \textcircled{$3$})  
    Suppose $a > b$.  
    There exists $p \in \mathbb{N}$ such that $a = b + p$.  
    Then $a + 1 = b + 1 + p$.  
    Therefore $[(a, b)] > \hat{0}$.  

    (\textcircled{$3$} \rightarrow \textcircled{$2$})  
    Suppose $[(a, b)] > \hat{0}$.  
    It follows that $a + 1 > b + 1$.  
    Thus there exists $p$ such that $a + 1 = b + 1 + p$.  
    Therefore $a = b + p$ and it follows that $a > b$.  
\end{proof}

\begin{proof}
    (\textcircled{$1$} \rightarrow \textcircled{$2$})  
    Suppose $[(a, b)] = [(1, m)]$ for some $m \in \mathbb{N}$ such that $m \ne 1$.  
    Then $a + m = b + 1$.  
    Since $m \ne 1$, $m > 1$.  
    There exists $p \in \mathbb{N}$ such that $s(p) = m$.  
    Then $a + s(p) = b + 1 \implies a + p + 1 = b + 1$.  
    It follows that $a = b - p$.  
    Thus $a < b$.  

    (\textcircled{$2$} \rightarrow \textcircled{$1$})  
    Suppose $a < b$.  
    There exists $p \in \mathbb{N}$ such that $b = a + p$ with $p \ne 0$.  
    Then $b + 1 = a + p + 1 = a + s(p)$.  
    Let $m = s(p)$. Then $m \ne 1$.  
    Therefore $[(a, b)] = [(1, m)]$ for some $m \in \mathbb{N}$ with $m \ne 1$.  

    (\textcircled{$2$} \rightarrow \textcircled{$3$})  
    Suppose $a < b$.  
    Then there exists $p \in \mathbb{N}$ such that $b = a + p$.  
    Then $b + 1 = a + 1 + p$.  
    Therefore $[(a, b)] < \hat{0}$.  

    (\textcircled{$3$} \rightarrow \textcircled{$2$})  
    Suppose $[(a, b)] < \hat{0}$.  
    It follows that $b + 1 > a + 1$.  
    Thus there exists $p \in \mathbb{N}$ such that $b + 1 = a + 1 + p$.  
    Therefore $b = a + p$, so $a < b$.  
\end{proof}

\begin{tcolorbox}[title=Problem 5, breakable]
    Prove Theorem $1.3.5$ ($1$) ($3$) ($4$) ($5$) ($6$)
    ($7$) ($8$) ($10$) ($11$) ($13$) ($14$).
\end{tcolorbox}

\begin{proof}
    Let $x, y, z \in \mathbb{Z}$.
    We must show $(x + y) + z = z + (x + y)$.
    Let $(x_1, x_2), (y_1, y_2), (z_1, z_2) \in \mathbb{N} \times \mathbb{N}$
        such that $x = (x_1, x_2)$, $y = (y_1, y_2)$
        and $z = (z_1, z_2)$.
    Then 
    \begin{align*}
        (x + y) + z 
            &= ([(x_1, x_2)] + [(y_1, y_2)]) + [(z_1, z_2)] \\
            &= [(x_1 + y_1), (x_2 + y_2)] + [(z_1, z_2)] \\
            &= [((x_1 + y_1) + z_1), ((x_2 + y_2) + z_2)] \\
            &= [(x_1 + (y_1 + z_1)), (x_2 + (y_2 + z_2))] \\
            &= [(x_1, x_2)] + [(y_1 + z_1), (y_2 + z_2)] \\
            &= [(x_1, x_2)] + ([y_1, y_2] +  [z_1, z_2]) \\
            &= x + (y + z)
    \end{align*}
\end{proof}

\begin{proof}
    We must show $x + \hat{0} = x$.
    Let $(x_1, x_2) \in \mathbb{N} \times \mathbb{N}$
        such that $x = [(x_1, x_2)]$.
    Then $x + \hat{0} = [(x_1, x_2)] + [(1, 1)] = [(x_1 + 1, x_2 + 1)]$.
    Now $x_1 + x_2 + 1 = x_1 + x_2 + 1$
    and rearranging shows $(x_1 + 1) + x_2 = (x_2 + 1) + x_1$.
    From which it follows
        $(x_1 + 1, x_2 + 1) \sim (x_1, x_2)$.
    Thus 
    \[[(x_1 + 1, x_2 + 1)] = [(x_1, x_2)] = x\]
\end{proof}

\begin{proof}
    Let $x \in \mathbb{N}$
    We must show $x + (-x) = \hat{0}$.
    Let $(x_1, x_2) \in \mathbb{N}$
        such that $x = [(x_1, x_2)]$.
    Then 
    \[x + (-x) = [(x_1, x_2)] + (-[(x_1, x_2)]) = [(x_1, x_2)] + [(x_2, x_1)] = [(x_1 + x_2, x_2 + x_1)]\]
    Now it is clearly $x_1 + x_2 + 1 = x_1 + x_2 + 1$
        and rearranging shows $(x_1 + x_2) + 1 = (x_2 + x_1) + 1$.
    Thus $(x_1 + x_2, x_2 + x_1) \sim (1, 1)$.
    Then
    \[[(x_1 + x_2, x_2 + x_1)] = [(1, 1)] = \hat{0}\]
\end{proof}

\begin{proof}
    Let $x, y, z \in \mathbb{Z}$.
    We must show $(xy)z = x(yz)$.
    Let $(x_1, x_2), (y_1, y_2), (z_1, z_2) \in \mathbb{N} \times \mathbb{N}$
        such that $x = [(x_1, x_2)], y = [(y_1, y_2)], z = [(z_1, z_2)]$.
    Then 
    \begin{align*}
        (xy)z 
            &= ([(x_1, x_2)] \cdot [(y_1, y_2)]) \cdot [(z_1, z_2)] \\
            &= [(x_1 y_1 + x_2 y_2, x_1 y_2 + x_2 y_1)] \cdot [(z_1, z_2)] \\
            &= [((x_1 y_1 + x_2 y_2) z_1 + (x_1 y_2 + x_2 y_1) z_2, (x_1 y_1 + x_2 y_2) z_2 + (x_1 y_2 + x_2 y_1) z_1)] \\
            &= [(x_1 y_1 z_1 + x_2 y_2 z_1 + x_1 y_2 z_2 + x_2 y_1 z_2, x_1 y_1 z_2 + x_2 y_2 z_2 + x_1 y_2 z_1 + x_2 y_1 z_1)] \\
            &= [(x_1(y_1 z_1 + y_2 z_2) + x_2(y_2 z_1 + y_1 z_2), x_1(y_1 z_2 + y_2 z_1) + x_2(y_2 z_2 + y_1 z_1))] \\
            &= [(x_1, x_2)] \cdot [(y_1 z_1 + y_2 z_2, y_1 z_2 + y_2 z_1)]  \\
            &= [(x_1, x_2)] \cdot ([(y_1, y_2)] \cdot [(z_1, z_2)]) \\
            &= x \cdot (y z)
    \end{align*}
\end{proof}

\begin{proof}
    Let $x, y \in \mathbb{N}$.
    We must show $xy = yx$.
    Let $(x_1, x_2), (y_1, y_2) \in \mathbb{N} \times \mathbb{N}$
        such that $x = [(x_1, x_2)], y = [(y_1, y_2)]$.
    Then 
    \begin{align*}
        xy 
            &= [(x_1, x_2)] \cdot [(y_1, y_2)] \\
            &= [(x_1 y_1 + x_2 y_2, x_1 y_2 + x_2 y_1)] \\
            &= [(x_2 y_2 + x_1 y_1 , x_2 y_1 + x_1 y_2)] \\
            &= [(y_1, y_2)] \cdot [(x_1, x_2)]  \\
            &= yx
    \end{align*}
\end{proof}

\begin{proof}
    Let $x \in \mathbb{Z}$.
    We must show $x \cdot \hat{1} = x$.
    Let $(x_1, x_2) \in \mathbb{N} \times \mathbb{N}$
        such that $x = [(x_1, x_2)]$.
    Then 
    \[x \cdot \hat{1} 
        = [(x_1, x_2)] \cdot [(1 + 1, 1)] 
        = [(x_1(1 + 1) + x_2 \cdot 1, x_1 \cdot 1 + x_2 \cdot 1)] 
        = [(2 x_1 + x_2, x_1 + x_2)]\]
    Now $2x_1 + 2x_2 = 2x_1 + 2x_2$.
    It follows that $(2 x_1 + x_2, x_1 + x_2) \sim (x_1, x_2)$.
    Therefore
    \[[(2 x_1 + x_2, x_1 + x_2)] = [(x_1, x_2)] = x\]
\end{proof}

\begin{proof}
    Let $x, y, z \in \mathbb{Z}$.
    We must show $x(y + z) = xy + xz$.
    Let $(x_1, x_2), (y_1, y_2), (z_1, z_2) \in \mathbb{N} \times \mathbb{N}$
        such that $x = [(x_1, x_2)], y = [(y_1, y_2)], z = [(z_1, z_2)]$.
    \begin{align*}
        x(y + z) 
            &= [(x_1, x_2)] \cdot ([(y_1, y_2)] + [(z_1, z_2)]) \\
            &= [(x_1, x_2)] \cdot [(y_1 + z_1, y_2 + z_2)] \\
            &= [(x_1(y_1 + z_1) + x_2(y_2 + z_2),\ x_1(y_2 + z_2) + x_2(y_1 + z_1))] \\
            &= [(x_1 y_1 + x_1 z_1 + x_2 y_2 + x_2 z_2,\ x_1 y_2 + x_1 z_2 + x_2 y_1 + x_2 z_1)] \\
            &= [(x_1 y_1 + x_2 y_2,\ x_1 y_2 + x_2 y_1) + (x_1 z_1 + x_2 z_2,\ x_1 z_2 + x_2 z_1)] \\
            &= xy + xz.
    \end{align*}
\end{proof}

\begin{proof}
    Let $x, y \in \mathbb{Z}$.
    We must show precisely one of $x < y$, $x = y$, or $x > y$ holds.
    Let $(x_1, x_2), (y_1, y_2) \in \mathbb{N} \times \mathbb{N}$
        such that $x = [(x_1, x_2)], y = [(y_1, y_2)]$.

    We first show no two hold simultaneously.

    Suppose $x < y$ and $x > y$.
    Then $x_1 + y_2 < x_2 + y_1$ and $x_1 + y_2 > x_2 + y_1$, 
    which is a contradiction.

    Suppose $x < y$ and $x = y$.
    Then $x_1 + y_2 < x_2 + y_1$ and $x_1 + y_2 = x_2 + y_1$, 
    which is a contradiction.

    Suppose $x > y$ and $x = y$.
    Then $x_1 + y_2 > x_2 + y_1$ and $x_1 + y_2 = x_2 + y_1$, 
    which is a contradiction.

    Thus no two hold simultaneously.

    We now show at least one holds.
    We know either $x_1 + y_2 < x_2 + y_1$, $x_1 + y_2 = x_2 + y_1$, 
        or $x_1 + y_2 > x_2 + y_1$.
    Thus at least one of $x < y$, $x = y$, or $x > y$ holds. 
\end{proof}

\begin{proof}
    Let $x, y, z \in \mathbb{Z}$.
    We must show if $x < y$ then $x + z < y + z$.
    Let $(x_1, x_2), (y_1, y_2), (z_1, z_2) \in \mathbb{N} \times \mathbb{N}$
        such that $x = [(x_1, x_2)], y = [(y_1, y_2)], z = [(z_1, z_2)]$.
    Suppose $x < y$.
    Then $x_1 + y_2 < x_2  + y_1$.
    There exists $p \in \mathbb{N}$ such that $x_1 + y_2 + p = x_2  + y_1$.
    It follows that $x_1 + y_2 + p + z_1 + z_2 = x_2  + y_1 + z_1 + z_2$.
    Rearranging terms $(x_1 + z_1) + (y_2 + z_2) + p = (x_2 + z_2) + (y_1 + z_1)$.
    Thus $(x_1 + z_1) + (y_2 + z_2) < (x_2 + z_2) + (y_1 + z_1)$.
    Then 
    \[[(x_1 + z_1, x_2 + z_2)] < [(y_1 + z_1, y_2 + z_2)] \iff [(x_1, x_2)] + [(z_1, z_2)] < [(y_1, y_2)]\]
    Therefore $x + z < y + z$.
\end{proof}

\begin{proof}
    Let $x, y, z \in \mathbb{Z}$.
    We must show if $x < y$ and $z > \hat{0}$, then $xz < yz$.
    Let $(x_1, x_2), (y_1, y_2), (z_1, z_2) \in \mathbb{N} \times \mathbb{N}$
        such that $x = [(x_1, x_2)], y = [(y_1, y_2)], z = [(z_1, z_2)]$.

    Suppose $x < y$ and $z > \hat{0}$.
    Then $x_1 + y_2 < x_2 + y_1$ and $z_1 > z_2$.
    Since $z_1 > z_2$, there exists $q \in \mathbb{N}$ such that $z_1 = z_2 + q$.
    From $x_1 + y_2 < x_2 + y_1$ there exists $p \in \mathbb{N}$
        such that $x_1 + y_2 + p = x_2 + y_1$.
    From $x_1 + y_2 + p = x_2 + y_1$ multiply by $z_1$, $x_1 z_1 + y_2 z_1 + p z_1 = x_2 z_1 + y_1 z_1$.
    From $x_1 + y_2 + p = x_2 + y_1$ multiply by $z_2$, $x_1 z_2 + y_2 z_2 + p z_2 = x_2 z_2 + y_1 z_2$.
    Taking sums
    \[(x_1 z_1 + x_2 z_2) + (y_1 z_2 + y_2 z_1) + p z_1 = (x_2 z_1 + x_1 z_2) + (y_2 z_1 + y_1 z_2)\]
    Rearranging terms gives
    \[(x_1 z_1 + x_2 z_2) + p z_1 < (x_2 z_1 + x_1 z_2)\]
    Thus $(x_1 z_1 + x_2 z_2) < (x_2 z_1 + x_1 z_2)$.
    Then
    \[[(x_1 z_1 + x_2 z_2, x_2 z_1 + x_1 z_2)] < [(y_1 z_1 + y_2 z_2, y_2 z_1 + y_1 z_2)] 
        \iff [(x_1, x_2)] \cdot [(z_1, z_2)] < [(y_1, y_2)] \cdot [(z_1, z_2)]\]
    Therefore $xz < yz$.
\end{proof}

\begin{proof}
    We must show $\hat{0} \ne \hat{1}$.
    For contradiction suppose $\hat{0} = \hat{1}$.
    Then $(1, 1) \sim (1 + 1, 1)$
        then $1 + 1 = 1 + 1 + 1$
    Let $p \in \mathbb{N}$ such that $p = 1 + 1$.
    It follows that $p + 1 = p$ which is a contradiction. 
\end{proof}

\begin{tcolorbox}[title=Problem 6, breakable]
    Prove Theorem $1.3.7$ ($1$) ($3$) ($4(b)$) ($4(c)$).
\end{tcolorbox}

\begin{theorem}
    Let $i : \mathbb{N} \rightarrow \mathbb{Z}$
        be defined by $i(n) = [(n + 1), 1]$
        for all $n \in \mathbb{N}$.
    \begin{enumerate}
        \item The function $i : \mathbb{N} \rightarrow \mathbb{Z}$ is injective.
        \item $i(\mathbb{N}) = \{x \in \mathbb{Z} \mid x > \hat{0}\}$.
        \item $i(1) = \hat{1}$.
        \item Let $a, b \in \mathbb{N}$. Then 
              \begin{enumerate}
                \item $i(a + b) = i(a) + i(b)$;
                \item $i(ab) = i(a)i(b)$;
                \item $a < b$ if and only if $i(a) < i(b)$.
              \end{enumerate}
    \end{enumerate}
\end{theorem}

\begin{proof}
    We must show $i : \mathbb{N} \rightarrow \mathbb{Z}$ is injective.
    Let $x_1, x_2 \in \mathbb{N}$
        such that $i(x_1) = i(x_2)$.
    We must show $x_1 = x_2$.
    Now, $[(x_1 + 1, 1)] = [(x_2 + 1, 1)]$.
    Thus $(x_1 + 1) + 1 = 1 + (x_2 + 1)$    
        and cancelling terms shows that $x_1 = x_2$.
\end{proof}

\begin{proof}
    We must show $i(1) = \hat{1}$.
    Now, $i(1) = [(1 + 1, 1)] = \hat{1}$.
\end{proof}

\begin{proof}
    We must show $i(ab) = i(a)i(b)$. 
    Now $i(ab) = [(ab + 1, 1)]$.
    We know that $ab + a + b + 3 = ab + a + b + 3$
        which is equivalent to $ab + 1 + a + 1 + b + 1 = 1 + (ab + a + b + 1) + 1$.
    Rearranging terms $(ab + 1) + ((a + 1) + (b + 1)) = 1 + ((a + 1)(b + 1) + 1)$.
    Thus $(ab + 1, 1) \sim ((a + 1)(b + 1) + 1, (a + 1) + (b + 1))$.
    Then $[(ab + 1, 1)] = [((a + 1)(b + 1) + 1, (a + 1) + (b + 1))]$
        and $[((a + 1)(b + 1) + 1, (a + 1) + (b + 1))] = [(a + 1, 1)] \cdot [(b + 1, 1)]$.
    It follows that $[(a + 1, 1)] \cdot [(b + 1, 1)] = i(a)i(b)$.
\end{proof}

\begin{proof}
    We must show $a < b$ if and only if $i(a) < i(b)$.

    Suppose $a < b$.
    It follows that $(a + 1) + 1 < 1 + (b + 1)$.
    Thus $[(a + 1, 1)] < [(b + 1, 1)]$.

    Suppose $i(a) < i(b)$.
    Then $[(a + 1, 1)] < [(b + 1, 1)]$.
    It follows that $(a + 1) + 1 < 1 + (b + 1)$.
    Cancelling terms shows $a < b$.
\end{proof}

\begin{tcolorbox}[title=Problem 7, breakable]
    Let $x, y, z \in \mathbb{Z}$
    \begin{enumerate}
        \item Prove that $x < y$ if and only if $-x > -y$.
        \item Prove that if $z < 0$, then $x < y$ if and only if $xz > yz$.
    \end{enumerate}
\end{tcolorbox}

\begin{proof}
    Suppose $x < y$ then
    \begin{align*}
        x < y 
            &\iff x + ((-x) + (-y)) < y + ((-x) + (-y)) && \text{by Theorem 1.3.5 part (12)} \\
            &\iff x + ((-x) + (-y)) < y + ((-y) + (-x)) && \text{by Theorem 1.3.5 part (2)} \\
            &\iff (x + (-x)) + (-y) < (y + (-y)) + (-x) && \text{by Theorem 1.3.5 part (1)} \\
            &\iff 0 + (-y) < 0 + (-x) && \text{by Theorem 1.3.5 part (4)} \\
            &\iff (-y) + 0 < (-x) + 0 && \text{by Theorem 1.3.5 part (2)} \\
            &\iff -y < -x && \text{by Theorem 1.3.5 (4)}
    \end{align*}
    Suppose $-y < -x$ then
    \begin{align*}
        -y < -x 
            &\iff (-y) + (x + y) < (-x) + (x + y) && \text{by Theorem 1.3.5 part (12)} \\
            &\iff (-y) + (y + x) < (-x) + (x + y) && \text{by Theorem 1.3.5 part (2)} \\
            &\iff ((-y) + y) + x < ((-x) + x) + y && \text{by Theorem 1.3.5 part (1)} \\
            &\iff 0 + x < 0 + y && \text{by Theorem 1.3.5 part (4)} \\
            &\iff x + 0 < y + 0 && \text{by Theorem 1.3.5 part (2)} \\
            &\iff x < y && \text{by Theorem 1.3.5 part (4)}
    \end{align*}
\end{proof}

\begin{proof}
    Suppose $z < 0$. It follows that $-z > 0$.
    
    Suppose $x < y$.
    By Theorem 1.3.5 part 13,2 it follows that $x(-z) < y(-z) \iff -zx < -zy$.
    By the previous problem, $zy > zx$.
    By Theorem 1.3.5 part 2, $xz > yz$,

    Suppose $xz > yz$.
    By the previous problem, $-xz < -yz$.
    By Theorem 1.3.5 part 2, $x(-z) < y(-z)$.
    By Theorem 1.3.5 part 13, $x < y$.
\end{proof}

\begin{tcolorbox}[title=Problem 8, breakable]
    Let $x \in \mathbb{Z}$. Prove that if $x > 0$ then $x \ge 1$.
    Prove that if $x < 0$ then $x \le -1$.
\end{tcolorbox}

\begin{proof}
    Suppose $x > 0$. For contradiction suppose $x < 1$.
    Then $0 < x < 1$ and it follows that $1 < x + 1 < 2$.
    Let $i$ be the bijective function in Theorem 1.3.7.
    It follows that $i(1) < i(x + 1) < i(2) = i(1) + i(1)$.
            contradicting Theorem $1.2.9$ part 9.
\end{proof}

\begin{proof}
    Suppose $x < 0$. For contradiction suppose $x > -1$.
    Then $-1 < x < 0$ and it follows that $1 < x + 2 < 2$.
    Let $i$ be the bijective function in Theorem 1.3.7.
    It follows that $i(1) < i(x + 2) < i(2) = i(1) + i(1)$,
        contradicting Theorem $1.2.9$ part 9.
\end{proof}

\begin{tcolorbox}[title=Problem 9, breakable]
    \begin{enumerate}
        \item Prove that $1 < 2$.
        \item Let $x \in \mathbb{Z}$. Prove that $2x \ne 1$.
    \end{enumerate}
\end{tcolorbox}

\begin{proof}
    For contradiction suppose $1 \ge 2$.
    Either $1 = 2$ or $1 > 2$.

    Suppose $1 = 2$.
    Let $i$ be the bijective function in Theorem 1.3.7.
    Then $i(1) = i(2) = i(1) + i(1)$, 
    which contradicts Theorem 1.2.7 part 6.

    Suppose $1 > 2$.
    Then $i(1) > i(1) + i(1)$.
    There exists $p \in \mathbb{N}$ such that $i(1) = p + i(1) + i(1)$.
    This also contradicts Theorem 1.2.7 part 6.

    It follows that $1 < 2$.
\end{proof}

\begin{proof}
    For contradiction suppose $2x = 1$
    Let $(x_1, x_2) \in \mathbb{N} \times \mathbb{N}$
        such that $x = [(x_1, x_2)]$.
    Then $[(3, 1)] \cdot [(x_1, x_2)] = [(1 + 1, 1)]
        \iff [(3x_1 + x_2, 3 x_2 + x_1)] = [(1 + 1, 1)]$
    It follows that $3x_1 + x_2 + 1 = 3 x_2 + x_1 + 1$.
    Cancelling terms shows $x_1 = x_2$.
    So $(x_1, x_2) \sim (1, 1)$ thus $2 \cdot \hat{0} = 0 \ne 1$.
\end{proof}

\begin{tcolorbox}[title=Problem 10, breakable]
    Prove that the Well-Order Principle (Theorem $1.2.10$),
    which was stated for $\mathbb{N}$ in Section $1.2$, still holds 
    when we think of $\mathbb{N}$ as the set of positive integers.
    That is, let $G \subseteq \{x \in \mathbb{Z} \mid x > 0\}$ be 
    a non-empty set. Prove that there is some $m \in G$ such that 
    $m \le g$ for all $g \in G$. Use Theorem $1.3.7$.
\end{tcolorbox}

\begin{proof}
    Let $G \subseteq \{x \in \mathbb{Z} \mid x > 0\}$
        such that $G \ne \emptyset$.
    Let $i$ be the bijective function in Theorem 1.3.7.
    By Theorem 1.2.10, since $i^{-1}(G) \subseteq \mathbb{N}$ there exists $n \in i^{-1}(G)$ 
    such that for all $x \in i^{-1}(G)$,
        $n \le x$.
    It follows that for all $x \in G$,
        $i(n) \le x$
\end{proof}

\begin{tcolorbox}[title=Problem 11, breakable]
    Prove Theorem $1.3.8$ ($1$) ($3$) ($4$) ($5$) ($7$) ($10$) ($11$).
\end{tcolorbox}

\begin{proof}
    We must show if $x + z = y + z$ then $x = y$.
    Suppose $x + z = y + z$.
    Then
    \begin{align*}
        &x + z = y + z \\ 
        \iff&(x + z) + (-z) = (y + z) + (-z)  \\
        \iff&x + (z + (-z)) = y + (z + (-z)) && \text{by Theorem 1.3.5 part (1)} \\
        \iff&x + 0 = y + z && \text{by Theorem 1.3.5 part (4)} \\
        \iff&x = y && \text{by Theorem 1.3.5 part (3)}
    \end{align*}
\end{proof}

\begin{proof}
    We must show $-(x + y) = (-x) + (-y)$.
    Then
    \begin{align*}
        &-(x + y) = (-x) + (-y) \\
        \iff&-(x + y) + (x + y) = (-x) + (-y) + (x + y) \\
        \iff&(x + y) + (-(x + y)) = (-x) + (x + y) + (-y) && \text{by Theorem 1.3.5 part (2)} \\
        \iff&(x + y) + (-(x + y)) = (-x) + x + (y + (-y)) && \text{by Theorem 1.3.5 part (5)} \\
        \iff&(x + y) + (-(x + y)) = x + (-x) + (y + (-y)) && \text{by Theorem 1.3.5 part (2)} \\
        \iff&0 = 0 + 0 && \text{by Theorem 1.3.5 part (4)} \\
        \iff&0 = 0 && \text{by Theorem 1.3.5 part (4)}
    \end{align*}
\end{proof}

\begin{proof}
    We must show $x\cdot 0 = 0$.
    \begin{align*}
        &(x\cdot 0) + (x\cdot 0) = x(0+0) && \text{by Theorem 1.3.5 part (8)} \\
        \iff&(x\cdot 0) + (x\cdot 0) = x\cdot 0 && \text{by Theorem 1.3.5 part (3)} \\
        \iff&(x\cdot 0) + (x\cdot 0) + (-(x\cdot 0)) = x\cdot 0 + (-(x\cdot 0)) \\
        \iff&(x\cdot 0) + \big( (x\cdot 0) + (-(x\cdot 0)) \big) = x\cdot 0 + (-(x\cdot 0)) && \text{by Theorem 1.3.5 part (1)} \\
        \iff&(x\cdot 0) + 0 = 0 && \text{by Theorem 1.3.5 part (4)} \\
        \iff&x\cdot 0 = 0 && \text{by Theorem 1.3.5 part (3)}
    \end{align*}
\end{proof}

\begin{proof}
    We must show that if $z \ne 0$ and $xz = yz$, then $x = y$.
    Suppose $z \ne 0$ and $xz = yz$. Then
    \begin{align*}
        xz = yz &\iff xz - yz = 0 \\
                &\iff (x - y)z = 0.
    \end{align*}
    Since $z \ne 0$, it follows that $x + (-y) = 0$, 
    so $x = y$.
\end{proof}

\begin{proof}
    We must show $xy = 1$ if and only if $x = 1 = y$ or $x = -1 = y$.

    ($\rightarrow$)
    Suppose $xy = 1$.
    For contradiction, suppose $x \ne 1, y \ne 1$, and $x \ne -1, y \ne -1$.

    To make things easier, we first show $x \ne 0, y \ne 0$.
    If $x = 0$ then $x y = 0 y$
        and from the 1.3.8 (4)
        it follows that $0 y = 0$
        contradicting that $x y = 1$.
    Similarly $y \ne 0$.

    \begin{enumerate}
        \item $x > 1, y > 1$.
        \item $x < 1, y < 1$ so $x < 0, y < 0$.
        \item $x > 1, y < 1$ so $y < 0$.
        \item $x < 1, y > 1$ so $x < 0$.
    \end{enumerate}

    Suppose $x > 1, y > 1$.
    Since $1 > 0$ it follows that $y > 0$ by Transitive Law.
    Since $1 < x$ and $y > 0$ it follows that 
        $1 \cdot y < x y$.
    Then from Identity Law for Multiplication
        it follows that $y < xy$
        showing $y < 1$ which is a contradiction.

    Suppose $x < 0$ and $y < 0$.
    Since $-x > 0$ and $-y > 0$, we have $(-x)(-y) = xy$.
    But $xy = 1$, so $(-x)(-y) = 1$.
    For contradiction suppose $-x \ne 1$.
    Then either $-x > 1$ or $-x < 1$.
    Suppose $-x > 1$.
    Since $-y > 0$ it follows that $1\cdot(-y) < (-x)(-y) = 1$.
    Then from Identity Law for Multiplication it follows that $-y < 1$.
    But $-y \in \mathbb{Z}$ and $-y > 0$ thus $-y \in \mathbb{N}$
        contradicting that $1$ is the lower bound of $\mathbb{N}$.
    Suppose $-x < 1$.
    So $-x \le 0$. Suppose $-x = 0$.
    Thus $x = 0$ which is a contradiction.
    Thus $-x < 0$.
    Since $-y > 0$ it follows that $(-x)(-y) < 0 \cdot (-y)$.
    From which it follows that $1 < 0$ which is a contradiction.

    Suppose $x > 1$ and $y < 0$.
    Since $1 < x$ and $y < 0$ it follows that $1\cdot y > x y$.
    Then from Identity Law for Multiplication
        it follows that $y > xy$.
    But $xy = 1$ so $y > 1$
        which contradicts $y < 1$.

    Suppose $x < 0$ and $y > 1$.
    Since $1 < y$ and $x < 0$ it follows that $x\cdot 1 > x y$.
    Then from Identity Law for Multiplication
        it follows that $x > x y$.
    But $x y = 1$ so $x > 1$
        which contradicts $x < 1$.

    ($\leftarrow$) Suppose $x = 1 = y$ or $x = -1 = y$.
    Suppose $x = 1 = y$.    
        Then $x y = 1 \cdot 1 = 1.$
    Suppose $x = -1 = y$.
        Then $xy = (-1)(-1)$.
    Then by 1.3.8 (6), $(-1)(-1) = 1(-(-1))$.
        and by 1.3.8 (2), $1(-(-1)) = 1 \cdot 1 = 1$.
    Thus $xy = 1$.
\end{proof}

\begin{proof}
    We must show if $x \le y$ and $y \le x$, then $x = y$.
    Suppose $x \le y$ and $y \le x$.
    For contradiction suppose $x \ne y$.
    Then either $x < y$ or $x > y$.
    Suppose $x < y$. This contradicts $y \le x$.
    Suppose $x > y$. This contradicts $x \le y$.
    Thus $x = y$.
\end{proof}

\begin{proof}
    We must show that if $x > 0$ and $y > 0$, then $xy > 0$, 
    and if $x > 0$ and $y < 0$, then $xy < 0$.

    Suppose $x, y > 0$ and for contradiction $xy \le 0$.
    Either $xy = 0$ or $xy < 0$.
    Suppose $xy = 0$ and it follows that $x = 0$ or $y = 0$
        contradicting $x, y >0$.
    Suppose $xy < 0$.
    It follows that $-xy > 0$.
    Thus $x(-y) > x \cdot 0$.
    Since $x > 0$ it follows that $-y > 0$ (Problem 7).
    Then $y < 0$ which is a contradiciton.

    Suppose $x > 0$ and $y < 0$ and for contradiction $xy \ge 0$.
    Either $xy = 0$ or $xy > 0$.
    Suppose $xy = 0$ and it follows that $x = 0$ or $y = 0$
        contradicting $x > 0, y < 0$.
    Suppose $xy > 0$.
    Since $-y > 0$ and $x > 0$ it follows that $x(-y) > 0$.
    Thus $-xy > 0$ and it follows that $xy < 0$.
\end{proof}

\subsection{Axioms for the Integers}

\begin{tcolorbox}[title=Problem 2, breakable]
    Let $n \in \mathbb{N}$. Prove that $n + 1 \in \mathbb{N}$.
\end{tcolorbox}

\begin{proof}
    Since $n \in \mathbb{N}$, $n \in \mathbb{Z}$ and $n > 0$.
    By Addition Law for Order, $n + 1 > 1$.
    By 1.4.5 (9), $n + 1 > 1 > 0$.
    By Transitive Law, $n + 1 > 0$.
    Since $n + 1 \in \mathbb{Z}$ and $n + 1 > 0$,
        by definition of $\mathbb{N}$, $n + 1 \in \mathbb{N}$.
\end{proof}

\begin{tcolorbox}[title=Problem 3, breakable]
    Let $x, y \in \mathbb{Z}$.
    Prove that $x \le y$ if and only if $-x \ge -y$.
\end{tcolorbox}

\begin{proof}
    ($\rightarrow$) 
    Suppose $x \le y$. Then
    \begin{align*}
        x \le y
        \iff &x + ((-x) + (-y)) \le y + ((-x) + (-y)) \\
        \iff &(x + (-x)) + (-y) \le y + ((-x) + (-y)) && \text{1.4.1 (a)} \\
        \iff &(x + (-x)) + (-y) \le y + ((-y) + (-x)) && \text{1.4.1 (b)} \\
        \iff &(x + (-x)) + (-y) \le (y + (-y)) + (-x) && \text{1.4.1 (a)} \\
        \iff &0 + (-y) \le 0 + (-x) && \text{1.4.1 (d)} \\
        \iff &-y + 0 \le -x + 0 && \text{1.4.1 (b)} \\
        \iff &-y \le -x && \text{1.4.1 (c)} 
    \end{align*}
    ($\leftarrow$)
    Suppose $-x \ge -y$. Then
    \begin{align*}
        -x \ge -y
        \iff &-x + (x + y) \ge -y + (x + y) \\
        \iff &(-x + x) + y \ge -y + (x + y) && \text{1.4.1 (a)} \\
        \iff &(-x + x) + y \ge -y + (y + x) && \text{1.4.1 (b)} \\
        \iff &(-x + x) + y \ge (-y + y) + x && \text{1.4.1 (a)} \\
        \iff &(x + (-x)) + y \ge (y + (-y)) + x && \text{1.4.1 (b)} \\
        \iff & 0 + y \ge 0 + x && \text{1.4.1 (d)} \\
        \iff &y + 0 \ge x + 0 && \text{1.4.1 (b)} \\
        \iff &y \ge x && \text{1.4.1 (c)} 
    \end{align*}
\end{proof}

\begin{tcolorbox}[title=Problem 4, breakable]
    Prove that $\mathbb{N} = \{x \in \mathbb{Z} \mid x \ge 1\}$.
\end{tcolorbox}

\begin{proof}
    Let $x \in \mathbb{N}$. By definition $x \in \mathbb{Z}$ and $x > 0$.
    For contradiction, suppose $x < 1$. Then $0 < x < 1$ contradicting 1.4.6.
    Thus $x \ge 1$. It follows that $x \in \{x \in \mathbb{Z} \mid x \ge 1\}$.
    Therefore $\mathbb{N} \subseteq \{x \in \mathbb{Z} \mid x \ge 1\}$.

    Let $x \in \{x \in \mathbb{Z} \mid x \ge 1\}$.
    Either $x = 1$ or $x > 1$. In either case $x > 0$.
    Thus $x \in \mathbb{N}$.
    Therefore $\{x \in \mathbb{Z} \mid x \ge 1\} \subseteq \mathbb{N}$.

    It follows that $\mathbb{N} = \{x \in \mathbb{Z} \mid x \ge 1\}$.
\end{proof}

\begin{tcolorbox}[title=Problem 5, breakable]
    Let $a, b \in \mathbb{Z}$.
    Prove that if $a < b$ then $a + 1 \le b$.
\end{tcolorbox}

\begin{proof}
    Suppose $a < b$. For contradiciton suppose $a + 1 > b$.
    Then $a + 1 > b > a$ contradicting 1.4.6.
\end{proof}

\begin{tcolorbox}[title=Problem 6, breakable]
    Let $n \in \mathbb{N}$.
    Suppose that $n \ne 1$.
    Prove that there is some $b \in \mathbb{N}$
        such that $b + 1 = n$.
\end{tcolorbox}

\begin{proof}
    (\textbf{Base Case}) Suppose $n = 2$.
    Then $s(1) = 1 + 1 = 2$.

    (\textbf{Induction Step}) Suppose the theorem holds for some $n \in \mathbb{N}$
        such that $n \ne 1$.
    Consider $n + 1 = s(n)$. By our hypothesis there exists $b \in \mathbb{N}$
        such that $s(b) = n$. Thus $n + 1 = s(s(b)) = s(b) + 1$.
        Thus proving our theorem.
\end{proof}

\begin{tcolorbox}[title=Problem 8, breakable]
    Let $a \in \mathbb{Z}$.
    \begin{enumerate}
        \item Let $G \subseteq \{x \in \mathbb{Z} \mid x \ge a\}$ be a set.
              Suppose that $a \in G$, and that if $g \in G$ then 
              $g + 1 \in G$. Prove that $G = \{x \in \mathbb{Z} \mid x \ge a\}$.
        \item Let $H \subseteq \{x \in \mathbb{Z} \mid x \le a\}$ be a set.
              Suppose that $a \in H$, and that if $h \in H$ then 
              $h + (-1) \in H$. Prove that $H = \{x \in \mathbb{Z} \mid x \le a\}$.
    \end{enumerate}
\end{tcolorbox}

\begin{proof}
    We must show for a fixed $a \in \mathbb{Z}$, $\{x \in \mathbb{Z} \mid x \ge a\} \subseteq G$.
    Now, $x \ge a \iff x + (-a) \ge 0 \iff x + (-a) + 1 \ge 1$.
    So we need to show $\{x \in \mathbb{Z} \mid x + (-a) + 1 \ge 1\} \subseteq G$.
    Which is equivalent to $\{x \in \mathbb{Z} \mid x + (-a) + 1 \in \mathbb{N}\}$.

    (\textbf{Base Case})  
    Let $x = a$. Then $x + (-a) + 1 = a + (-a) + 1 = 1 \in \mathbb{N}$.
    Thus $x = a \in G$.

    (\textbf{Induction Step})  
    Suppose for some $x \in \mathbb{Z}$ that $x + (-a) + 1 \in \mathbb{N}$ and $x \in G$.  
    Then consider $x+1$. We have
    \[
        (x+1) + (-a) + 1 = (x + (-a) + 1) + 1 \in \mathbb{N}.
    \]
    By the definition of $G$, since $x \in G$, we have $x+1 \in G$.

    Thus $G = \{x \in \mathbb{Z} \mid x \ge a\}$.
\end{proof}

\begin{proof}
    We must show for a fixed $a \in \mathbb{Z}$, $\{x \in \mathbb{Z} \mid x \le a\} \subseteq H$.
    Now, $x \le a \iff a - x \ge 0 \iff a - x + 1 \ge 1$.
    So we need to show $\{x \in \mathbb{Z} \mid a - x + 1 \ge 1\} \subseteq H$.
    Which is equivalent to $\{x \in \mathbb{Z} \mid a - x + 1 \in \mathbb{N}\}$.

    (\textbf{Base Case})  
    Let $x = a$. Then $a - x + 1 = a - a + 1 = 1 \in \mathbb{N}$.
    Thus $x = a \in H$.

    (\textbf{Induction Step})  
    Suppose for some $x \in \mathbb{Z}$ that $a - x + 1 \in \mathbb{N}$ and $x \in H$.  
    Then consider $x - 1$. We have
    \[
        a - (x-1) + 1 = (a - x + 1) + 1 \in \mathbb{N}.
    \]
    By the definition of $H$, since $x \in H$, it follows that $x - 1 \in H$.

    Thus $H = \{x \in \mathbb{Z} \mid x \le a\}$.
\end{proof}


\begin{tcolorbox}[title=Extra Problem]
    There is a ``unique'' ordered integral domain 
        that satisfies Axiom 1.4.4. 
    Formulate this rigorously and prove it.
\end{tcolorbox}

\begin{theorem}
    Let $A$ and $A'$ be ordered integral domains satisfying Axiom 1.4.4.
    Let $0, 1 \in A$ and $0', 1' \in A'$
        such that $0 < 1$ and $0' < 1'$
        and for all $x \in A$, $x + 0 = x$ and for all $x' \in B$, $x' + 0' = x'$.
    Then there exists a bijective function 
    \[i: A \rightarrow B\] 
    such that $i(1) = 1'$  and for all $x, y \in A$
        the following equations and relation holds.
    \begin{enumerate}
        \item $i(x + y) = i(x) + i(y)$.
        \item $i(x - y) = i(x) - i(y)$.
        \item $i(x \cdot y) = i(x) \cdot i(y)$.
        \item $x < y \iff i(x) < i(y)$.
    \end{enumerate}
\end{theorem}

\begin{proof}
    We can apply the recursive construction to the set $B$, the element $1' \in B$, 
    and the function $s(x) = x + 1'$ on $B$, 
    to deduce that there is a unique function $i: A \rightarrow B$
    such that $i(x + 1) = i(x) + 1'$ for all $x \in A$ and $i(1) = 1'$.

    Similarly, we can construct a function $i' : B \rightarrow A$
    such that  $i'(y + 1') = i'(y) + 1$ for all $y \in B$ and $i'(1') = 1$.

    Now we show that $i'$ is the inverse of $i$.

    Consider $i' \circ i$.  
    Let $x \in A$.

    \textbf{Base case:} $x = 1$.  
    \[(i' \circ i)(1) = i'(i(1)) = i'(1') = 1 = x\]

    \textbf{Inductive step:} Suppose $x > 1$.  
    By the properties of the domain, there exists $y \in A$ such that $y + 1 = x$.  
    Suppose for $y \in A$ with $y < x$ we have $(i' \circ i)(y) = y$. Then
    \begin{align*}
        (i' \circ i)(x) 
        &= i'(i(y + 1)) \\
        &= i'(i(y) + 1')  \\
        &= i'(i(y)) + 1 \\
        &= y + 1 && \text{(induction hypothesis)} \\
        &= x
    \end{align*}

    Now consider $i \circ i'$.  
    Let $y \in B$.

    \textbf{Base case:} $y = 1'$.  
    \[(i \circ i')(1') = i(i'(1')) = i(1) = 1' = y\]

    \textbf{Inductive step:} Suppose $y > 1'$.  
    By the properties of the domain, there exists $y_0 \in B$ such that $y_0 + 1' = y$.  
    Suppose for $y_0 < y$ we have $(i \circ i')(y_0) = y_0$. Then
    \begin{align*}
        (i \circ i')(y) 
        &= i(i'(y_0 + 1')) \\
        &= i(i'(y_0) + 1) \\
        &= i(i'(y_0)) + 1' \\
        &= y_0 + 1' && \text{(induction hypothesis)} \\
        &= y
    \end{align*}

    Since $(i' \circ i)(x) = x$ and $(i \circ i')(y) = y$, we conclude that $i'$ is the inverse of $i$.  
    Thus $i$ is bijective.

    Finally, we check that $i$ preserves all operations:

    \begin{itemize}
        \item Addition: By construction, $i(x + 1) = i(x) + 1'$; using induction on sums $x + y$, we get $i(x + y) = i(x) + i(y)$ for all $x, y \in A$.
        \item Subtraction: $i(x - y) = i(x) - i(y)$ follows from the additive inverse and induction.
        \item Multiplication: Using induction on $y$, $i(x \cdot 1) = i(x) \cdot 1'$ and $i(x \cdot (y + 1)) = i(x \cdot y) + i(x)$, giving $i(x \cdot y) = i(x) \cdot i(y)$.
        \item Order: By definition, $x < y \iff \exists z \neq 0$ with $x + z = y$; using preservation of addition and $i(0) = 0'$, we have $i(x) < i(y) \iff x < y$.
    \end{itemize}
\end{proof}

\subsection{Constructing the Rational Numbers}

\begin{tcolorbox}[title=Problem 1, breakable]
    Complete the proof of Lemma 1.5.2. That is,
    prove that the relation $\asymp$ is reflexive 
    and symmetric.
\end{tcolorbox}

\begin{proof}
    Let $(a, b), (c, d) \in \mathbb{Q} \times \mathbb{Q}^*$.

    We must show $(a, b) \asymp (a, b)$.
    But $ab = ab$ thus $(a, b) \asymp (a, b)$.

    Suppose $(a, b) \asymp (c, d)$.
    We must show $(c, d) \asymp (a, b)$.
    But $ad = bc \iff cb = da$ thus $(c, d) \asymp (a, b)$.
\end{proof}

\begin{tcolorbox}[title=Redefining $<$]
    Let $[(a, b)] \in \mathbb{Q}$.
    Define $[(a,b)]$ to be in $P$ iff both $a,b>0$ or both $a,b<0$.
    \begin{enumerate}
        \item If $[(a,b)] = [(c,d)]$ then $[(a,b)]$ in $P$ iff $[(c,d)]$ in $P$.
        \item Define $x<y$ if and only if $y-x$ in $P$.
        \item Show $x<y$ if and only if Definition 1.5.3 is satisfied 
             (this simplifies the proof of < being well-defined).
        \item Show that if $x,y$ in $P$, then $x+y$ and $xy$ in $P$.
        \item Show that for any nonzero $x$ in $\mathbb{Q}$, 
                exactly one of $x$, $-x$ is in $P$.
    \end{enumerate}
\end{tcolorbox}

\begin{proof}
    Let $(x_1, x_2), (y_1, y_2) \in \mathbb{Q} \times \mathbb{Q}^*$
        and let $x, y \in \mathbb{Q}$ such that $x = [(x_1, x_2)]$
            and $y = [(y_1, y_2)]$.
    We first show $x<y$ if and only if Definition 1.5.3 is satisfied.
    Definition 1.5.3 states $<$ on $\mathbb{Q}$ is defined by 
    \begin{align*}
    [(x_1, x_2)] < [(y_1, y_2)] 
        &\iff (x_2 > 0 \wedge y_2 > 0 \wedge x_1 y_2 < y_1 x_2) &&\textcircled{1} \\
        &\quad \vee (x_2 < 0 \wedge y_2 < 0 \wedge x_1 y_2 < y_1 x_2) &&\textcircled{2} \\
        &\quad \vee (x_2 > 0 \wedge y_2 < 0 \wedge x_1 y_2 > y_1 x_2) &&\textcircled{3} \\
        &\quad \vee (x_2 < 0 \wedge y_2 > 0 \wedge x_1 y_2 > y_1 x_2) &&\textcircled{4}
    \end{align*}

    ($\longrightarrow$)
    Suppose $x < y$. It follows that $y - x \in P$. Then
    \[[(y_1, y_2)] - [(x_1, x_2)] 
        = [(y_1, y_2)] + [(-x_1, x_2)] 
        = [(y_1 x_2 - x_1y_2, y_2 x_2)] \in P\]
    There are two cases.
    Either  $(y_1 x_2 - x_1 y_2), (y_2 x_2) > 0$ or
        or  $(y_1 x_2 - x_1 y_2), (y_2 x_2) < 0$.

    \textbf{(Case 1)} Suppose $(y_1 x_2 - x_1 y_2), (y_2 x_2) < 0$.
    Since $y_2 x_2 > 0$ either $y_2, x_2 > 0$ or $y_2 x_2 < 0$.
    So, suppose $y_2, x_2 > 0$ we see \textcircled{1} holds.
    Similarly, suppose $y_2, x_2 < 0$ we see \textcircled{2} holds.

    \textbf{(Case 2)} Suppose $(y_1 x_2 - x_1 y_2), (y_2 x_2) < 0$.
    Since $y_2 x_2 < 0$ either $y_2 > 0$ and $x_2 < 0$ or $y_2 < 0$ and $x_2 > 0$.
    So, suppose $y_2 > 0, x_2 < 0$ we see \textcircled{3} holds.
    Similarly, suppose $y_2 < 0, x_2 > 0$ we see \textcircled{4} holds.

    So we have shown if $x < y$ then one case of Definition 1.5.3 holds.

    ($\longleftarrow$)
    Suppose $[(x_1, x_2)] < [(y_1, y_2)]$ then one of 
        \textcircled{1}, \textcircled{2}, \textcircled{3}, \textcircled{4} holds.
    
    \textbf{Case \textcircled{1}}: $x_2 > 0$, $y_2 > 0$, and $x_1 y_2 < y_1 x_2$.
    Then $y_1 x_2 - x_1 y_2 > 0$ and $y_2 x_2 > 0$, so $[(y_1 x_2 - x_1 y_2, y_2 x_2)] \in P$.  
    Thus $y - x \in P$ and therefore $x < y$.

    \textbf{Case \textcircled{2}}: $x_2 < 0$, $y_2 < 0$, and $x_1 y_2 < y_1 x_2$.
    Then $y_1 x_2 - x_1 y_2 > 0$ and $y_2 x_2 > 0$, so $[(y_1 x_2 - x_1 y_2, y_2 x_2)] \in P$.  
    Thus $y - x \in P$ and therefore $x < y$.

    \textbf{Case \textcircled{3}}: $x_2 > 0$, $y_2 < 0$, and $x_1 y_2 > y_1 x_2$.
    Then $y_1 x_2 - x_1 y_2 < 0$ and $y_2 x_2 < 0$, so $[(y_1 x_2 - x_1 y_2, y_2 x_2)] \in P$.  
    Thus $y - x \in P$ and therefore $x < y$.

    \textbf{Case \textcircled{4}}: $x_2 < 0$, $y_2 > 0$, and $x_1 y_2 > y_1 x_2$.
    Then $y_1 x_2 - x_1 y_2 < 0$ and $y_2 x_2 < 0$, so $[(y_1 x_2 - x_1 y_2, y_2 x_2)] \in P$.  
    Thus $y - x \in P$ and therefore $x < y$.

    Thus proving our theorem.
\end{proof}

\begin{proof}
    Let $x, y \in \mathbb{Z}$.
    Suppose $x, y \in P$.
    Let $[(x_1, x_2)] = x$ and $[(y_1, y_2)] = y$
        such that $x_1, x_2, y_1, y_2 \in \mathbb{Z}$.
    By definition of $+$,
        $x + y = [(x_1 y_2 + y_1 x_2, x_2 y_2)]$.
    There are four cases.

    \textbf{Case ($x_1, x_2, y_1, y_2 > 0$)}
    Now, $y_1 > 0$ and since $x_2 > 0$,
        $y_1 x_2 > 0 \cdot x_2 \iff y_1 x_2 > 0$.
    Similarly $x_1 y_2 > 0$.
    Then $y_1 x_2 + x_1 y_2 > 0 + x_1 y_2 \iff y_1 x_2 + x_1 y_2 > x_1 y_2 > 0$.
    Thus $x_1 y_2 + y_1 x_2 > 0$.
    Now, $x_2 > 0$ and since $y_2 > 0$, $x_2 y_2 > 0 \cdot x_2 \iff x_2 y_2 > 0$.
    Since $x_1 y_2 + y_1 x_2 > 0$ and $x_2 y_2 > 0$,
        it follows that $x + y > 0$.

    \textbf{Case ($x_1, x_2, y_1, y_2 < 0$)}
    Now, $y_1 < 0$ and since $x_2 < 0$,
        $y_1 x_2 < 0 \cdot x_2 \iff y_1 x_2 > 0$.
    Similarly $x_1 y_2 < 0 \cdot y_2 \iff x_1 y_2 > 0$.
    Then $y_1 x_2 + x_1 y_2 > 0 + x_1 y_2 \iff y_1 x_2 + x_1 y_2 > x_1 y_2 > 0$.
    Now, $x_2 < 0$ and since $y_2 < 0$, $x_2 y_2 < 0 \cdot x_2 \iff x_2 y_2 > 0$.
    Since $x_1 y_2 + y_1 x_2 > 0$ and $x_2 y_2 > 0$,
        it follows that $x + y > 0$.

    \textbf{Case ($x_1, x_2 > 0$, $y_1, y_2 < 0$)}
    Now, $y_1 < 0$ and since $x_2 > 0$, $y_1 x_2 < 0 \cdot x_2 \iff y_1 x_2 < 0$.
    Similarly, $x_1 y_2 > 0 \cdot y_2 \iff x_1 y_2 < 0$.
    Then $y_1 x_2 + x_1 y_2 < 0 + x_1 y_2 \iff y_1 x_2 + x_1 y_2 < x_1 y_2 < 0$.
    Now, $x_2 > 0$ and $y_2 < 0$, so $x_2 y_2 > 0 \cdot y_2 \iff x_2 y_2 < 0$.
    Since $y_1 x_2 + x_1 y_2 < 0$ and $x_2 y_2 < 0$,
        it follows that $x + y > 0$.

    \textbf{Case ($x_1, x_2 < 0$, $y_1, y_2 > 0$)}
    Now, $y_1 > 0$ and $x_2 < 0$, so $y_1 x_2 < 0 \cdot x_2 \iff y_1 x_2 < 0$.
    Similarly, $x_1 y_2 < 0 \cdot y_2 \iff x_1 y_2 < 0$.
    Then $y_1 x_2 + x_1 y_2 < 0 + x_1 y_2 \iff y_1 x_2 + x_1 y_2 < x_1 y_2 < 0$.
    Now, $x_2 < 0$ and $y_2 > 0$, so $x_2 y_2 < 0 \cdot y_2 \iff x_2 y_2 < 0$.
    Since $y_1 x_2 + x_1 y_2 < 0$ and $x_2 y_2 < 0$,
        it follows that $x + y > 0$.
\end{proof}

\begin{proof}
    Let $x, y \in \mathbb{Z}$.
    Suppose $x, y \in P$.
    Let $[(x_1, x_2)] = x$ and $[(y_1, y_2)] = y$
        such that $x_1, x_2, y_1, y_2 \in \mathbb{Z}$.
    By definition of $\cdot$,
        $xy = [(x_1 y_1, x_2 y_2)]$.
    There are four cases.

    \textbf{Case ($x_1, x_2, y_1, y_2 > 0$)}
    Since $x_1 > 0$ and $y_1 > 0$ it follows that $x_1 y_1 > 0$.
    Since $x_2 > 0$ and $y_2 > 0$ it follows that $x_2 y_2 > 0$.
    Therefore $xy = [(x_1 y_1, x_2 y_2)] \in P$.

    \textbf{Case ($x_1, x_2, y_1, y_2 < 0$)}
    Since $x_1 < 0$ and $y_1 < 0$ it follows that $x_1 y_1 > 0$.
    Since $x_2 < 0$ and $y_2 < 0$ it follows that $x_2 y_2 > 0$.
    Therefore $xy = [(x_1 y_1, x_2 y_2)] \in P$.

    \textbf{Case ($x_1, x_2 > 0$, $y_1, y_2 < 0$)}
    Since $x_1 > 0$ and $y_1 < 0$ it follows that $x_1 y_1 < 0$.
    Since $x_2 > 0$ and $y_2 < 0$ it follows that $x_2 y_2 < 0$.
    Therefore $xy = [(x_1 y_1, x_2 y_2)] \in P$.

    \textbf{Case ($x_1, x_2 < 0$, $y_1, y_2 > 0$)}
    Since $x_1 < 0$ and $y_1 > 0$ it follows that $x_1 y_1 < 0$.
    Since $x_2 < 0$ and $y_2 > 0$ it follows that $x_2 y_2 < 0$.
    Therefore $xy = [(x_1 y_1, x_2 y_2)] \in P$.
\end{proof}

\begin{proof}
    Let $x \in \mathbb{Q}$ such that $x \ne 0$.
    Let $x = [(x_1, x_2)]$ such that $x_1, x_2 \in \mathbb{Z}$.
    We want to show that exactly one of $x$ or $-x$ is in $P$.
    Consider $-x = [(-x_1, x_2)]$.
    There are four cases.

    \textbf{Case ($x_1, x_2 > 0$)} 
    Then $x \in P$ by definition.  
    For $-x = [(-x_1, x_2)]$, so $-x_1 < 0$ and $x_2 > 0$, so $-x \notin P$.  
    Thus exactly one of $x$ or $-x$ is in $P$.

    \textbf{Case ($x_1, x_2 < 0$)}
    Then $x \in P$ by definition.  
    For $-x = [(-x_1, x_2)]$, so $-x_1 > 0$ and $x_2 < 0$, so $-x \notin P$.  
    Thus exactly one of $x$ or $-x$ is in $P$.

    \textbf{Case ($x_1 > 0, x_2 < 0$)}
    Then $x \notin P$ by definition.  
    For $-x = [(-x_1, x_2)]$, so $-x_1 < 0$ and $x_2 < 0$, so $-x \in P$.  
    Thus exactly one of $x$ or $-x$ is in $P$.

    \textbf{Case ($x_1 < 0, x_2 > 0$)}
    Then $x \notin P$ by definition.  
    For $-x = [(-x_1, x_2)]$, so $-x_1 > 0$ and $x_2 > 0$, so $-x \in P$.  
    Thus exactly one of $x$ or $-x$ is in $P$.
\end{proof}


\begin{tcolorbox}[title=Problem 2, breakable]
    Complete the proof of Lemma 1.5.4. That is,
    prove that the binary relation $+$,
    the unary operation $^{-1}$ and the relation 
    $<$, on all $\mathbb{Q}$, are well-defined.
\end{tcolorbox}

\begin{proof}
    Let $(a, b), (c, d), (x, y), (z, w) \in \mathbb{Q} \times \mathbb{Q}^*$.
    Suppose $(a, b) \asymp (c, d)$ and $(x, y) \asymp (z, w)$.
    Thus $ad = bc$ and $xw = zy$.

    Then 
    \begin{align*}
        &[(a, b)] + [(x, y)] = [(c, d)] + [(z, w)] \\
        \iff &(ay + bx, by) \asymp (cw + dz, dw) \\
        \iff &(ay + bx)dw = (cw + dz)by \\
        \iff &adyw + bxdw = cbyw + dzby.
    \end{align*}
    Since $ad = bc$ and $xw = zy$ it follows that $bcyw + b zyd = bcyw + bdzy$
    which holds. Thus $(ay + bx, by) \asymp (cw + dz, dw)$.

    Suppose $a \ne 0$ and $c \ne 0$.
    $[(a, b)]^{-1} = [(b, a)]$ and $[(c, d)]^{-1} = [(d, c)]$.
    Then $(b, a) \asymp (d, c) \iff bc = da$ which holds.
    Thus $[(a, b)]^{-1} = [(b, a)] = [(d, c)] = [(c, d)]^{-1}$

    Suppose $[(a, b)] < [(x, y)]$.
    Then $[(a, b)] - [(x, y)] \in P$.
    But $[(a, b)] = [(c, d)]$ and $[(x, y)] = [(z, w)]$
        so $[(c, d)] - [(z, w)] \in P$.
    It follows that $[(c, d)] < [(z, w)]$.
\end{proof}

\begin{tcolorbox}[title=Problem 3, breakable]
    Let $x \in \mathbb{Z}$ and $y \in \mathbb{Z}^*$.
    \begin{enumerate}
        \item Prove that $[(x, y)] = \bar{0}$ if and only if $x = 0$.
        \item Prove that $[(x, y)] = \bar{1}$ if and only if $x = y$.
        \item Prove that $\bar{0} < [(x, y)]$ if and only if $0 < xy$.
    \end{enumerate}
\end{tcolorbox}

\begin{proof}
    Suppose $[(x, y)] = \bar{0}$.
    Then $[(x, y)] = [(0, 1)]$.
    It follows that $x \cdot 1 = y \cdot 0$.
    Thus $x = 0$.

    Suppose $x = 0$.
    Then $x \cdot 1 = y \cdot 0$
    It follows that $[(x, y)] = [(0, 1)]$.
    Thus $[(x, y)] = \bar{0}$.
\end{proof}

\begin{proof}
    Suppose $[(x, y)] = \bar{1}$.
    Then $[(x, y)] = [(1, 1)]$.
    It follows that $x \cdot 1 = y \cdot 1$.
    Thus $x = y$.

    Suppose $x = y$.
    Then $x \cdot 1 = y \cdot 1$.
    It follows that  $[(x, y)] = [(1, 1)]$.
    Thus $[(x, y)] = \bar{1}$.
\end{proof}

\begin{proof}
    Suppose $\bar{0} < [(x, y)]$.
    By definition, $[(x, y)] - \bar{0} \in P$.
    But $\bar{0} = [(0,1)]$, so
    \[[(x, y)] - [(0, 1)] = [(x \cdot 1 - 0 \cdot y, y \cdot 1)] = [(x, y)] \in P\]
    Thus either $x, y > 0$ or $x, y < 0$, so in either case $xy > 0$.

    Suppose $xy > 0$.
    Then either $x, y > 0$ or $x, y < 0$.
    Thus $[(x, y)] \in P$, so $\bar{0} < [(x, y)]$.
\end{proof}

\begin{tcolorbox}[title=Problem 4, breakable]
    Prove Theorem 1.5.5 (1) (2) (3) (5) (6) (8) (9) (11) (12) (14).
\end{tcolorbox}

\begin{proof}
    Let $r, s, t \in \mathbb{Q}$.
    We must show $(r + s) + t = r + (s + t)$.
    Let $(r_1, r_2), (s_1, s_2), (t_1, t_2) \in \mathbb{Z} \times \mathbb{Z}^*$.
    Then 
    \begin{align*}
        (r + s) + t 
            &= ([(r_1, r_2)] + [(s_1, s_2)]) + [(t_1, t_2)] \\
            &= [(r_1 s_2 + r_2 s_1, r_2 s_2)] + [(t_1, t_2)] \\
            &= [((r_1 s_2 + r_2 s_1) t_2 + t_1 (r_2 s_2), r_2 s_2 t_2)] \\
            &= [(r_1 s_2 t_2 + r_2 s_1 t_2 + t_1 r_2 s_2, r_2 s_2 t_2)] \\
            &= [(r_1 s_2 t_2 + (r_2 s_1 t_2 + t_1 r_2 s_2), r_2 s_2 t_2)] \\
            &= [(r_1 s_2 t_2 + r_2 (s_1 t_2 + s_2 t_1), r_2 s_2 t_2)] \\
            &= [(r_1, r_2)] + [(s_1 t_2 + s_2 t_1, s_2 t_2)] \\
            &= [(r_1, r_2)] + ([(s_1, s_2)] + [(t_1, t_2)]) \\
            &= r + (s + t)
    \end{align*}
\end{proof}

\begin{proof}
    Let $r, s \in \mathbb{Q}$.
    We must show $r + s = s + r$.
    Let $(r_1, r_2), (s_1, s_2)\in \mathbb{Z} \times \mathbb{Z}^*$.
    Then 
    \begin{align*}
        r + s 
            &= [(r_1, r_2)] + [(s_1, s_2)] \\
            &= [(r_1 s_2 + s_1 r_2, r_2 s_2)] \\
            &= [(s_1 r_2 + r_1 s_2, s_2 r_2)] \\
            &= [(s_1, s_2)] + [(r_1, r_2)] \\
            &= s + r
    \end{align*}
\end{proof}

\begin{proof}
    Let $r \in \mathbb{Q}$.
    We must show $r + \bar{0} = r$.
    Let $(r_1, r_2) \in \mathbb{Z} \times \mathbb{Z}^*$.
    Then 
    \begin{align*}
        r + \bar{0} = 
            &= [(r_1, r_2)] + [(0, 1)] \\
            &= [(r_1 \cdot 1 + 0 \cdot r_2, r_2 \cdot 1)] \\
            &= [(r_1, r_2)] \\
            &= r
    \end{align*}
\end{proof}

\begin{proof}
    Let $r, s, t \in \mathbb{Q}$.
    We must show $(rs)t = r(st)$.
    Let $(r_1, r_2), (s_1, s_2), (t_1, t_2) \in \mathbb{Z} \times \mathbb{Z}^*$.
    Then 
    \begin{align*}
        (rs)t 
            &= ([(r_1, r_2)] \cdot [(s_1, s_2)]) \cdot [(t_1, t_2)] \\
            &= [(r_1 s_1, r_2 s_2)] \cdot [(t_1, t_2)] \\
            &= [(r_1 s_1 t_1, r_2 s_2 t_2)] \\
            &= [(r_1, r_2)] \cdot [(s_1 t_1, s_2 t_2)] \\
            &= [(r_1, r_2)] \cdot ([(s_1, s_2)] \cdot [(t_1, t_2)]) \\
            &= r (s t)
    \end{align*}
\end{proof}

\begin{proof}
    Let $r, s \in \mathbb{Q}$.
    We must show $rs = sr$.
    Let $(r_1, r_2), (s_1, s_2) \in \mathbb{Z} \times \mathbb{Z}^*$.
    Then 
    \begin{align*}
        rs &= [(r_1, r_2)] \cdot [(s_1, s_2)] \\
           &= [(r_1 s_1, r_2 s_2)] \\
           &= [(s_1 r_1, s_2 r_2)] \\
           &= [(s_1, s_2)] \cdot [(r_1, r_2)] \\
           &= sr
    \end{align*}
\end{proof}

\begin{proof}
    Let $r \in \mathbb{Q}$.
    We must show if $r \ne \bar{0}$, then $r \cdot r^{-1} = \bar{1}$.
    Let $(r_1, r_2) \in \mathbb{Z} \times \mathbb{Z}^*$.
    Suppose $r \ne \bar{0}$.
    Then 
    \begin{align*}
        r \cdot r^{-1}
            &= [(r_1, r_2)] \cdot [(r_1, r_2)]^{-1} \\
            &= [(r_1, r_2)] \cdot [(r_2, r_1)] && \text{$r \ne \bar{0} \implies r_1 \ne 0 \implies r_1 \in \mathbb{Z}^*$} \\
            &= [(r_1 r_2, r_2 r_1)] \\
            &= \bar{1} && \text{Problem 3 b}
    \end{align*}
\end{proof}

\begin{proof}
    Let $r, s, t \in \mathbb{Q}$.
    We must show $r(s + t) = rs + rt$.
    Let $(r_1, r_2), (s_1, s_2), (t_1, t_2) \in \mathbb{Z} \times \mathbb{Z}^*$.
    \begin{align*}
        r(s + t) &= [(r_1, r_2)]([(s_1, s_2)] + [(t_1, t_2)]) \\ 
                 &= [(r_1, r_2)]([s_1 t_2 + t_1 s_2, s_2 t_2]) \\ 
                 &= [(r_1(s_1 t_2 + t_1 s_2), r_2 (s_2 t_2))] \\ 
                 &= [(r_1 s_1 t_2 + r_1 t_1 s_2, r_2 s_2 t_2)] \\ 
                 &= [(r_1 s_1, r_2 s_2)] + [(r_1 t_1, r_2 t_2)] \\ 
                 &= rs + rt.
    \end{align*}
\end{proof}

\begin{proof}
    Let $r, s, t \in \mathbb{Q}$.
    We must show if $r < s$ and $s < t$, then $r < t$.
    Suppose $r < s$ and $s < t$.
    It follows that $s - r \in P$ and $t - s \in P$.
    It follows that $(s - r) + (t - s) = t - r \in P$.
    Thus $r < t$.
\end{proof}

\begin{proof}
    Let $r, s, t \in \mathbb{Q}$.
    We must show if $r < s$ then $r + t < s + t$.
    Supose $r < s$. 
    Then $s - r = s - r + 0 = s - r + t + (-t) = (s + t) - (r + t) \in P$.
    Thus $r + t < s + t$.
\end{proof}

\begin{proof}
    We must show $\bar{0} \ne \bar{1}$.
    Suppose $\bar{0} = \bar{1}$.
    Then $[(0, 1)] = [(1, 1)] \iff 0 \cdot 1 = 1 \cdot 1 \iff 0 = 1$
        which is a contradiction.
    Thus $\bar{0} \ne \bar{1}$.
\end{proof}

\begin{tcolorbox}[title=Problem 5, breakable]
    Prove Theorem 1.5.6 (1) (2) (3).
\end{tcolorbox}

\begin{theorem}
    Let $i : \mathbb{Z}  \longrightarrow  \mathbb{Q}$
        be defined by $i(x) = [(x, 1)]$ for all $x \in \mathbb{Z}$.
    \begin{enumerate}
        \item Then function $i: \mathbb{Z} \longrightarrow \mathbb{Q}$ is injective.
        \item $i(0) = \bar{0}$ and $i(1) = \bar{1}$.
        \item Let $x, y \in \mathbb{Z}$. Then 
              \begin{enumerate}
                \item $i(x + y) = i(x) + i(y)$;
                \item $i(-x) = -i(x)$;
                \item $i(xy) = i(x) i(y)$;
                \item $x < y$ if and only if $i(x) < i(y)$
              \end{enumerate}
        \item For each $r \in \mathbb{Q}$ there are $x, y \in \mathbb{Z}$
              such that $y \ne 0$ and $r = i(x)(i(y))^{-1}$.
    \end{enumerate}
\end{theorem}

\begin{proof}
    Let $x, y \in \mathbb{Z}$.
    Suppose $i(x) = i(y)$.
    Thus $[(x, 1)] = [(y, 1)]$ so $(x, 1) \asymp (y, 1)$.
    It follows that $x \cdot 1 = y \cdot 1$.
    From the Identity Law for Multiplication $x = y$.
    Thus $i$ is injective.
\end{proof}

\begin{proof}
    Notice $i(0) = [(0, 1)] = \bar{0}$
        and $i(1) = [(1, 1)] = \bar{1}$.
\end{proof}

\begin{proof}
    Let $x, y \in \mathbb{Z}$.
    Then 
    \[i(x + y) = [(x + y, 1)] = [(x \cdot 1 + y \cdot 1, 1 \cdot 1)] = [(x, 1)] + [(y, 1)] = i(x) + i(y)\]
    Similarly 
    \[i(-x) = [(-x, 1)] = -[(x, 1)] = -i(x)\]
    Similarly 
    \[i(xy) = [(xy, 1)] = [(xy, 1 \cdot 1)] = [(x, 1)] \cdot [(y, 1)] = i(x) i(y)\]
    Finally suppose $x < y$. Then $i(y - x) = [(y - x, 1)] \in P$ since $y - x \in P$ and $1 \in P$.
    Thus 
    \[[(y - x, 1)] =  [(y \cdot 1 - x \cdot 1, 1 \cdot 1)] = [(y, 1)] - [(x, 1)] = i(y) - i(x) \in P\]
    It follows that $i(x) < i(y)$.
\end{proof}

\begin{tcolorbox}[title=Problem 6, breakable]
    Let $r, s, p, q \in \mathbb{Q}$.
    \begin{enumerate}
        \item Prove that $-1 < 0 < 1$.
        \item Prove that if $r < s$ then $-s < -r$.
        \item Prove that $r \cdot 0 = 0$.
        \item Prove that if $r > 0$ and $s > 0$, then $r + s > 0$ and $rs > 0$.
        \item Prove that if $r > 0$, then $\frac{1}{r} > 0$.
        \item Prove that if $0 < r < s$, then $\frac{1}{s} < \frac{1}{r}$.
        \item Prove that if $0 < r < p$ and $0 < s < q$, then $rs < pq$.
    \end{enumerate}
\end{tcolorbox}

\begin{proof}
    By Theorem 1.5.5 Part 14,
        $0 \ne 1$.
    By Theorem 1.5.5 Part 10, either $1 < 0$ or $0 < 1$.
    By our definition of $<$
        either $-1 \in P$ or $1 \in P$.
    Suppose $-1 \in P$.
    Again, by our definition of $<$, $(-1) \cdot (-1) \in P$.
    But $(-1) \cdot (-1) = 1 \in P$, which is a contradiction.
    Thus $1 \in P$ and $-1 \notin P$. 
\end{proof}

\begin{proof}
    Suppose $r < s$.
    Then $s - r \in P$.
    But $(-r) - (-s) = s - r \in P$.
    Therefore $-s < -r$.
\end{proof}

\begin{proof}
    Let $r = [(a, b)]$ such that $a, b \in \mathbb{Z}$, $b \neq 0$.
    \[
        r \cdot 0 = [(a, b)] \cdot [(0, 1)] = [(a \cdot 0, b \cdot 1)] = [(0, b)] = [(0,1)] = 0
    \]
\end{proof}

\begin{proof}
    Suppose $r > 0$ and $s > 0$.
    Our definition of $<$ showed that $r + s > 0$ and $rs > 0$.
\end{proof}

\begin{proof}
    Suppose $r > 0$ and for contradiction $\frac{1}{r} < 0$.
    Then $-\frac{1}{r} > 0$.
    Then $r \cdot -\frac{1}{r} = -\frac{r}{r} = -1 \notin P$
        which is a contradiction. 
\end{proof}

\begin{proof}
    Suppose $0 < r < s$.
    Since $\frac{1}{r} > 0$ by the previous part, by Theorem 1.5.5 Part 13,
        $r \cdot \frac{1}{r} < s \cdot \frac{1}{r} \iff 1 < s \cdot \frac{1}{r}$.
    Similarly, multiplying both sides by $\frac{1}{s} > 0$ shows
        $\frac{1}{s} < \frac{1}{r}$.
\end{proof}

\begin{proof}
    Suppose $0 < r < p$ and $0 < s < q$.
    Since $r < p$ and $0 < s$, by Theorem 1.5.5 Part 13,
        $r s < p s$.
    Similarly, since $s < q$ and $p > 0$,
        it follows that $p s < p q$.
    Then by transitive law, $r s < p q$ as required.
\end{proof}

\begin{tcolorbox}[title=Problem 7, breakable]
    \begin{enumerate}
        \item Prove that $1 < 2$.
        \item Let $s,t \in \mathbb{Q}$. Suppose $s < t$.
              Prove that $\frac{s + t}{2} \in \mathbb{Q}$, 
              and that $s < \frac{s + t}{2} < t$.
    \end{enumerate}
\end{tcolorbox}

\begin{proof}
    By Problem 6 Part 1, $0 < 1$. 
    By Addition Law for Order, $0 + 1 < 1 + 1 \iff 1 < 2$.
\end{proof}

\begin{proof}
    Now $\frac{1}{2} \in \mathbb{Q}$.
    Since $\mathbb{Q}$ is closed under 
        multiplication and addition 
        $(s + t) \cdot \frac{1}{2} = (s + t)2^{-1} = \frac{s + t}{2} \in \mathbb{Q}$.

    We now show $\frac{s + t}{2} < t$.
    First, notice $t - \frac{s + t}{2} = \frac{t}{1} + \frac{-(s + t)}{2} = \frac{2t + (-(s + t)) 1}{2 \cdot 1}$.
    Clearly $2 \in P$ and $2t + (-(s + t)) 1 = 2t -s - t = t - s$. 
    Since $s < t$ it follows that $t - s \in P$.
    Thus $t - \frac{s + t}{2} \in P$.
    Therefore $\frac{s + t}{2} < t$.

    We now show $\frac{s + t}{2} > s$.
    First, notice $\frac{s + t}{2} - s = \frac{s + t}{2} + \frac{-s}{1} = \frac{(s + t)1 + (-2)s}{2 \cdot 1}$.
    Clearly $2 \in P$ and $(s + t)1 + (-2)s = t - s$.
    Since $s < t$ it follows that $t - s \in P$.
    Thus $\frac{s + t}{2} - s \in P$.
    Therefore $\frac{s + t}{2} > s$.

    It follows that $s < \frac{s + t}{2} < t$.
\end{proof}

\begin{tcolorbox}[title=Problem 8, breakable]
    Let $r \in \mathbb{Q}$. Suppose that $r > 0$.
    \begin{enumerate}
        \item Prove that if $r = \frac{a}{b}$ for some $a, b \in \mathbb{Z}$
              such that $b \ne 0$, then either $a > 0$ and $b > 0$,
              or $a < 0$ and $b < 0$.
        \item Prove that $r = \frac{m}{n}$ for some $m, n \in \mathbb{Z}$
              such that $m > 0$ and $n > 0$.
    \end{enumerate}
\end{tcolorbox}

\begin{proof}
    Suppose $r = \frac{a}{b}$ for some $a, b \in \mathbb{Z}$
        such that $b \ne 0$.
    Since $r > 0$ it follows that $a, b \in P$ or $a, b \notin P$.
    Suppose $a, b \in P$. Then $a = a - 0 \in P$ and $b = b - 0 \in P$.
    Thus $a > 0$ and $b > 0$.
    Suppose $a, b \notin P$. Then $a = a - 0 \notin P$ and $b = b - 0 \notin P$.
    Thus $-(a - 0) = -a + 0 = 0 - a \in P$ and $-(b - 0) = -b + 0 = 0 - b \in P$.
    Thus $0 > a$ and $0 > b$.
\end{proof}

\begin{proof}
    Suppose $r = \frac{a}{b}$ for some $a,b \in \mathbb{Z}$ with $b \ne 0$.
    By part (1), either $a,b>0$ or $a,b<0$.
    If $a,b>0$, let $m = a$ and $n = b$.
    Suppose $a,b<0$. Then $-a>0$ and $-b>0$.
    Also, $(-a)(-b) = -(-ab) = ab$ so $(a,b)\asymp(-a,-b)$.  
    Thus let $m=-a$ and $n=-b$.
\end{proof}

\begin{tcolorbox}[title=Problem 9, breakable]
    Let $r, s \in \mathbb{Q}$.
    \begin{enumerate}
        \item Suppose $r > 0$ and $s > 0$. Prove that there is some $n \in \mathbb{N}$
              such that $s < nr$.
        \item Suppose that $r > 0$. Prove that there is some $m \in \mathbb{N}$
              such that $\frac{1}{m} < r$.
        \item For each $x \in \mathbb{Q}$, let $x^2$ denote $x \cdot x$.
              Suppose that $r > 0$ and $s > 0$. Prove that if $r^2 < p$,
              then there is some $k \in \mathbb{N}$ such that 
              $(r + \frac{1}{k})^2 < p$.
    \end{enumerate}
\end{tcolorbox}

\begin{proof}
    Since $r, s > 0$, let $r = \frac{a}{b}$ and $s = \frac{c}{d}$ such that $a,b,c,d \in \mathbb{N}$.
    Furthermore, $n = bc + 1$.
    Then $nr - s 
        = n \cdot \frac{a}{b} - \frac{c}{d}
        = n \cdot \left(\frac{a}{b} \cdot \frac{d}{d}\right) - \left(\frac{c}{d} \cdot \frac{b}{b}\right)
        = \frac{adn}{bd} - \frac{bc}{bd} 
        = \frac{adn - bc}{bd} 
        = \frac{ad(bc + 1) - bc}{bd}
        = \frac{ad(bc) + ad - bc}{bd}
        = \frac{bc(ad - 1) + ad}{bd}$.
    Now $a, d \in \mathbb{N}$ thus $ad \ge 1$.
    Thus $bc(ad - 1) + ad > 0$.
    It follows that $s < nr$ as required.
\end{proof}

\begin{proof}
    Since $r > 0$, let $r = \frac{a}{b}$ such that $a, b \in \mathbb{N}$.
    Furthermore, let $m = b + 1$.
    Then $r - \frac{1}{m} 
        = \frac{a}{b} - \frac{1}{b + 1} 
        = \frac{a(b + 1)}{b(b + 1)} - \frac{b}{b(b + 1)}
        = \frac{ab - b + a}{b(b + 1)}$.
    Now, $a,b \in \mathbb{N}$ thus $ab > b$.
    Thus $ab - b + a > 0$.
    It follows that $\frac{1}{m} < r$ as required.
\end{proof}

\begin{proof}
    Let $k$ be an arbitrary natural number. Notice
    $(r + \frac{1}{k})^2 < p 
        \iff r^2  + \frac{2r}{k} + \frac{1}{k^2} < p$.
    Now $r^2  + \frac{2r}{k} + \frac{1}{k^2} < r^2  + \frac{2r}{k} + \frac{1}{k}$
        thus it remains to show that $r^2  + \frac{2r}{k} + \frac{1}{k} < p$.
    Now 
    \begin{align*}
        p - (r^2  + \frac{2r}{k} + \frac{1}{k})
            &= \frac{pk}{k} - \frac{kr^2}{k} - \frac{2r}{k} + \frac{1}{k} \\
            &= \frac{pk - kr^2 - 2r - 1}{k}
    \end{align*}
    Clearly $k > 0$ since $k \in \mathbb{N}$ so it remains 
        to show that $pk - kr^2 - 2r - 1 > 0$ for some $k \in \mathbb{N}$.
    Now, since $r, p > 0$, let $r = \frac{a}{b}$ and $p = \frac{c}{d}$
        such that $a, b, c, d \in \mathbb{N}$.
    Then
    \begin{align*}
        pk - kr^2 - 2r - 1 
            &= \frac{c}{d} k - k \frac{a^2}{b^2} - 2 \frac{a}{b} - 1 \\
            &= \frac{k c b^2 - k a^2 d - 2 a b d - b^2 d}{d b^2}
    \end{align*}
    Now $d, b \in \mathbb{N}$ thus $d b^2 > 0$.
    It remains to show that there is some $k \in \mathbb{N}$
        such that $k(b^2c - a^2 d) - 2a b d - b^2 d > 0$.
    Now since $r^2 < p \iff \frac{a^2}{b^2} < \frac{c}{d} \iff a^2d < b^2 c$ it follows that $b^2 c - a^2 d \ge 1$.
    Then $k(b^2 c - a^2 d) - 2a b d - b^2 d \ge k - 2a b d - b^2 d$.
    Let $k = 2 a b d + b^2 d + 1$.
    Then, clearly $k(b^2 c - a^2 d) - 2a b d - b^2 d > 0$ as required.
\end{proof}

\subsection{Dedekind Cuts}

\begin{tcolorbox}[title=Problem 1, breakable]
    Let $A, B \subseteq \mathbb{Q}$ be Dedekind cuts.
    Suppose that $A \subset B$.
    Prove that $B - A$ has more than one element.
    If you are familiar with the cardinality of sets, prove that $B - A$
    is countably infinite.
\end{tcolorbox}

\begin{proof}
    Since $A \subset B$, $B - A \ne \emptyset$.
    Then, let $x \in B - A$ so $x \in B$ and $x \notin A$.
    By 1.6.1 Part (c) there exists $c < x$ such that $c \in B$.
    Now $x \in \mathbb{Q} - A$ and since $c < x$ by Lemma 1.6.5 Part (2),
        $c \in \mathbb{Q} - A$.
    Thus $c \in B$ and $c \notin A$ thus $c \in B - A$.
\end{proof}

\begin{proof}
    The reader is not familiar with the cardinality of sets as required.
\end{proof}

\begin{tcolorbox}[title=Problem 2, breakable]
    Let $T$ be the set defined in Equation $1.6.1$.
    \begin{enumerate}
        \item Prove that $T$ is Dedekind cut.
        \item Prove that if $T = D_r$ for some $r \in \mathbb{Q}$, then $r^2 = 2$.
              [Use Exercise $1.5.7$ and Exercise $1.5.9 (3)$]
    \end{enumerate}
\end{tcolorbox}

\textbf{Definition of $T$} 
\[T = \{x \mid x > 0 \text{ and } x^2 > 2\}\]
\begin{proof}
    Clearly $1 \in \mathbb{Q}$, $1 > 0$, and $1^2 = 1 < 2$ thus $1 \notin T$. Therefore $T \ne \mathbb{Q}$.
    Also, $3 \in \mathbb{Q}$, $3 > 0$, and $3^2 = 9 > 2$ thus $3 \in T$. Therefore $T \ne \emptyset$.
    Satisyfing 1.6.1 Part (a).

    Suppose $x \in A$ and $y \in \mathbb{Q}$ such that $y \ge x$.
    Clearly $y^2 \ge x^2 > 2$ and $y \ge x > 0$ thus $y \in T$.
    Satisyfing 1.6.1 Part (b).

    Let $x \in T$ such that $x > 0$ and $x^2 > 2$.
    By 1.5.9 Part (2) there exists $k \in \mathbb{N}$
        such that $\left(x - \frac{1}{k}\right)^2 > 2$.
    Clearly $x > x - \frac{1}{k}$ and $x - \frac{1}{k} \in T$.
    Satisyfing 1.6.1 Part (c).
\end{proof}

\begin{proof}
    Suppose $T = D_r$ for some $r \in \mathbb{Q}$.
    Then $T =  \{x \mid x > 0 \text{ and } x^2 > 2\} = \{x \mid x > r\}$ for some $r \in \mathbb{Q}$.
    Now, we know that $r \notin D_r = T$, thus $r^2 \le 2$.
    For contradiction, suppose $r^2 < 2$.
    By 1.5.9 Part (2) there exists $k \in \mathbb{N}$
        such that $(r^2 + \frac{1}{k})^2 < 2$.
    Thus $(r^2 + \frac{1}{k})^2 \notin T$ but $r < r + \frac{1}{k}$ thus 
        $r + \frac{1}{k} \in D_r = T$ which is a contradicton.
    Therefore $r^2 = 2$ as required.
\end{proof}

\begin{tcolorbox}[title=Problem 3, breakable]
    Prove Lemma $1.6.8 (3)$.
\end{tcolorbox}

\begin{proof}
    Let $A, B \subseteq \mathbb{Q}$ be a Dedekind cuts.
    Suppose that $0 \in \mathbb{Q} - A$ and $0 \in \mathbb{Q} - B$.
    Let 
    \[M = \{r \in \mathbb{Q} \mid r = ab \text{ for some } a \in A \text{ and } b \in B\}\]
    (\textbf{a}) Now $0 \in M$ implies $ab = 0$ thus $a = 0 \in A$ or $b = 0 \in B$ which is a contradiction.
    Thus $M \ne \mathbb{Q}$.
    We know that $A, B \ne \emptyset$ and $A, B \ne \mathbb{Q}$.
    Let $a \in A$ and $b \in B$.
    Then $a b \in M$ thus $M \ne \emptyset$.

    (\textbf{b}) Let $x \in M$. Suppose $y \in \mathbb{Q}$ such that $y \ge x$.
    Now $x = ab$ for some $a, b \in \mathbb{Q}$ and $a \ne 0, b \ne 0$.
    Then, since $y > x = ab$ it follows $\frac{y}{b} > a$.
    By the definition of Dedekind cuts Part (b), $\frac{y}{b} \in A$.
    Then $\frac{y}{b} \cdot b = y$ thus $y \in M$.

    (\textbf{c}) Let $x \in M$. Now $x = ab$ for some $a \in A$ and $b \in B$.
    By the definition of Dedekind cuts Part (c), there exists $c < a$ such that $c \in A$.
    Then $cb < ab = x$ and since $c \in A$ and $b \in B$, $cb \in M$ as required.
\end{proof}

\begin{tcolorbox}[title=Problem 4, breakable]
    Let $A \subseteq \mathbb{Q}$ be a Dedekind cut, and let $r \in \mathbb{Q}$.
    \begin{enumerate}
        \item Prove that $A \subset D_r$ if and only if there is some $q \in \mathbb{Q} - A$ such that $r < q$.
        \item Prove that \textcircled{1} $A \subseteq D_r$ 
                if and only if \textcircled{2} $r \in \mathbb{Q} - A$ 
                if and only if \textcircled{3} $r < a$ for all $a \in A$.
    \end{enumerate}
\end{tcolorbox}

\begin{proof}
    ($\longrightarrow$) Suppose $A \subset D_r$. Let $x \in D_r - A$. Now, $x \notin A$ thus $x \in \mathbb{Q} - A$. 
    Since $x \in D_r$, $x > r$ as required.

    ($\longleftarrow$) Suppose there is some $q \in \mathbb{Q} - A$ such that $r < q$.
    Let $x \in A$. Then $x > q > r$ thus $x > r$ and it follows that $x \in D_r$.
    Thus $A \subset D_r$.
\end{proof}

\begin{proof}
    (\textbf{\textcircled{1} $\implies$ \textcircled{2}})
    Suppose $A \subseteq D_r$. Suppose $x \in A$ then $x \in D_r$ thus $x > r$.
    Suppose $r \in A$ then $r > r$ which is a contradiction.
    Thus $r \in \mathbb{Q} - A$ as required.

    (\textbf{\textcircled{2} $\implies$ \textcircled{3}})
    Suppose $r \in \mathbb{Q} - A$.
    Let $a \in A$ and suppose $r > a$.
    By the definition of Dedekind cuts Part (b), $r \in A$ which is a contradicion.
    Thus for all $a \in A$, $r < a$ as required.

    (\textbf{\textcircled{3} $\implies$ \textcircled{1}})
    Suppose $r < a$ for all $a \in A$.
    Let $x \in A$ then $r < a$ by the definition of Dedekind cuts Part (c)
        $x \in D_r$. Thus $A \subseteq D_r$ as required.
\end{proof}

\begin{tcolorbox}[title=Problem 5, breakable]
    What we call a Dedekind cut is often called an ``upper cut'', to 
    differentiate it from the analogous ``lower cut''. Both types of cuts are 
    equally valid, and are mirror images of each other, though upper cuts are 
    slightly simpler to use because the product of positive numbers is positive,
    whereas the product of negative numbers is not negative.
    \begin{enumerate}
        \item Write a precise definition of lower cuts, modeled upon Definition $1.6.1$.
        \item Let $A \subseteq Q$ be a Dedekind cut. Find an example to show $\mathbb{Q} - A$ is not 
              necessarily a lower cut.
        \item Let $A \subseteq \mathbb{Q}$ be a Dedekind cut. Prove that if $\mathbb{Q} - A$ is not a lower cut,
              then $m \in \mathbb{Q} - A$ such that $x \le m$ for all $x \in \mathbb{Q} - A$.
        \item Let $A \subseteq \mathbb{Q}$ be a Dedekind cut. Suppose that $\mathbb{Q} - A$ is not a lower cut.
              Prove that there is a unique element $k \in \mathbb{Q} - A$ such that $\mathbb{Q} - (A \cup \{k\})$
              is a lower cut.
        \item Let $A \subseteq Q$ be a Dedekind cut. Suppose that $\mathbb{Q} - A$ is not a lower cut.
              Let $k$ be as in Part (4) of this lemma. Prove that $k \le x$ for all $x \in A \cup \{k\}$.
        \item Let $\mathcal{D}^u$ denote the set of of all Dedekind cuts of $\mathbb{Q}$, and 
              let $\mathcal{D}^l$ denote the set of all lower cuts of $\mathbb{Q}$. Prove that there 
              is a bijective function $\phi : \mathcal{D}^u \longrightarrow \mathcal{D}^l$ such that 
              $A \subseteq B$ implies $\phi(A) \supseteq \phi(B)$ for all $A, B \in \mathcal{D}^u$. Because 
              lower cuts are completely analogous to Dedekind cuts, you may assume that the analog of everything
              that has been previously proved about Dedekind cuts and lower cuts holds with the roles of 
              Dedekind cuts and lower cuts reversed. [Use Exercise 1.6.1.]
    \end{enumerate}
\end{tcolorbox}

\begin{definition}[Lower Cuts]
    Let $A \subseteq \mathbb{Q}$ be a set. The set $A$ is a \textbf{Dedikind cut}
    if the following three properties hold.
    \begin{enumerate}
        \item $A \ne \emptyset$ and $A \ne \mathbb{Q}$
        \item Let $x \in A$. If $y \in \mathbb{Q}$ and $y \le x$, then $y \in A$.
        \item Let $x \in A$. Then there is some $y \in A$ such that $y > x$.
    \end{enumerate} 
\end{definition}

\textbf{Solution (2):}  
Consider the upper cut $D_0 = \{x \in \mathbb{Q} \mid x > 0\}$.  
Now, $0 \notin D_0$, so $0 \in \mathbb{Q} - D_0$.  
For $\mathbb{Q} - D_0$ to be a lower cut, it must satisfy that for every $x \in \mathbb{Q} - D_0$ 
    there exists $y \in \mathbb{Q} - D_0$ with $y > x$.  
But $0 \in \mathbb{Q} - D_0$ is the largest element, so there is no $y \in \mathbb{Q} - D_0$ with $y > 0$.  
Thus, $\mathbb{Q} - D_0$ is not a lower cut.

\begin{proof}
    Suppose $\mathbb{Q} - A$ is not a lower cut.  
    For contradiction, suppose there does not exist a maximal element in $\mathbb{Q} - A$.

    (\textbf{a}) Now, clearly $\mathbb{Q} - A \ne \emptyset$ since $A \ne \mathbb{Q}$.
    Take $a \in A$ it follows that $a \notin \mathbb{Q} - A$ thus $L \ne \mathbb{Q}$.  
    
    (\textbf{b}) Let $x \in \mathbb{Q} - A$. Then $x - 1 < x$ and $x - 1 \in \mathbb{Q} - A$.  
    
    (\textbf{c}) Finally, suppose $x \in \mathbb{Q} - A$. There must exist $b \in \mathbb{Q} - A$ 
    such that $x < b$. Otherwise, $x$ would be the greatest element in $\mathbb{Q} - A$.  
    Thus $\mathbb{Q} - A$ is a lower cut, which is a contradiction.
\end{proof}

\begin{proof}
    Let $k$ be the maximal element in $\mathbb{Q} - A$.
    We now show  $L = \mathbb{Q} - (A \cup \{k\})$ is a lower cut.
    
    (\textbf{a}) Now $k - 1 \in L$ thus $L \ne \emptyset$.
    Take $a \ne k \in A$ it follows that $a \notin L$ thus $L \ne \mathbb{Q}$.

    (\textbf{b}) Let $x \in L$. Then $x - 1 < x$ and $x - 1 \in L$.

    (\textbf{c}) Let $x \in L$. $x < \frac{x + k}{2} <  k$ thus $\frac{x + k}{2} \in L$.
\end{proof}

\begin{proof}
    Let $x \in A \cup \{k\}$. If $x = k$ then clearly $k \le x$.
    If $x \ne k$ then $x > k$ since $k \in \mathbb{Q} - A$. Thus $k \le x$ as required.
\end{proof}

\begin{proof}
    Let $\phi : \mathcal{D}^u \longrightarrow \mathcal{D}^l$ be a mapping defined as 
        $\phi(A) = \{-x : x \in A\}$.

    We first show $\phi$ is a function. 
    This is trivial since the inverse mapping $B = \{-x : x \in B\}$ exists, so $\phi$ is a bijection.

    Let $A$ be an arbitrary upper cut. We now show $\phi(A)$ is a lower cut.
    This is also trivial since the argument is identical to the upper cut 
        but flipping the inequality.

    Suppose $A, B \in \mathcal{D}^u$ and $A \subseteq B$.
    Let $x \in B$, if $x \notin A$ then $x \in B-A$, 
        but all elements of $A$ are in $B$ thus $\{-x : x \in A\} \supseteq \{-x : x \in B\}$.
    Therefore $\phi(A) \supseteq \phi(B)$.
\end{proof}

\begin{tcolorbox}[title=Problem 6, breakable]
    In Definition 1.6.1, Dedekind cuts were defined as subsets of the 
    set $\mathbb{Q}$. However, an examination of this definition reveals 
    that it does not make use of the full features of $\mathbb{Q}$, but 
    only the order relation $<$ on $\mathbb{Q}$. Thus, it is possible 
    to define Dedekind cuts on sets equipped with only order relations,
    but not necessarily with binary operations such as addition and multiplication.

    Let $S$ be a non-empty set, and let $<$ be a relation on $S$.
    The relation $<$ is an \textbf{order relation} if it satisfies 
    the Trichotemy Law and the Transitive Law, as stated,
    for example, in Theorem 1.5.5 (10) (11); the set $S$ is an \textbf{ordered set}
    if $<$ is an order relation. For example, the natural numbers, the integers and 
    the rational numbers are all ordered sets. Dedekind cuts can be defined for any 
    ordered set exactly as in Definition $1.6.1$.
    \begin{enumerate}
        \item Give an example of an ordered set for which the analog of Lemma 1.6.2 does not hold.
        \item Find criteria on an ordered set that would guarantee that the analog of Lemma 1.6.2 holds.
             The criteria must be defined strictly in terms of the order relation.
        \item Verify that the analog of Lemma 1.6.7 holds for arbitrary ordered sets.
    \end{enumerate}
\end{tcolorbox}

\begin{proof}
    Consider the set $\mathbb{Z}$ equipped with the standard order relation $<$.  
    Let $a \in \mathbb{Z}$ and define
    \[
    D_a = \{x \in \mathbb{Z} \mid x > a\}.
    \]  
    Now $a + 1 > a$, thus $a + 1 \in D_a$.  
    Clearly, the definition of Dedekind cuts Part (3) cannot hold since there does not exist $k \in \mathbb{Z}$ such that $a < k < a + 1$.
\end{proof}

\begin{lemma}[1.6.1]
    Let $A \subseteq \mathbb{Q}$ be a set. The set $A$ is a \textbf{Dedekind cut} if the following 
    three properties hold.
    \begin{enumerate}
        \item $A \ne \emptyset$ and $A \ne \mathbb{Q}$.
        \item Let $x \in A$. If $y \in \mathbb{Q}$ and $y \ge x$, then $y \in A$.
        \item Let $x \in A$. Then there is some $y \in A$ such that $y < x$.
    \end{enumerate}
\end{lemma}

The following criteria are required of an ordered set for Lemma 1.6.2 to hold.
Let $A$ be an ordered set. 
\begin{enumerate}
    \item $A \ne \emptyset$.
    \item Let $a, b \in A$. There must exist $k \in A$ such that $a < k < b$.
    \item Let $a \in A$. There exists $b \in A$ such that $b < a$.
\end{enumerate}

\textbf{Solution (3):}  
Clearly the proof provided only makes use of the ordering relation and not 
properties specific to $\mathbb{Q}$.

\subsection{Construction of the Real Numbers}

\begin{tcolorbox}[title=Problem 1, breakable]
    Let $r \in \mathbb{Q}$
    \begin{enumerate}
        \item Prove that $D_{-1} = -D_r$, using the Definition 1.6.4 and Definition 1.7.3.
        \item Prove that $D_{r^{-1}} = [D_r]^{-1}$, using only Definition 1.7.5 and Definition 
              1.7.3.
    \end{enumerate}
\end{tcolorbox}

\begin{proof}
    According to Definition 1.7.3, $D_{-r} = \{x \in \mathbb{Q} \mid x > -r\}$.
    Futhermore, according to definition 1.7.3,
        $-D_r = \{x \in \mathbb{Q} \mid -x < c \text{ for some } c \in \mathbb{Q} - D_r\}$.
    Now suppose $x \in D_{-r}$. Then $x > -r$ and it follows that $-x < r$.
    Since $r \notin D_r$ it follows that $r \in \mathbb{Q} - D_r$.
    Thus $x \in -D_r$. Therefore $D_{-r} \subseteq -D_r$.
    Now suppose $x \in -D_r$. Then $-x < c$ for some $c \in \mathbb{Q} - D_r$.
    Now $c \le r$ thus $x > -c > -r$. It follows that $x \in D_{-r}$.
    Therefore $-D_r \subseteq D_{-r}$.
    It follows that $D_{-r} = -D_r$.
\end{proof}

\begin{proof}
    According to Definition 1.7.5, 
    if $A > D_0$ then 
        $A^{-1} = \{r \in \mathbb{Q} \mid r > 0 \text{ and } \frac{1}{r} < c \text{ for some } c \in \mathbb{Q} - A\}$; 
    if $A < D_0$ then 
        $A^{-1} = -(-A)^{-1}$.
    According to Definition 1.7.3 
        $-A = \{r \in \mathbb{Q} \mid -r < c \text{ for some } c \in \mathbb{Q} - A\}$.
    There are two cases.
    Suppose $D_r > D_0$.  
    Now suppose $x \in D_{r^{-1}}$.  
    Then $x > \frac{1}{r} > 0$. Since $r > 0$, we have $\frac{1}{x} < r$.
    Since $r \in \mathbb{Q} - D_r$, it follows that $x \in [D_r]^{-1}$.
    Now suppose $x \in [D_r]^{-1}$. 
    Then $x > 0$ and $\frac{1}{x} < c \le r$ for some $c \in \mathbb{Q} - D_r$.
    It follows that $x > \frac{1}{r}$, thus $x \in D_{r^{-1}}$.
    Thus if $D_r > D_0$ then $D_{r^{-1}} = [D_r]^{-1}$.
    Suppose $D_r < D_0$. 
    Now $-D_r = D_{-r}$ by part 1. 
    Then, since $-r > 0$, we can apply case one to find
        $(D_{-r})^{-1} = D_{(-r)^{-1}}$.
    Therefore
        $[D_r]^{-1} = -[D_{-r}]^{-1} = -D_{(-r)^{-1}} = D_{r^{-1}}$.
    Thus $D_{r^{-1}} = [D_r]^{-1}$.
\end{proof}

\begin{tcolorbox}[title=Problem 2, breakable]
    Let $A, B \in \mathbb{R}$. Suppose that $A > D_0$ and 
    $B > D_0$. For this exercise, you may use results prior to 
    Theorem 1.7.6.
    \begin{enumerate}
        \item Prove that $AB > D_0$.
        \item Prove that $A^{-1} > D_0$.
    \end{enumerate}
\end{tcolorbox}

\begin{proof}
    Suppose $A > D_0$ and $B > D_0$. Thus $D_0 \supset A, B$.
    Let $x \in A B$. Then $x = a b$ for some $a \in A$ and $b \in B$.
    Since $D_0 \supset A, B$ it follows that $a, b \in D_0$ and therefore $a, b > 0$.
    Thus $x = a b > 0$ and it follows that $x \in D_0$.
    Therefore $D_0 \supset AB$ thus $AB > D_0$.
\end{proof}

\begin{proof}
    Suppose $A > D_0$. Thus $D_0 \supset A$.
    Let $x \in A^{-1}$. Since $A > D_0$ it follows from the definition of inverse 
        that $x > 0$ and $\frac{1}{x} < c$ for some $c \in \mathbb{Q} - A$.
    Since $x > 0$ it follows that $x \in D_0$.
    Thus $A^{-1} \subset D_0$ thus $A^{-1} > D_0$.
\end{proof}

\begin{tcolorbox}[title=Problem 3, breakable]
    Prove Theorem 1.7.6 (14). For this exercise you may use only results 
    prior to Theorem 1.7.6. [Use Exercise 1.5.6 (1).]
\end{tcolorbox}

\begin{theorem}[1.7.6 Part (14)]
    $D_0 < D_1$ (\emph{Non-Triviality})
\end{theorem}

\begin{proof}
    Clearly, $D_1 \ne D_0$ since $1 > \frac{1}{2} > 0$ thus $\frac{1}{2} \notin D_1$ and $\frac{1}{2} \in D_0$.
    Let $x \in D_1$ thus $x > 1 > 0$.
    Therefore $x > 0$ and it follows that $x \in D_0$.
    Thus $D_0 \supset D_1$ therefore $D_0 < D_1$.
\end{proof}

\begin{tcolorbox}[title=Problem 4, breakable]
    For this exercise, use only the properties of the real numers 
    stated in Theorem 1.7.6 (1) (2) (3) (4) (10) (11) (12) (14);
    it is not necessary to use the definition of real numbers as 
    Dedekind cuts. Let $A, B \in \mathbb{R}$.
    \begin{enumerate}
        \item Prove that \textcircled{1} $A > D_0$ 
              if and only if \textcircled{2} $-A < D_0$,
              and that \textcircled{3} $A < D_0$
              if and only if \textcircled{4} $-A > D_0$.
        \item Prove that $-(-A) = A$.
        \item Prove that $-(A + B) = (-A) + (-B)$.
        \item Prove that if $A > D_0$ and $B > D_0$, then $A + B > D_0$,
              and that if $A < D_0$ and $B < D_0$, then $A + B < D_0$.
        \item Prove that $A = (-B) + (A + B) = B + [A + (-B)]$ and $-A = B + [-(B + A)]$.
    \end{enumerate}
\end{tcolorbox}

\begin{proof}
    (\textbf{\textcircled{1} \longrightarrow \textcircled{2}})
    Suppose $A > D_0$.
    Then 
    \begin{align*}
        A > D_0 
        \implies &A + (-A) > D_0 + (-A) && \text{(12)} \\
        \iff &D_0 > D_0 + (-A) && \text{(4)} \\
        \iff &D_0 > (-A) + D_0 && \text{(2)} \\
        \iff &D_0 > -A && \text{(3)}
    \end{align*}
    (\textbf{\textcircled{2} \longrightarrow \textcircled{1}})
    Suppose $-A < D_0$.
    \begin{align*}
        -A < D_0 
        \implies &-A + A < D_0 + A && \text{(12)} \\
        \iff &A + (-A) < D_0 + A && \text{(2)} \\
        \iff &D_0 < D_0 + A && \text{(4)} \\
        \iff &D_0 < A && \text{(3)}
    \end{align*}
    (\textbf{\textcircled{3} \longrightarrow \textcircled{4}})
    Suppose $A < D_0$.
    \begin{align*}
        A < D_0 
        \implies &A + (-A) < D_0 + (-A) && \text{(12)} \\
        \iff &D_0 < D_0 + (-A) && \text{(4)} \\
        \iff &D_0 < (-A) + D_0 && \text{(2)} \\
        \iff &D_0 < -A && \text{(3)}
    \end{align*}
    (\textbf{\textcircled{4} \longrightarrow \textcircled{3}})
        \begin{align*}
        -A > D_0 
        \implies &-A + A > D_0 + A && \text{(12)} \\
        \iff &A + (-A) > D_0 + A && \text{(2)} \\
        \iff &D_0 > D_0 + A && \text{(4)} \\
        \iff &D_0 > A && \text{(3)}
    \end{align*}
\end{proof}

\begin{proof}
    By (4) we know $A + (-A) = D_0$ and $(-(-A)) + (-A) = D_0$.
    Thus $A + (-A) = (-(-A)) + (-A)$.
    Adding $-(-A)$ to both sides and applying (4)
        shows $A = -(-A)$.
\end{proof}

\begin{proof}
    Notice
    \begin{align*}
        A + B + ((-A) + (-B)) 
        &= A + ((-A) + (B + (-B))) && \text{(2)} \\
        &= (A + (-A)) + (B + (-B)) && \text{(2)} \\
        &= D_0 + D_0 && \text{(4)} \\
        &= D_0 && \text{(3)}
    \end{align*}
    Thus $-(A + B) = (-A) + (-B)$.
\end{proof}

\begin{proof}
    Suppose $A > D_0$ and $B > D_0$.
    Then, by (12),
    \begin{align*}
        A > D_0 
        &\implies A + (B + (-B)) > D_0 && \text{(12)} \\
        &\implies (A + B) + (-B) > D_0 && \text{(2)} \\
    \end{align*}
    Adding $B$ to both sides and cancelling shows $A + B > D_0 + B = B > D_0$.
    Thus $A + B > D_0$.
\end{proof}

\begin{proof}
    Suppose $A < D_0$ and $B < D_0$.
    Then, by (12),
    \begin{align*}
        A < D_0 
        &\implies A + (B + (-B)) < D_0 && \text{(12)} \\
        &\implies (A + B) + (-B) < D_0 && \text{(2)} \\
    \end{align*}
    Adding $B$ to both sides and cancelling shows $A + B < D_0 + B = B < D_0$.
    Thus $A + B < D_0$.
\end{proof}

\begin{proof}
    Notice 
    \begin{align*}
        A &= A + D_0 && \text{(3)} \\
          &= A + (B + (-B)) && \text{(4)} \\
          &= (A + B) + (-B) && \text{(1)} \\
          &= (-B) + (A + B) && \text{(2)} \\
          &= ((-B) + A) + B && \text{(1)} \\
          &= (A + (-B)) + B && \text{(2)} \\
          &= B + (A + (-B)) && \text{(2)}
    \end{align*}
    Thus $A = (-B) + (A + B) = B + (A + (-B))$.
    Similarly 
        \begin{align*}
        -A &= -A + D_0 && \text{(3)} \\
          &= -A + (B + (-B)) && \text{(4)} \\
          &= -A + ((-B) + B) && \text{(2)} \\
          &= (-A + (-B)) + B && \text{(1)} \\
          &= (-B + (-A)) + B && \text{(2)} \\
          &= -(B + A) + B && \text{(Part 3)} \\
          &= B + (-(B + A)) && \text{(2)}
    \end{align*}
    Thus $-A = B + (-(B + A))$.
\end{proof}

\newpage
\begin{tcolorbox}[title=Problem 5, breakable]
    Prove Theorem 1.7.6 (5) (7). For this exercise, you may use only 
    Parts (1) (2) (3) (4) (10) (11) (12) (14) of the theorem,
    and anything prior to the theorem. [Use Exercise 1.7.4]
\end{tcolorbox}

\begin{theorem}[1.7.4 Part (5)]
    Let $A, B, C \in \mathbb{R}$.
    \[(AB)C = A(BC)\]
\end{theorem}

\begin{theorem}[1.7.4 Part (7)]
    Let $A \in \mathbb{R}$.
    \[A \cdot D_1 = A\]
\end{theorem}

\begin{proof}
    Let $A, B, C \in \mathbb{R}$.  
    There are $8$ cases depending on whether each of $A, B, C$ is $>D_0$ or $<D_0$.  

    (\textbf{Case 1: $A, B > D_0, C > D_0$})  
    Suppose $x \in (AB)C$. Then $x = dc$ for some $d \in AB$ and $c \in C$.  
    Now $d = ab$ for some $a \in A$ and $b \in B$. 
    Then $x = dc = (ab)c = a(bc)$.  
    Since $b \in B$ and $c \in C$, $bc \in BC$.
    Since $a \in A$ and $bc \in BC$, it follows that $x \in A(BC)$.  

    Suppose $x \in A(BC)$. Then $x = ad$ for some $a \in A$ and $d \in BC$.
    Now $d = bc$ for some $b \in B$ and $c \in C$.
    Then $x = ad = a(bc) = (ab)c$.
    Since $a \in A$ and $b \in B$, $ab \in AB$.
    Since $ab \in AB$ and $c \in C$ it follows that $x \in (AB)C$.

    Thus $(AB)C \subseteq A(BC)$.

    (\textbf{Case 2: $A < D_0, B > D_0, C > D_0$})  
    Since $A < D_0$, $-A > D_0$.
    Now, part Part $1$, $(-A)(BC) = ((-A)B)C$.
    Mulitplying by $-1$ shows $-[(-A)(BC)] = [-((-A)B)C]$.
    Then by problem $4$, $A(BC) = (AB)C$.

    (\textbf{Case 3: $A > D_0, B < D_0, C > D_0$})  
    Since $B < D_0$, $-B > D_0$.  
    By Case 1, $A((-B)C) = (A(-B))C$.  
    Multiplying by $-1$ shows $A(BC) = (AB)C$.

    (\textbf{Case 4: $A > D_0, B > D_0, C < D_0$})  
    Since $C < D_0$, $-C > D_0$.  
    By Case 1, $(AB)(-C) = A(B(-C))$.  
    Multiplying by $-1$ shows $(AB)C = A(BC)$.

    (\textbf{Case 5: $A < D_0, B < D_0, C > D_0$})  
    Since $A < D_0$ and $B < D_0$, $-A > D_0$ and $-B > D_0$.  
    By Case 1, $((-A)(-B))C = (-A)((-B)C)$.  
    Multiplying by $-1$ twice shows $A(BC) = (AB)C$.

    (\textbf{Case 6: $A < D_0, B > D_0, C < D_0$})  
    Since $A < D_0$ and $C < D_0$, $-A > D_0$ and $-C > D_0$.  
    By Case 1, $((-A)B)(-C) = (-A)(B(-C))$.  
    Multiplying by $-1$ twice shows $A(BC) = (AB)C$.

    (\textbf{Case 7: $A > D_0, B < D_0, C < D_0$})  
    Since $B < D_0$ and $C < D_0$, $-B > D_0$ and $-C > D_0$.  
    By Case 1, $A((-B)(-C)) = (A(-B))(-C)$.  
    Multiplying by $-1$ twice shows $A(BC) = (AB)C$.

    (\textbf{Case 8: $A < D_0, B < D_0, C < D_0$})  
    Since $A < D_0$, $B < D_0$, $C < D_0$, we have $-A > D_0$, $-B > D_0$, $-C > D_0$.  
    By Case 1, $((-A)(-B))(-C) = (-A)((-B)(-C))$.  
    Multiplying by $-1$ three times shows $A(BC) = (AB)C$.

    Thus, in all $8$ cases, $(AB)C = A(BC)$.
\end{proof}

\begin{proof}
    Let $A \in \mathbb{R}$.  

    (\textbf{Case 1: $A > D_0$})  
    Suppose $x \in A D_1$. Then $x = ad$ for some $a \in A$ and $d \in D_1$.
    Since $d > 1$, $ad > a$ thus $ad \in A$.
    Therefore $A D_1 \subseteq A$.

    Suppose $x \in A$.  
    Let $d \in D_1$. Then $d > 1$.
    Now $\frac{x}{d} < x$. Since $d > 0$ and $x > 0$, $\frac{x}{d} > 0$
        thus $\frac{x}{d} \in A$.
    Then $x = \frac{x}{d} \cdot d \in A D_1$.
    Thus $A \subseteq A D_1$.

    (\textbf{Case 2: $A < D_0$})
    Now $-A > D_0$ and by \textbf{Case 1}, $-A = -A D_1$.
    Multiplying by $-1$ shows $A = A D_1$.
\end{proof}

\newpage
\begin{tcolorbox}[title=Problem 6, breakable]
    Prove the remaining four cases in the proof 
    of Theorem 1.7.6 (9). [Use Exercise 1.7.4]
\end{tcolorbox}

\begin{proof}
    There four 8 FUCKING cases again! Unfortunately
        Bloch is making us do 4 of them for some unkown reason...

    (\textbf{})
\end{proof}

\begin{tcolorbox}[title=Problem 7, breakable]
    Prove Theorem 1.7.10. [Use Exercise 1.7.1]
\end{tcolorbox}

\begin{tcolorbox}[title=Problem 8, breakable]
    This exercise makes use of Exercise 1.6.6.
    Let $S$ be a non-empty ordered set.
    The \textbf{Dedekind set} of $S$, denoted $S^D$, is defined by 
    \[S^D = \{A \subseteq S \mid A \text{ is a Dedekind cut}\}.\]
    For example, we know by definition of $\mathbb{Q}^D = \mathbb{R}$.
    The order relation $<$ on $S^D$ is defined analogously to 
    Definition $1.7.2$.
    \begin{enumerate}
        \item Find an example of an ordered set $T$ for which $T^D = \emptyset$.
              It is sufficient to state informally the reasion why your example works.
        \item Find an example of an ordered set $U$ for which $U^D$ has exactly one element.
              It is sufficient to state informally the reason why your example works.
        \item Verify that $S^D$ satisfies the Least Upper Bound Property.
        \item What can you say about $\mathbb{R}^D$? It is sufficient to answer the question informally.
    \end{enumerate}
\end{tcolorbox}

\textbf{Solution (1):}
Clearly $\mathbb{Z}$ is an ordered set which has no Dedekind cuts.
Thus $T^D = \emptyset$.