\subsection{Axioms for the Natural Numbers}

\begin{tcolorbox}[title=Problem 1, breakable]
    Fill in the missing details in the proof of Theorem $1.2.6$.
\end{tcolorbox}

\begin{proof}
    We must show the uniquness of the binary operation $\cdot : \mathbb{N} \times \mathbb{N} \rightarrow \mathbb{N}$
    that satisfies the following two properties for all $n, m \in \mathbb{N}$.
    \begin{enumerate}[label=\textbf{\alph*.}]
        \item $n \cdot 1 = n$.
        \item $n \cdot s(m) = (n \cdot m) + n$.
    \end{enumerate}
    Suppose there are two binary operations $\cdot$ and $\times$
        on $\mathbb{N}$ that satisfy the two properties for all $n, m \in \mathbb{N}$.
    Let 
    \[G = \{x \in \mathbb{N} \mid n \cdot x = n \times x \text{ for all } n \in \mathbb{N}\}\]
    We will prove that $G = \mathbb{N}$, which will imply that $\cdot$ and $\times$ are 
        the same binary operation.
    It is clear that $G \subseteq \mathbb{N}$.
    By part (a) applied to each of $\cdot$ and $\times$ we see that 
        $n \cdot 1 = n = n \times 1$ for all $n \in \mathbb{N}$
        and hence $1 \in G$.
    Now let $q \in G$. Let $n \in \mathbb{N}$.
    Then $n \cdot q = n \times q$ by hypothesis on $q$.
    It then follows from part (b) that $n \cdot s(q) = (n \cdot q) + n = (n \times q) + n = n \times s(q)$.
    Hence $s(q) \in G$.
    By part (c) of the Peano Postulates we conclude that $G = \mathbb{N}$.
\end{proof}

\begin{proof}
    We must show the two properties hold.
    Now, $n \cdot 1 = g_n(1) = n$, which is part (a),
    and $n \cdot s(m) 
        = g_n(s(m)) 
        = (g_n \circ s)(m) 
        = (h_n \circ g_n)(m)
        = g_n(m) + n
        = (n \cdot m) + n$, which is part (b).
\end{proof}

\newpage
\begin{tcolorbox}[title=Problem 2, breakable]
    Prove Theorem $1.2.7$ ($2$) ($3$) ($4$) ($7$) ($8$) ($9$) ($10$) ($11$) ($13$).
\end{tcolorbox}

\begin{proof}
    Let $a, b, c \in \mathbb{N}$.
    We must show $(a + b) + c = a + (b + c)$.
    Consider the set 
    \[G = \{z \in \mathbb{N} \mid \text{if } x,y \in \mathbb{N} \text{ then } (x + y) + z = x + (y + z)\}\]
    We will show $G = \mathbb{N}$.
    Clearly $G \subseteq \mathbb{N}$.
    We first show $1 \in G$.
    Suppose $z \in G$.
    Consider 
    \[(x + y) + 1 = s(x + y) = x + s(y) = x + (y + 1)\]
    Thus $1 \in G$.
    Futher let $x, y, z \in \mathbb{N}$, and consider 
    \[(x + y) + s(z) = s((x + y) + z)\]
    By our hypothesis on $z$, $(x + y) + z = x + (y + z)$ so
    \[s((x + y) + z) = s(x + (y + z)) = x + s(y + z) = x + (y + s(z))\]
    So $s(z) \in G$. Thus $G = \mathbb{N}$ by part (c) of the Peano Postulates.
\end{proof}

\begin{proof}
    Let $a \in \mathbb{N}$.
    We must show $1 + a = s(a) = a + 1$.
    Consider the set 
    \[G = \{a \in \mathbb{N} \mid 1 + a = s(a) = a + 1\}\]
    We will show $G = \mathbb{N}$.
    Clearly $G \subseteq \mathbb{N}$.
    We first show $1 \in G$.
    Let $a \in \mathbb{N}$ such that $a = 1$.
    \[1 + a = s(a) = s(1) = 1 + 1 = a + 1\]
    Thus $1 \in G$.
    Suppose $x \in \mathbb{N}$ and $x \in G$.
    By our hypothesis, $1 + x = x + 1$.
    Then 
    \[1 + s(x) = s(1 + x) = s(x + 1) = s(x) + 1\]
    So $s(x) \in G$. Thus $G = \mathbb{N}$ by part (c) of the Peano Postulates.
\end{proof}

\begin{proof}
    Let $a, b \in \mathbb{N}$.
    We must show $a + b = b + a$.
    Consider the set 
    \[G = \{x \in \mathbb{N} \mid \text{if } y \in \mathbb{N} \text{ then } x + y = y + x\}\]
    We will show $G = \mathbb{N}$.
    Clearly $G \subseteq \mathbb{N}$.
    We first show $1 \in G$.
    Let $x \in \mathbb{N}$.
    By Theorem $1.2.7$ part ($3$), $1 + x = x + 1$.
    Thus $1 \in G$.
    Now suppose $x \in G$.
    Let $y \in \mathbb{N}$.
    First note by Theorem $1.2.7$ part ($2$), $1 + (x + y) = (1 + x) + y$.
    Consider
    \[y + s(x) = s(y + x) = s(x + y) \text{ hypothesis on $x$} = 1 + (x + y) = (1 + x) + y = s(x) + y\]
    So $s(x) \in G$. Thus $G = \mathbb{N}$ by part (c) of the Peano Postulates.
\end{proof}

\begin{proof}
    Let $a \in \mathbb{N}$.
    We must show $a \cdot 1 = a = 1 \cdot a$.
    Consider the set 
    \[G = \{x \in \mathbb{N} \mid x \cdot 1 = x = 1 \cdot x\}\]
    We will show $G = \mathbb{N}$.
    Clearly $G \subseteq \mathbb{N}$.
    We first show $1 \in G$. Consider
    \[
    \begin{aligned}
    x \cdot 1 
        &= x && \text{Theorem 1.2.6 part (a)} \\
        &= 1 \\
        &= 1 \cdot 1 \\
        &= x \cdot 1
    \end{aligned}
    \]
    Thus $1 \in G$.
    Consider 
    \[
    \begin{aligned}
    s(x) \cdot 1 
        &= s(x) && \text{Theorem 1.2.6 part (a)} \\
        &= x + 1 && \text{Theorem 1.2.5 part (a)} \\
        &= x \cdot 1 + 1 && \text{Theorem 1.2.6 part (a)} \\
        &= 1 \cdot x + 1 && \text{Induction hypothesis} \\
        &= 1 \cdot s(x) && \text{Theorem 1.2.6 part (b)}
    \end{aligned}
    \]
    So $s(x) \in G$. Thus $G = \mathbb{N}$ by part (c) of the Peano Postulates.
\end{proof}

\begin{proof}
    Let $a, b, c \in \mathbb{N}$.
    We must show $(a + b)c = ac + bc$.
    Consider the set 
    \[G = \{c \in \mathbb{N} \mid \text{if } a, b \in \mathbb{N} \text{ then } (a + b)c = ac + bc\}\]
    We will show $G = \mathbb{N}$.
    Clearly $G \subseteq \mathbb{N}$.
    We first show $1 \in G$. Let $a, b \in \mathbb{N}$. Then
    \begin{align*}
        (a + b)1 
            &= a + b \quad &&\text{(Theorem $1.2.6$ part (a))} \\
            &= a \cdot 1 + b \cdot 1 \quad &&\text{(Theorem $1.2.6$ part (a))}
    \end{align*}
    Suppose $a, b, c \in \mathbb{N}$ and $c \in G$.
    Then
    \begin{align*}
        (a + b) \cdot s(c) 
            &= ((a + b) c) + (a + b) \quad &&\text{(Theorem 1.2.6 part (a))} \\
            &= (ac + b c + a + b) \quad &&\text{(Induction Hypothesis)} \\
            &= (ac + a + bc + b) \quad &&\text{(Theorem 1.2.7 part (4))} \\
            &= a \cdot s(c) + b \cdot s(c) \quad &&\text{(Theorem 1.2.5 part (a))}
    \end{align*}
    So $s(c) \in G$. Thus $G = \mathbb{N}$ by part (c) of the Peano Postulates.
\end{proof}

\begin{proof}
    Let $a, b \in \mathbb{N}$.
    We must show $ab = ba$.
        Consider the set 
    \[G = \{a \in \mathbb{N} \mid \text{if } b \in \mathbb{N} \text{ then } ab = ba\}\]
    We will show $G = \mathbb{N}$.
    Clearly $G \subseteq \mathbb{N}$.
    We first show $1 \in G$. By Theorem $1.2.7$ part ($7$), $a \cdot 1 = 1 \cdot a$.
    Thus $1 \in G$.
    Suppose $a, b \in \mathbb{N}$ and $a \in G$.
    \begin{align*}
        s(a) \cdot b 
            &= (a + 1)b \quad &&\text{(Theorem $1.2.5$ part (a))} \\
            &= ab + 1 b \quad &&\text{(Theorem $1.2.7$ part ($8$))} \\
            &= ab + b 1 \quad &&\text{(Theorem $1.2.7$ part ($7$))} \\
            &= ab + b \quad &&\text{(Theorem $1.2.6$ part ($7$))} \\
            &= ba + b \quad &&\text{(Induction Hypothesis)} \\
            &= b \cdot s(a) \quad &&\text{(Theorem $1.2.6$ part (b))} 
    \end{align*}
    So $s(a) \in G$. Thus $G = \mathbb{N}$ by part (c) of the Peano Postulates.
\end{proof}

\begin{proof}
    Let $a, b \in \mathbb{N}$.
    We must show $c(a + b) = ca + cb$.
    By Theorem $1.2.7$ part ($9$), $c(a + b) = (a + b)c$.
    By Theorem $1.2.7$ part ($8$), $(a + b)c = ac + bc$.
     By Theorem $1.2.7$ part ($9$), $ac + bc = ca + cb$.
\end{proof}

\newpage
\begin{proof}
    Let $a, b, c \in \mathbb{N}$.
    We must show $(ab)c = a(bc)$.
\end{proof}

\begin{proof}
    Let $a, b, c \in \mathbb{N}$.
    We must show $(ab)c = a(bc)$.
    Consider the set 
    \[G = \{c \in \mathbb{N} \mid \text{if } a, b \in \mathbb{N} \text{ then } (ab)c = a(bc)\}\]
    We will show $G = \mathbb{N}$.
    Clearly $G \subseteq \mathbb{N}$.
    We first show $1 \in G$. Let $a, b \in \mathbb{N}$. Then
    \[(ab)1 = ab \text{ (Theorem $1.2.7$ part ($7$)) } = a (b \cdot 1) \text{ (Theorem $1.2.6$ part (a)) }\]
    Thus $1 \in G$.
    Suppose $a, b, c \in \mathbb{N}$ and $c \in G$.
    Then
    \begin{align*}
        (ab) \cdot s(c)
            &= (ab)(c + 1) \quad &&\text{(Theorem 1.2.5 part (a))} \\
            &= (ab)c + (ab) 1 \quad &&\text{(Theorem 1.2.7 part ($10$))} \\
            &= a(bc) + (ab) 1 \quad &&\text{(Induction Hypothesis)} \\
            &= a(bc) + ab \quad &&\text{(Theorem 1.2.7 part ($7$))} \\
            &= a(bc + b) \quad &&\text{(Theorem 1.2.7 part ($8$))} \\
            &= a(bc + b \cdot 1) \quad &&\text{(Theorem 1.2.7 part ($7$))} \\
            &= a(b(c + 1)) \quad &&\text{(Theorem 1.2.7 part ($8$))} \\
            &= a(b \cdot s(c)) \quad &&\text{(Theorem 1.2.5 part (a))}
    \end{align*}
    So $s(c) \in G$. Thus $G = \mathbb{N}$ by part (c) of the Peano Postulates.
\end{proof}

\begin{proof}
    Let $a, b \in \mathbb{N}$.
    We must show $ab = 1$ if and only if $a = 1 = b$.

    Suppose $ab = 1$. For contradiction, suppose $a \ne 1 $ or $b \ne 1$.
    Suppose $a \ne 1$.
    By Lemma $1.2.3$ there exists $c \in \mathbb{N}$ such that $s(c) = a$.
    Then
    \[ab = s(c) b = (c + 1)b \text{ (Theorem $1.2.5$ part (a)) } = c b + b \text{ (Theorem $1.2.7$ part ($8$)) } = 1\]
    Contradicting Theorem $1.2.7$ part ($5$).
    Suppose $b \ne 1$.
    By Lemma $1.2.3$ there exists $c \in \mathbb{N}$ such that $s(c) = b$.
    Then
    \[ab = a \cdot s(c) = a(c + 1) \text{ (Theorem $1.2.5$ part (a)) } = ac + a \text{ (Theorem $1.2.7$ part ($10$)) } = 1\]
    Contradicting Theorem $1.2.7$ part ($5$).

    Suppose $a = 1 = b$.
    Then $a b = a \cdot 1 = a = 1$ by Theorem $1.2.6$ part (a).
\end{proof}

\begin{tcolorbox}[title=Problem 3, breakable]
    Let $a, b \in \mathbb{N}$.
    Suppose $a < b$. 
    Prove that there is a unique $p \in \mathbb{N}$
        such that $a + p = b$
\end{tcolorbox}

\begin{proof}
    We first prove uniqueness.
    Let $a, b \in \mathbb{N}$ such that $a < b$.
    Suppose $x, y \in \mathbb{N}$
        such that $a + x = b$ and $a + y = b$.
    Then $a + x = a + y$.
    By Theorem $1.2.7$ part ($4$),
        $x + a = y + a$.
    Then by Theorem $1.2.7$ part ($1$), $x = y$.

    We now prove existence.
    Since $a < b$, by definition of $<$ 
        there exists $p \in \mathbb{N}$
        such that $a + p = b$.
\end{proof}

\begin{tcolorbox}[title=Problem 4, breakable]
    Prove Theorem $1.2.9$ ($1$) ($3$) ($4$) ($5$) ($11$).
\end{tcolorbox}

\begin{proof}
    Let $a \in \mathbb{N}$.
    We must show $a \le a$, and $a \not < a$, and $a < a + 1$.

    To show $a \le a$, suppose for contradiction $a = a$. Thus $a \le a$.
    To show $a \not < a$, first, suppose $a < a$.
    By definition of $<$, there exists $p \in \mathbb{N}$ such that $a + p = a$
        contradicting Theorem $1.2.7$ part ($6$).
    To show $a < a + 1$ consider $s(a) = a + 1 = a + 1$ thus $a < a + 1$.
\end{proof}

\begin{proof}
    Let $a, b, c \in \mathbb{N}$.
    We must show if $a < b$ and $b < c$, then $a < c$;
        if $a \le b$ and $b < c$, then $a < c$;
        if $a < b$ and $b \le c$, then $a < c$;
        if $a \le b$ and $b \le c$, then $a \le c$.

    \textcircled{1} 
    Suppose $a < b$ and $b < c$.
    By definition of $<$, there exists $p_1, p_2 \in \mathbb{N}$
        such that $a + p_1 = b$ and $b + p_2 = c$.
    Then $b + p_2 = (a + p_1) + p_2 = c$.
    By definition of $<$, $a < c$.

    \textcircled{2}
    Suppose $a \le b$ and $b < c$.
    By definition of $\le$,
        either $a = b$ or $a < b$.
    Suppose $a < b$. By \textcircled{1}, $a < c$.
    Suppose $a = b$.
    By definition of $<$, there exists $p \in \mathbb{N}$
        such that $b + p = c$.
    Then $b + p = a + p = c$.
    By definition of $<$, $a < c$.

    \textcircled{3}
    Suppose $a < b$ and $b \le c$.
    By definition of $\le$,
        either $b = c$ or $b < c$.
    Suppose $b < c$. By \textcircled{1}, $a < c$.
    Suppose $b = c$.
    By definition of $<$, there exists $p \in \mathbb{N}$
        such that $a + p = b$.
    Then $b = a + p = c$ thus, by definition of $<$, $a < c$.

    Suppose $a \le b$ and $b \le c$.
    There are four cases:
    \begin{enumerate}
        \item Suppose $a < b$ and $b < c$. By \textcircled{1}, $a < c$.
        \item Suppose $a \le b$ and $b < c$. By \textcircled{2}, $a < c$.
        \item Suppose $a < b$ and $b \le c$. By \textcircled{3}, $a < c$.
        \item Suppose $a \le b$ and $b \le c$. 
              There are four cases:
              \begin{enumerate}
                \item Suppose $a = b$ and $b < c$.
                      By definition of $<$, there exists $p \in \mathbb{N}$
                      such that $b + p = c$.
                      Then $b + p = a + p = c$ so $a < c$.
                \item Suppose $a < b$ and $b < c$. 
                      By \textcircled{1}, $a < c$.
                \item Suppose $a = b$ and $b = c$. 
                      Clearly $a = b = c$ thus $a = c$.
                \item Suppose $a < b$ and $b = c$. 
                    By definition of $<$, there exists $p \in \mathbb{N}$
                        such that $a + p = b$.
                      Then $a + p = b = c$ so $a < c$.
              \end{enumerate}
    \end{enumerate}
    Thus either $a < c$ or $a = c$ thus, by definition of $\le$, $a \le c$.
\end{proof}

\begin{proof}
    Let $a, b, c \in \mathbb{N}$.
    We must show if $a < b$ if and only if $a + c < b + c$.
    
    Suppose $a < b$. By definition of $<$, 
        there exists $p \in \mathbb{N}$ such that $a + p = b$.
    By Theorem $1.2.7$ part ($1$), $(a + p) + c = b + c$.
    By Theorem $1.2.7$ part ($2$), $a + (p + c) = b + c$.
    By Theorem $1.2.7$ part ($4$), $a + (c + p) = b + c$.
    By Theorem $1.2.7$ part ($2$), $(a + c) + p = b + c$.
    Thus by definition of $<$, $a + c < b + c$.

    Suppose $a + c < b + c$. There exists $p \in \mathbb{N}$
        such that $(a + c) + p = b + c$.
    By Theorem $1.2.7$ part ($4$), $p + (a + c) = b + c$.
    By Theorem $1.2.7$ part ($2$), $(p + a) + c = b + c$.
    By Theorem $1.2.7$ part ($1$), $p + a = b$ so,
        by Theorem $1.2.7$ part ($4$),  $a + p = b$.
    Thus by definition of $<$, $a < b$.
\end{proof}
\begin{proof}
    Let $a, b, c \in \mathbb{N}$.
    We must show $a < b$ if and only if $ac < bc$.
    
    Suppose $a < b$.
    For contradiction, suppose $ac \ge bc$.
    By definition of $\ge$, either $ac = bc$ or $ac > bc$.

    Suppose $ac = bc$.
    By Theorem $1.2.7$ part ($12$), $a = b$.
    But $a = b < b$ contradicting
        Theorem $1.2.9$ part ($1$).

    Suppose $ac > bc$.
    By definition of $<$, there exists $p_1, p_2 \in \mathbb{N}$
        such that $a + p_1 = b$ and $bc + p_2 = ac$.
    Then $bc + p_2 = (a + p_1)c + p_2 = ac + p_1 c + p_2 \;(\text{by Theorem $1.2.8$ part (8) for distributivity}) = ac$.
    By definition of $<$, $ac < ac$ contradicting
        Theorem $1.2.9$ part ($1$).

    Suppose $ac < bc$.
    For contradiction, suppose $a \ge b$. 
    By definition of $\ge$, either $a = b$ or $a > b$ 

    Suppose $a = b$.
    Then $ac = bc < bc$ which contradicts Theorem $1.2.9$ part ($1$).

    Suppose $a > b$.
    By definition of $<$, there exists $p \in \mathbb{N}$ such that $b + p = a$.
    Then, by Theorem $1.2.8$ part ($8$), $ac = (b + p)c = bc + pc$.
    By definition of $<$, $bc < ac$.
\end{proof}

\begin{proof}
    Let $a, b \in \mathbb{N}$.
    We must show $a < b$ if and only if $a + 1 \le b$.

    Suppose $a < b$.
    For contradiction, suppose $a + 1 > b$.
    By definition of $<$, there exists $p_2 \in \mathbb{N}$ such that $a + p_2 = b$.
    Since $a + 1 > b$, there exists $p_1 \in \mathbb{N}$ such that $b + p_1 = a + 1$.
    Then $b + p_1 = (a + p_2) + p_1 = a + 1$.
    By Theorem $1.2.7$ part ($4$), $p_1 + (a + p_2) = 1 + a$.
    By Theorem $1.2.7$ part ($4$), $p_1 + (p_2 + a) = 1 + a$.
    By Theorem $1.2.7$ part ($2$), $(p_1 + p_2) + a = 1 + a$.
    By Theorem $1.2.7$ part ($1$), $p_1 + p_2 = 1$
        contradicting Theorem $1.2.7$ part ($5$).

    Suppose $a + 1 \le b$.
    By definition of $\le$, either $a + 1 = b$ or $a + 1 < b$.
    
    Suppose $a + 1 = b$.
    By definition of $<$, $a < b$.

    Suppose $a + 1 < b$.
    For contradiction, suppose $a \ge b$.
    By definition of $\ge$, either $a = b$ or $a > b$.
    Suppose $a = b$, then $a + 1 = b + 1 > b$
        contradicting Theorem $1.2.7$ part ($6$).
    Suppose $a > b$.
    By definition of $<$, there exists $p_1, p_2 \in \mathbb{N}$ 
        such that $(a + 1) + p_1 = b$ and $b + p_2 = a$.
    Then $(a + 1) + p_1 = ((b + p_2) + 1) + p_1 = b$.
    By definition of $<$, $b < b$ contradicting
        Theorem $1.2.9$ part ($1$).
\end{proof}

\begin{tcolorbox}[title=Problem 5, breakable]
    Let $a, b \in \mathbb{N}$. 
    Prove that if $a + a = b + b$, then $a = b$.
\end{tcolorbox}

\begin{proof}
    Suppose $a + a = b + b$.
    First, by Theorem $1.2.6$ part (a),
         $a + a = a \cdot 1 + a \cdot 1$.
    Then, by Theorem $1.2.7$ part ($10$),
        $a \cdot 1 + a \cdot 1 = a(1 + 1) = a\cdot 2$.
    Similarly $b + b = b\cdot 2$.
    Then, by Theorem $1.2.7$ part ($12$),
        since $a\cdot 2 = b\cdot 2$, $a = b$.
\end{proof}

\begin{tcolorbox}[title=Problem 6, breakable]
    Let $b \in \mathbb{N}$. Prove that 
    \[\{n \in \mathbb{N} \mid 1 \le n \le b\} \cup \{n \in \mathbb{N} \mid b + 1 \le n\} = \mathbb{N}\]
    \[\{n \in \mathbb{N} \mid 1 \le n \le b\} \cap \{n \in \mathbb{N} \mid b + 1 \le n\} = \emptyset\]
\end{tcolorbox}

\begin{proof}
    Let $A = \{n \in \mathbb{N} \mid 1 \le n \le b\}$ and $B = \{n \in \mathbb{N} \mid b + 1 \le n\}$.
    It is clear that $A \subseteq \mathbb{N}$ and $B \subseteq \mathbb{N}$.
    Thus $A \cup B \subseteq \mathbb{N}$.
    Now let $x$ be an arbitrary element in $\mathbb{N}$.
    By Theorem $1.2.9$ part ($6$), either $x < b$, $x = b$, or $x > b$.
    Suppose $x < b$. Then $x \in A$, so $x \in A \cup B$.
    Suppose $x = b$. Then $x \in A$, so $x \in A \cup B$.
    Suppose $x > b$. Then $x \in B$, so $x \in A \cup B$.
    Therefore $\mathbb{N} \subseteq A \cup B$.
    It follows that $A \cup B = \mathbb{N}$.

    Suppose $A \cap B \ne \emptyset$.
    Let $x \in A \cap B$.
    Then $1 \le x \le b$ and $b + 1 \le x$.
    By Theorem $1.2.9$ part ($3$), $b + 1 \le x \le b$
        contradicting Theorem $1.2.9$ part ($9$).
\end{proof}

\begin{tcolorbox}[title=Problem 7, breakable]
    Let $A \subseteq N$ be a set.
    The set $A$ is \textbf{closed} if $a \in A$ implies $a + 1 \in A$.
    Suppose $A$ is closed.
    \begin{enumerate}
        \item Prove that if $a \in A$ and $n \in \mathbb{N}$, then $a + n \in A$.
        \item Prove that if $a \in A$, then $\{x \in \mathbb{N} \mid x \ge a\} \subseteq A$.
    \end{enumerate}
\end{tcolorbox}

\begin{proof}
    If $A = \emptyset$ then clearly the implication vacuously holds.
    Suppose $A \ne \emptyset$.
    Consider the set
    \[G = \{x \in \mathbb{N} \mid a + x \in A\}.\]
    We will show $G = \mathbb{N}$, proving our implication.
    Now, since $a \in A$ and $A$ is closed, $a + 1 \in A$, thus $1 \in G$.
    Suppose $x \in \mathbb{N}$ and $x \in G$.
    Then consider $a + s(x) = a + (x + 1)$.
    By Theorem $1.2.7$ part ($2$), $a + (x + 1) = (a + x) + 1$.
    By our hypothesis, $a + x \in A$.
    But since $A$ is closed, $(a + x) + 1 \in A$.
    Thus $s(x) \in G$.
    By the part (c) of the Peano Postulates, we conclude that $G = \mathbb{N}$.
\end{proof}

\begin{proof}
    Suppose $a \in A$. Let $x \in \mathbb{N}$ such that $x \ge a$.
    Either $x = a$ or $a < x$.
    Suppose $x = a$, then trivially $x = a \in A$.
    Suppose $a < x$.
    By definition of $<$, there exists $p \in \mathbb{N}$ 
        such that $a + p = x$.
    By the previous proof, $a + p = x \in A$.
\end{proof}

\begin{tcolorbox}[title=Problem 8, breakable]
    Suppose that the set $\mathbb{N}$ together with the element 
        $1 \in \mathbb{N}$ and the function $s : \mathbb{N} \rightarrow \mathbb{N}$,
        and the set $\mathbb{N}'$ together with the element $1' \in \mathbb{N}$
        and the function $s' : \mathbb{N}' \rightarrow \mathbb{N}'$, both satisfy 
        the Peano Postulates. Prove that there is a bijective function 
        $f : \mathbb{N} \rightarrow \mathbb{N}'$ such that $f(1) = 1'$
        and $f \circ s = s' \circ f$. The existence of such a bijective function.
\end{tcolorbox}

\begin{proof}
    We can apply Theorem $1.2.4$
        to the set $\mathbb{N}'$, the element $1'$
        and the function  $s' : \mathbb{N}' \rightarrow \mathbb{N}'$,
        to deduce that there is a unique function $f : \mathbb{N} \rightarrow \mathbb{N}'$ such that 
        $f \circ s = s' \circ f$ and $f(1) = 1'$.

    We can apply Theorem $1.2.4$ again,
        to the set $\mathbb{N}$, the element $1$
        and the function  $s : \mathbb{N} \rightarrow \mathbb{N}$,
        to deduce that there is a unique function $f' : \mathbb{N}' \rightarrow \mathbb{N}$ such that 
        $f' \circ s' = s \circ f'$ and $f'(1') = 1$.

    Now we must show $f'$ is the inverse of $f$.

    Consider $f' \circ f$.  
    Let $x \in \mathbb{N}$.  

    \textbf{Base case:} $x = 1$.  
    \[(f' \circ f)(x) = f'(f(1)) = f'(1') = 1 = x\]  

    \textbf{Inductive step:} Suppose $x > 1$. By Lemma $1.2.3$ there exists $y \in \mathbb{N}$ such that $s(y) = x$.  
    Suppose for $y \in \mathbb{N}$ such that $y < x$, $(f' \circ f)(y) = y$.
    Then 
    \begin{align*}
    (f' \circ f)(x) 
    &= f'(f(s(y))) \\
    &= f'(s'(f(y))) && \text{(by $f \circ s = s' \circ f$)} \\
    &= s(f'(f(y))) && \text{(by $f' \circ s' = s \circ f'$)} \\
    &= s(y) && \text{$y < x$} \\
    &= x
    \end{align*}

    Consider $f \circ f'$.  
    Let $x' \in \mathbb{N}'$.  

    \textbf{Base case:} $x' = 1'$.  
    \[(f \circ f')(x') = f(f'(1')) = f(1) = 1' = x'\]  

    \textbf{Inductive step:} Suppose $x' > 1'$. By Lemma $1.2.3$ there exists $y' \in \mathbb{N}'$ such that $s'(y') = x'$.  
    Suppose for $y' \in \mathbb{N}'$ such that $y' < x'$, $(f \circ f')(y') = y'$.
    Then
    \begin{align*}
    (f \circ f')(x') 
    &= f(f'(s'(y'))) \\
    &= f(s(f'(y'))) && \text{(by $f' \circ s' = s \circ f'$)} \\
    &= s'(f(f'(y'))) && \text{(by $f \circ s = s' \circ f$)} \\
    &= s'(y') && \text{(induction hypothesis)} \\
    &= x'
    \end{align*}

    Since $(f' \circ f)(x) = x$ and $(f \circ f')(x') = x'$, we conclude that $f'$ is the inverse of $f$.  
    Thus $f$ is bijective.
\end{proof}

\newpage
\begin{tcolorbox}[title=Extra Problem, breakable]
    Show the Peano axioms are independent. 
    That is, for any two Peano axioms, 
        find a structure that satisfies them but not the third. 
    You may assume the regular math of $\mathbb{Z}$, $\mathbb{Q}$, $\mathbb{R}$.
\end{tcolorbox}

\begin{axiom}[Peano Postulates]
    There exists a set $\mathbb{N}$ with an element $1 \in \mathbb{N}$
        and a function $s : \mathbb{N} \rightarrow \mathbb{N}$
        that satisfy the following three properties.
    \begin{enumerate}[label=\textbf{\alph*.}]
        \item There is no $n \in \mathbb{N}$ such that $s(n) = 1$.
        \item The function $s$ is injective.
        \item Let $G \subseteq \mathbb{N}$. Suppose that $1 \in G$, and
              that if $g \in G$ then $s(g) \in G$. Then $G = \mathbb{N}$.
    \end{enumerate}
\end{axiom}

\begin{proof}
    (\textbf{a., b.})
    Let $s : \mathbb{N} \rightarrow \mathbb{N}$ be defined by $s(x) = x + 2$.
    Let $G = \{x \mid \exists k \in \mathbb{Z}, x = 2k + 1\}$.
    Clearly $s$ is injective, $1 \in G$, and $G \subseteq \mathbb{N}$.
    But $G \ne \mathbb{N}$, and if $g \in G$ then $s(g) = g + 2 \in G$.
    Clearly \textbf{a., b.} 
        hold while \textbf{c.} does not hold.

    (\textbf{a., c.}) 
    Let $M = \{1, p\}$ and let $s : M \rightarrow M$ be defined
        by $s(1) = p$ and $s(p) = p$. Clearly \textbf{a., c.} 
        hold while \textbf{b.} does not hold.

    (\textbf{b., c.}) Let $M = \{1, p\}$ and let $s : M \rightarrow M$ be defined
        by $s(1) = p$ and $s(p) = 1$. Clearly \textbf{b., c} 
        hold while \textbf{a.} does not hold.
\end{proof}

\subsection{Constructing the Integers}

\begin{tcolorbox}[title=Problem 2, breakable]
    Complete the proof of Lemma $1.3.2$.
    That is, prove that the relation $\sim$ is transitive.
\end{tcolorbox}

\begin{proof}
    Let $(a, b), (c, d), (e, f) \in \mathbb{N} \times \mathbb{N}$.
    Assume $(a, b) \sim (c, d)$ and $(c, d) \sim (e, f)$.
    By definition of $\sim$,
        $a + d = b + c$ and $c + f = d + e$.
    Then taking sums shows $a + d + c + f = b + c + d + e$.
    Cancelling terms $a + f = b + e$.
    Thus, by definition of $\sim$, $(a, b) \sim (f, e)$.
    Since $\sim$ is symmetric, $(a, b) \sim (e, f)$.
\end{proof}

\begin{tcolorbox}[title=Problem 3, breakable]
    Complete the proof of Lemma $1.3.4$.
    That is, prove that $\cdot$ and $-$ for $\mathbb{Z}$
    are well-defined. The proof for $\cdot$ is a bit more complicated
    than might be expected. [Use Exercise $1.2.5$.]
\end{tcolorbox}

\begin{proof}
    Let $(a, b), (c, d), (x, y), (z, w) \in \mathbb{N} \times \mathbb{N}$.
    Suppose $(a, b) \sim (c, d)$ and $(x, y) \sim (z, w)$.
    So $a + d = b + c$ and $x + w = y + z$.
   

    Therefore, $(a, b) \cdot (x, y) \sim (c, d) \cdot (z, w)$, and multiplication is well-defined.
\end{proof}

\begin{proof}
    Let $(a, b), (c, d), (x, y), (z, w) \in \mathbb{N} \times \mathbb{N}$.
    Suppose $(a, b) \sim (c, d)$ and $(x, y) \sim (z, w)$.
    So $a + d = b + c$ and $x + w = y + z$.
    Summing shows $a + y + d + z = b + x + c + w$.
    Which is to say $(a + y, b + x) \sim (c + w, d + z)$.
    Therefore  $(a, b) + (y, x) \sim (c, d) + (w, z)$.
    It then follows that $(a, b) - (x, y) \sim (c, d) - (z, w)$.
    Thus $-$ is well defined.
\end{proof}

\begin{tcolorbox}[title=Problem 4, breakable]
    Let $a, b \in \mathbb{N}$.
    \begin{enumerate}
        \item Prove that $[(a,b)] = \hat{0}$ if and only if $a = b$.
        \item Prove that $[(a, b)] = \hat{1}$ if and only if $a = b + 1$.
        \item Prove that \textcircled{$1$} $[(a, b)] = [(n, 1)]$ for some $n \in \mathbb{N}$
              such that $n \ne 1$ if and only if \textcircled{$2$} $a > b$
              if and only if \textcircled{$3$} $[(a, b)] > \hat{0}$.
        \item Prove that \textcircled{$1$} $[(a, b)] = [(1, m)]$ for some $m \in \mathbb{N}$
              such that $m \ne 1$ if and only if \textcircled{$2$} $a < b$ if and only if 
              \textcircled{$3$} $[(a, b)] < \hat{0}$.
    \end{enumerate}
\end{tcolorbox}

\begin{proof}
    Suppose $[(a,b)] = \hat{0}$.
    Thus $(a, b) \sim (1, 1)$.
    Therefore $a + 1 = b + 1$.
    It follows that $a = b$.

    Suppose $a = b$.
    Then $a + 1 = b + 1$
    Therefore $(a, b) \sim (1, 1)$.
    It follows that $[(a,b)] = \hat{0}$.
\end{proof}

\begin{proof}
    Suppose $[(a, b)] = \hat{1}$.
    Thus $(a, b) \sim (1 + 1, 1)$.
    Therefore $a + 1 = b + (1 + 1)$.
    It follows that $a = b + 1$.

    Suppose $a = b + 1$.
    Thus $a + 1 = b + (1 + 1)$.
    Thus $(a, b) \sim (1 + 1, 1)$.
    It follows that $[(a,b)] = \hat{1}$.
\end{proof}

\begin{proof}
    (\textcircled{$1$} \rightarrow \textcircled{$2$})  
    Suppose $[(a, b)] = [(n, 1)]$ for some $n \in \mathbb{N}$ such that $n \ne 1$.  
    Thus $a + 1 = b + n$.  
    Since $n \ne 1$, $n > 1$.  
    There exists $p \in \mathbb{N}$ such that $s(p) = n$.  
    Then $a + 1 = b + s(p) = b + p + 1$.  
    It follows that $a = b + p$.  
    Thus $b < a$.  

    (\textcircled{$2$} \rightarrow \textcircled{$1$})  
    Suppose $a > b$.  
    There exists $p \in \mathbb{N}$ such that $a = b + p$.  
    Then $a + 1 = b + p + 1$.  
    It follows that $a + 1 = b + s(p)$.  
    Let $n = s(p)$.  
    Therefore $[(a, b)] = [(n, 1)]$ for some $n \in \mathbb{N}$ such that $n \ne 1$.  

    (\textcircled{$2$} \rightarrow \textcircled{$3$})  
    Suppose $a > b$.  
    There exists $p \in \mathbb{N}$ such that $a = b + p$.  
    Then $a + 1 = b + 1 + p$.  
    Therefore $[(a, b)] > \hat{0}$.  

    (\textcircled{$3$} \rightarrow \textcircled{$2$})  
    Suppose $[(a, b)] > \hat{0}$.  
    It follows that $a + 1 > b + 1$.  
    Thus there exists $p$ such that $a + 1 = b + 1 + p$.  
    Therefore $a = b + p$ and it follows that $a > b$.  
\end{proof}

\begin{proof}
    (\textcircled{$1$} \rightarrow \textcircled{$2$})  
    Suppose $[(a, b)] = [(1, m)]$ for some $m \in \mathbb{N}$ such that $m \ne 1$.  
    Then $a + m = b + 1$.  
    Since $m \ne 1$, $m > 1$.  
    There exists $p \in \mathbb{N}$ such that $s(p) = m$.  
    Then $a + s(p) = b + 1 \implies a + p + 1 = b + 1$.  
    It follows that $a = b - p$.  
    Thus $a < b$.  

    (\textcircled{$2$} \rightarrow \textcircled{$1$})  
    Suppose $a < b$.  
    There exists $p \in \mathbb{N}$ such that $b = a + p$ with $p \ne 0$.  
    Then $b + 1 = a + p + 1 = a + s(p)$.  
    Let $m = s(p)$. Then $m \ne 1$.  
    Therefore $[(a, b)] = [(1, m)]$ for some $m \in \mathbb{N}$ with $m \ne 1$.  

    (\textcircled{$2$} \rightarrow \textcircled{$3$})  
    Suppose $a < b$.  
    Then there exists $p \in \mathbb{N}$ such that $b = a + p$.  
    Then $b + 1 = a + 1 + p$.  
    Therefore $[(a, b)] < \hat{0}$.  

    (\textcircled{$3$} \rightarrow \textcircled{$2$})  
    Suppose $[(a, b)] < \hat{0}$.  
    It follows that $b + 1 > a + 1$.  
    Thus there exists $p \in \mathbb{N}$ such that $b + 1 = a + 1 + p$.  
    Therefore $b = a + p$, so $a < b$.  
\end{proof}

\begin{tcolorbox}[title=Problem 5, breakable]
    Prove Theorem $1.3.5$ ($1$) ($3$) ($4$) ($5$) ($6$)
    ($7$) ($8$) ($10$) ($11$) ($13$) ($14$).
\end{tcolorbox}

\begin{proof}
    Let $x, y, z \in \mathbb{Z}$.
    We must show $(x + y) + z = z + (x + y)$.
    Let $(x_1, x_2), (y_1, y_2), (z_1, z_2) \in \mathbb{N} \times \mathbb{N}$
        such that $x = (x_1, x_2)$, $y = (y_1, y_2)$
        and $z = (z_1, z_2)$.
    Then 
    \begin{align*}
        (x + y) + z 
            &= ([(x_1, x_2)] + [(y_1, y_2)]) + [(z_1, z_2)] \\
            &= [(x_1 + y_1), (x_2 + y_2)] + [(z_1, z_2)] \\
            &= [((x_1 + y_1) + z_1), ((x_2 + y_2) + z_2)] \\
            &= [(x_1 + (y_1 + z_1)), (x_2 + (y_2 + z_2))] \\
            &= [(x_1, x_2)] + [(y_1 + z_1), (y_2 + z_2)] \\
            &= [(x_1, x_2)] + ([y_1, y_2] +  [z_1, z_2]) \\
            &= x + (y + z)
    \end{align*}
\end{proof}

\begin{proof}
    We must show $x + \hat{0} = x$.
    Let $(x_1, x_2) \in \mathbb{N} \times \mathbb{N}$
        such that $x = (x_1, x_2)$.
    Then $x + \hat{0} = [(x_1, x_2)] + [(1, 1)] = [(x_1 + 1, x_2 + 1)]$.
    It follows that $x_1 + 1 + x_1 = x_2 + 1 + x_2$.

\end{proof}

\begin{tcolorbox}[title=Problem 6, breakable]
    Prove Theorem $1.3.7$ ($1$) ($3$) ($4(b)$) ($4(c)$).
\end{tcolorbox}

\begin{tcolorbox}[title=Problem 7, breakable]
    Let $x, y, z \in \mathbb{Z}$
    \begin{enumerate}
        \item Prove that $x < y$ if and only if $-x > -y$.
        \item Prove that if $z < 0$, then $x < y$ if and only if $xz > yz$.
    \end{enumerate}
\end{tcolorbox}

\begin{tcolorbox}[title=Problem 8, breakable]
    Let $x \in \mathbb{Z}$. Prove that if $x > 0$ then $x \ge 1$.
    Prove that if $x < 0$ then $x \le -1$.
\end{tcolorbox}

\begin{tcolorbox}[title=Problem 9, breakable]
    \begin{enumerate}
        \item Prove that $1 < 2$.
        \item Let $x \in \mathbb{Z}$. Prove that $2x \ne 1$.
    \end{enumerate}
\end{tcolorbox}

\begin{tcolorbox}[title=Problem 10, breakable]
    Prove that the Well-Order Principle (Theorem $1.2.10$),
    which was stated for $\mathbb{N}$ in Section $1.2$, still holds 
    when we think of $\mathbb{N}$ as the set of positive integers.
    That is, let $G \subseteq \{x \in \mathbb{Z} \mid x > 0\}$ be 
    a non-empty set. Prove that there is some $m \in G$ such that 
    $m \le g$ for all $g \in G$. Use Theorem $1.3.7$.
\end{tcolorbox}

\begin{tcolorbox}[title=Problem 11, breakable]
    Prove Theorem $1.3.8$ ($1$) ($3$) ($4$) ($5$) ($7$) ($10$) ($11$).
\end{tcolorbox}