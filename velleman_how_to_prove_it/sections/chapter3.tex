\subsection{Proof Strategies}

\begin{tcolorbox}[title=Problem 2, breakable]
    Consider the following theorem. (The theorem is correct, but we will not ask you to 
    prove it here.) \\ \\
    \textbf{Theorem.} Suppose that $b^2 > 4ac$. Then the quadratic equation 
    $ax^2 + bx + c = 0$ has exactly two real solutions. \\ \\
    (a) Identify the hypothesis and conclusion of the theorem. \\
    (b) To give an instance of the theorem, you must specify values for $a$, $b$,
    and $c$, but not $x$. Why? \\
    (c) What can you conclude from the theorem in the case $a = 2$, $b = -5$,
    $c = 3$? Check directly that this conclusion is correct. \\
    (d) What can you conclude from the theorem in the case $a = 2$, $b = 4$,
    $c = 3$? Check directly that this conclusion is correct.
\end{tcolorbox}

\textbf{Solution 1 (a)} \\
Hypothesis: $b^2 > 4ac$ and $a$, $b$, $c$ are real numbers. \\
Conclusion: $ax^2 + bx + c = 0$ has two real solutions. \\

\textbf{Solution 1 (b)} \\
$a$, $b$, and $c$ are free and $x$ is bound.

\textbf{Solution 1 (c)} \\
If $a = 2$, $b = -5$, and $c = 3$ \\
Since $(-5)^2 > 4(2)(3) \leftrightarrow 25 > 24$ is true \\
$2x^2 + (-5)x + 3 = 0$ has exactly two real roots. \\
\begin{align}
    2x^2 + (-5)x + 3                      & = 0  &  & \\
    \leftrightarrow 2x^2 - 3x - 2x + 3    & = 0  &  & \\
    \leftrightarrow  2x(x - 1) - 3(x - 1) & = 0  &  & \\
    \leftrightarrow (2x-3)(x - 1)         & = 0
\end{align}
Therefore $x = 1$ and $x = \frac{3}{2}$. \\

\textbf{Solution 1 (d)} \\
If $a = 2$, $b = 4$, and $c = 3$ \\
Since $4^2 > 4(2)(3) \leftrightarrow 16 > 24$ is false \\
$2x^2 + 4x + 3$ may or may not have exactly two real roots.

\begin{tcolorbox}[title=Problem 3, breakable]
    Consider the following theorem. \\ \\
    \textbf{Theorem.} Suppose $n$ is a natural number larger than $2$, and 
    $n$ is not a prime number. Then $2n + 13$ is not a prime number. \\ \\
    What are the hypotheses and conclusion of this theorem? Show that the 
    theorem theorem is incorrect by finding a counterexample.
\end{tcolorbox}

\textbf{Solution}
Hypotheses 
\begin{enumerate}
    \item $n \in \mathbb{N}$
    \item $n > 2$
    \item $n$ is not a prime number
\end{enumerate}
Conclusion: $2n + 13$ is not a prime number. \\
Counterexample: Let $n = 9$, $2(9) + 13 = 31$ which is prime.
Therefore the theorem is incorrect.

\begin{tcolorbox}[title=Problem 4, breakable]
    Complete the following alternative proof of the theorem in Example $3.1.2$.
    \begin{proof}
        Suppose $0 < a < b$. Then $b - a > 0$. \\   
        (Fill in a proof of $b^2 - a^2 > 0$ here) \\
        Since $b^2 - a^2 > 0$, it follows that $a^2 < b^2$. Therefore, 
        if $0 < a < b$ then $a^2 < b^2$.
    \end{proof}
\end{tcolorbox}

\begin{proof}
    Suppose $0 < a < b$. Then $b - a > 0$. \\  
    Since $b - a > 0$ we can multiply each side whithout flipping the inequality by $b+a$ which is a 
    positive quantity because $a$ and $b$ are greater than $0$. 
    We then get $(b + a)(b - a) > (b + a)0$ and $b^2 - a^2 > 0$. \\
    Since $b^2 - a^2 > 0$, it follows that $a^2 < b^2$. Therefore, if $0 < a < b$ then $a^2 < b^2$.
\end{proof}

\begin{tcolorbox}[title=Problem 5, breakable]
    Suppose $a$ and $b$ are real numbers. Prove that if $a < b < 0$ then $a^2 > b^2$.
\end{tcolorbox}

\begin{proof}
    Since $a < b$, then $a - b < 0$. Note that $a < 0$ and $b < 0$ so $a + b < 0$. 
    Then $(a + b)(a - b) > (a + b)0$ and $a^2 - b^2 > 0$. It follows that $a^2 > b^2$.
    Therefore supposing $a$ and $b$ are real numbers, if $a < b < 0$ then $a^2 > b^2$.
\end{proof}

\begin{tcolorbox}[title=Problem 6, breakable]
    Suppose $a$ and $b$ are real numbers. Prove that if $a < b < 0$ then $\frac{1}{b} < \frac{1}{a}$.
\end{tcolorbox}

\begin{proof}
    Since $a < b$, then $a - b < 0$. Note that $a < 0$ and $b < 0$ therefore $ab > 0$. Then
    \begin{align*}
        \frac{a - b}{ab}                            & < \frac{0}{ab} \\
        \leftrightarrow \frac{a}{ab} - \frac{b}{ab} & < 0            \\
        \leftrightarrow \frac{1}{b} - \frac{1}{a}   & < 0 
    \end{align*}
    It then follows that $\frac{1}{b} < \frac{1}{a}$. Therefore, 
    supposing $a$ and $b$ are real numbers, if $a < b < 0$ then $\frac{1}{b} < \frac{1}{a}$.
\end{proof}

\begin{tcolorbox}[title=Problem 7, breakable]
    Suppose $a$ is a real number. Prove that if $a^3 > a$ then $a^5 > a$.
\end{tcolorbox}

\begin{proof}
    Suppose $a^3 > a$. Then $a^3 - a > 0$. Note that $a^2 + 1 > 0$. Then 
    \begin{align*}
        a^3 - a                          & > 0          \\
        \leftrightarrow a^3 - a(a^2 + 1) & > (a^2 + 1)0 \\
        \leftrightarrow a^5 - a          & > 0 
    \end{align*}
    It then follows that $a^5 > a$. Therefore, supposing $a$ is a real number
    if $a^3 > a$ then $a^5 > a$.
\end{proof}

\begin{tcolorbox}[title=Problem 8, breakable]
    Suppose $A \setminus B \subseteq C \cap D$ and $x \in A$. Prove that if 
    $x \not \in D$ then $x \in B$.
\end{tcolorbox}

\begin{proof}
    Suppose $x \not \in D$. By contrapositive if $x \not \in C$ or $ x \not \in D$
    then $x \not \in A$ or  $x \in B$. Since $x \not \in D$, either $x \not \in A$ or $x \in B$, 
    but $x \in A$ thus $x \in B$. Therefore supposing $A \setminus B \subseteq C \cap D$ and $x \in A$,
    if $x \not \in D$ then $x \in B$.
\end{proof}

\begin{tcolorbox}[title=Problem 9, breakable]
    Suppose $A \cap C \subseteq C \setminus D$. Prove that if $x \in A$, then if $x \in D$
    then $x \not \in B$.
\end{tcolorbox}

\begin{proof}
    Suppose $x \in A$ and $x \in D$. By the contrapositive 
    if $x \not \in C$ or $x \in D$ then $x \not \in A$ or $x \not \in B$.
    Since $x \in D$, $x \not \in A$ or $x \not \in B$ but $x \in A$ thus $x \not \in B$.
    Therefore, supposing $A \cap C \subseteq C \setminus D$, if $x \in A$, then if $x \in D$
    then $x \not \in B$.
\end{proof}

\begin{tcolorbox}[title=Problem 11, breakable]
    Suppose $x$ is a real number and $x \not = 0$. Prove that if 
    $(\sqrt[3](x) + 5) / (x^2 + 6) = 1 / x$ then $x \not = 8$.
\end{tcolorbox}

\begin{proof}
    By contrapositive if $x = 8$ then  $(\sqrt[3](x) + 5) / (x^2 + 6) \not = 1 / x$. Suppose $x = 8$,
    $(\sqrt[3](8) + 5) / (8^2 + 6) \not = 1 / 8$ if and only if $7 / 72 \not = 1 / 8$. Therefore,
    supposing $x$ is a real number,
    if $(\sqrt[3](x) + 5) / (x^2 + 6) = 1 / x$ then $x \not = 8$. 
\end{proof}

\begin{tcolorbox}[title=Problem 12, breakable]
    Suppose $a$, $b$, $c$, and $d$ are real numbers $0 < a < b$, and $d > 0$. Prove that 
    if $ac \ge bd$ then $c > d$.
\end{tcolorbox}

\begin{proof}
    Suppose $ac \ge bd$. Note that $d > 0$ and $a > 0$. Since $a < b$, $ad < bd$ and $ac \ge bd > ad$. It follows
    that $ac > ad$ and $c > d$. Therefore supposing $a$, $b$, $c$, and $d$ are real numbers $0 < a < b$, and $d > 0$,
    if $ac \ge bd$ then $c > d$.
\end{proof}

\begin{tcolorbox}[title=Problem 14, breakable]
    Suppose that $x$ and $y$ are real numbers. Prove that if $x^2 + y = -3$ and 
    $2x - y = 2$ then $x= -1$.
\end{tcolorbox}

\begin{proof}
    Assume $x^2 + y = -3$ and $2x - y = 2$. Then $y = 2x - 2$ and 
    $x^2 + 2x - 2 = -3$ and finally $x^2 + 2x + 1 = 0$ which has two factors 
    both of which are $x + 1 = 0$ therefore $x = -1$. Therefore supposing 
    $x$ and $y$ are real numbers, if $x^2 + y = -3$ and 
    $2x - y = 2$ then $x= -1$.
\end{proof}

\begin{tcolorbox}[title=Problem 15, breakable]
    Prove the first theorem in Example $3.1.1$. (Hint: You might find it useful
    to apply the theorem from Example $3.1.2$). \\ \\
    \textbf{Theorem 3.1.1} Suppose $x > 3$ and $y < 2$. Then $x^2 - 2y > 5$.
\end{tcolorbox}

\begin{proof}
    Since $x > 3$ then $x^2 > 9$ by theorem $3.1.2$. 
    Also $y < 2$ then $2y < 4$ and $0 < 4 - 2y$.
    It follows that $x^2 > 9 > 4 > 2y$. Subtracting $2y$ and $5$ gives 
    $x^2 - 2y - 5 > 4 - 2y > -1 - 2y > -5$. Since $4 - 2y > 0$, 
    $x^2 - 2y - 5 > 0$ and $x^2 - 2y > 5$. Therefore supposing
    $x > 3$ and $y < 2$. Then $x^2 - 2y > 5$.
\end{proof}

\begin{tcolorbox}[title=Problem 16, breakable]
    Consider the following theorem. \\ \\
    \textbf{Theorem.} Suppose $x$ is a real number and $x \not = 4$. If 
    $(2x - 5)/(x - 4) = 3$ then $x = 7$.
    \begin{proof}
        Suppose $x = 7$. Then $(2x - 5)/(x - 4) = (2(7) - 5)/(7 - 4) = 9/3 = 3$.
        Therefore if $(2x - 5)(x - 4) = 3$ then $x = 7$.
    \end{proof}
\end{tcolorbox}

\textbf{Solution 16 (a)} \\
You cannot assume that the consequent of the conclusion is true.

\begin{proof}
    Suppose $(2x - 5)/(x - 4) = 3$; it follows that $2x - 5 = 3x - 12$ and $-5 = x - 12$. Thus $x = 7$.
    Therefore supposing $x$ is a real number and $x \not = 4$. If 
    $(2x - 5)/(x - 4) = 3$ then $x = 7$.
\end{proof}

\begin{tcolorbox}[title=Problem 17, breakable]
    Consider the following incorrect theorem: \\ \\
    \textbf{Incorrect Theorem.} Suppose that $x$ and $y$ are real numbers 
    and $x \not = 3$. If $x^2y = 9y$ then $y = 0$. \\ \\
    (a) What's wrong with the following proof of the theorem. \\
    \textbf{Proof}. Suppose that $x^2y = 9y$. Then $x^2 - y = 0$. Since
    $x \not = 3$, $x^2 \not = 9$, so $x^2 -9 \not = 0$. Therefore we can divide
    both sides of the equation $(x^2 - 9)y = 0$ by $x^2 - 9$,
    which leads to the conclusion that $y = 0$. Thus, if $x^2y =
        9y$ then $y = 0$. \\
    (b) Show that the theorem is incorrect by finding a counterexample.
\end{tcolorbox}

\textbf{Solution 17 (a)} \\
$x = -3$ is also a solution to $x^2 = 9$.

\textbf{Solution 17 (b)} \\
Suppose $x = -3$ and $y = 1$. Then ${(-3)}^2(1) = 9(1)$ which gives 
us $9 = 9$ which is true. So supposing $x$ and $y$ are real numbers  if $x^2y =
    9y$ then $y = 0$ is a false statement.

\subsection{Proofs Involving Negations and Conditionals}

\begin{tcolorbox}[title=Problem 1, breakable]
    This problem could be solved by using truth tables,
    but don’t do it that way. Instead, use the methods
    for writing proofs discussed so far in this chapter.
    (See Example $3.2.4.$) \\
    (a) Suppose $P \rightarrow Q$ and $Q \rightarrow R$ are both true. Prove that
    $P \rightarrow R$ is true. \\
    (b) Suppose $\neg R \rightarrow (P \rightarrow \neg Q)$ is true. Prove that $P \rightarrow (Q
        \rightarrow R)$ is true.
\end{tcolorbox}

\begin{proof}
    Suppose $P$ therefore $Q$ and since $Q$ then $R$. Therefore,
    supposing $P \rightarrow Q$ and $Q \rightarrow R$ is true, $P \rightarrow R$
    is true. Therefore, supposing $\neg R \rightarrow (P \rightarrow \neg Q)$ is true $P \rightarrow (Q
        \rightarrow R)$ is true.
\end{proof}

\begin{proof}
    Suppose $P$ and $Q$ are true. By the contrapositive $\neg R \rightarrow (P \rightarrow \neg Q)$
    is $\neg(P \rightarrow \neg Q) \rightarrow R$. This is equivalent to 
    $(P \wedge Q) \rightarrow R$ and since $P \wedge Q$ therefore $R$.
    Therefore supposing $\neg R \rightarrow (P \rightarrow \neg Q)$ is true, $P \rightarrow (Q
        \rightarrow R)$ is true.
\end{proof}

\begin{tcolorbox}[title=Problem 2, breakable]
    This problem could be solved by using truth tables,
    but don’t do it that way. Instead, use the methods
    for writing proofs discussed so far in this chapter.
    (See Example $3.2.4.$) \\
    (a) Suppose $P \rightarrow  Q$ and $R \rightarrow  \neg Q$ are both true. Prove
    that $P \rightarrow  \neg R$ is true. \\
    (b) Suppose that $P$ is true. Prove that $Q \rightarrow  \neg (Q \rightarrow  \neg P)$ is
    true.
\end{tcolorbox}

\begin{proof}
    Suppose $P$ then $Q$, and the contrapostive of $R \rightarrow \neg Q$ is  
    $Q \rightarrow \neg R$. Since $Q$ then $\neg R$. Therefore, supposing 
    $P \rightarrow  Q$ and $R \rightarrow  \neg Q$ are both true $P \rightarrow  \neg R$ is true. 
\end{proof}

\begin{proof}
    Suppose $Q$, we now need to show $\neg(Q \rightarrow \neg P)$ which is equivalent to $Q \wedge P$.
    Since $Q$ is true and $P$ is true, $Q$ and $P$ is true. Therefore,
    supposing that $P$ is true  $Q \rightarrow  \neg (Q \rightarrow  \neg P)$ is
    true.
\end{proof}

\begin{tcolorbox}[title=Problem 3, breakable]
    Suppose $A \subseteq C$, and $B$ and $C$ are disjoint. Prove that
    if $x \in A$ then $x \not\in B$.
\end{tcolorbox}

\begin{proof}
    Suppose $x \in A$. Since $A$ is a subset of $C$ then $x \in C$. It follows from
    $C$ being disjoint from $B$ that since $x \in C$, $x \not \in B$. Therefore, 
    supposing $A \subseteq C$, and $B$ and $C$ are disjoint, if $x \in A$ then $x \not\in B$.
\end{proof}

\begin{tcolorbox}[title=Problem 4, breakable]
    Suppose that $A \setminus B$ is disjoint from $C$ and $x \in A$.
    Prove that if $x \in C$ then $x \in B$.
\end{tcolorbox}

\begin{proof}
    Suppose $x \in C$ and assume for contradiction $x \not \in B$.
    Since $x \in A$ and $x \not \in B$, $x \in A \setminus B$, but
    $x \in C$ and $A \setminus B$ is disjoint from $C$ therefore $x$
    cannot be in both creating a contradiction. Therefore, 
    supposing $A \setminus B$ is disjoint from $C$ and $x \in A$ if $x \in C$ then $x \in B$.
\end{proof}

\begin{tcolorbox}[title=Problem 5, breakable]
    Prove that it cannot be the case that $x \in A \setminus B$ and $x
        \in B \setminus C$.
\end{tcolorbox}

\begin{proof}
    Assume for contradiction $x \in A \setminus B$. This means $x \in A$ and $x \not \in B$.
    But $x$ is also in $B$ because $x \in B \setminus C$ which is a contradiction. Therefore,
    it cannot be the case that $x \in A \setminus B$ and $x\in B \setminus C$.
\end{proof}

\begin{tcolorbox}[title=Problem 6, breakable]
    Use the method of proof by contradiction to prove
    the theorem in Example $3.2.1$. \\ \\
    \textbf{Theorem 3.2.1} Suppose $A \cap C \subseteq B$ and $x \in C$.
    Prove that $x \not \in A \setminus B$.
\end{tcolorbox}

\begin{proof}
    Assume for contradiction $x \in A \setminus B$. 
    Therefore $x \in A$ and $x \not \in B$, but since 
    $x \in A$ and $x \in C$ and $A \cap C \subseteq B$, $x \in B$. This is a contradiction.
    Therefore supposing $A \cap C \subseteq B$ and $x \in C$, $x \not \in A \setminus B$.
\end{proof}

\begin{tcolorbox}[title=Problem 7, breakable]
    Use the method of proof by contradiction to prove
    the theorem in Example $3.2.5$. \\ \\
    \textbf{Theorem 3.2.5} Suppose $A \subseteq B$, $x \in A$ and $x \not \in B \setminus C$.
    Prove that $x \in C$.
\end{tcolorbox}

\begin{proof}
    Assume $x \not \in C$. Since $x \not \in B \setminus C$, $x \not \in B$
    or $x \not \in C$, but $x \not \in C$ therefore $x \not \in B$. Since $x \in A$
    and $A \subseteq B$, $x \in B$ which leads to a contradiction. Therefore,
    supposing $A \subseteq B$, $x \in A$ and $x \not \in B \setminus C$, $x \in C$.
\end{proof}

\begin{tcolorbox}[title=Problem 8, breakable]
    Suppose that $y + x = 2y - x$, and $x$ and $y$ are not
    both zero. Prove that $y \not = 0$.
\end{tcolorbox}

\begin{proof}
    Assume $y = 0$ this means that $x \not = 0$. Then $y + x = 2y - x$
    and setting $y = 0$ gives us $x = -x$. This is only possible
    if $x = 0$ which contradicts that $x \not = 0$ thus $y \not = 0$. Therefore,
    supposing $y + x = 2y - x$, and $x$ and $y$ are not
    both zero, $y \not = 0$.
\end{proof}

\begin{tcolorbox}[title=Problem 9, breakable]
    Suppose that $a$ and $b$ are nonzero real numbers.
    Prove that if $a < 1/a < b < 1/b$ then $a < -1$.
\end{tcolorbox}

\begin{proof}
    Suppose $a < 1/a < b < 1/b$ and assume for contradiction $a \ge -1$.
    There are two cases \\
    \textbf{Case 1 $a > 0$}
    \begin{adjustwidth}{2em}{}
        Since $a > 0$ it follows that $a^2 > 0$. Since $0 < a < 1/a < b < 1/b$
        we can multiply by $a$ to get $0 < a^2 < 1 < ab < a/b$.
        Note that $b > 0$ and $1 < ab$. Dividing by $b$ we get 
        $1/b < a$ but we supposed that $a < 1/b$. Therefore 
        $a \le 0$.
    \end{adjustwidth}
    \textbf{Case 2 $a < 0$ and $a \ge -1$}
    \begin{adjustwidth}{2em}{}
        Note that since $a < 0$ and $a \ge -1$, $a^2 \le 1$.
        Since $a < 1/a$ and $a < 0$ we can multiply by $a$
        and get $a^2 > 1$. But $a^2 > 1$ and $a^2 \le 1$ which is a contradiction.
        Therefore $a > 0$ or $a < -1$.
    \end{adjustwidth}
    Since $a$ being greater than or equal to $-1$ leads to a contradiction in all cases,
    $a$ must be less than $1$. Therefore, supposing that $a$ and $b$ are nonzero real numbers
    if $a < 1/a < b < 1/b$ then $a < -1$.
\end{proof}

\begin{tcolorbox}[title=Problem 11, breakable]
    Suppose that $x$ and $y$ are real numbers. Prove that
    if $x \not = 0$, then if $y = (3x^2 + 2y)/(x^2 + 2)$ then $y = 3$.
\end{tcolorbox}

\begin{proof}
    By the contrapositive $(y = (3x^2 + 2y)/(x^2 + 2) \wedge y \not = 3) \rightarrow x = 0$.
    Assume for contradiction $x \not = 0$ and suppose $y = (3x^2 + 2y)/(x^2 + 2)$ and $y \not = 3$.
    \begin{align*}
        y         & = (3x^2 + 2y)/(x^2 + 2) &  & \\
        yx^2 + 2y & = 3x^2 + 2y             &  & \\
        yx^2      & = 3x^2                  &  & \\
        y         & = 3
    \end{align*}
    Since this is a contradiction $x = 0$. Therefore supposing $x$ and $y$ are real numbers 
    if $x \not = 0$, then if $y = (3x^2 + 2y)/(x^2 + 2)$ then $y = 3$.
\end{proof}

\begin{tcolorbox}[title=Problem 12, breakable]
    Consider the following incorrect theorem:
    Incorrect Theorem. Suppose $x$ and $y$ are real
    numbers and $x + y = 10$. Then $x \not = 3$ and $y \not = 8$. \\
    (a) What’s wrong with the following proof of the
    theorem?
    \begin{proof}
        Suppose the conclusion of the theorem is
        false. Then $x = 3$ and $y = 8$. But then $x + y = 11$,
        which contradicts the given information that $x + y
            = 10$. Therefore the conclusion must be true.
    \end{proof}
    (b) Show that the theorem is incorrect by finding a
    counterexample.
\end{tcolorbox}

\textbf{Solution 12 (a)}
The negation of the conclusion is $x = 3$ or $y = 8$. It only rules out
the case where $x = 3$ and $y = 8$. 

\textbf{Solution 12 (b)} Let $x = 3$ and $y = 7$ then $x + y = 10$.

\begin{tcolorbox}[title=Problem 13, breakable]Consider the following incorrect theorem:
    Incorrect Theorem. Suppose that $A \subseteq C$, $B \subseteq C$,
    and $x \in A$. Then $x \in B$. \\
    (a) What’s wrong with the following proof of the
    theorem?
    \begin{proof}
        Suppose that $x \not\in B$. Since $x \in A$ and $A \subseteq C$, $x
            \in C$. Since $x \not\in B$ and $B \subseteq C$, $x \not\in C$. But now we
        have proven both $x \in C$ and $x \not\in C$, so we have
        reached a contradiction. Therefore $x \in B$.
    \end{proof}
    (b) Show that the theorem is incorrect by finding a counter example.
\end{tcolorbox}

\textbf{Solution 13 (a)}
This does not follow ``since $x \not\in B$ and $B \subseteq C$, $x \not\in C$''.
$B$ is a subset of $C$ means if $x$ is in $B$ then it is in $C$. 
Since $x$ is not in $B$ it may or may not be in $C$.

\textbf{Solution 13 (b)} \\
Let $A = \{6\}$ \\
Let $B = \{4, 5\}$ \\
Let $C = \{4, 5, 6\}$ \\
Now let $x = 6$. 

\begin{tcolorbox}[title=Problem 18, breakable]
    Can the proof in Example 3.2.2 be modified to
    prove that if $x^2 + y = 13$ and $x \not = 3$ then $y \not = 4$
    Explain.
\end{tcolorbox}

\textbf{Solution 18}

No it cannot be modified to prove the new claim. Counterexample: $y = 4$ and $x
    = -3$ so $x \not = 3$ then $x^2 + y = {(-3)}^2 + 4 = 9 + 4 = 13$. So the
premises are true and $y = 4$ so the implication is invalid.

\subsection{Proofs Involving Quantifiers}

\begin{tcolorbox}[title=Problem 1, breakable]
    In exercise $7$ of Section $2.2$ you used logical
    equivalences to show that $\exists{x}(P(x) \rightarrow Q(x))$ is
    equivalent to $\forall{x}P(x) \rightarrow \exists{x}Q(x)$. Now use the
    methods of this section to prove that if $\exists{x}(P(x) \rightarrow
        Q(x))$ is true, then $\forall{x}P(x) \rightarrow \exists{x}Q(x)$ is true. (Note:
    The other direction of the equivalence is quite a bit
    harder to prove. See exercise $30$ of Section $3.5$.)
\end{tcolorbox}

\begin{proof}
    Suppose $x_0$ exists such that $P(x_0)$ implies $Q(x_0)$ is true.
    Assume for contradiction for all $z$ $P(z)$ is true, and for all $y$ $Q(y)$ is false. 
    This is a contradiction because $P$ and $Q$ are true for $x_0$. Therefore, 
    if $\exists{x}(P(x) \rightarrow Q(x))$ is true, then 
    $\forall{x}P(x) \rightarrow \exists{x}Q(x)$ is true.
\end{proof}

\begin{tcolorbox}[title=Problem 2, breakable]
    Prove that if $A$ and $B \setminus C$ are disjoint, then $A \cap B \subseteq
        C$.
\end{tcolorbox}

\begin{proof}
    Suppose $A$ and $B \setminus C$ are disjoint. Let $x$ be an arbitrary element such
    that $x \in A \cap B$. Since $x \in A$ and $x \in B$, either $x \not \in B$
    or $x \in C$. Since $x \in B$ it follows that $x \in C$.
    Therefore if $A$ and $B \setminus C$ are disjoint, then $A \cap B \subseteq
        C$.
\end{proof}

\begin{tcolorbox}[title=Problem 3, breakable]
    Prove that if $A \subseteq B \setminus C$ then $A$ and $C$ are disjoint.
\end{tcolorbox}

\begin{proof}
    Suppose $A \subseteq B \setminus C$. Let $x$ be an arbitrary element in $A$.
    It follows that $x \in B$ and $x \not \in C$. Therefore $x \in A$ and $x \not \in C$.
    Since $x$ was arbitrary then for all $x \in A$, $x \not \in C$.
\end{proof}

\begin{tcolorbox}[title=Problem 4, breakable]
    Suppose $A \subseteq  \mathcal{P}(A)$. Prove that $\mathcal{P}(A) \subseteq  \mathcal{P}(\mathcal{P}(A))$.
\end{tcolorbox}

\begin{proof}
    Suppose $x$, $y$ are arbitrary and $x \in \mathcal{P}(A)$ and $y \in x$. 
    Since $A \subseteq \mathcal{P}(A)$, 
    $y \in x$, and $x \subseteq A$,
    $y \in \mathcal{P}(A)$. We know $y \in \mathcal{P}(A)$ for all $y \in x$
    because $y$ is arbitrary. So $x$ is a set of elements that are all in $\mathcal{P}(A)$.
    This subset must be an element of $\mathcal{P}(\mathcal{P}(A))$.
    Since $x$ was arbitrary, for all $x \in \mathcal{P}(A)$,  
    $x \in \mathcal{P}(\mathcal{P}(A))$. Therefore
    $\mathcal{P}(A) \subseteq \mathcal{P}(\mathcal{P}(A))$.
\end{proof}

\begin{tcolorbox}[title=Problem 5, breakable]
    The hypothesis of the theorem proven in exercise $4$
    is $A \subseteq  P(A)$. \\ 
    (a) Can you think of a set $A$ for which this hypothesis is
    true? \\
    (b) Can you think of another?
\end{tcolorbox}

\textbf{Solution 5(a)} \\
Let $A = \emptyset$. Then $\mathcal{P}(A) = \emptyset$. In this case $A \subseteq \mathcal{P}(A)$. \\
\textbf{Solution 5(b)} \\
Let $A = \{\emptyset\}$. Then $\mathcal{P}(A) = \{\emptyset, \{\emptyset\}\}$. In this case $A \subseteq \mathcal{P}(A)$.

\begin{tcolorbox}[title=Problem 6, breakable]
    Suppose $x$ is a real number. \\
    (a) Prove that if $x \not =  1$ then there is a real number $y$
    such that $\frac{y + 1}{y - 2} = x$ \\
    (b) Prove that if there is a real number $y$ such that $\frac{y + 1}{y - 2} = x$
    then $x \not =  1$.
\end{tcolorbox}

\begin{proof}
    Suppose $x \not = 1$ and let $y = \frac{-1 - 2x}{1 - x}$. Since $x \not = 1$
    $y$ is defined. Then 
    \begin{align*}
        \frac{y + 1}{y - 2}
         & = \frac{\frac{-1 - 2x}{1 - x} + 1}{\frac{-1 - 2x}{1 - x} - 2}                           &  & \\
         & = \frac{1 - x}{1 - x} \cdot \frac{\frac{-1 - 2x}{1 - x} + 1}{\frac{-1 - 2x}{1 - x} - 2} &  & \\
         & = \frac{1 - 2x + 1 - x}{-1 - 2x - 2 + 2x}                                               &  & \\
         & = \frac{1 - 2x + 1 - x}{-1 - 2x - 2 + 2x}                                               &  & \\
         & = \frac{-3x}{-3}                                                                        &  & \\
         & = x
    \end{align*}
    Since $\frac{y + 1}{y - 2} = x$ then there is a real number $y$
    such that $\frac{y + 1}{y - 2} = x$.
\end{proof}

\begin{proof}
    Suppose there is a real number $y$ such that $\frac{y + 1}{y - 2} = x$ and assume 
    $x = 1$ for contradiction. Then $\frac{y + 1}{y - 2} = 1$ and multiplying 
    by $y - 2$ on each sides gives $y + 1 = y - 2$. This is a contradiction, so 
    $x \not = 1$. Therefore, if there is a real number $y$ such that $\frac{y + 1}{y - 2} = x$
    then $x \not =  1$.
\end{proof}

\begin{tcolorbox}[title=Problem 7, breakable]
    Prove that for every real number $x$, if $x > 2$ then
    there is a real number $y$ such that $y + 1/y = x$.
\end{tcolorbox}

\begin{proof}
    Suppose $x > 2$. Then 
    \begin{align*}
        y + \frac{1}{y} & = x  &  & \\
        y^2 + 1         & = xy &  & \\
        y^2 - xy + 1    & = 0
    \end{align*}
    We can then analyze the discriminant $b^2 - 4ac$
    where $a = 1$, $b = -x$ and $c = 1$. So ${(-x)}^2 - 4(1)(1)$
    and simplifying gives ${(-x)}^2 - 4$. Since $x > 2$ it follows
    that ${(-x)}^2 > 4$ and therefore ${(-x)}^2 - 4 > 0$. Since the 
    discriminant is greater than $0$, there are two distinct real roots
    $y$ such that  $y + 1/y = x$.

    We need to now verify that $y \not = 0$ since $y + \frac{1}{y} = x$ where $y =
        0$ is undefined. By the quadratic equation we need to show that $-b \pm
        \sqrt{b^2 - 4ac} \not = 0$. Note that $b \not = 0$ and $\sqrt{b^2 - 4ac} \not =
        0$.

    Suppose $-b + \sqrt{b^2 - 4ac} = 0$ where $a = 1$, $b = -x$ where $x > 2$, and
    $c = 1$. Then 
    \begin{align*}
        -b  & = \sqrt{b^2 - 4ac}     &  & \\
        -b  & = \sqrt{b^2 - 4(1)(1)} &  & \\
        -b  & = \sqrt{b^2 - 4}       &  & \\
        b^2 & = b^2 - 4              &  & \\
        0   & = -4
    \end{align*}
    This is a contradiction so $-b + \sqrt{b^2 - 4ac} \not = 0$.

    Suppose $-b - \sqrt{b^2 - 4ac} = 0$ where $a = 1$, $b = -x$ where $x > 2$, and
    $c = 1$. Then 
    \begin{align*}
        b   & = \sqrt{b^2 - 4ac}     &  & \\
        b   & = \sqrt{b^2 - 4(1)(1)} &  & \\
        b   & = \sqrt{b^2 - 4}       &  & \\
        b^2 & = b^2 - 4              &  & \\
        0   & = -4
    \end{align*}
    This is a contradiction so $-b - \sqrt{b^2 - 4ac} \not = 0$.

    Therefore, if $x > 2$ then there is a real number $y$ such that $y + 1/y = x$.
\end{proof}

\begin{tcolorbox}[title=Problem 8, breakable]
    Prove that if $F$ is a family of sets and $A \in  F$, then $A
        \subseteq  \bigcup F$.
\end{tcolorbox}

\begin{proof}
    Suppose $x \in A$. Since $x \in A$ and $A \in F$,
    $x \in \bigcup F$ by the definition of a union of 
    a family of sets. Therefore, if $F$ is a family of sets and $A \in  F$, then $A
        \subseteq  \bigcup F$.
\end{proof}

\begin{tcolorbox}[title=Problem 9, breakable]
    Prove that if $F$ is a family of sets and $A \in  F$, then
    $\bigcap F \subseteq  A$.
\end{tcolorbox}

\begin{proof}
    Suppose $A \in F$, and $y \in \bigcap F$. For all sets $B$
    in $F$, $y \in B$. It follows that $y \in A$ since $A \in F$. 
    Therefore, if $F$ is a family of sets and $A \in  F$, then
    $\bigcap F \subseteq  A$.
\end{proof}

\begin{tcolorbox}[title=Problem 10, breakable]
    Suppose that $F$ is a nonempty family of sets, $B$ is a
    set, and $\forall{A} \in  F (B \subseteq  A)$. Prove that $B \subseteq  \bigcap F$.
\end{tcolorbox}

\begin{proof}
    Suppose $x \in B$. Since $x \in B$ and for all sets $A$ in $F$ $B \subseteq  A$, it must 
    be the case that $x \in A$. Therefore for every set $A$ in $F$ $x \in A$. 
    It follows by the definition of the intersection of a family of sets 
    that $x \in \bigcap F$.
\end{proof}

\begin{tcolorbox}[title=Problem 11, breakable]
    Suppose that $F$ is a family of sets. Prove that if $\emptyset \in
        F$ then $\bigcap F = \emptyset$.
\end{tcolorbox}

\begin{proof}
    Suppose $\emptyset \in F$ and assume for contradiction $\bigcap F \not = \emptyset$.
    Let $x_0$ be an element of $\bigcap F$. $x_0$ must be in all
    sets within $F$ but $x_0 \not \in \emptyset$ which is a conradiction. 
    Therefore, if $\emptyset \in  F$ then $\bigcap F = \emptyset$.
\end{proof}

\begin{tcolorbox}[title=Problem 12, breakable]
    Suppose $\mathcal{F}$ and $\mathcal{G}$ are families of sets. Prove that if
    $\mathcal{F} \subseteq  \mathcal{G}$ then $\bigcup \mathcal{F} \subseteq \bigcup \mathcal{G}$. 
\end{tcolorbox}

\begin{proof}
    Suppose $\mathcal{F} \subseteq \mathcal{G}$
    and assume for contradiction $\bigcup \mathcal{F} \not \subseteq \bigcup \mathcal{G}$.
    This means there exists $x_0$
    such that $x_0 \in \bigcup \mathcal{F}$ and $x_0 \not \in \bigcup \mathcal{G}$. 
    Since $x_0 \in \bigcup \mathcal{F}$, there is a set $A \in \mathcal{F}$ such that $x_0 \in A$.
    Note that $A \not \in \mathcal{G}$ or $x_0$ would be in $\bigcup{\mathcal{G}}$. This means
    $A \in \mathcal{F}$ and $A \not \in \mathcal{G}$ which is a contradiction since
    $\mathcal{F} \subseteq \mathcal{G}$. 
    Therefore, if $\mathcal{F} \subseteq  \mathcal{G}$ then $\bigcup \mathcal{F} \subseteq \bigcup \mathcal{G}$. 
\end{proof}

\begin{tcolorbox}[title=Problem 13, breakable]
    Suppose $\mathcal{F}$ and $\mathcal{G}$ are nonempty families of sets.
    Prove that if $\mathcal{F} \subseteq  \mathcal{G}$ then $\bigcap \mathcal{G} \subseteq  \bigcap \mathcal{F}$.
\end{tcolorbox}

\begin{proof}
    Suppose $\mathcal{F} \subseteq  \mathcal{G}$
    and assume for contradiction $\bigcap \mathcal{G} \not \subseteq \bigcap \mathcal{F}$.
    Let $x_0$ be the element such that $x_0 \in \bigcap \mathcal{G}$ and $x_0 \not \in \bigcap \mathcal{F}$.
    This means $x_0$ is in every set in $\mathcal{G}$ and is not in at least one set in $\mathcal{F}$.
    Let $A$ be the set such that $x_0 \not \in A$ and $A \in \mathcal{F}$.
    But all sets in $\mathcal{G}$ contain $x_0$ so $A \not \in \mathcal{G}$.
    This is contradiction since $\mathcal{F} \subseteq  \mathcal{G}$.
    Therefore if $\mathcal{F} \subseteq  \mathcal{G}$ then $\bigcap \mathcal{G} \subseteq  \bigcap \mathcal{F}$.
\end{proof}

\begin{tcolorbox}[title=Problem 14, breakable]
    Suppose that $\{A_i | i \in  I\}$ is an indexed family of
    sets. Prove that $\bigcup_{i \in I} \mathcal{P}(A_i) \subseteq \mathcal{P}(\bigcup_{i \in I} A_i)$
    (Hint: First make sure you know what all the notation means!).
\end{tcolorbox}

\begin{proof}
    Assume for contradiction $\bigcup_{i \in I} \mathcal{P}(A_i) \not \subseteq \mathcal{P}(\bigcup_{i \in I} A_i)$.
    Let $A \in \bigcup_{i \in I} \mathcal{P}(A_i)$ and $A \not \in \mathcal{P}(\bigcup_{i \in I} A_i)$.
    Since $A \in \bigcup_{i \in I} \mathcal{P}(A_i)$, $A \subseteq A_i$ for some $i \in I$.
    It follows that $A \subseteq \bigcup_{i \in I} A_i$
    and therefore $A \in \mathcal{P}(\bigcup_{i \in I} A_i)$ which is a contradiction.
    Therefore $\bigcup_{i \in I} \mathcal{P}(A_i) \subseteq \mathcal{P}(\bigcup_{i \in I} A_i)$.
\end{proof}

\begin{tcolorbox}[title=Problem 15, breakable]
    Suppose $\{A_i | i \in  I\}$ is an indexed family of sets and
    $I \not = \emptyset$. Prove that  $\bigcap_{i \in I} A_i \in \bigcap_{i \in I} \mathcal{P}(A_i)$.
\end{tcolorbox}

\begin{proof}
    First note that $\bigcap_{i \in I} \mathcal{P}(A_i)$
    consists of the sets that are subsets of every $A_i$, where $i \in I$.
    $\bigcap_{i \in I} A_i$ is one such set since it is a subset of every $A_i$ by definition of the intersection.
    Therefore, $\bigcap_{i \in I} A_i \in \bigcap_{i \in I} \mathcal{P}(A_i)$.
\end{proof}

\begin{tcolorbox}[title=Problem 16, breakable]
    Prove the converse of the statement proven in
    Example $3.3.5$. In other words, prove that if $\mathcal{F} \subseteq
        \mathcal{P}(B)$ then $\bigcup \mathcal{F} \subseteq B$.
\end{tcolorbox}

\begin{proof}
    Suppose $\mathcal{F} \subseteq \mathcal{P}(B)$.
    Let $x_0$ be an arbitrary element in $\bigcup \mathcal{F}$.
    $x_0$ must be in a set $A \in \mathcal{F}$. 
    Since $\mathcal{F} \subseteq \mathcal{P}(B)$, $A \in \mathcal{P}(B)$.
    It then follows that $A \subseteq B$.
    Since $x_0 \in A$, $x_0 \in B$.
    Therefore, if $\mathcal{F} \subseteq
        \mathcal{P}(B)$ then $\bigcup \mathcal{F} \subseteq B$.
\end{proof}

\begin{tcolorbox}[title=Problem 17, breakable]
    Suppose $\mathcal{F}$ and $\mathcal{G}$ are nonempty families of sets,
    and every element of $\mathcal{F}$ is a subset of every element
    of $\mathcal{G}$. Prove that $\bigcup \mathcal{F} \subseteq  \bigcap \mathcal{G}$.
\end{tcolorbox}

\begin{proof}
    Let $x_0$ be an arbitrary element such that $x_0 \in \bigcup \mathcal{F}$.
    There must exist $A \in \mathcal{F}$ such that $x_0 \in A$.
    Since $A$ is a subset of all sets in $\mathcal{G}$,
    $x_0$ is in all sets in $\mathcal{G}$. It follows that $x_0 \in \bigcap \mathcal{G}$.
    Therefore, $\bigcup \mathcal{F} \subseteq  \bigcap \mathcal{G}$.
\end{proof}

\begin{tcolorbox}[title=Problem 20, breakable]
    Consider the following theorem:
    Theorem. For every real number $x$, $x^2 \ge 0$.
    What’s wrong with the following proof of the
    theorem?
    \begin{proof}
        Suppose not. Then for every real number $x$,
        $x^2 < 0$. In particular, plugging in $x = 3$ we would get
        $9 < 0$, which is clearly false. This contradiction
        shows that for every number $x$, $x^2 \ge 0$.
    \end{proof}
\end{tcolorbox}

\textbf{Solution 20}
The proof assumes an incorrect negation of for every real number $x$, $x^2 \ge 0$.
The negation should be there exists $x$ such that $x^2 < 0$.

\begin{tcolorbox}[title=Problem 21, breakable]
    Consider the following incorrect theorem:
    Incorrect Theorem. If $\forall{x} \in  A(x \not =  0)$ and $A \subseteq  B$
    then $\forall{x} \in  B(x \not =  0)$. \\
    (a) What’s wrong with the following proof of the
    theorem?
    \begin{proof}
        Suppose that $\forall{x} \in  A(x \not =  0)$ and $A \subseteq  B$. Let $x$
        be an arbitrary element of $A$. Since $\forall{x} \in  A(x \not =  0)$,
        we can conclude that $x \not =  0$. Also, since $A \subseteq  B$, $x \in
            B$. Since $x \in  B$, $x \not =  0$, and $x$ was arbitrary, we can
        conclude that $\forall{x} \in  B(x \not =  0)$.
    \end{proof}
    (b) Find a counterexample to the theorem. In other
    words, find an example of sets $A$ and $B$ for which
    the hypotheses of the theorem are true but the
    conclusion is false.
\end{tcolorbox}

\textbf{Solution 21 (a)}

The conclusion is incorrect. Just because $x \in B$ does not mean $0 \not \in
    B$.

\textbf{Solution 21 (b)}

Let $A = \{1, 2, 3\}$.

Let $B = \{0, 1, 2, 3\}$.

The premises $\forall{x} \in A(x \not = 0)$ and $A \subseteq B$ are true yet
the conclusion, $\forall{x} \in B(x \not = 0)$ is false.

\begin{tcolorbox}[title=Problem 22, breakable]
    Consider the following incorrect theorem:
    Incorrect Theorem. $\exists{x} \in  R\forall{y} \in  R(xy^2 = y - x)$.
    What’s wrong with the following proof of the
    theorem?
    \begin{proof}
        Let $x = y/(y^2 + 1)$. Then \\
        $$y - x = y - \frac{y}{y^2 + 1} = \frac{y^3}{y^2 + 1} = \frac{y}{y^2 + 1} \cdot y = xy^2$$
    \end{proof}
\end{tcolorbox}

\textbf{Problem 22}

It assumes that $y \not = 0$.

\begin{tcolorbox}[title=Problem 23, breakable]
    Consider the following incorrect theorem:
    Incorrect Theorem. Suppose $F$ and $G$ are families
    of sets. If $\bigcup F$ and $\bigcup G$ are disjoint, then so are $F$ and
    $G$. \\ 
    (a) What’s wrong with the following proof of the
    theorem?
    \begin{proof}
        Suppose $\bigcup F$ and $\bigcup G$ are disjoint. Suppose $F$
        and $G$ are not disjoint. Then we can choose some
        set $A$ such that $A \in  F$ and $A \in  G$. Since $A \in  F$, by
        exercise $8$, $A \subseteq  \bigcup F$, so every element of $A$ is in $\bigcup F$.
        Similarly, since $A \in  G$, every element of $A$ is in $\bigcup G$.
        But then every element of $A$ is in both $\bigcup F$ and $\bigcup G$,
        and this is impossible since $\bigcup F$ and $\bigcup G$ are disjoint.
        Thus, we have reached a contradiction, so $F$ and $G$
        must be disjoint.
    \end{proof}
    (b) Find a counterexample to the theorem.
\end{tcolorbox}

\textbf{Solution 23 (a)}

It is possible for every element in $A$ to be in $\bigcup \mathcal{F}$ and
$\bigcup \mathcal{G}$ if $A = \emptyset$.

\textbf{Solution 23 (b)}

Let $\mathcal{F} = \{\emptyset, \{1\}\}$

Let $\mathcal{G} = \{\emptyset, \{2\}\}$

The premise $\bigcup \mathcal{F} = \{1\}$ is disjoint $\bigcup \mathcal{G} =
    \{2\}$ is true, yet the conclusion $\mathcal{F}$ is disjoint from $\mathcal{G}$
is false since $\mathcal{F} \cap \mathcal{G} = \{\emptyset\}$.

\begin{tcolorbox}[title=Problem 24, breakable]
    Consider the following putative theorem: \\
    Theorem. For all real numbers $x$ and $y$, $x^2 + xy -
        2y^2 = 0$. \\
    (a) What’s wrong with the following proof of the
    theorem?
    \begin{proof}
        Let $x$ and $y$ be equal to some arbitrary real
        number $r$. Then
        $x^2 + xy - 2y^2 = r2 + r \cdot r - 2r^2 = 0$.
        Since $x$ and $y$ were both arbitrary, this shows that
        for all real numbers $x$ and $y$, $x^2 + xy - 2y^2 = 0$.
    \end{proof}
    (b) Is the theorem correct? Justify your answer with
    either a proof or a counterexample.
\end{tcolorbox}

\textbf{Solution 24 (a)}

Assumes that $x$ and $y$ are both equal to $r$ or in other words only considers
cases where $x = y$.

\textbf{Solution 24 (b)}

Let $x = 1$ and $y = 2$ then 

\begin{align*}
    x^2 + xy - 2y^2             & = 0  &  & \\
    {(1)}^2 + (1)(2) - 2{(2)}^2 & = 0  &  & \\
    1 + 2 - 8                   & = 0  &  & \\
    -5                          & = 0
\end{align*}

This conclusion is false while the premise is true so the theorem is incorrect.

\begin{tcolorbox}[title=Problem 25, breakable]
    Prove that for every real number $x$ there is a real
    number $y$ such that for every real number $z$, $yz = {(x
        + z)}^2 - (x^2 + z^2)$.
\end{tcolorbox}

\begin{proof}
    Let $y = 2x$. Then 
    \begin{align*}
        2x(z) & = {(x + z)}^2 - (x^2 + z^2)   &  & \\
        2xz   & = x^2 + 2xz + z^2 - x^2 - z^2 &  & \\
        2xz   & = 2xz
    \end{align*}
    Therefore, for every real number $x$ there is a real
    number $y$ such that for every real number $z$, $yz = {(x + z)}^2 - (x^2 + z^2)$.
\end{proof}

\subsection{Proofs Involving Conjunctions and Biconditionals}

\begin{tcolorbox}[title=Problem 1, breakable]
    Use the methods of this chapter to prove that $\forall{x}(P(x) \wedge Q(x))$
    is equivalent to $\forall{x}P(x) \wedge \forall{x}Q(x)$.
\end{tcolorbox}

\begin{proof}
    ($\rightarrow$) Assume for all $x$ $P(x)$ and $Q(x)$ are true.
    Let $x_0$ be arbitrary, since for all $x$ $P(x)$ and $Q(x)$ are true, $P(x_0)$ is true.
    Let $y_0$ be arbitrary, since for all $x$ $P(x)$ and $Q(x)$ are true, $Q(y_0)$ is true.
    Since $x_0$ and $y_0$ were arbitrary, for all $x$ $P(x)$ is true, and for all $x$ $Q(x)$ is true.

    ($\leftarrow$) Assume for all $x$ $P(x)$ is true, and 
    for all $x$ $Q(x)$ is true.
    Let $x_0$ be arbitrary. Since, for all $x$ $P(x)$ is true,
    $P(x_0)$ is true. Also, since for all $x$ $Q(x)$ is true,
    $Q(x_0)$ is true. It then follows that $P(x_0)$ and $Q(x_0)$ are true.
    Therefore, since $x_0$ was arbitrary, for all $x$
    $P(x)$ and $Q(x)$ are true.
\end{proof}

\begin{tcolorbox}[title=Problem 2, breakable]
    Prove that if $A \subseteq B$ and $A \subseteq C$ then $A \subseteq B \cap C$.
\end{tcolorbox}

\begin{proof}
    Suppose $A \subseteq B$ and $A \subseteq C$.
    Let $x_0$ be an arbitrary element such that $x_0 \in A$.
    Since $A \subseteq B$, $x_0 \in B$.
    Also, since $A \subseteq C$, $x_0 \in C$.
    Since, $x_0 \in B$ and $x_0 \in C$, $x_0 \in B \cap C$.
    Therefore, if $A \subseteq B$ and $A \subseteq C$ then $A \subseteq B \cap C$.
\end{proof}

\begin{tcolorbox}[title=Problem 3, breakable]
    Suppose $A \subseteq B$. Prove that for every set $C$, $C \setminus B \subseteq C \setminus A$.
\end{tcolorbox}

\begin{proof}
    Let $x_0$ be an arbitrary element such that $x_0 \in C \setminus B$.
    It follows that $x \in C$ and $x \not \in B$.
    Since $A \subseteq B$ and $x \not \in B$, $x \not \in A$.
    Finally, since $x \in C$ and $x \not \in A$, $x \in C \setminus A$.
\end{proof}

\begin{tcolorbox}[title=Problem 4, breakable]
    Prove that if $A \subseteq B$ and $A \not \subseteq C$ then $B \not \subset C$.
\end{tcolorbox}

\begin{proof}
    Suppose $A \subseteq B$ and $A \not \subseteq C$.
    Let $x_0$ be an element such that $x_0 \in A$.
    Since $A \subseteq B$, $x_0 \in B$.
    Also, since $A \not \subseteq C$, $x_0 \not \in C$.
    Since $x \in B$ and $x \not \in C$, $B \not \subseteq C$.
\end{proof}

\begin{tcolorbox}[title=Problem 5, breakable]
    Prove that if $A \subseteq B \setminus C$ and $A \not = \emptyset$ then $B \not \subseteq C$.
\end{tcolorbox}

\begin{proof}
    Suppose $A \subseteq B \setminus C$ and $A \not = \emptyset$.
    Let $x_0$ be an element such that $x_0 \in A$.
    Since $A \subseteq B \setminus C$, $x_0 \in B$ and $x_0 \not \in C$.
    It then follows that $B \not \subseteq C$.
\end{proof}

\begin{tcolorbox}[title=Problem 6, breakable]
    Prove that for any sets $A$, $B$, and $C$, $A \setminus (B \cap C) = (A \setminus B) \cup (A \setminus C)$,
    by finding a string of equivalences starting with $x \in A \setminus (B \cap C)$ and 
    ending with $x \in (A \setminus B) \cup A \setminus C$.
\end{tcolorbox}

\begin{proof}
    Let $x$ be arbitrary. Then 
    \begin{align*}
        x \in A \setminus (B \cap C) & \leftrightarrow x \in A \wedge x \not \in B \cap C                               &  & \\
                                     & \leftrightarrow x \in A \wedge (x \not \in B \vee x \not \in C)                  &  & \\
                                     & \leftrightarrow (x \in A \wedge x \not \in B) \vee (x \in A \wedge x \not \in C) &  & \\
                                     & \leftrightarrow (x \in A \setminus B) \vee (x \in A \setminus C)                 &  & \\
                                     & \leftrightarrow x \in (A \setminus B) \cup (A \setminus C) 
    \end{align*}
\end{proof}

\begin{tcolorbox}[title=Problem 7, breakable]
    Use the methods of this chapter to prove that for any sets $A$ and $B$,
    $\mathcal{P}(A \cap B) = \mathcal{P}(A) \cap \mathcal{P}(B)$.
\end{tcolorbox}

\begin{proof}
    Let $C$ be arbitrary. Then
    \begin{align*}
        C \in \mathcal{P}(A \cap B) & \leftrightarrow C \subseteq A \cap B                             &  & \\
                                    & \leftrightarrow C \subseteq A \wedge C \subseteq B               &  & \\
                                    & \leftrightarrow C \in \mathcal{P}(A) \wedge C \in \mathcal{P}(B) &  & \\
                                    & \leftrightarrow C \in \mathcal{P}(A) \cap \mathcal{P}(B)
    \end{align*}
\end{proof}

\begin{tcolorbox}[title=Problem 8, breakable]
    Prove that $A \subseteq B$ iff $\mathcal{P}(A) \subseteq \mathcal{P}(B)$.
\end{tcolorbox}

\begin{proof}
    ($\rightarrow$) Suppose $A \subseteq B$. 
    Let $y$ be an arbitrary set such that $y \in \mathcal{P}(A)$. 
    Since $A \subseteq B$ for an arbitrary element  $x \in A$, $x \in B$. 
    So for all $x \in y$, $x \in B$ and therefore $y \subseteq B$.
    It follows that $y \in \mathcal{P}(B)$.

    ($\leftarrow$) 
    Let $x$ be an arbitrary element such that $x \in A$.
    Now let $y$ be an arbitrary set such that $y \subseteq A$ and $x \in y$.
    Since $y \subseteq A$, $y \in \mathcal{P}(A)$.
    Then, since $\mathcal{P}(A) \subseteq \mathcal{P}(B)$, $y \in \mathcal{P}(B)$.
    Since $y \in \mathcal{P}(B)$ it follows that $y \subseteq B$.
    Finally since $x \in y$ and $y \subseteq B$, $x \in B$.

    Therefore $A \subseteq B$ iff $\mathcal{P}(A) \subseteq \mathcal{P}(B)$.
\end{proof}

\begin{tcolorbox}[title=Problem 11, breakable]
    Prove that for every integer $n$, $n^3$ is even iff $n$ is even.
\end{tcolorbox}

\begin{proof}
    ($\rightarrow$) Suppose $n$ is even. It can be written in the form $2k$ where $k \in \mathbb{Z}$.
    Then 
    \begin{align*}
        n^3 & = n \cdot n \cdot n    &  & \\
            & = 2k \cdot 2k \cdot 2k &  & \\
            & = 2(2^2 \cdot k^3)
    \end{align*}
    ($\leftarrow$) We will prove the contrapositive. Suppose $n$ is odd and can therefore be written 
    in the form $2k + 1$ where $k \in \mathbb{Z}$.
    Then 
    \begin{align*}
        n^3 & = n \cdot n \cdot n        &  & \\
            & = (2k + 1)(2k + 1)(2k + 1) &  & \\
            & = (4k^2 + 4k + 1)(2k + 1)  &  & \\
            & = 8k^3 + 12k^2 + 6k + 1    &  & \\
            & = 2(4k^3 + 6k^2 + 3k) + 1
    \end{align*}
    Therefore, for every integer $n$, $n^3$ is even iff $n$ is even.
\end{proof}

\begin{tcolorbox}[title=Problem 12, breakable]
    Consider the following putative theorem:

    \textbf{Theorem?} Suppose $m$ is an even integer and $n$ is an odd integer.
    Then $n^2 - m^2 = n + m$.

    (a) What's wrong with the following proof of the theorem?
    \begin{proof}
        Since $m$ is even, we can choose some integer $k$ such that 
        $m = 2k$. Similarly, since $n$ is odd we have $n = 2k + 1$. Therefore 
        \begin{align*}
            n^2 - m^2 & = (2k + 1)^2 - (2k)^2 = 4k^2 + 4k + 1 - 4k^2 = 4k + 1 &  & \\
                      & = (2k + 1) + (2k) = n + m
        \end{align*}
    \end{proof}
    (b) Is the theorem correct? Justify your answer with either a proof 
    or a counterexample.
\end{tcolorbox}

\textbf{Solution 12 (a)}

The $2k$ in the equation for $m$ is not necessarily equivalent to the $2k$ in
the equation for $n$.

\textbf{Solution 12 (b)}

Let $m = 4$ and $n = 1$.
\begin{align*}
     & n^2 - m^2 = n + m                 &  & \\
     & \leftrightarrow 1^2 - 4^2 = 1 + 4 &  & \\
     & \leftrightarrow 1 - 16 = 5        &  & \\
     & \leftrightarrow -15 = 5
\end{align*}
This equation is false, and therefore, the theorem is false.

\begin{tcolorbox}[title=Problem 13, breakable]
    Prove that $\forall{x \in \mathbb{R}}[\exists{y \in \mathbb{R}}(x + y = xy) \leftrightarrow x \not = 1]$.
\end{tcolorbox}

\begin{proof}
    ($\rightarrow$) Suppose $x + y = xy$. Assume for contradiction
    $x = 1$. Then 
    \begin{align*}
        x + y = xy \leftrightarrow 1 + y = (1)y \leftrightarrow 1 + y = y \leftrightarrow 1 = 0
    \end{align*}
    Which is a contradiction. Therefore if $x + y = xy$ then $x \not = 1$.

    ($\leftarrow$) Suppose $x \not = 1$. Let $y = \frac{x}{x - 1}$. Then 
    \begin{align*}
         & x + y = xy                                               &  & \\
         & \leftrightarrow x + \frac{x}{x - 1} = x(\frac{x}{x - 1}) &  & \\
         & \leftrightarrow (x - 1)x + x = x^2                       &  & \\
         & \leftrightarrow x^2 - x + x = x^2                        &  & \\
         & \leftrightarrow x^2 = x^2
    \end{align*}
    Therefore, $\forall{x \in \mathbb{R}}[\exists{y \in \mathbb{R}}(x + y = xy) \leftrightarrow x \not = 1]$.
\end{proof}

\begin{tcolorbox}[title=Problem 14, breakable]
    Prove that $\exists{z} \in \mathbb{R} \forall{x} \in \mathbb{R^{+}}
        [\exists{y} \in \mathbb{R}(y - x = y / x) \leftrightarrow x \not = z]$.
\end{tcolorbox}

\begin{proof}
    ($\rightarrow$) Suppose $y - x = y / x$. Let $z = 1$ and assume for contradiction $x = z = 1$.
    Then
    \begin{align*}
        y - x & = y / x &  & \\
        y - 1 & = y / 1 &  & \\
        y - 1 & = y     &  & \\
        -1    & = 0
    \end{align*}
    Which is a contradiction.

    ($\leftarrow$) Suppose $x \not = z$. Let $z = 1$ and therefore $x \not = 1$.
    Let $y = \frac{x^2}{x - 1}$. Then 
    \begin{align*}
        y - x                                      & = y / x                                       &  & \\
        \frac{x^2}{x - 1} - x                      & = \frac{x^2}{x - 1} / x                       &  & \\
        (x - 1) \cdot \frac{x^2}{x - 1} - x(x - 1) & = (x - 1) \cdot \frac{(\frac{x^2}{x - 1})}{x} &  & \\
        x^2 - x^2 + x                              & = x^2 / x                                     &  & \\
        x^3 - x^3 + x^2                            & = x^3 / x                                     &  & \\
        x^2                                        & = x^2
    \end{align*}
    Therefore, $\exists{z} \in \mathbb{R} \forall{x} \in \mathbb{R^{+}}
        [\exists{y} \in \mathbb{R}(y - x = y / x) \leftrightarrow x \not = z]$.
\end{proof}

\begin{tcolorbox}[title=Problem 15, breakable]
    Suppose $B$ is a set and $\mathcal{F}$ is a family of sets.
    Prove that $\bigcup\{A \setminus B \mid A \in \mathcal{F}\}
        \subseteq \bigcup (\mathcal{F} \setminus \mathcal{P}(B))$.
\end{tcolorbox}

\begin{proof}
    Let \(x_0\) be arbitrary such that \(x_0 \in \bigcup \{A \setminus B \mid A \in \mathcal{F}\}\).  
    It follows that \(x_0 \in A\) where \(A \in \mathcal{F}\) and \(x_0 \notin B\).  
    Since \(A \in \mathcal{F}\) and since \(x_0 \notin B\), 
    it follows that  \(A \not\subseteq B\), so \(A \notin \mathcal{P}(B)\).  
    Therefore, \(x_0 \in \bigcup (\mathcal{F} \setminus \mathcal{P}(B))\).
    Since $x_0$ was arbitrary, for all $x$, if \(x \in \bigcup\{A \setminus B \mid A \in \mathcal{F}\}\) then
    \(x \in \bigcup (\mathcal{F} \setminus \mathcal{P}(B))\). Therefore 
    \(\bigcup\{A \setminus B \mid A \in \mathcal{F}\} 
    \subseteq \bigcup (\mathcal{F} \setminus \mathcal{P}(B))\).
\end{proof}

\begin{tcolorbox}[title=Problem 17, breakable]
    Prove that for any set $A$, $A = \bigcup \mathcal{P}(A)$.
\end{tcolorbox}

\begin{proof}
    Let $x_0$ be arbitrary such that $x_0 \in A$.
    Let $Y = \{x_0\} \subseteq A$.
    It follows that $Y \in \mathcal{P}(A)$.
    From the definition of the union of a family of sets $x_0 \in \bigcup \mathcal{P}(A)$.
    Therefore, since $x_0$ was arbitrary, for all $x \in A$, $x \in \bigcup \mathcal{P}(A)$.

    Now let $y_0$ be arbitrary such that $y_0 \in \bigcup \mathcal{P}(A)$. By the
    definition of the union of a family of sets there is a set $Z$ such that $y_0
        \in Z$ and $Z \in \mathcal{P}(A)$. It follows that $Z \subseteq A$, and
    therefore, $y_0 \in A$. Therefore, since $y_0$ was arbitrary, for all $y \in
        \bigcup \mathcal{P}(A)$, $y \in A$.

    Finally, since, for all $x \in A$, $x \in \bigcup \mathcal{P}(A)$, and, for all
    $y \in \bigcup \mathcal{P}(A)$, $y \in A$, $A = \bigcup \mathcal{P}(A)$.
\end{proof}

\begin{tcolorbox}[title=Problem 18, breakable]
    Suppose $\mathcal{F}$ and $\mathcal{G}$ are families of sets.

    (a) Prove $\bigcup (\mathcal{F} \cap \mathcal{G}) \subseteq (\mathcal{F}) \cap (\mathcal{G})$.

    (b) What's wrong with the following proof that 
    $(\bigcup \mathcal{F}) \cap (\bigcup \mathcal{G}) \subseteq \bigcup(\mathcal{F} \cap \mathcal{G})$?
    \begin{proof}
        Suppose $x \in (\bigcup \mathcal{F}) \cap (\bigcup \mathcal{G})$. This means $x \in \bigcup \mathcal{F}$
        and $x \in \bigcup \mathcal{G}$, so $\exists A \in \mathcal{F}(x \in A)$ and $\exists A \in \mathcal{G}(x \in A)$.
        Thus, we can choose a set $A$ such that $A \in \mathcal{F}$, $A \in \mathcal{G}$, and $x \in A$. Since 
        $A \in \mathcal{F}$ and $A \in \mathcal{G}$, $A \in \mathcal{F} \cap \mathcal{G}$. 
        Therefore $\exists{A} \in \mathcal{F} \cap \mathcal{G}(x \in A)$,
        so $x \in \bigcup(\mathcal{F} \cap \mathcal{G})$. Since $x$ was arbitrary, we can conclude that 
        $(\bigcup \mathcal{F}) \cap (\bigcup \mathcal{G}) \subseteq \bigcup(\mathcal{F} \cap \mathcal{G})$.
    \end{proof}

    (c) Find an example of famlies of sets $\mathcal{F}$ and $\mathcal{G}$ for which 
    $\bigcup(\mathcal{F} \cap \mathcal{G}) \not = (\bigcup \mathcal{F}) \cap (\bigcup \mathcal{G})$.
\end{tcolorbox}

\begin{proof}
    Suppose $x \in \bigcup (\mathcal{F} \cap \mathcal{G})$.
    There exists a set $A$ such that $x \in A$ and $A \in (\mathcal{F} \cap \mathcal{G})$.
    Since $A \in (\mathcal{F} \cap \mathcal{G})$, $A \in \mathcal{F}$ and $A \in \mathcal{G}$.
    It follows that $x \in \bigcup \mathcal{F}$ and $x \in \bigcup \mathcal{G}$.
\end{proof}

\textbf{Solution 18 (b)}

The two sets $A \in \mathcal{F}$ and $A \in \mathcal{G}$ are not necessarily
equivalent.

\textbf{Solution 18 (c)}

$\mathcal{F} = \{\{1\}\}$

$\mathcal{G} = \{\{1, 2\}\}$

$(\bigcup \mathcal{F}) \cap (\bigcup \mathcal{G}) = \{1\}$

$\bigcup(\mathcal{F} \cap \mathcal{G}) = \emptyset$

\begin{tcolorbox}[title=Problem 19, breakable]
    Suppose $\mathcal{F}$ and $\mathcal{G}$ are families of sets. Prove 
    that $(\bigcup \mathcal{F}) \cap (\bigcup \mathcal{G}) \subseteq \bigcup(\mathcal{F} \cap \mathcal{G})$
    iff $\forall{A} \in \mathcal{F}\forall{B} \in \mathcal{G}(A \cap B \subseteq \bigcup(\mathcal{F} \cap \mathcal{G}))$.
\end{tcolorbox}

\begin{proof}
    ($\rightarrow$) Suppose $(\bigcup \mathcal{F}) \cap (\bigcup \mathcal{G}) \subseteq \bigcup(\mathcal{F} \cap \mathcal{G})$.
    Let $A$, $B$ be arbitrary sets such that $A \in \mathcal{F}$ and $B \in \mathcal{G}$.
    Let $x_0$ be an arbitrary element such that $x_0 \in A \cap B$.
    It follows that $x_0 \in A$ and $x_0 \in B$.
    Since $x_0 \in A$ and $A \in \mathcal{F}$, $x_0 \in (\bigcup \mathcal{F})$.
    Also, since $x_0 \in B$ and $B \in \mathcal{G}$, $x_0 \in (\bigcup \mathcal{G})$.
    Note that since $x_0 \in (\bigcup \mathcal{F})$ and $x_0 \in (\bigcup \mathcal{G})$,
    $x_0 \in (\bigcup \mathcal{F}) \cap (\bigcup \mathcal{G})$.
    Finally, since $(\bigcup \mathcal{F}) \cap (\bigcup \mathcal{G}) \subseteq \bigcup(\mathcal{F} \cap \mathcal{G})$,
    $x_0 \in \bigcup(\mathcal{F} \cap \mathcal{G})$.

    ($\leftarrow$) Suppose $\forall{A} \in \mathcal{F}\forall{B}
        \in \mathcal{G}(A \cap B \subseteq \bigcup(\mathcal{F} \cap \mathcal{G}))$.
    Let $x_0$ be arbitrary such that $x_0 \in (\bigcup \mathcal{F}) \cap (\bigcup \mathcal{G})$.
    It follows that $x_0 \in (\bigcup \mathcal{F})$ and $x_0 \in (\bigcup \mathcal{G})$.
    Since $x_0 \in (\bigcup \mathcal{F})$, 
    there exists a set $A$ such that $x_0 \in A$ and $A \in \mathcal{F}$.
    Also, since $x_0 \in (\bigcup \mathcal{G})$, 
    there exists a set $B$ such that $x_0 \in B$ and $B \in \mathcal{G}$.
    Note that $x_0 \in A$ and $x_0 \in B$ so $x_0 \in A \cap B$.
    Finally, since $x_0 \in A$ and $x_0 \in B$ so $x_0 \in A \cap B$ and $\forall{A} \in \mathcal{F}\forall{B}
        \in \mathcal{G}(A \cap B \subseteq \bigcup(\mathcal{F} \cap \mathcal{G}))$,
    $x_0 \in \bigcup(\mathcal{F} \cap \mathcal{G})$.
\end{proof}

\begin{tcolorbox}[title=Problem 20, breakable]
    Suppose $\mathcal{F}$ and $\mathcal{G}$ are families of sets. Prove that $\bigcup \mathcal{F}$ and $\bigcup \mathcal{G}$
    are disjoint iff for all $A \in \mathcal{F}$ and $B \in \mathcal{G}$, $A$ and $B$ are disjoint.
\end{tcolorbox}

\begin{proof}
    ($\rightarrow$) Suppose $\bigcup \mathcal{F}$ and $\bigcup \mathcal{G}$ are disjoint.
    Assume for contradiction there exists $A$ and $B$
    such that $A \in \mathcal{F}$ and $B \in \mathcal{G}$ and $A$ and $B$ are not disjoint.
    It follows there must exists an element $x$ such that $x \in A$ and $x \in B$.
    Since $x \in A$ and $A \in \mathcal{F}$, $x \in \bigcup \mathcal{F}$.
    Also, since $x \in B$ and $B \in \mathcal{G}$, $x \in \bigcup \mathcal{G}$.
    But $\bigcup \mathcal{F}$ and $\bigcup \mathcal{G}$ are disjoint which is a contradiction.

    ($\leftarrow$) Suppose for all $A \in \mathcal{F}$ and $B \in \mathcal{G}$, $A$ and $B$ are disjoint.
    Assume for contradiction $\bigcup \mathcal{F}$ and $\bigcup \mathcal{G}$ are not disjoint.
    Since $\bigcup \mathcal{F}$ and $\bigcup \mathcal{G}$ are not disjoint, there exists an element $x$
    such that $x \in \bigcup \mathcal{F}$ and $x \in \bigcup \mathcal{G}$. Furthermore, 
    there exists a set $A$ such that $x \in A$ and $A \in \mathcal{F}$ and there exists a set $B$
    such that $x \in B$ and $B \in \mathcal{G}$. The sets $A$ and $B$ are clearly not disjoint
    which is a contradiction.

    Therefore, $\bigcup \mathcal{F}$ and $\bigcup \mathcal{G}$ are disjoint iff for
    all $A \in \mathcal{F}$ and $B \in \mathcal{G}$, $A$ and $B$ are disjoint.
\end{proof}

\begin{tcolorbox}[title=Problem 21, breakable]
    Suppose $\mathcal{F}$ and $\mathcal{G}$ are families of sets. \\
    (a) Prove that $(\bigcup \mathcal{F}) \setminus (\bigcup \mathcal{G})
        \subseteq \bigcup(\mathcal{F} \setminus \mathcal{G})$. \\
    (b) What's wrong with the following proof that $\bigcup (\mathcal{F} \setminus \mathcal{G})
        \subseteq (\bigcup \mathcal{F}) \setminus (\bigcup \mathcal{G})$.
    \begin{proof}
        Suppose $x \in \bigcup(\mathcal{F} \setminus \mathcal{G})$. Then we can 
        choose some $A \in \mathcal{F} \setminus \mathcal{G}$ such that $x \in A$.
        Since $A \in \mathcal{F} \setminus \mathcal{G}$, $A \in \mathcal{F}$
        and $A \not \in \mathcal{G}$. Therefore $x \in (\bigcup \mathcal{F}) \setminus (\bigcup \mathcal{G})$.
    \end{proof}
    (c) Prove that $\bigcup (\mathcal{F} \setminus \mathcal{G}) \subseteq (\bigcup \mathcal{F}) \setminus (\bigcup \mathcal{G})$
    iff $\forall{A} \in (\mathcal{F} \setminus \mathcal{G})\forall{B} \in \mathcal{G}(A \cap B = \emptyset)$. \\
    (d) Find an example of families of sets $\mathcal{F}$ and $\mathcal{G}$ for which 
    $\bigcup(\mathcal{F} \setminus \mathcal{G}) \not = (\bigcup \mathcal{F}) \setminus (\bigcup \mathcal{G})$
\end{tcolorbox}

\begin{proof}
    Let $x_0$ be an arbitrary element such that $x_0 \in (\bigcup \mathcal{F}) \setminus (\bigcup \mathcal{G})$.
    It follows that $x_0 \in (\bigcup \mathcal{F})$ and $x_0 \not \in (\bigcup \mathcal{G})$.
    There exists a set $A$ such that $x_0 \in A$ and $A \in \mathcal{F}$.
    There does not exist a set $B$ such that $x_0 \in B$ and $B \in \mathcal{G}$.
    It follows that $A \in (\mathcal{F} \setminus \mathcal{G})$.
    By definition of union of a family of sets since $x_0 \in A$, $x_0 \in \bigcup (\mathcal{F} \setminus \mathcal{G})$.
\end{proof}

\textbf{Solution 21 (b)}

There could be a set $B$ such that $x \in B$ and $B \in \mathcal{G}$ where $A
    \not = B$. It follows that $x \in \bigcup \mathcal{G}$.

\begin{proof}
    ($\rightarrow$) Suppose $\bigcup (\mathcal{F} \setminus \mathcal{G})
        \subseteq (\bigcup \mathcal{F}) \setminus (\bigcup \mathcal{G})$.
    Assume for contradiction there exists sets $A$, $B$ such that $A \in \mathcal{F} \setminus \mathcal{G}$, 
    $B \in \mathcal{G}$, and $A \cap B \not = \emptyset$.
    It follows that there exists $x_0$ such that $x_0 \in A$ and $x_0 \in B$.
    Since $x_0 \in A$ and $A \in \mathcal{F}$, $x_0 \in (\bigcup \mathcal{F})$.
    Since $x_0 \in B$, and $B \in \mathcal{G}$, $x_0 \in (\bigcup\mathcal{G})$.
    It follows that $x_0 \not \in (\bigcup \mathcal{F}) \setminus(\bigcup \mathcal{G})$.
    Since $x_0 \in A$ and $A \in \mathcal{F} \setminus \mathcal{G}$, 
    it follows that $x_0 \in \bigcup (\mathcal{F} \setminus \mathcal{G})$.
    But $\bigcup (\mathcal{F} \setminus \mathcal{G})
        \subseteq (\bigcup \mathcal{F}) \setminus (\bigcup \mathcal{G})$
    which is a contradiction.
    So, if $\bigcup (\mathcal{F} \setminus \mathcal{G})
        \subseteq (\bigcup \mathcal{F}) \setminus (\bigcup \mathcal{G})$
    then $\forall{A} \in (\mathcal{F} \setminus \mathcal{G})\forall{B}
        \in \mathcal{G}(A \cap B = \emptyset)$

    ($\leftarrow$) Suppose $\forall{A} \in (\mathcal{F} \setminus \mathcal{G})\forall{B} \in \mathcal{G}(A \cap B = \emptyset)$.
    Let $x_0$ be an arbitrary element such that $x_0 \in \bigcup (\mathcal{F} \setminus \mathcal{G})$.
    There must be a set $A$ such that $x_0 \in A$ and $A \in \mathcal{F} \setminus \mathcal{G}$.
    It follows $A \in \mathcal{F}$ and $A \not \in \mathcal{G}$.
    Since $x_0 \in A$ and $A \in \mathcal{F}$, $x_0 \in \bigcup \mathcal{F}$.
    Since $A \in \mathcal{F} \setminus \mathcal{G}$ and all sets in $\mathcal{G}$
    are disjoint from $A$, $x_0 \not \in (\bigcup \mathcal{G})$.
    Since $x_0 \in (\bigcup \mathcal{F})$ and $x_0 \not \in (\bigcup \mathcal{G})$,
    $x_0 \in (\bigcup \mathcal{F}) \setminus (\bigcup \mathcal{G})$.

    Therefore, $\bigcup (\mathcal{F} \setminus \mathcal{G}) \subseteq (\bigcup
        \mathcal{F}) \setminus (\bigcup \mathcal{G})$ iff $\forall{A} \in (\mathcal{F}
        \setminus \mathcal{G})\forall{B} \in \mathcal{G}(A \cap B = \emptyset)$.
\end{proof}

\textbf{Solution 21 (d)}

$\mathcal{F} = \{\{1\}, \{2\}\}$

$\mathcal{G} = \{\{1, 2\}\}$

$\mathcal{F} \setminus \mathcal{G} = \{\{1\}, \{2\}\}$

$\bigcup(\mathcal{F} \setminus \mathcal{G}) = \{1, 2\}$

$(\bigcup \mathcal{F}) = \{1, 2\}$

$(\bigcup \mathcal{G}) = \{1, 2\}$

$(\bigcup \mathcal{F}) \setminus (\bigcup \mathcal{G}) = \emptyset$

\begin{tcolorbox}[title=Problem 23, breakable]
    Suppose $B$ is a set, $\{A_i \mid i \in I\}$ is an indexed family of sets,
    and $I \not = \emptyset$.

    (a) What proof strategies are used in the following proof of the equation 
    $B \cap (\bigcup_{i \in I} A_i) = \bigcup_{i \in I}(B \cap A_i)$
    \begin{proof}
        Let $x$ be arbitrary. Suppose $x \in B \cap (\bigcup_{i \in I} A_i)$.
        Then $x \in B$ and $x \in \bigcup_{i \in I} A_i$, so we can choose some 
        $i_0 \in I$ such that $x \in A_{i_0}$. Since $x \in B$ and $x \in A_{i_0}$,
        $x \in B \cap A_{i_0}$. Therefore $x \in \bigcup_{i \in I} (B \cap A_i)$.

        Now suppose $x \in \bigcup_{i \in I} (B \cap A_i)$. Then we can choose some
        $i_0 \in I$ such that $x \in B \cap A_{i_0}$. Therefore $x \in B$ and $x \in
            A_{i_0}$. Since $x \in A_{i_0}$, $x \in \bigcup_{i \in I} A_i$. Since $x \in B$
        and $x \in \bigcup_{i \in I} A_i$, $x \in B \cap (\bigcup_{i \in I} A_i)$.

        Since $x$ was arbitrary, we have shown that $\forall{x}[x \in B \cap
                (\bigcup_{i \in I} A_i) \leftrightarrow x \in \bigcup_{i \in I}(B \cap A_i)]$,
        so $B \cap (\bigcup_{i \in I}) = \bigcup_{i \in I} (B \setminus A_i)$.
    \end{proof}
    (b) Prove that $B \setminus (\bigcup_{i \in I} A_i) = \bigcap_{i \in I} (B \setminus A_i)$.

    (c) Can you discover and prove a similar theorem about $B \setminus (\bigcap_{i \in I} A_i)$?
    (Hint: Try to guess the theorem, and then try to prove it. If you can't finish the proof,
    it might be because your guess was wrong. Change your guess and try again.)
\end{tcolorbox}

\textbf{Solution 23(a)}

The proof uses a direct proof strategy based on mutual inclusion: it shows that
each side of the equation is a subset of the other. For any two sets $A$ and
$B$, if $A \subseteq B$ and $B \subseteq A$ then $A = B$.

\textbf{Solution 23(b)}
\begin{proof}
    ($\rightarrow$) Let $x_0$ be an arbitrary element such that $x_0 \in B \setminus (\bigcup_{i \in I} A_i)$.
    So $x_0 \in B$ and $x_0 \not \in (\bigcup_{i \in I} A_i)$.
    So $x_0 \not \in A_i$ for any $i \in I$.
    It follows that $x_0 \in \bigcap_{i \in I} (B \setminus A_i)$.

    ($\leftarrow$) Let $x_0$ be an arbitrary element such that $x_0 \in \bigcap_{i \in I}(B \setminus A_i)$.
    So $x_0 \in B$ and $x_0 \not \in A_i$ for any $i \in I$.
    It follows that $x_0 \not \in (\bigcup_{i \in I} A_i)$.
    So $x_0 \in B$ and $x_0 \not \in (\bigcup_{i \in I} A_i)$, so $x_0 \in B \setminus (\bigcup_{i \in I} A_i)$.

    Therefore, $B \setminus (\bigcup_{i \in I} A_i) = \bigcap_{i \in I} (B
        \setminus A_i)$.
\end{proof}

\textbf{Solution 23(c)}

$B \setminus (\bigcap_{i \in I} A_i) = \bigcup_{i \in I} (B \setminus A_i)$

\begin{proof}
    ($\rightarrow$) Let $x_0$ be an arbitrary element such that $x_0 \in B \setminus (\bigcap_{i \in I} A_i)$.
    So $x_0 \in B$ and $x_0 \not \in (\bigcap_{i \in I} A_i)$.
    It follows that since $x_0 \in B$ and $x_0 \not \in A_i$ for some $i \in I$, $x_0 \in \bigcup_{i \in I} (B \setminus A_i)$.

    ($\leftarrow$)  Let $x_0$ be an arbitrary element such that $x_0 \in \bigcup_{i \in I} (B \setminus A_i)$.
    So $x_0 \in B$ and $x_0 \not \in A_i$ for some $i \in I$.
    It follows that $x_0 \not \in (\bigcap_{i \in I} A_i)$ and, therefore,
    $x_0 \in B \setminus (\bigcap_{i \in I} A_i)$.

    Therefore $B \setminus (\bigcap_{i \in I} A_i) = \bigcup_{i \in I} (B \setminus
        A_i)$.
\end{proof}

\begin{tcolorbox}[title=Problem 24, breakable]
    Suppose $\{A_i \mid i \in I\}$ and $\{B_i \mid i \in I\}$ are indexed family of sets and 
    $I \not = \emptyset$.

    (a) Prove that $\bigcup_{i \in I} (A_i \setminus B_i)
        \subseteq (\bigcup_{i \in I} A_i) \setminus (\bigcap_{i \in I} B_i)$.

    (b) Find and example for which $\bigcup_{i \in I} (A_i \setminus B_i)
        \not = (\bigcup_{i \in I} A_i) \setminus (\bigcap_{i \in I} B_i)$.
\end{tcolorbox}

\begin{proof}
    Let $x_0$ be an arbitrary element such that $x_0 \in \bigcup_{i \in I} (A_i \setminus B_i)$.
    For some $i \in I$, $x_0 \in A_i$ and $x_0 \not \in B_i$.
    Since $x_0 \in A_i$ for some $i \in I$, $x_0 \in \bigcup_{i \in I} A_i$.
    Since $x_0 \not \in B_i$ for some $i \in I$, $x_0 \not \in \bigcap_{i \in I} B_i$.
    It follows, since $x_0 \in \bigcup_{i \in I} A_i$ and $x_0 \not \in \bigcap_{i \in I} B_i$,
    $x_0 \in \bigcup_{i \in I} A_i \setminus \bigcap_{i \in I} B_i$.
    Therefore,  $\bigcup_{i \in I} (A_i \setminus B_i)
        \subseteq (\bigcup_{i \in I} A_i) \setminus (\bigcap_{i \in I} B_i)$.
\end{proof}

\textbf{Solution 24(b)}

$I = \{1, 2\}$

$A_1 = \{1, 2\}$, $A_2 = \{3\}$

$B_1 = \{1\}$, $B_2 = \{2\}$

$\bigcup_{i \in I} (A_i \setminus B_i) = \{2, 3\}$

$(\bigcup_{i \in I} A_i) \setminus (\bigcap_{i \in I} B_i) = \{1, 2, 3\}$

\begin{tcolorbox}[title=Problem 25, breakable]
    Suppose $\{A_i \mid i \in I\}$ and $\{B_i \mid i \in I\}$ are indexed family of sets and 
    $I \not = \emptyset$.

    (a) Prove that $\bigcup_{i \in I}(A_i \cap B_i)
        \subseteq (\bigcup_{i \in I} A_i) \cap (\bigcup_{i \in I} B_i)$.

    (b) Find an example for which $\bigcup_{i \in I} (A_i \cap B_i)
        \not = (\bigcup_{i \in I} A_i) \cap (\bigcup_{i \in I} B_i)$.
\end{tcolorbox}

\begin{proof}
    Let $x_0$ be arbitrary such that $x_0 \in \bigcup_{i \in I} (A_i \cap B_i)$.
    For some $i \in I$, $x_0 \in A_i$ and $x_0 \in B_i$.
    Since $x_0$ in $A_i$ for some $i \in I$, $x_0 \in (\bigcup_{i \in I} A_i)$.
    Since $x_0$ in $B_i$ for some $i \in I$, $x_0 \in (\bigcup_{i \in I} B_i)$.
    So $x_0 \in (\bigcup_{i \in I} A_i)$ and $x_0 \in (\bigcup_{i \in I} B_i)$.
    It follows $x_0 \in (\bigcup_{i \in I} A_i) \cap (\bigcup_{i \in I} B_i)$.
    Therefore, $\bigcup_{i \in I}(A_i \cap B_i)
        \subseteq (\bigcup_{i \in I} A_i) \cap (\bigcup_{i \in I} B_i)$.
\end{proof}

\textbf{Solution 25(b)}

$I = \{1, 2\}$

$A_1 = \{1\}$,  $A_2 = \{2\}$

$B_1 =\{2\}$, $B_2 = \{1\}$

$\bigcup_{i \in I} (A_i \cap B_i) = \emptyset$

$(\bigcup_{i \in I} A_i) \cap (\bigcup_{i \in I} B_i) = \{1, 2\}$

\begin{tcolorbox}[title=Problem 27, breakable]
    (a) Prove that for every integer $n$, $15 \mid n$ iff $3 \mid n$ and $5 \mid n$. \\
    (b) Prove that it is not true that for every integer $n$, $60 \mid n$ iff $6 \mid n$
    and $10 \mid n$.
\end{tcolorbox}

\begin{proof}
    ($\rightarrow$) Suppose $15 \mid n$. Then $n = 15k$ where $k \in \mathbb{Z}$.
    Then $n = 3(5k)$ and $n = 5(3k)$ showing $3 \mid n$ and $5 \mid n$.

    ($\leftarrow$) Suppose $3 \mid n$ and $5 \mid n$.
    So $n = 3k_1$ and $n = 5k_2$ for $k_1, k_2 \in \mathbb{Z}$.
    \begin{align*}
        15(2k_2 - k_1) & = 30k_2 - 15k_1     &  & \\
                       & = 6(5k_2) - 5(3k_1) &  & \\
                       & = 6n - 5n = n
    \end{align*}
    So $15 \mid n$.
    Therefore, for every integer $n$, $15 \mid n$ iff $3 \mid n$ and $5 \mid n$.
\end{proof}

\textbf{Solution 27 (b)}

Counter example for backwards implication ($\leftarrow$).

Let $n = 30$. $n$ is divisible by $6$ and $10$ but not by $60$.

\subsection{Proofs Involving Disjunctions}

\begin{tcolorbox}[title=Problem 1, breakable]
    Suppose $A$, $B$, and $C$ are sets. 
    Prove that $A \cap (B \cup C) \subseteq (A \cap B) \cup C$.
\end{tcolorbox}

\begin{proof}
    Let $x_0$ be an arbitrary element such that $x_0 \in A \cap (B \cup C)$.
    So $x_0 \in A$ and $x_0 \in B \cup C$ and therefore, $x_0 \in B$ or $x_0 \in C$.

    Case ($x_0 \in B$): Since $x_0 \in A$ and $x_0 \in B$, $x_0 \in A \cap B$. It
    follows that $x_0 \in (A \cap B) \cup C$.

    Case ($x_0 \in C$): It immediately follows $x_0 \in (A \cap B) \cup C$.

    Since these cases were exhaustive, $A \cap (B \cup C) \subseteq (A \cap B) \cup
        C$.
\end{proof}

\begin{tcolorbox}[title=Problem 4, breakable]
    Suppose $A$, $B$, and $C$ are sets.
    Prove that $A \setminus (B \setminus C) = (A \setminus B) \cup (A \cap C)$.
\end{tcolorbox}

\begin{proof}
    We first show $A \setminus (B \setminus C) \subseteq (A \setminus B) \cup (A \cap C)$
    Let $x_0$ be an arbitrary element such that $x_0 \in A \setminus (B \setminus C)$.
    So $x_0 \in A$ and $x_0 \not \in B \setminus C$.
    It follows that $x_0 \not \in B$ or $x_0 \in C$.

    Case ($x_0 \not \in B$): Since $x_0 \in A$ and $x_0 \not \in B$, $x_0 \in A
        \setminus B$. It follows that $x_0 \in (A \setminus B) \cup (A \cap C)$.

    Case ($x_0 \in C$): Since $x_0 \in A$ and $x_0 \in C$, $x_0 \in A \cap C$. It
    follows that $x_0 \in (A \setminus B) \cup (A \cap C)$.

    Since these cases were exhaustive, $A \setminus (B \setminus C) \subseteq (A
        \setminus B) \cup (A \cap C)$.

    We now show $(A \setminus B) \cup (A \cap C) \subseteq A \setminus (B \setminus
        C)$. Let $x_0$ be an arbitrary element such that $x_0 \in (A \setminus B) \cup
        (A \cap C)$. So $x_0 \in A \setminus B$ or $x_0 \in A \cap C$.

    Case ($x_0 \in A \cap C$): So $x_0 \in A$ and $x_0 \in C$. Since $x_0 \in C$ it
    follows $x_0 \not \in B \setminus C$. Since $x_0 \in A$ and $x_0 \not \in B
        \setminus C$, $x_0 \in A \setminus (B \setminus C)$.

    Case ($x_0 \in A \setminus B$): So $x_0 \in A$ and $x_0 \not \in B$. Since $x_0
        \not \in B$, $x_0 \not \in B \setminus C$. Since $x_0 \in A$ and $x_0 \not \in
        B \setminus C$, $x_0 \in A \setminus (B \setminus C)$.

    Since these cases were exhaustive, $(A \setminus B) \cup (A \cap C) \subseteq A
        \setminus (B \setminus C)$.

    Since $A \setminus (B \setminus C) \subseteq (A \setminus B) \cup (A \cap C)$
    and $(A \setminus B) \cup (A \cap C) \subseteq A \setminus (B \setminus C)$, $A
        \setminus (B \setminus C) = (A \setminus B) \cup (A \cap C)$.
\end{proof}

\begin{tcolorbox}[title=Problem 6, breakable]
    Recall from Section $1.4$ that the symmetric difference
    of two sets $A$ and $B$ is the set 
    $A \triangle B = (A \setminus B) \cup (B \setminus A)
        = (A \cup B) \setminus (A \cap B)$. Prove 
    that if $A \triangle B \subseteq A$ then $B \subseteq A$.
\end{tcolorbox}

\begin{proof}
    Suppose $A \triangle B \subseteq A$ and assume for contradiction $B \not \subseteq A$.
    Let $x_0$ be an element such that $x_0 \in B$ and $x_0 \not \in A$.
    Since $A \triangle B \subseteq A$, $(A \setminus B) \cup (B \setminus A) \subseteq A$.
    Since $x_0 \in B$ and $x_0 \not \in A$, $x_0 \in B \setminus A$ and therefore $x_0 \in (A \setminus B) \cup (B \setminus A)$.
    Since $x_0 \in (A \setminus B) \cup (B \setminus A)$ and $(A \setminus B) \cup (B \setminus A) \subseteq A$, $x_0 \in A$
    which is a contradiction.
    Therefore, if $A \triangle B \subseteq A$ then $B \subseteq A$.
\end{proof}

\begin{tcolorbox}[title=Problem 8, breakable]
    Prove that for any sets $A$ and $B$, $\mathcal{P}(A) \cup \mathcal{P}(B) \subseteq \mathcal{P}(A \cup B)$.
\end{tcolorbox}

\begin{proof}
    Let $x_0$ be an arbitrary element such that $x_0 \in \mathcal{P}(A) \cup \mathcal{P}(B)$.
    So $x_0 \in \mathcal{P}(A)$ or $x_0 \in \mathcal{P}(B)$.

    Case ($x_0 \in \mathcal{P}(A)$) So $x_0 \subseteq A$ and it follows that $x_0
        \subseteq A \cup B$. Since $x_0 \subseteq A \cup B$, $x_0 \in \mathcal{P}(A
        \cup B)$.

    Case ($x_0 \in \mathcal{P}(B)$) So $x_0 \subseteq B$ and it follows that $x_0
        \subseteq A \cup B$. Since $x_0 \subseteq A \cup B$, $x_0 \in \mathcal{P}(A
        \cup B)$.

    Since these cases were exhaustive, for any sets $A$ and $B$, $\mathcal{P}(A)
        \cup \mathcal{P}(B) \subseteq \mathcal{P}(A \cup B)$.
\end{proof}

\begin{tcolorbox}[title=Problem 9, breakable]
    Prove that for any sets $A$ and $B$, if $\mathcal{P}(A) \cup \mathcal{P}(B) = \mathcal{P}(A \cup B)$
    then either $A \subseteq B$ or $B \subseteq A$.
\end{tcolorbox}

\begin{proof}
    Suppose $\mathcal{P}(A) \cup \mathcal{P}(B) = \mathcal{P}(A \cup B)$.
    Assume for contradiction $A \not \subseteq B$ and $B \not \subseteq A$.
    There exists elements $x_0$, $y_0$ such that $x_0 \in A \setminus B$
    and $y_0 \in B \setminus A$.
    Now, $\{x_0, y_0\} \subseteq A \cup B$ and therefore
    $\{x_0, y_0\} \in \mathcal{P}(A \cup B)$.
    But $\{x_0, y_0\} \not \in \mathcal{P}(A)$ and 
    $\{x_0, y_0\} \not \in \mathcal{P}(B)$ which is a contradiction.
    Therefore, for any sets $A$ and $B$, if $\mathcal{P}(A) \cup \mathcal{P}(B)
        = \mathcal{P}(A \cup B)$
    then either $A \subseteq B$ or $B \subseteq A$.
\end{proof}

\begin{tcolorbox}[title=Problem 10, breakable]
    Suppose $x$ and $y$ are real numbers and $x \not = 0$. Prove that $y + 1/x  = 1 + y/x$ iff
    either $x = 1$ or $y = 1$.
\end{tcolorbox}

\begin{proof}
    ($\rightarrow$) Suppose $y + 1/x  = 1 + y/x$.
    Then:
    \begin{align*}
        y + 1/x          & = 1 + y/x &  & \\
        xy + 1           & = x + y   &  & \\
        x - 1 - y(x - 1) & = 0       &  & \\
        (x - 1)(1 - y)   & = 0 
    \end{align*}
    From this it follows that $x = 1$ or $y = 1$.

    ($\leftarrow$) Suppose $x = 1$ or $y = 1$.
    Suppose $x = 1$, then $y + 1/x = 1 + y/x$ and $y + 1 = y + 1$.
    Suppose $y = 1$, then $y + 1/x = 1 + y/x$ and $1 + 1/x = 1 + 1/x$.

    Therefore, $y + 1/x = 1 + y/x$ iff either $x = 1$ or $y = 1$.
\end{proof}

\begin{tcolorbox}[title=Problem 13, breakable]
    (a) Prove that for all real numbers $a$ and $b$, $|a| \le b$ iff $-b \le a \le b$.

    (b) Prove that for any real number $x$, $-|x| \le x \le |x|$. (Hint: Use part (a).)

    (c) Prove that for all real numbers $x$ and $y$, $|x + y| \le |x| + |y|$. (This
    is called the triangle inequality. One way to do this is to combine parts (a)
    and (b), but you can also do it by considering a number of cases.)

    (d) Prove that for all real numbers $x$ and $y$, $|x + y| \ge |x| - |y|$. (Hint:
    Start with the equation $|x| = |(x + y) + (-y)|$ and then apply the 
    triangle inequality to the right hand side.)
\end{tcolorbox}

\begin{proof}
    ($\rightarrow$) Suppose $|a| \le b$. Either $a \ge 0$ or $a \le 0$.

    Case ($a \ge 0$). Since $a \ge 0$, $|a| = a$. Plugging into $|a| \le b$ we get
    $a \le b$. It follows that $a \ge -b$ and therefore $-b \le a \le b$.

    Case ($a \le 0$). Since $a \le 0$, $|a| = -a$. Plugging into $|a| \le b$ we get
    $-a \le b$. Multiplying by $-1$ gives $a \ge -b$. It then follows that $a \le
        b$ and therefore $-b \le a \le b$. 

    ($\leftarrow$) Suppose $-b \le a \le b$. Either $a \ge 0$ or $a \le 0$.

    Case ($a \ge 0$). Since $a \ge 0$, $|a| = a$. Plugging into $a \le b$ we get
    $|a| \le b$.

    Case ($a \le 0$). Since $a \le 0$, $|a| = -a$. From $a \ge -b$ we can multiply
    by $-1$ and get $-a \le b$. Then plugging in we get $|a| \le b$.

    Therefore for all real numbers $a$ and $b$, $|a| \le b$ iff $-b \le a \le b$.
\end{proof}

\begin{proof}
    Since $x \le |x|$ using part $i$,
    $-|x| \le x \le |x|$.
\end{proof}

\begin{proof}
    Since $x \le |x|$ and $y \le  |y|$ we can can take their sum and get
    $x + y \le  |x| + |y|$. Now, since $-x \le |x|$ and $-y \le |y|$
    we can take their sum and get $-(x + y) \le |x| + |y|$.
    Since $x + y \le |x| + |y|$ and $-(x + y) \le |x| + |y|$,
    $|x + y| \le |x| + |y|$.
\end{proof}

\begin{proof}
    First:
    \begin{align*}
        |x| & = |x|              &  &   \\
            & = |x + 0|          &  &   \\
            & = |x + y - y|      &  &   \\
            & = |(x + y) + (-y)| &  & 
    \end{align*}
    From part $iii$ (triangle inequality):
    \[
        |x + y| + |-y| \ge |(x + y) + (-y)|
    \]
    So:
    \begin{align*}
        |x + y| + |-y| & \ge |(x + y) + (-y)| &  &                                    \\
        |x + y| + |y|  & \ge |x|              &  & \text{(since } |-y| = |y| \text{)} \\
        |x + y|        & \ge |x| - |y|        &  & 
    \end{align*}
\end{proof}

\begin{tcolorbox}[title=Problem 14, breakable]
    Prove that for every integer $x$, $x^2 + x$ is even.
\end{tcolorbox}

\begin{proof}
    Let $x \in \mathbb{Z}$. There are two cases. Either $x = 2k$
    or $x = 2k + 1$ for some $k \in \mathbb{Z}$.

    Case ($x = 2k$)
    \begin{align*}
        x + x^2 & = (2k) + {(2k)}^2 &  & \\
                & = 2k + 4k^2       &  & \\
                & = 2(k + 2k^2)     &  & \\
    \end{align*}
    Case ($x = 2k + 1$)
    \begin{align*}
        x + x^2 & = (2k + 1) + {(2k + 1)}^2 &  & \\
                & = 2k + 1 + 4k^2 + 4k + 1  &  & \\
                & = 4k^2 + 6k + 2           &  & \\
                & = 2(2k^2 + 3k + 1)
    \end{align*}
    Therefore, for every integer $x$, $x^2 + x$ is even.
\end{proof}

\begin{tcolorbox}[title=Problem 15, breakable]
    Prove that for every integer $x$, the remainder when $x^4$ is divided by $8$
    is either $0$ or $1$.
\end{tcolorbox}

\begin{proof}
    Either $x$ is even or odd.
    Suppose $x$ is even. Then $x = 2k$ where $k \in \mathbb{Z}$.
    Then ${(2k)}^4 = 16k^4 = 8(2k^4)$ which when divided by $8$ has a remainder of $0$.

    Suppose $x$ is odd. Then $x = 2k + 1$ where $k \in \mathbb{Z}$. Then:
    \begin{align*}
        {(2k + 1)}^4 & = 16k^4 + 32k^3 + 24k^2 + 8k + 1  &  & \\
                     & = 8(2k^4 + 4k^3 + 3k^2 + k) + 1 
    \end{align*}
    which when divided by $8$ has a remainder of $1$.

    Since these cases were exhaustive, for every integer $x$, the remainder when
    $x^4$ is divided by $8$ is either $0$ or $1$.
\end{proof}

\begin{tcolorbox}[title=Problem 16, breakable]
    Suppose $\mathcal{F}$ and $\mathcal{G}$ are nonempty families of sets.

    (a) Prove that $\bigcup(\mathcal{F} \cup \mathcal{G}) = (\bigcup \mathcal{F}) \cup (\bigcup \mathcal{G})$.

    (b) Can you discover and prove a similar theorem about $\bigcap(\mathcal{F} \cup \mathcal{G})$.
\end{tcolorbox}

\begin{proof}
    First we show $\bigcup(\mathcal{F} \cup \mathcal{G}) \subseteq (\bigcup \mathcal{F}) \cup (\bigcup \mathcal{G})$.
    Let $x_0$ be an arbitrary element such that $x_0 \in \bigcup(\mathcal{F} \cup \mathcal{G})$.
    There exists a set $A$ such that $x_0 \in A$ and either $A \in \mathcal{F}$ or $A \in \mathcal{G}$.

    \textbf{Case $A \in \mathcal{F}$}: Then, since $x_0 \in A$, 
    $x_0 \in (\bigcup \mathcal{F})$ and $x_0 \in (\bigcup \mathcal{F}) \cup (\bigcup \mathcal{G})$.

    \textbf{Case $A \in \mathcal{G}$}: Then, since $x_0 \in A$, 
    $x_0 \in (\bigcup \mathcal{G})$ and $x_0 \in (\bigcup \mathcal{F}) \cup (\bigcup \mathcal{G})$.

    Since these cases are exhaustive, it follows that $\bigcup(\mathcal{F} \cup
        \mathcal{G}) \subseteq (\bigcup \mathcal{F}) \cup (\bigcup \mathcal{G})$.

    We now show that $(\bigcup \mathcal{F}) \cup (\bigcup \mathcal{G}) \subseteq
        \bigcup(\mathcal{F} \cup \mathcal{G})$. Let $x_0$ be an arbitrary element such
    that $x_0 \in (\bigcup \mathcal{F}) \cup (\bigcup \mathcal{G})$. Either $x_0
        \in (\bigcup \mathcal{F})$ or $x_0 \in (\bigcup \mathcal{G})$.

    \textbf{Case $x_0 \in (\bigcup\mathcal{F})$} There exists a set $A$ such that $x_0 \in A$ and $A \in \mathcal{F}$.
    It follows that $A \in \mathcal{F} \cup \mathcal{G}$.
    Since $x_0 \in A$ and $A \in \mathcal{F} \cup \mathcal{G}$, $x_0 \in \bigcup(\mathcal{F} \cup \mathcal{G})$.

    \textbf{Case $x_0 \in (\bigcup\mathcal{G})$} There exists a set $A$ such that $x_0 \in A$ and $A \in \mathcal{G}$.
    It follows that $A \in \mathcal{F} \cup \mathcal{G}$.
    Since $x_0 \in A$ and $A \in \mathcal{F} \cup \mathcal{G}$, $x_0 \in \bigcup(\mathcal{F} \cup \mathcal{G})$.

    Since these cases are exhaustive, it follows that $(\bigcup \mathcal{F}) \cup
        (\bigcup \mathcal{G}) \subseteq \bigcup(\mathcal{F} \cup \mathcal{G})$.

    Since $\bigcup(\mathcal{F} \cup \mathcal{G}) \subseteq (\bigcup \mathcal{F})
        \cup (\bigcup \mathcal{G})$ and $(\bigcup \mathcal{F}) \cup (\bigcup
        \mathcal{G}) \subseteq \bigcup(\mathcal{F} \cup \mathcal{G})$,
    $\bigcup(\mathcal{F} \cup \mathcal{G}) = (\bigcup \mathcal{F}) \cup (\bigcup
        \mathcal{G})$.
\end{proof}

A similar theorem is $\bigcap(\mathcal{F} \cup \mathcal{G}) =
    (\bigcap{\mathcal{F}}) \cap (\bigcap{\mathcal{G}})$.
\begin{proof}
    We first show $\bigcap(\mathcal{F} \cup \mathcal{G}) \subseteq (\bigcap{\mathcal{F}}) \cap (\bigcap{\mathcal{G}})$.
    Let $x_0$ be an arbitrary element such that $x_0 \in \bigcap(\mathcal{F} \cup \mathcal{G})$.
    For all sets $T \in \mathcal{F} \cup \mathcal{G}$, $x_0 \in T$.
    It follows that for all sets $A \in \mathcal{F}$, $x_0 \in A$; 
    and for all sets $B \in \mathcal{G}$, $x_0 \in B$.
    Since $x_0 \in A$ for all $A \in \mathcal{F}$, it follows that $x_0 \in \bigcap \mathcal{F}$.
    Since $x_0 \in B$ for all $B \in \mathcal{G}$, it follows that $x_0 \in \bigcap \mathcal{G}$.
    Since $x_0 \in (\bigcap\mathcal{F})$ and $x_0 \in (\bigcap\mathcal{G})$,
    $x_0 \in (\bigcap\mathcal{F}) \cap (\bigcap\mathcal{G})$.

    We now show $(\bigcap{\mathcal{F}}) \cap (\bigcap{\mathcal{G}}) \subseteq
        \bigcap(\mathcal{F} \cup \mathcal{G})$. Let $x_0$ be an arbitrary element such
    that $x_0 \in (\bigcap{\mathcal{F}}) \cap (\bigcap{\mathcal{G}})$. It follows
    that $x_0 \in (\bigcap{\mathcal{F}})$ and $x_0 \in (\bigcap{\mathcal{G}})$.
    Since $x_0 \in A$ for all $A \in \mathcal{F}$ and $x_0 \in B$ for all $B \in
        \mathcal{G}$, it follows that $x_0 \in T$ for all $T \in \mathcal{F} \cup
        \mathcal{G}$. Therefore, $x_0 \in \bigcap(\mathcal{F} \cup \mathcal{G})$.

    Therefore, $\bigcap(\mathcal{F} \cup \mathcal{G}) = (\bigcap{\mathcal{F}}) \cap
        (\bigcap{\mathcal{G}})$.
\end{proof}

\begin{tcolorbox}[title=Problem 17, breakable]
    Suppose $\mathcal{F}$ is a nonempty family of sets and $B$ is a set.

    (a) Prove that $B \cup (\bigcup \mathcal{F}) = \bigcup(\mathcal{F} \cup \{B\})$.

    (b) Prove that $B \cup (\bigcap \mathcal{F}) = \bigcap_{A \in \mathcal{F}}(B \cup A)$.

    (c) Can you discover and prove similar theorems about $B \cap (\bigcup \mathcal{F})$
    and $B \cap (\bigcap \mathcal{F})$.
\end{tcolorbox}

\begin{proof}
    We first show $B \cup (\bigcup \mathcal{F}) \subseteq \bigcup(\mathcal{F} \cup \{B\})$.
    Let $x_0$ be an arbitrary element such that $x_0 \in B \cup (\bigcup \mathcal{F})$.
    Either $x_0 \in B$ or $x_0 \in (\bigcup \mathcal{F})$.

    Suppose $x_0 \in B$ it immediately follows that $x_0 \in \bigcup(\mathcal{F}
        \cup \{B\})$.

    Suppose $x_0 \in (\bigcup \mathcal{F})$. There exists a set $A$ such that $x_0
        \in A$ and $A \in \mathcal{F}$. It follows that $A \in \mathcal{F} \cup \{B\}$.
    Since $x_0 \in A$ then $x_0 \in \bigcup(\mathcal{F} \cup \{B\})$.

    We now show $\bigcup(\mathcal{F} \cup \{B\}) \subseteq B \cup (\bigcup
        \mathcal{F})$. Let $x_0$ be an arbitrary element such that $x_0 \in
        \bigcup(\mathcal{F} \cup \{B\})$. There exists a set $A$ such that $A \in
        \mathcal{F} \cup \{B\}$. So $A \in \mathcal{F}$ or $A \in \{B\}$.

    Suppose $A \in \mathcal{F}$. Since $x_0 \in A$, $x_0 \in
        (\bigcup{\mathcal{F}})$. It follows that $x_0 \in B \cup (\bigcup
        \mathcal{F})$.

    Suppose $A \in \{B\}$. Since $x_0 \in A$, $x_0 \in B$. It follows that $x_0 \in
        B \cup (\bigcup \mathcal{F})$.

    Therefore, $B \cup (\bigcup \mathcal{F}) = \bigcup(\mathcal{F} \cup \{B\})$.
\end{proof}

\begin{proof}
    We first show $B \cup (\bigcap \mathcal{F}) \subseteq \bigcap_{A \in \mathcal{F}}(B \cup A)$.
    Let $x_0$ be an arbitrary element such that $x_0 \in B \cup (\bigcap \mathcal{F})$.
    So $x_0 \in B$ or $x_0 \in (\bigcap{\mathcal{F}})$.

    Suppose $x_0 \in B$. It follows that for all $A \in \mathcal{F}$ $x_0 \in B
        \cup A$. Therefore, $x_0 \in \bigcap_{A \in \mathcal{F}}(B \cup A)$.

    Suppose $x_0 \in (\bigcap \mathcal{F})$. For all $A \in \mathcal{F}$, $x_0 \in
        A$ and, therefore, $x_0 \in B \cup A$. Therefore $x_0 \in \bigcap_{A \in
            \mathcal{F}}(B \cup A)$.

    We now show $\bigcap_{A \in \mathcal{F}}(B \cup A) \subseteq B \cup (\bigcap
        \mathcal{F})$. Let $x_0$ be an arbitrary element such that $x_0 \in \bigcap_{A
            \in \mathcal{F}}(B \cup A)$. Either $x_0 \in B$ or for all sets $A \in
        \mathcal{F}$, $x_0 \in A$. Meaning $x_0 \in (\bigcap\mathcal{F})$.

    Suppose $x_0 \in B$. It immediately follows that $x_0 \in B \cup
        (\bigcap\mathcal{F})$.

    Suppose $x_0 \in (\bigcap\mathcal{F})$. It immediately follows that $x_0 \in B
        \cup (\bigcap\mathcal{F})$.

    Therefore, $B \cup (\bigcap \mathcal{F}) = \bigcap_{A \in \mathcal{F}}(B \cup
        A)$.
\end{proof}

We now prove $B \cap (\bigcup \mathcal{F}) = \bigcup_{A \in \mathcal{F}}(B \cap
    A)$.

\begin{proof}
    We first show $B \cap (\bigcup \mathcal{F}) \subseteq \bigcup_{A \in \mathcal{F}}(B \cap A)$.
    Let $x_0$ be an arbitrary element such that $x_0 \in B \cap (\bigcup \mathcal{F})$.
    So $x_0 \in B$ and $x_0 \in \bigcup \mathcal{F}$.
    Since $x_0 \in \bigcup \mathcal{F}$, there exists a set $A$
    such that $A \in \mathcal{F}$ and $x_0 \in A$.
    It follows that $x_0 \in B \cap A$ and therefore, $x_0 \in \bigcup_{A \in \mathcal{F}}(B \cap A)$.

    We now show $\bigcup_{A \in \mathcal{F}}(B \cap A) \subseteq B \cap (\bigcup
        \mathcal{F})$. Let $x_0$ be an arbitrary element such that $x_0 \in \bigcup_{A
            \in \mathcal{F}}(B \cap A)$. Since $x_0 \in \bigcup_{A \in \mathcal{F}}(B \cap
        A)$, $x_0 \in B$ and $x_0 \in A$ for some $A \in \mathcal{F}$. Since $x_0 \in
        A$ for some $A \in \mathcal{F}$, $x_0 \in \bigcup(\mathcal{F})$. Since $x_0 \in
        B$ and $x_0 \in \bigcup(\mathcal{F})$, $x_0 \in B \cap \bigcup(\mathcal{F})$.

    Therefore, $B \cap (\bigcup \mathcal{F}) = \bigcup_{A \in \mathcal{F}}(B \cap
        A)$.
\end{proof}

\begin{tcolorbox}[title=Problem 18, breakable]
    Suppose $\mathcal{F}$, $\mathcal{G}$, and $\mathcal{H}$
    are nonempty families of sets and for every $A \in \mathcal{F}$
    and every $B \in \mathcal{G}$, $A \cup B \in \mathcal{H}$. 
    Prove that $\bigcap{\mathcal{H}} \subseteq (\bigcap \mathcal{F}) \cup (\bigcap \mathcal{G})$.
\end{tcolorbox}

\begin{proof}
    Assume for contradiction $\bigcap{\mathcal{H}} \not \subseteq (\bigcap \mathcal{F}) \cup (\bigcap \mathcal{G})$.
    Let $x_0$ be an element such that $x_0 \in \bigcap{\mathcal{H}}$
    and $x_0 \not \in (\bigcap \mathcal{F}) \cup (\bigcap \mathcal{G})$.
    There is a set $A \in \mathcal{F}$ such that $x_0 \not \in A$
    and there is a set $B \in \mathcal{G}$ such that $x_0 \not \in B$.
    $A \cup B \in \mathcal{H}$, but $x_0 \not \in A \cup B$ and $x_0 \in T$ for all $T \in \mathcal{H}$
    which is a contradiction.
    Therefore, $\bigcap{\mathcal{H}}
        \subseteq (\bigcap \mathcal{F}) \cup (\bigcap \mathcal{G})$.
\end{proof}

\begin{tcolorbox}[title=Problem 26, breakable]
    Suppose $A$, $B$, and $C$ are sets. Consider the sets $(A \setminus B) \triangle C$ and 
    $(A \triangle C) \setminus (B \triangle C)$. Can you prove that either is a subset of the other?
    Justify your conclusions with either proofs or counterexamples.
\end{tcolorbox}

We will prove that $(A \triangle C) \setminus (B \triangle C) \subseteq (A
    \setminus B) \triangle C$.
\begin{proof}
    Let $x_0$ be an arbitrary element such that $x_0 \in (A \triangle C) \setminus (B \triangle C)$.
    So $x_0 \in (A \triangle C)$ and $x_0 \not \in B \triangle C$.
    Since $x_0 \in A \triangle C$, either $x_0 \in A \setminus C$ or $x_0 \in C \setminus A$.
    Since $x_0 \notin B \triangle C$, it follows that either $x_0 \in B \cap C$ or $x_0 \notin B \cup C$. There
    are four cases.

    Case ($x_0 \in A \setminus C$ and $x_0 \in B \cap C$) This isn't possible since
    it would require $x_0 \not \in C$ since $x_0 \in A \setminus C$ and $x_0 \in C$
    since $x_0 \in B \cap C$.

    Case ($x_0 \in A \setminus C$ and $x_0 \notin B \cup C$) Since $x_0 \not\in B
        \cup C$, $x_0 \not\in C$. Since $x_0 \not\in C$, $x_0 \notin (A \setminus B)
        \cap C$. Since $x_0 \in A \setminus C$, $x_0 \in A$. Since $x_0 \not \in B \cup
        C$, $x_0 \not \in B$. Since $x_0 \in A$ and $x_0 \not\in B$, $x_0 \in A
        \setminus B$. It follows that $x_0 \in (A \setminus B) \cup C$. Since $x_0 \in
        (A \setminus B) \cup C$ and $x_0 \not \in (A \setminus B) \cap C$, $x_0 \in (A
        \setminus B) \triangle C$.

    Case ($x_0 \in C \setminus A$ and $x_0 \in B \cap C$) Since $x_0 \in B \cap C$,
    $x_0 \in C$. Since $x_0 \in C$, $x_0 \in (A \setminus B) \cup C$. Since $x_0
        \in B \cap C$, $x_0 \in B$. Since $x_0 \in B$, $x_0 \not \in A \setminus B$.
    Since $x_0 \not \in A \setminus B$, $x_0 \not \in (A \setminus B) \cap C$.
    Since $x_0 \in (A \setminus B) \cup C$ and $x_0 \not \in (A \setminus B) \cap
        C$, $x_0 \in (A \setminus B) \triangle C$.

    Case ($x_0 \in C \setminus A$ and $x_0 \notin B \cup C$) Since $x_0 \in C
        \setminus A$, $x_0 \in C$. Since $x_0 \not\in B \cup C$, $x_0 \not \in C$ which
    is a contradiction.

    Therefore, $(A \triangle C) \setminus (B \triangle C) \subseteq (A \setminus B)
        \triangle C$.
\end{proof}

It is not true that $(A \setminus B) \triangle C \subseteq (A \triangle C)
    \setminus (B \triangle C)$. Counterexample:

$A = \emptyset$

$B = \emptyset$

$C = \{1\}$

$A \setminus B = \emptyset$

$(A \setminus B) \triangle C = \{1\}$

$A \triangle C = \{1\}$

$B \triangle C = \{1\}$

$(A \triangle C) \setminus (B \triangle C) = \emptyset$

Clearly $(A \setminus B) \triangle C = \{1\} \not \subseteq (A \triangle C)
    \setminus (B \triangle C) = \emptyset$.

\begin{tcolorbox}[title=Problem 27, breakable]
    Consider the following putative theorem.

    \text{Theorem?} \emph{For every real number $x$, if $|x - 3| < 3$ then $0 < x < 6$.}

    Is the following proof correct? If so, what proof strategies does it use? If
    not, can it be fixed? Is the theorem correct?

    \begin{proof}
        Let $x$ be an arbitrary real number, and suppose $|x - 3| < 3$. We consider
        two cases.

        Case $1$. $x - 3 \ge 0$. Then $|x - 3| = x - 3$. Plugging this into the
        assumption that $|x - 3| < 3$, we get $x - 3 < 3$, so clearly $x < 6$.

        Case $2$. $x - 3 < 0$. Then $|x - 3| = 3 - x$, so the assumption $|x - 3| < 3$
        means that $3 - x < 3$. Therefore $3 < 3 + x$, so $0 < x$.

        Since we have proven both $0 < x$ and $x < 6$, we can conclude that $0 < x <
            6$.
    \end{proof}
\end{tcolorbox}

The theorem is correct. Proof is wrong it's missing the upper or lower bound of
$x$.

\begin{proof}
    Let $x$ be an arbitrary real number, and suppose $|x - 3| < 3$. We consider
    two cases.

    Case $1$. $x - 3 \ge 0$. Since $x - 3 \ge 0$ it follows that $x \ge 3$ so $x
        \ge 0$. Then $|x - 3| = x - 3$. Plugging this into the assumption that $|x - 3|
        < 3$, we get $x - 3 < 3$, so clearly $x < 6$.

    Case $2$. $x - 3 < 0$. Since $x - 3 < 0$ it follows that $x < 3$ so $x < 6$.
    Then $|x - 3| = 3 - x$, so the assumption $|x - 3| < 3$ means that $3 - x < 3$.
    Therefore $3 < 3 + x$, so $0 < x$.

    Since we have proven both $0 < x$ and $x < 6$, we can conclude that $0 < x <
        6$.
\end{proof}

\begin{tcolorbox}[title=Problem 28, breakable]
    Consider the following putative theorem.

    \textbf{Theorem?} \emph{For any sets $A$, $B$, and $C$, if $A \setminus B \subseteq C$
        and $A \not \subseteq C$ then $A \cap B \not = \emptyset$.}

    Is the following proof correct? If so, what proof strategies does it use? If
    not, can it be fixed? Is the theorem correct?

    \begin{proof}
        Suppose $A \setminus B \subseteq C$ and $A \not \subseteq C$. Since $A \not \subseteq C$,
        so we can choose some $x$ such that $x \in A$ and $x \not \in C$. Since $x \not \in C$
        and $A \setminus B \subseteq C$, $x \not \in A \setminus B$. Therefore either 
        $x \not \in A$ or $x \in B$. But we already know that $x \in A$, so it follows 
        that $x \in B$. Since $x \in A$ and $x \in B$, $x \in A \cap B$.
        Therefore $A \cap B \not = \emptyset$.
    \end{proof}
\end{tcolorbox}

\textbf{Solution 28}

The proof is correct. It uses direct reasoning based on the definition of set
difference. The theorem is correct.

\begin{tcolorbox}[title=Problem 29, breakable]
    Consider the following putative theorem.

    \textbf{Theorem?} \emph{$\forall{x \in \mathbb{R} \exists{y \in \mathbb{R}}(xy^2 \not = y - x)}$}

    Is the following proof correct? If so, what proof strategies does it use? If
    not, can it be fixed? Is the theorem correct?

    \begin{proof}
        Let $x$ be an arbitrary real number.

        Case $1$. $x = 0$. Let $y = 1$. Then $xy^2 = 0$ and $y - x = 1 - 0 = 1$, so
        $xy^2 \not = y - x$.

        Case $2$. $x \not = 0$. Let $y = 0$. Then $xy^2 = 0$ and $y - x = -x \not = 0$,
        so $xy^2 \not = y - x$.

        Since these cases are exhaustive, we have shown that $\exists{y \in
                \mathbb{R}}(xy^2 \not = y - x)$. Since $x$ was arbitrary, this shows that
        $\forall{x \in \mathbb{R}}\exists{y \in \mathbb{R}}(xy^2 \not = y - x)$.
    \end{proof}
\end{tcolorbox}

\textbf{Solution 29}

The proof is correct and it uses proof by cases. The theorem is correct.

\begin{tcolorbox}[title=Problem 31, breakable]
    Consider the following putative theorem.

    \textbf{Theorem?} \emph{Suppose $A$, $B$, and $C$ are sets and $A \subseteq B \cup C$.
        Then either $A \subseteq B$ or $A \subseteq C$.}

    Is the following proof correct? If so, what proof strategies does it use? If
    not, can it be fixed? Is the theorem correct?

    \begin{proof}
        Let $x$ be an arbitrary element of $A$. Since $A \subseteq B \cup C$, it follows
        that either $x \in B$ or $x \in C$.

        Case $1$. $x \in B$. Since $x$ was an arbitrary element of $A$, it follows that
        $\forall{x \in A}(x \in B)$, which means that $A \subseteq B$.

        Case $2$. $x \in C$. Similarly, since $x$ was an arbitrary element of $A$, we
        can conclude that $A \subseteq C$.

        Thus, either $A \subseteq B$ or $A \subseteq C$.
    \end{proof}
\end{tcolorbox}

\textbf{Solution 31}

The theorem is incorrect. It attempts to use proof by cases. From $A \subseteq
    B \cup C$, it does NOT follow that either $A \subseteq B$ or $A \subseteq C$.
Proof is not fixable as the theorem is incorrect. In the first case, the proof
shows that one particular $x \in A$ lies in $B$, but that does not imply all
elements of $A$ are in $B$. The same issue applies to Case 2.

Counterexample:

Let $B = \{1\}$.

Let $C = \{2\}$.

Let $A = \{1, 2\}$,

Obviously $A \subseteq B \cup C$ but $A \not \subseteq B$ and $A \not \subseteq
    C$.

\begin{tcolorbox}[title=Problem 33, breakable]
    Prove that $\exists{x}(P(x) \rightarrow \forall{y}P(y))$. (Note: Assume the universe of discourse
    is not the empty set.)
\end{tcolorbox}

\begin{proof}
    Since the universe of discourse is not empty
    there are two cases:

    Case 1: $P(x)$ is false for some $x$. Then the implication $P(x) \rightarrow
        \forall{y}P(y)$ is true.

    Case 2: $P(x)$ is true for all $x$. Then for all $y$, $P(y)$ is true, so the
    implication holds for every $x$.

    Therefore, $\exists{x}(P(x) \rightarrow \forall{y}P(y))$.
\end{proof}

\subsection{Existence and Uniqueness Proof}

\begin{tcolorbox}[title=Problem 1, breakable]
    Prove that for every real number $x$ there is a unique
    real number $y$ such that $x^2y = x - y$.
\end{tcolorbox}

\begin{proof}
    We first prove existence. Let $y = \frac{x}{x^2 + 1}$.
    Then:
    \begin{align*}
        x^2y                                        & = x - y                 &  & \\
        \leftrightarrow x^2 \cdot \frac{x}{x^2 + 1} & = x - \frac{x}{x^2 + 1} &  & \\
        \leftrightarrow x^2 \cdot x                 & = x(x^2 + 1) - x        &  & \\
        \leftrightarrow x^2 \cdot x                 & = x^3 + x - x           &  & \\
        \leftrightarrow x^3                         & = x^3                   &  & \\
        \leftrightarrow x                           & = x
    \end{align*}
    We now prove uniqueness. Suppose $x^2y + y = x$.
    Then:
    \begin{align*}
        x^2y                       & = x - y             &  &   \\
        \leftrightarrow x^2y + y   & = x                 &  &   \\
        \leftrightarrow y(x^2 + 1) & = x                 &  &   \\
        \leftrightarrow y          & = \frac{x}{x^2 + 1} &  & 
    \end{align*}
\end{proof}

\begin{tcolorbox}[title=Problem 2, breakable]
    Prove that there is a unique real number $x$ such that 
    for every real number $y$, $xy + x - 4 = 4y$.
\end{tcolorbox}

\begin{proof}
    We first prove existence. Let $x = 4$. Then:
    \begin{align*}
        xy + x - 4                 & = 4y &  &   \\
        \leftrightarrow 4y + 4 - 4 & = 4y &  &   \\
        \leftrightarrow 4y         & = 4y &  &   \\
        \leftrightarrow y          & = y  &  & 
    \end{align*}
    We now prove uniqueness.
    \begin{align*}
        xy + x - 4 - 4y                     & = 0 &  &   \\
        \leftrightarrow x(y + 1) - 4(y + 1) & = 0 &  &   \\
        \leftrightarrow (y + 1)(x - 4)      & = 0 &  & 
    \end{align*}
    So there is exactly one value of $x$, namely $x = 4$, which is a solution for all $y$.
\end{proof}

\begin{tcolorbox}[title=Problem 3, breakable]
    Prove that for ever real number $x$, if $x \not = 0$ and $x \not = 1$
    then there is a unique real number $y$ such that $y/x = y - x$.
\end{tcolorbox}

\begin{proof}
    We first prove existence.
    Suppose $x \not = 0$ and $x \not = 1$.
    Let $y = \frac{-x^2}{1 - x}$.
    Then:
    \begin{align*}
        \frac{y}{x}                                    & = y - x                                                    &  &                                                 \\
        \leftrightarrow \frac{(\frac{-x^2}{1 - x})}{x} & = \frac{-x^2}{1 - x} - x                                   &  &                                                 \\
        \leftrightarrow \frac{(\frac{-x^2}{1 - x})}{x} \cdot (x)(1 - x)
                                                       & = \frac{-x^2}{1 - x} \cdot (x)(1 - x) - x \cdot (x)(1 - x) &  & \quad\text{since $x \not = 0$ and $x \not = 1$} \\
        \leftrightarrow \frac{-x^2}{1 - x} \cdot (1 - x)
                                                       & = -x^2 \cdot x - x \cdot (x)(1 - x)                        &  &                                                 \\
        \leftrightarrow -x^2                           & = -x^3 - (x^2)(1 - x)                                      &  &                                                 \\
        \leftrightarrow -x^2                           & = -x^3 - x^2 + x^3                                         &  &                                                 \\
        \leftrightarrow -x^2                           & = - x^2                                                    &  &                                                 \\
        \leftrightarrow x                              & = x                                                        &  & 
    \end{align*}
    We now prove uniqueness.
    \begin{align*}
        \frac{y}{x}                 & = y - x              &  &                                \\
        \leftrightarrow \frac{y}{x} & = y - x              &  &                                \\
        \leftrightarrow y           & = xy - x^2           &  & \quad\text{Since $x \not = 0$} \\
        \leftrightarrow y - xy      & = - x^2              &  &                                \\
        \leftrightarrow y(1 - x)    & = - x^2              &  &                                \\
        \leftrightarrow y           & = \frac{-x^2}{1 - x} &  & 
    \end{align*}
\end{proof}

\begin{tcolorbox}[title=Problem 4, breakable]
    Prove that for every real number $x$, if $x \not = 0$ then there is 
    a unique real number $y$ such that for every real number $z$, $zy = z / x$.
\end{tcolorbox}

\begin{proof}
    We first prove existence. 
    Suppose $x \not = 0$.
    Let $y = \frac{1}{x}$ which is defined since $x \not = 0$.
    Then:
    \begin{align*}
        zy                                  & = \frac{z}{x} &  &   \\
        \leftrightarrow z \cdot \frac{1}{x} & = \frac{z}{x} &  &   \\
        \leftrightarrow \frac{z}{x}         & = \frac{z}{x} &  & 
    \end{align*}
    We now prove uniqueness.
    \begin{align*}
        zy                        & = \frac{z}{x} &  &                                \\
        \leftrightarrow xzy       & = z           &  & \quad\text{Since $x \not = 0$} \\
        \leftrightarrow xzy - z   & = 0           &  &                                \\
        \leftrightarrow z(xy - 1) & = 0           &  & 
    \end{align*}
    It then follows, since $xy - 1  = 0$ is a solution, that $y = \frac{1}{x}$.
\end{proof}

\begin{tcolorbox}[title=Problem 6, breakable]
    Let $U$ be any set.

    (a) Prove that there is a unique $A \in \mathcal{P}(U)$ such 
    that for every $B \in \mathcal{P}(U)$, $A \cup B  = B$.

    (b) Prove that there is a unique $A \in \mathcal{P}(U)$ such 
    that for every $B \in \mathcal{P}(U)$, $A \cup B = A$.
\end{tcolorbox}

\begin{proof}
    We first show existence.
    Let $A = \emptyset$.
    Since $A \in \mathcal{P}(U)$, $A \subseteq U$.
    It trivially follows that $A \cup B = B$.

    We now show uniqueness. Let $C$ and $D$ be arbitrary sets such that $C \cup B =
        B$, $C \in \mathcal{P}(U)$ and $D \cup B = B$, $D \in \mathcal{P}(U)$. Let $B =
        D$, then $C \cup D = D$. Let $B = C$, then $D \cup C = C$. Since $C \cup D = D
        \cup C$, $D = C$.
\end{proof}

\begin{tcolorbox}[title=Problem 7, breakable]
    Let $U$ be any set.

    (a) Prove that there is a unique $A \in \mathcal{P}(U)$
    such that for every $B \in \mathcal{P}(U)$, $A \cap B = B$.

    (b) Prove that there is a unique $A \in \mathcal{P}(U)$
    such that for every $B \in \mathcal{P}(U)$, $A \cap B = A$.
\end{tcolorbox}

\begin{proof}
    We first show existence. Let $A = U$. 
    Clearly $A \subseteq U$, 
    therefore $A \in \mathcal{P}(U)$.
    Let $B$ be an arbitrary set such that $B \in \mathcal{P}(U)$.
    Since $B \in \mathcal{P}(U)$, $B \subseteq U$.
    It follows that for all $x \in B$, $x \in A$.
    Therefore, $A \cap B = B$.

    We now show uniqueness. Let $C$ and $D$ be arbitrary sets such that for all $B
        \in \mathcal{P}(U)$, $C \cap B = B$ and $D \cap B = B$. Let $B = D$, then $C
        \cap D = D$. Let $B = C$, then $D \cap C = C$. Since $C \cap D = D \cap C$, $D
        = C$.
\end{proof}

\begin{proof}
    We first show existence.
    Let $A = \emptyset$. Clearly $A \subseteq U$,
    therefore $A \in \mathcal{P}(U)$.
    Let $B$ be an arbitrary set such that $B \in \mathcal{P}(U)$.
    Trivially $A \cap B = A$.

    We now show uniqueness. Let $C$ and $D$ be arbitrary sets such that for all $B
        \in \mathcal{P}(U)$, $C \cap B = C$ and $D \cap B = D$. Let $B = D$, then $C
        \cap D = C$. Let $B = C$, then $D \cap C = D$. Since $C \cap D = D \cap C$, $C
        = D$.
\end{proof}

\begin{tcolorbox}[title=Problem 9, breakable]
    Recall that you showed in excersize $14$ of Section $1.4$
    that symmetric difference is associative; in other words,
    for all sets $A, B$ and $C, A \triangle (B \triangle C)
        = (A \triangle B) \triangle C$. You may also find 
    it useful in this problem to note that the symmetric 
    difference is clearly commutative; in other words,
    for all sets $A$ and $B$, $A \triangle B = B \triangle A$.

    (a) Prove that there is a unique identity element for symmetric 
    difference. In other words, there is a unique set $X$
    such that for every set $A$, $A \triangle X = A$.

    (b) Prove that every set has a unique inverse for the operation
    of symmetric difference. In other words, for every set $A$ there 
    is a unique set $B$ such that $A \triangle B = X$, where $X$
    is the identity element from part (a).

    (c) Prove that for any sets $A$ and $B$ there is a unique 
    set $C$ such that $A \triangle C = B$.

    (d) Prove that for every set $A$ there is a unique set $B \subseteq A$
    such that for every set $C \subseteq A$, $B \triangle C = A \setminus C$.
\end{tcolorbox}

\begin{proof}
    We first show existence.
    Let $X = \emptyset$ and let $A$ be an arbitrary set.
    By the definition of the symmetric difference 
    $A \triangle X = (A \cup X) \setminus (A \cap X)$.
    Cleary $A \cup X = A \cup \emptyset = A$,
    and $A \cap X = A \cap \emptyset = \emptyset$.
    Therefore, $A \triangle X = (A \cup X) \setminus (A \cap X) =  A \setminus \emptyset = A$.

    We now show uniqueness. Let $B$ and $C$ be arbitrary sets such that for all
    $A$, $A \triangle B = A$ and $A \triangle C = A$. Let $A = B$, then $B
        \triangle C = B$. Let $A = C$, then $C \triangle B = C$. So, $B = B \triangle C
        = C \triangle B = C$ showing $B = C$.
\end{proof}

\begin{proof}
    We first show existence.
    Let $A$ be an arbitrary set.
    Let $B$ be a set such that $B = A$.
    By the definition of the symmetric difference 
    $A \triangle B = (A \cup B) \setminus (A \cap B)$.
    Clearly, $A \cup B = A \cup A = A$ and $A \cap B = A \cap A = A$.
    It follows that $A \triangle B = (A \cup B) \setminus (A \cap B) = A \setminus A = \emptyset$.

    We now show uniqueness. Suppose there exists a set $B$ such that $A \triangle B
        = \emptyset$ and $A \not = B$. Either $A \not \subseteq B$ or $B \not \subseteq
        A$.

    Suppose $A \not \subseteq B$. Let $x$ be an element such that $x \in A$ and $x
        \not \in B$. Since $x \in A$, $x \in A \cup B$. Also, since $x \in A$ and $x
        \not \in B$, $x \not \in A \cap B$. Since $x \in A \cup B$ and $x \not \in A
        \cap B$, $x \in (A \cup B) \setminus (A \cap B) = A \triangle B$. Contradicting
    $A \triangle B = \emptyset$.

    Suppose $B \not \subseteq A$. Let $x$ be an element such that $x \in B$ and $x
        \not \in A$. Since $x \in B$, $x \in A \cup B$. Also, since $x \in B$ and $x
        \not \in A$, $x \not \in A \cap B$. Since $x \in A \cup B$ and $x \not \in A
        \cap B$, $x \in (A \cup B) \setminus (A \cap B) = A \triangle B$. Contradicting
    $A \triangle B = \emptyset$.
\end{proof}

\begin{proof}
    We first prove existence.
    Let $A$ and $B$ be arbitrary sets.
    Let $C = A \triangle B$.
    Now $A \triangle C = A \triangle (A \triangle B) = (A \triangle A) \triangle B$.
    By previous proof $A \triangle A = \emptyset$,
    therefore $(A \triangle A) \triangle B = \emptyset \triangle B$.
    Now, by the previous previous proof $\emptyset \triangle B = B$.

    We now show uniqueness. Suppose $X$ is another set such that $A \triangle X =
        B$. Then
    \[
        X  = (A \triangle A) \triangle X = A \triangle (A \triangle X) = A \triangle B = C.
    \]
    So $X = C$.
\end{proof}

\begin{proof}
    We first prove existence. Let $B = A$.
    Then for any $C \subseteq A$:
    \[
        B \triangle C = A \triangle C = (A \setminus C) \cup (C \setminus A) = A \setminus C,
    \]
    since $C \subseteq A$ implies $C \setminus A = \emptyset$.

    We now prove uniqueness. Suppose $X$ and $Y$ are sets such that for every $C
        \subseteq A$, $X \triangle C = A \setminus C$ and $Y \triangle C = A \setminus
        C$. Take $C = \emptyset$, then
    \[
        X \triangle \emptyset = A \setminus \emptyset = A \quad \text{and} \quad
        Y \triangle \emptyset = A \setminus \emptyset = A.
    \]
    Therefore $X = Y = A$.
\end{proof}

\begin{tcolorbox}[title=Problem 10, breakable]
    Suppose $A$ is a set, and for every family of sets $\mathcal{F}$,
    if $\bigcup \mathcal{F} = A$ then $A \in \mathcal{F}$. Prove that 
    $A$ has exactly one element.
\end{tcolorbox}

\begin{proof}
    Let $A$ be an arbitrary set.
    For contradiction, assume for every family of sets $\mathcal{F}$,
    if $\bigcup \mathcal{F} = A$ then $A \in \mathcal{F}$ and $A$ does not have exactly one element.
    Either $A = \emptyset$ or $A$ has more than $1$ element.

    Suppose $A = \emptyset$. Let $\mathcal{F} = \emptyset$. Clearly $\bigcup
        \mathcal{F} = A$, but $A \not \in \mathcal{F}$ so $A \not = \emptyset$.

    Suppose $A$ has more than one element. Let $\mathcal{T} = \{\{x\} \mid x \in
        A\}$. Clearly $\bigcup \mathcal{T} = A$, but $A \not \in \mathcal{T}$ which is
    a contradiction.
\end{proof}

\begin{tcolorbox}[title=Problem 13, breakable]
    (a) Prove that there is a unique real number $c$ such that there 
    is a unique real number $x$ such that $x^2 + 3x + c = 0$.
    (In other words, there is a unique real number $c$ such that 
    the equation $x^2 + 3x + c = 0$ has exactly one solution.)

    (b) Show that it is not the case that there is a unique real number 
    $x$ such that there is a unique real number $c$ such that 
    $x^2 + 3x + c = 0$. (Hint: You should be able to prove that 
    for every real number $x$ there is a unique real number $c$
    such that $x^2 + 3x + c = 0$.)
\end{tcolorbox}

\begin{proof}
    We first show existence.
    Let $c = \frac{9}{4}$.
    Then: $x^2 + 3x + \frac{9}{4} = 0 \iff (x + \frac{3}{2})(x + \frac{3}{2}) = 0$.
    There is a single solution, namely $x = \frac{-3}{2}$.

    We now show uniqueness. Given a polynomial of degree $2$ there is a single
    solution if and only if the discriminant is zero. Letting $a = 1, b = 3$ we get
    $\sqrt{3^2 - 4c} = 0 \iff 3^2 - 4c = 0 \iff c = \frac{9}{4}$.
\end{proof}

\begin{proof}
    We first show existence.
    Let $x$ be an arbitrary real number.
    Let $c = -x^2 - 3x$.
    Then: $x^2 + 3x + c = 0 \iff x^2 + 3x - x^2 - 3x = 0 \iff 0 = 0$.

    We now show uniqueness: $x^2 + 3x + c = 0 \iff x^2 + c = -3x \iff c = -x^2 -
        3x$.
\end{proof}

\subsection{More Examples of Proofs}

\begin{tcolorbox}[title=Problem 2, breakable]
    Prove that there is a unique positive real number $m$ that has the following two properties:

    (a) For every positive real number $x$, $\frac{x}{x + 1} < m$.

    (b) If $y$ is any positive real number with the property that for every positive real number $x$,
    $\frac{x}{x + 1} < y$, then $m \le y$.
\end{tcolorbox}

\begin{proof}
    We first show existence.
    Let $m = 1$. 
    First note that $\frac{x}{x + 1} < 1$.
    Trivially $\frac{x}{x + 1} < 1 = m$.
    Suppose $y < 1$ and $y > \frac{x}{x + 1}$ for all $x > 0$.
    Let $x = \frac{y}{1 - y}$.
    Then:
    \begin{align*}
        \frac{x}{x + 1} = \frac{\frac{y}{1-y}}{\frac{y}{1-y} + 1} = \frac{y}{y + 1 - y} = y
    \end{align*}
    So $y \ge 1 = m$.

    We now prove uniqueness. Suppose $m_1$, $m_2$ satisfy both properties. Then
    $m_1 \le y$ and $m_2 \le y$. Applying this to $y = m_1$ and $y = m_2$ gives
    $m_1 \le m_2$ and $m_2 \le m_1$, therefore, $m_1 = m_2$.
\end{proof}

\begin{tcolorbox}[title=Problem 6, breakable]
    Suppose $\mathcal{F}$ is a nonempty family of sets. Let $I = \bigcup \mathcal{F}$ and $J = \bigcap \mathcal{F}$.
    Suppose also that $J \not = \emptyset$, and notice that if follows that for every $X \in \mathcal{F}$,
    $X \not = \emptyset$, and also that $I \not = \emptyset$. Finally, suppose that $\{A_i \mid i \in I\}$
    is an indexed family of sets.

    (a) Prove that $\bigcup_{i \in I} A_i = \bigcup{X \in \mathcal{F}}(\bigcup_{i \in X} A_i)$.

    (b) Prove that $\bigcap_{i \in I} A_i = \bigcup_{X \in \mathcal{F}}(\bigcap_{i \in X} A_i)$.

    (c) Prove that $\bigcup_{i \in J} A_i \subseteq \bigcap_{X \in \mathcal{F}}(\bigcup_{i \in X} A_i)$. 
    Is it always the case that $\bigcup_{i \in J} A_i = \bigcap_{X \in \mathcal{F}}(\bigcup_{i \in X} A_i)$?
    Give either a proof or a counterexample to justify your answer.

    (d) Discover and prove a theorem relating $\bigcap_{i \in J} A_i$ and $\bigcup_{X \in \mathcal{F}}(\bigcap_{i \in X} A_i)$.
\end{tcolorbox}

\begin{proof}
    Suppose $x$ is an arbitrary element such that $x \in \bigcup_{i \in I} A_i$. 
    Let $i$ be the index of the set $x$ is in. 
\end{proof}

\begin{tcolorbox}[title=Problem 10, breakable]
    Is the following proof correct? If so, what proof strategies does it use?
    If not, can it be fixed? Is the theorem correct? (Note: The proof uses
    the fact that $\sqrt{2}$ is irrational, which we'll prove in Chapter 6 -
    see Theorem $6.4.5$)
    \begin{proof}
        Either $\sqrt{2}^{\sqrt{2}}$ is ration or it's irrational.

        Suppose $\sqrt{2}^{\sqrt{2}}$ is rational. Let $a = \sqrt{2}$ and $b =
            \sqrt{2}$. Then $a^b = \sqrt{2}^{\sqrt{2}}$, which is what we're assuming in
        this case is rational.

        Suppose $\sqrt{2}^{\sqrt{2}}$ is irrational. Let $a = \sqrt{2}^{\sqrt{2}}$ and
        $b = \sqrt{2}$. The $a$ is irrational by assumption, and we know that $b$ is
        irrational. Also,
        \begin{align*}
            a^b = (\sqrt{2}^{\sqrt{2}})^{\sqrt{2}} = \sqrt{2}^{\sqrt{2} \cdot \sqrt{2}} = \sqrt{2}^2 = 2
        \end{align*}
        which is rational.
    \end{proof}
\end{tcolorbox}

\textbf{Solution:}

Yes the proof is correct. It uses proof by cases. Since the proof is valid the
theorem is correct.
