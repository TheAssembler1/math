\subsection{Quantifiers}

\begin{tcolorbox}[title=Problem 1, breakable]
Analyize the logical forms of the following statements: \\
(a) Anyone who as forgiven at least one person is a saint. \\
(b)  Nobody in the calculus class is smarter than everybody in the discrete
math class. \\
(c) Everyone likes Mary, except Mary herself. \\
(d) Jane saw a police officer, and Roger saw one too. \\
(e) Jane saw a police officer, and Roger saw him too. 
\end{tcolorbox}

\textbf{Solution 1 (a)} \\
Let $P(x, y)$ mean ``$x$ has forgiven $y$''. \\
Let $S(x)$ mean ``$x$ is a saint''. \\
\begin{align*}
    \forall{x}(\exists{y}P(x, y) \rightarrow S(x))
\end{align*}
\textbf{Solution 1 (b)} \\
Let $S(x, y)$ mean ``$x$ is smarter than $y$''. \\
Let $M(x)$ mean ``$x$ is in the calculus class''. \\ 
Let $D(x)$ mean ``$x$ is in the discrete math class''. \\
\begin{align*}
    \forall{x}(M(x) \rightarrow (\exists{y}(D(y) \wedge S(y, x))))
\end{align*}
\textbf{Solution 1 (c)} \\
Let $L(x)$ mean ``$x$ likes Mary''. \\
Let $M(x)$ mean ``$x$ is Mary''. \\
\begin{align*}
    \forall{x}((\neg M(x) \rightarrow L(x)) \wedge (M(x) \rightarrow \neg L(x)))
\end{align*}
\textbf{Solution 1 (d)} \\
Let $J(x)$ mean ``$x$ is Jane''. \\
Let $R(x)$ mean ``$x$ is Roger''. \\
Let $P(x)$ mean ``$x$ saw a police officer'' \\
\begin{align*}
    \forall{x}((J(x) \rightarrow P(x)) \wedge (R(x) \rightarrow P(x)))
\end{align*}
\textbf{Solution 1 (e)} \\
Let $J(x)$ mean ``$x$ is Jane''. \\
Let $R(x)$ mean ``$x$ is Roger''. \\
Let $P(x, y)$ mean ``$x$ saw a police officer y'' \\
\begin{align*}
    \exists{y}(\forall{x}((J(x) \rightarrow P(x, y)) \wedge (R(x) \rightarrow P(x, y))))
\end{align*}

\begin{tcolorbox}[title=Problem 3, breakable]
Analyze the logical forms of the following statements. The universe of discourse
is $\mathbb{R}$. What are the free variables in each statement? \\
(a) Every number that is larger than $x$ is larger than $y$ \\
(b) For every number $a$, the equation $ax^2 + 4x - 2 = 0$ has at least one
 solution iff $a >= -2$. \\
(c) All solutions of the inequality $x^3 - 3x < 3$ are smaller than $10$. \\
(d) If there is a number $x$ such that $x^2 + 5x = w$ and there is a number $y$
such that $4 - y^2 = 2$ then $w$ is strictly between $-10$ and $10$.
\end{tcolorbox}

\textbf{Solution 3 (a)} \\
\begin{align*}
    \forall{t}((t > x) \rightarrow (t > y))
\end{align*}
Free variables are $x$ and $y$. \\
\textbf{Solution 3 (b)} \\
\begin{align*}
    \forall{t}(\exists{x}(tx^2 +4x -2 = 0) \leftrightarrow t \ge -2)
\end{align*}
No free variables. \\
\textbf{Solution 3 (c)} \\
\begin{align*}
    \forall{x}(x^3 - 3x < 3 \rightarrow x < 10)
\end{align*}
No free variables. \\
\textbf{Solution 3 (d)} \\
\begin{align*}
    (\exists{x}(x^2 + 5x = w) \wedge \exists{y}(4 - y^2 = 2)) \rightarrow -10 < w < 10
\end{align*}
Free variable is $w$.

\begin{tcolorbox}[title=Problem 5, breakable]
Translate the following statements into idiomatic mathematical English. \\
(a) $\forall{x}[(P(x) \wedge \neg(x = 2)) \rightarrow O(x)]$, where $P(x)$ means ``x
is a prime number'' and $O(x)$ means ``x is odd''. \\
(b) $\exists{x}[P(x) \wedge \forall{y}(P(y) \rightarrow y \leq x)]$, where $P(x)$ means
``x is a perfect number''.
\end{tcolorbox}

\textbf{Solution 5 (a)} \\
For all $x$, if $x$ is a prime number and $x$ is not equal to $2$ then
$x$ is odd. \\
\textbf{Solution 5 (b)} \\
There exists $x$ such that $x$ is a perfect number and
for all $y$, if $y$ is a perfect number then $y$ is less than 
or equal to $x$.

\textbf{Solution 5 (a)}

\begin{tcolorbox}[title=Problem 8, breakable]
Are these statements true or false? The universe of discourse is $\mathbb{N}$. \\
(a) $\forall{x}\exists{y}(2x - y = 0)$. \\
(b) $\exists{y}\forall{x}(2x - y = 0)$. \\
(c) $\forall{x}\exists{y}(x - 2y = 0)$. \\
(d) $\forall{x}(x < 10 \rightarrow \forall{y}(y < x \rightarrow y < 9))$. \\
(e) $\exists{y}\exists{z}(y + z = 100)$. \\
(f) $\forall{x}\exists{y}(y > x \wedge \exists{z}(y + z = 100))$
\end{tcolorbox}

\textbf{Solution 8 (a)} \\
True. \\
\textbf{Solution 8 (b)} \\
False. \\
\textbf{Solution 8 (c)} \\
False. \\
\textbf{Solution 8 (d)} \\
True. \\
\textbf{Solution 8 (e)} \\
True. \\
\textbf{Solution 8 (f)} \\
False.

\begin{tcolorbox}[title=Problem 9, breakable]
Same excersize as $8$ but with $\mathbb{R}$ as the universe of discourse.
\end{tcolorbox}

\textbf{Solution 9 (a)} \\
True. \\
\textbf{Solution 9 (b)} \\
False. \\
\textbf{Solution 9 (c)} \\
True. \\
\textbf{Solution 9 (d)} \\
False. \\
\textbf{Solution 9 (e)} \\
True. \\
\textbf{Solution 9 (f)} \\
True. 


\begin{tcolorbox}[title=Problem 10, breakable]
Same excersize as $8$ but with $\mathbb{Z}$ as the universe of discourse.
\end{tcolorbox}

\textbf{Solution 9 (a)} \\
True \\
\textbf{Solution 9 (b)} \\
False \\
\textbf{Solution 9 (c)} \\
False \\
\textbf{Solution 9 (d)} \\
True \\
\textbf{Solution 9 (e)} \\
True \\
\textbf{Solution 9 (f)} \\
True

\subsection{Equivalences Involving Quantifiers}

\begin{tcolorbox}[title=Problem 1, breakable]
    Negate these statements and then reexpress the results as equivalent positive statements.
    (See Example $2.2.1$.) \\
    (a) Everyone who is majoring in math has a friend who needs helps with his or her homework. \\
    (b) Everyone has a roomate who dislikes everyone. \\
    (c) $A \cup B \subseteq C \setminus D$ \\
    (d) $\exists{x}\forall{x}[y > x \rightarrow \exists{z}(z^2 + 5z = y)]$
\end{tcolorbox}

\textbf{Solution 1(a)} \\
Let $M(x)$ mean ``$x$ is majoring in math''. \\
Let $F(x, y)$ mean ``$x$ and $y$ are friends''. \\
Let $H(x)$ mean ``$x$ needs help with his or her homework''. \\
Original statement means $\forall{x}(M(x) \rightarrow \exists{y}(F(x, y) \wedge H(y)))$.
\begin{align*}
& \neg \forall{x}(M(x) \rightarrow \exists{y}(F(x, y) \wedge H(y))) && \\
&\leftrightarrow    \exists{x} \neg (M(x) \rightarrow \exists{y}(F(x, y) \wedge H(y))) && \\
&\leftrightarrow    \exists{x} \neg (\neg M(x) \vee \exists{y}(F(x, y) \wedge H(y))) && \\
&\leftrightarrow    \exists{x} (M(x) \wedge \neg\exists{y}(F(x, y) \wedge H(y))) && \\
&\leftrightarrow    \exists{x} (M(x) \wedge \forall{y}\neg(F(x, y) \wedge H(y))) && \\
&\leftrightarrow    \exists{x} (M(x) \wedge \forall{y}(\neg F(x, y) \vee \neg H(y))) && \\
&\leftrightarrow    \exists{x} (M(x) \wedge \forall{y}(F(x, y) \rightarrow \neg H(y)))
\end{align*}
There is a math student for which all the students friends do not need help with their homework. \\
\textbf{Solution 1(b)} \\
Let $R(x, y)$ mean ``$x$ and $y$ are roomates''. \\
Let $D(x, y)$ mean ``$x$ dislikes $y$''. \\
Original statement means $\forall{x}\exists{y}(R(x, y) \wedge \forall{z}D(y, z))$.
\begin{align*}
&    \neg \forall{x}\exists{y}(R(x, y) \wedge \forall{z}D(y, z)) && \\
&    \leftrightarrow    \exists{x}\neg\exists{y}(R(x, y) \wedge \forall{z}D(y, z)) && \\
&    \leftrightarrow    \exists{x}\forall{y}\neg(R(x, y) \wedge \forall{z}D(y, z)) && \\
&    \leftrightarrow    \exists{x}\forall{y}(\neg R(x, y) \vee \neg\forall{z}D(y, z)) && \\
&    \leftrightarrow    \exists{x}\forall{y}(\neg R(x, y) \vee \exists{z}\neg D(y, z))
\end{align*}
There exists a student for which everyone is not his roomate or 
he has a roomate and that roomate likes someone. \\
\textbf{Solution 1(c)} \\
Original statement means $\forall{x}(x \in A \cup B \rightarrow C \setminus D)$.
\begin{align*}
&    \neg \forall{x}(x \in A \cup B \rightarrow x \in C \setminus D) && \\
&    \leftrightarrow    \exists{x}\neg(\neg(x \in A \cup B) \vee x\ in C \setminus D) && \\
&    \leftrightarrow    \exists{x}((x \in A \cup B) \wedge \neg(x \in C \setminus D)) && \\
&    \leftrightarrow    \exists{x}((x \in A \cup B) \wedge \neg(x \in C \wedge x \not \in D)) && \\
&    \leftrightarrow    \exists{x}((x \in A \cup B) \wedge (x \not \in C \vee x \in D))
\end{align*}
There exists $x$ in set $A$ or $B$ that is either not in $C$ or is in $D$. \\
\textbf{Solution 1(d)} \\
\begin{align*}
&    \neg \exists{x}\forall{y}[y > x \rightarrow \exists{z}(z^2 + 5z = y)] && \\
&    \leftrightarrow    \forall{x}\neg\forall{y}[y > x \rightarrow \exists{z}(z^2 + 5z = y)] && \\
&    \leftrightarrow   \forall{x}\exists{y}\neg[y > x \rightarrow \exists{z}(z^2 + 5z = y)] && \\
&    \leftrightarrow   \forall{x}\exists{y}\neg[\neg(y > x) \vee \exists{z}(z^2 + 5z = y)] && \\
&    \leftrightarrow    \forall{x}\exists{y}[y > x \wedge \neg\exists{z}(z^2 + 5z = y)] && \\
&    \leftrightarrow    \forall{x}\exists{y}[y > x \wedge \exists{z}\neg(z^2 + 5z = y)] && \\
&    \leftrightarrow    \forall{x}\exists{y}[y > x \wedge \exists{z}(z^2 + 5z \not = y)] 
\end{align*}
For all $x$ there exists $y$ such that $y$ is greater than $x$ and 
there is a number $z$ such that $z^2 + 5z$ is not equal to $y$.

\begin{tcolorbox}[title=Problem 2, breakable]
    Negate these statements and then reexpress the results as equivalent positive statements.
    (See Examples $2.2.1$.) \\
    (a) There is someone in the freshman class who doesn't have a room-mate. \\
    (b) Everyone likes someone, but no ones likes everyone. \\
    (c) $\forall{a \in A} \exists{b \in B} (a \in C \leftrightarrow b \in C)$. \\
    (d) $\forall{y > 0 \exists{x}}(ax^2 + bx + c = y)$.
\end{tcolorbox}

\textbf{Solution 2(a)} \\
Let $F(x)$ mean ``$x$ is a freshman''. \\
Let $R(x)$ mean ``$x$ has a room-mate''. \\
Original statement means $\exists{x}(F(x) \wedge \neg R(x))$.
\begin{align*}
    & \neg \exists{x}(F(x) \wedge \neg R(x)) && \\
    &\leftrightarrow \forall{x}\neg(F(x) \wedge \neg R(x)) && \\
    &\leftrightarrow \forall{x}(\neg F(x) \vee R(x))
\end{align*}
For all people either they are not a freshmen or they have a roomate. \\
\textbf{Solution 2(b)} \\
Let $L(x, y)$ mean ``$x$ likes $y$''. \\
Orignal statement means $\forall(x)(\exists{y}L(x, y) \wedge \exists{z} \neg L(x, z))$.
\begin{align*}
    & \neg \forall x(\exists{y}L(x, y) \wedge \exists{z} \neg L(x, z)) && \\
    & \leftrightarrow \exists x\neg(\exists{y}L(x, y) \wedge \exists{z} \neg L(x, z)) && \\
    & \leftrightarrow \exists x(\neg\exists{y}L(x, y) \vee \neg\exists{z} \neg L(x, z)) && \\
    & \leftrightarrow \exists x(\forall{y}\neg L(x, y) \vee \forall{z} L(x, z))
\end{align*}
There is someone who either everyone dislikes or everyone likes. \\
\textbf{Solution 2(c)} \\
\begin{align*}
    & \neg \forall{a \in A} \exists{b \in B} (a \in C \leftrightarrow b \in C) && \\
    & \leftrightarrow \exists{a \in A} \neg \exists{b \in B} (a \in C \leftrightarrow b \in C) && \\
    & \leftrightarrow \exists{a \in A} \forall{b \in B} \neg (a \in C \leftrightarrow b \in C) && \\
    & \leftrightarrow \exists{a \in A} \forall{b \in B} 
        \neg ((a \in C \rightarrow b \in C) \wedge (b \in C \rightarrow a \in C)) && \\
    & \leftrightarrow \exists{a \in A} \forall{b \in B} 
        (\neg(a \in C \rightarrow b \in C) \vee \neg(b \in C \rightarrow a \in C)) && \\
    & \leftrightarrow \exists{a \in A} \forall{b \in B} 
        (\neg(\neg(a \in C) \vee b \in C) \vee \neg(\neg(b \in C) \vee a \in C)) && \\
    & \leftrightarrow \exists{a \in A} \forall{b \in B} 
        ((a \in C) \wedge \neg(b \in C)) \vee ((b \in C) \wedge \neg(a \in C)) && \\
    & \leftrightarrow \exists{a \in A} \forall{b \in B} 
        (a \in C \wedge b \not \in C) \vee (b \in C \wedge a \not \in C) 
\end{align*}
There exists $a$ in set $A$ for which all $b$ in set $B$ either $a$ in set $C$ and $b$ is not
in set $C$ or $a$ is not in $C$ and $b$ is in $C$. \\
\textbf{Solution 2(d)} \\
\begin{align*}
    & \neg \forall{y > 0} \exists{x}(ax^2 + bx + c = y) && \\
    & \leftrightarrow \exists{y > 0} \neg \exists{x}(ax^2 + bx + c = y) && \\
    & \leftrightarrow \exists{y > 0} \forall{x}\neg(ax^2 + bx + c = y) && \\
    & \leftrightarrow \exists{y > 0} \forall{x}(ax^2 + bx + c \not = y)
\end{align*}
There exists a number $y$ such that for all $x$, $ax^2 + bx + c$ is not equal to $y$.

\begin{tcolorbox}[title=Problem 3, breakable]
    Are these statements true or false? The universe of discourse is $\mathbb{N}$. \\
    (a) $\forall{x}(x < 7 \rightarrow \exists{a}\exists{b}\exists{c}(a^2 + b^2 + c^2 = x))$. \\
    (b) $\exists{!x}(x^2 + 3 = 4x)$. \\
    (c) $\exists{!x}(x^2 = 4x + 5)$. \\
    (d) $\exists{x}\exists{y}(x^2 = 4x + 5 \wedge y^2 = 4y + 5)$.
\end{tcolorbox}

\textbf{Solution 3(a)} \\
True. \\
\textbf{Solution 3(b)} \\
False. \\
\textbf{Solution 3(c)} \\
True. \\
\textbf{Solution 3(d)} \\
True

\begin{tcolorbox}[title=Problem 4, breakable]
    Show that the second quantifier negation law, which says that $\neg\forall{x}P(x)$ is
    equivalent to $\exists{x}\neg P(x)$, can be derived from the first, which says that
    $\neg \exists{x}P(x)$ is equivalent to $\forall{x}\neg P(x)$. (Hint: Use the double 
    negation law.)
\end{tcolorbox}

\begin{proof}
\begin{align*}
    \neg\forall{x}P(x) &\equiv \neg\forall{x}\neg(\neg P(x)) && \\
    &\equiv \neg\neg\exists{x}(\neg P(x)) && \quad \text{first quantifier negation law} \\
    &\equiv \exists{x}\neg P(x)
\end{align*}
\end{proof}

\begin{tcolorbox}[title=Problem 5, breakable]   
    Show that $\neg \exists{x} \in A P(x)$ is equivalent to $\forall{x} \in A \neg P(x)$.
\end{tcolorbox}

\begin{proof}
    \begin{align*}
        \neg \exists{x} \in A P(x) &= \forall{x} \in A \neg P(x) && \quad \text{second quantifier negation law}
    \end{align*}
\end{proof}

\begin{tcolorbox}[title=Problem 6, breakable]
    Show that the existential quantifier distributes over disjunction. In other words, 
    show that $\exists{x}(P(x) \vee Q(x))$ is equivalent to 
    $\exists{x}P(x) \vee \exists{x}Q(x)$. (Hint: Use the fact, discussed in this section, that 
    the universal quantifier distributes over conjunction.)
\end{tcolorbox}

\begin{proof}
    \begin{align*}
        \exists{x}(P(x) \vee Q(x)) &\equiv \neg\neg \exists{x}(P(x) \vee Q(x)) && \\
        &\equiv \neg \forall{x}\neg(P(x) \vee Q(x)) && \\
        &\equiv \neg \forall{x}(\neg P(x) \wedge \neg Q(x)) && \\
        &\equiv \neg (\forall{x}\neg P(x) \wedge \forall{x}\neg Q(x)) && \\
        &\equiv (\neg \forall{x}\neg P(x) \vee \neg \forall{x}\neg Q(x)) && \\
        &\equiv (\exists{x}\neg\neg P(x) \vee \exists{x}\neg\neg Q(x)) && \\
        &\equiv \exists{x} P(x) \vee \exists{x} Q(x)
    \end{align*}
\end{proof}

\begin{tcolorbox}[title=Problem 7, breakable]
    Show that $\exists{x}(P(x) \rightarrow Q(x))$ is equivalent to 
    $\forall{x}P(x) \rightarrow \exists{x}Q(x)$.
\end{tcolorbox}

\begin{proof}
    \begin{align*}
        \exists{x}(P(x) \rightarrow Q(x)) &\equiv \exists{x}(\neg P(x) \vee Q(x)) && \\
        &\equiv (\exists{x}\neg P(x) \vee \exists{x}Q(x)) && \quad \text{from prob. $6$} \\
        &\equiv (\neg \forall{x} P(x) \vee \exists{x}Q(x)) && \\
        &\equiv \forall{x} P(x) \rightarrow \exists{x}Q(x)
    \end{align*}
\end{proof}


\begin{tcolorbox}[title=Problem 8, breakable]
    Show that $(\forall{x} \in AP(x)) \wedge (\forall{x} \in B P(x))$ is equivalent 
    $\forall{x} \in (A \cup B)P(x)$. (Hint: Start by writing out the meanings of
    the bounded quantifiers in terms of unbounded quantifiers.)
\end{tcolorbox}

\begin{proof}
    \begin{align*}
        (\forall{x} \in AP(x)) \wedge (\forall{x} \in B P(x)) 
            &\equiv \forall{x}(x \in A \rightarrow P(x)) \wedge \forall{x}(x \in B \rightarrow P(x))  && \\
        &\equiv \forall{x}((x \in A \rightarrow P(x)) \wedge (x \in B \rightarrow P(x)))  && \\
        &\equiv \forall{x}((\neg(x \in A) \vee P(x)) \wedge (\neg(x \in B) \vee P(x)))  && \\
        &\equiv \forall{x}((\neg(x \in A) \wedge (\neg(x \in B)) \vee P(x)))  && \\
        &\equiv \forall{x}(\neg(x \in A \vee x \in B) \vee P(x))  && \\
        &\equiv \forall{x}(x \in A \vee x \in B \rightarrow P(x))  && \\
        &\equiv \forall{x}(x \in A \vee B \rightarrow P(x))  && \\
        &\equiv \forall{x}(x \in A \vee B)P(x)  && \\
        &\equiv \forall{x} \in (A \cup B)P(x)
    \end{align*}
\end{proof}

\begin{tcolorbox}[title=Problem 9, breakable]
    Is $\forall{x}(P(x) \vee Q(x))$ equivalent to $\forall{x}P(x) \vee \forall{x} Q(x)$? 
    Explain. (Hint: Try assigning meanings to $P(x)$ and $Q(x)$.)
\end{tcolorbox}

\textbf{Solution}
No they are not equal. The first says that for every $x$, $P$ or $Q$ is true. In contrast
the second statement means $P$ is true for all $x$ or $Q$ is true for all $x$. If we 
let $P$ mean ``$x$ goes to the store'' and Q mean ``$x$ goes to the gym'' then the first
statement means for each person they go to the store or they go to the gym. The second
statement means all people go the store or all people go to the gym.

\begin{tcolorbox}[title=Problem 10, breakable]
    (a) Show that $\exists{x} \in A P(x) \vee \exists{x} \in B P(x)$ is equivalent to 
    $\exists{x} \in (A \cup B) P(x)$. \\
    (b) Is $\exists{x} \in A P(x) \wedge \exists{x} \in B P(x)$ equivalent to 
    $\exists{x} \in (A \cap B) P(x)$? Explain.
\end{tcolorbox}

\begin{proof}
    \begin{align*}
        \exists{x} \in A P(x) \vee \exists{x} \in B P(x) 
            &\equiv \exists{x} (x \in A P(x) \vee x \in B P(x)) && \quad \text {prev prob} \\
            &\equiv \exists{x} ((x \in A  \vee x \in B) \wedge P(x)) && \quad \text {} \\
            &\equiv \exists{x} (x \in (A \cup B) \wedge P(x)) && \quad \text {} \\
            &\equiv \exists{x}\in (A \cup B) P(x)
    \end{align*}
\end{proof}

\textbf{Solution 10(b)}
False. The first statement means that there exists $x$ in set $A$ for which $P(x)$ is true
and there exists another $x$ possibly distinct from the former in set $B$ for which $P(x)$
is true. The second statement says that there exists $x$ in the set $A$ and is also in the 
set $B$ for which $P(x)$ is true.

\begin{tcolorbox}[title=Problem 11, breakable]
    Show that the statements $A \subseteq B$ and $A \setminus B = \emptyset$ are equivalent
    by writing each in logical symbols and then showing that the resulting formulas are 
    equivalent.
\end{tcolorbox}

\begin{proof}
    Lhs is equavalent to $\forall{x}(x \in A \rightarrow x \in B)$ \\
    For the rhs: \\
    \begin{align*}
        \neg \exists{x}(x \in A \wedge x \not \in B)
            &\equiv \forall{x} \neg(x \in A \wedge x \not \in B) && \\
            &\equiv \forall{x} (x \not \in A \vee x \in B) && \\
            &\equiv \forall{x} (x \in A \rightarrow x \in B) 
    \end{align*}
\end{proof}

\begin{tcolorbox}[title=Problem 12, breakable]
    Show that the statements $C \subseteq A \cup B$ and $C \setminus A \subseteq B$ are
    equivalent writing each in logical symbols and then showing that the resulting formulas 
    are equivalent.
\end{tcolorbox}

\begin{proof}
    The lhs means $\forall{x}(x \in C \rightarrow x \in A \cup B)$. \\
    For the rhs
    \begin{align*}
        \forall{x}(x \in C \setminus A \rightarrow x \in B) 
            &\equiv \forall{x}(\neg(x \in C \setminus A) \rightarrow x \in B) && \\
            &\equiv \forall{x}(\neg(x \in C \wedge x \not \in A) \vee x \in B) && \\
            &\equiv \forall{x}((x \not \in C \vee x \in A) \vee x \in B) && \\
            &\equiv \forall{x}(x \not \in C \vee (x \in A \vee x \in B)) && \\
            &\equiv \forall{x}(\neg(x \in C) \vee (x \in A \cup B)) && \\
            &\equiv \forall{x}(x \in C \rightarrow x \in A \cup B)
    \end{align*}
\end{proof}

\begin{tcolorbox}[title=Problem 13, breakable]
    (a) Show that the statements $A \subseteq B$ and $A \cup B = B$ are equivalent by
    writing each in logical symbols and then showing that the resulting formulas are 
    equivalent. (Hint: You may find excersize $11$ from Section $1.5$ useful.)
\end{tcolorbox}

\begin{proof}
    First note sec. $1$ prob. $11$ says $x \in P \vee x \in Q \leftrightarrow x \in Q \equiv x \in P \rightarrow x \in Q$. \\
    The lhs means $\forall{x}(x \in A \rightarrow x \in B)$. \\
    For the rhs
    \begin{align*}
        \forall{x}(x \in A \vee x \in B \leftrightarrow x \in B)
            &\equiv \forall{x}(x \in A \rightarrow x \in B) \quad \text{from sec. $1$, prob. $11$} &&
    \end{align*}
\end{proof}

\begin{tcolorbox}[title=Problem 14, breakable]
    Show that the statements $A \cap B = \emptyset$ and $A \setminus B = A$ are equivalent.
\end{tcolorbox}

\begin{proof}
    The lhs means $\neg \exists{x}(x \in A \wedge x \in B)$ \\
    For the rhs
    \begin{align*}
        \forall{x}(x \in (A \setminus B) \leftrightarrow x \in A)
            &\equiv \forall{x}((x \in (A \setminus B)) \rightarrow x \in A) 
                \wedge (x \in A \rightarrow (x \in (A \setminus B))) && \\
            &\equiv \forall{x}(\neg(x \in (A \setminus B)) \vee x \in A) 
                \wedge (\neg(x \in A) \vee (x \in (A \setminus B))) && \\
            &\equiv \forall{x}(\neg(x \in A \wedge x \not \in B) \vee x \in A) 
                \wedge (\neg(x \in A) \vee (x \in (A \setminus B))) && \\
             &\equiv \forall{x}(x \not \in A \vee x \in B \vee x \in A) 
                \wedge (\neg(x \in A) \vee (x \in (A \setminus B))) && \\
            &\equiv \forall{x}(\neg(x \in A) \vee (x \in (A \setminus B))) && \quad \text{tautology} \\
            &\equiv \forall{x}(x \not \in A \vee (x \in (A \setminus B))) && \\
            &\equiv \forall{x}(x \not \in A \vee  (x \in A \wedge x \not \in B)) && \\
            &\equiv \forall{x}((x \not \in A \vee  x \in A) \wedge (x \not \in A \vee x \not \in B)) && \\
            &\equiv \forall{x}(x \not \in A \vee x \not \in B) && \quad \text {tautology} \\
            &\equiv \forall{x}\neg (x \in A \wedge x \in B) && \\
            &\equiv \neg \exists{x}(x \in A \wedge x \in B) 
    \end{align*}
\end{proof}

\begin{tcolorbox}[title=Problem 15, breakable]
    Let $T(x, y)$ mean ``$x$ is a teacher of $y$'' What do the following statements mean?
    Under what circumstances would each one be true? Are any of them equivalent to each
    other? \\
    (a) $\exists{!yT(x, y)}$. \\
    (b) $\exists{x}\exists{!yT(x, y)}$. \\
    (c) $\exists{!x}\exists{y}T(x, y)$. \\
    (d) $\exists{y}\exists{!x}T(x, y)$. \\
    (e) $\exists{!x}\exists{!y}T(x, y)$. \\
    (f) $\exists{x}\exists{y}[T(x, y) 
        \wedge \neg \exists{u}\exists{v}(T(u, v) \wedge (u \not = x \vee v \not = y))]$.
\end{tcolorbox}

\textbf{Solution 15 (a)} \\
There is one student that $x$ is a teacher of. \\
It is true when the teacher has only one student. \\
\textbf{Solution 15 (b)} \\
There is a teacher that has one student. \\
It is true if any teacher has one student. \\
\textbf{Solution 15 (c)} \\
There is one teacher that has a student. \\
It is true when there is a single teacher than has any students. \\
\textbf{Solution 15 (d)} \\
There is a student that has only one teacher. \\
It is true if any student has one teacher. \\
\textbf{Solution 15 (e)} \\
There is one student that has one teacher. \\
It is true if there is one student with one teacher. \\
\textbf{Solution 15 (f)} \\
There is a student and a teacher and there isn't another teacher with any other student
other than the former. \\
It is true if there is a student and a teacher and there isn't another teacher with any other student
other than the former. 

None are equivalent.

\subsection{More Operations On Sets}

\begin{tcolorbox}[title=Problem 1, breakable]
    Analyze the logical forms of the following statements. You may use the symbols
    $\in$, $\not \in$, $=$, $\not =$, $\wedge$, $\vee$, $\rightarrow$,
    $\leftrightarrow$, and $\exists$ in your answers, but not
    $\subseteq$, $\not \subseteq$,  $\mathcal{P}$, $\cap$, $\cup$,
    $\setminus$, $\{$, $\}$, or $\neg$. (Thus you must write out the definitions of some
    theory notation, and you must use equivalences to get rid of any occurences
    of $\neg$). \\
    (a) $\mathcal{F} \subseteq \mathcal{P}{(A)}$ \\
    (b) $A \subseteq \{2n + 1 | n \in \mathbb{N}\}$ \\
    (c) $\{n^2 + n + 1 | n \in \mathbb{N}\} \subseteq \{2n + 1 | n \in \mathbb{N}\}$ \\
    (d) $\mathcal{P}(\bigcup_{i \in I} A_i) \not \subseteq \bigcup_{i \in I} \mathcal{P}(A_i)$
\end{tcolorbox}

\textbf{Solution 1 (a)} \\
    \begin{align*}
        \mathcal{F} \subseteq \mathcal{P}{(A)} 
        &= \forall{x}(x \in \mathcal{F} \rightarrow x \in \mathcal{P}(A)) && \\
        &= \forall{x}(x \in \mathcal{F} 
            \rightarrow \forall{y}(y \in x \rightarrow y \in A))
    \end{align*}
\textbf{Solution 1 (b)} \\
    \begin{align*}
        A \subseteq \{2n + 1 | n \in \mathbb{N}\}
            &= \forall{x}(x \in A \rightarrow x \in \{2n + 1 | n \in \mathbb{N}\}) && \\
        &= \forall{x}(x \in A \rightarrow \exists{n}\in\mathbb{N}(x = 2n + 1))
    \end{align*}
\textbf{Solution 1 (c)} \\
    \begin{align*}
        \{n^2 + n + 1 | n \in \mathbb{N}\} \subseteq \{2n + 1 | n \in \mathbb{N}\}
            &= \forall{x}(x \in \{n^2 + n + 1 | n \in \mathbb{N}\} 
                \rightarrow x \in \{2n + 1 | n \in \mathbb{N}\}) && \\
        &= \forall{x}(\exists{n}\in\mathbb{N}(x = n^2 + n + 1) 
            \rightarrow \exists{n}\in\mathbb{N}(x = 2n + 1))
    \end{align*}
\textbf{Solution 1 (d)}
    \begin{align*}
        \mathcal{P}\left(\bigcup_{i \in I} A_i\right) \not \subseteq \bigcup_{i \in I} \mathcal{P}(A_i)
            &=  \exists{x}\left(x \in \mathcal{P}\left(\bigcup_{i \in I} A_i\right) 
                \wedge \neg\left(x \in \bigcup_{i \in I} \mathcal{P}(A_i)\right)\right) && \\
                &= \exists{x} \left( 
                    \forall{y} \left( y \in x \rightarrow y \in \mathcal{P}\left( \bigcup_{i \in I} A_i \right) \right)
                    \wedge 
                    \neg\left( x \in \bigcup_{i \in I} \mathcal{P}(A_i) \right)
                  \right) && \\              
            &=  \exists{x}\left((\forall{y}(y \in x \rightarrow \exists{i \in I}(y \in A_i)))
                \wedge \neg\left(\exists{i \in I}(x \in \mathcal{P}(A_i))\right)\right) && \\
            &=  \exists{x}\left((\forall{y}(y \in x \rightarrow \exists{i \in I}(y \in A_i)))
                \wedge \neg\left(\exists{i \in I}(\forall{y}(y \in x \rightarrow y \in A_i))\right)\right) && \\
            &=  \exists{x}\left((\forall{y}(y \in x \rightarrow \exists{i \in I}(y \in A_i)))
                \wedge \left(\forall{i \in I}(\exists{y}(y \in x \wedge y \not \in A_i))\right)\right) 
    \end{align*}

\begin{tcolorbox}[title=Problem 2, breakable]
    Analyze the logical forms of the following statements. You may use the symbols
    $\in$, $\not \in$, $=$, $\not =$, $\wedge$, $\vee$, $\rightarrow$,
    $\leftrightarrow$, and $\exists$ in your answers, but not
    $\subseteq$, $\not \subseteq$,  $\mathcal{P}$, $\cap$, $\cup$,
    $\setminus$, $\{$, $\}$, or $\neg$. (Thus you must write out the definitions of some
    theory notation, and you must use equivalences to get rid of any occurences
    of $\neg$). \\
    (a) $x \in \bigcup \mathcal{F} \setminus \bigcup \mathcal{G}$ \\
    (b) $\{x \in B | x \not \in C\} \in \mathcal{P}(A)$ \\
    (c) $x \in \bigcap_{i \in I}(A_i \cup B_i)$ \\
    (d) $x \in (\bigcap_{i \in I} A_i) \cup (\bigcap_{i \in I} B_i)$
\end{tcolorbox}

\textbf{Solution 2(a)} \\
    \begin{align*}
        x \in\bigcup\mathcal{F} \setminus\bigcup\mathcal{G}
            &\equiv x \in \bigcup\mathcal{F} \wedge \neg\left(x \in \bigcup \mathcal{G}\right) && \\
        &\equiv \exists{A}(A \in F \wedge x \in A) \wedge \neg(\exists{A}(A \in G \wedge x \in A)) && \\
        &\equiv \exists{A}(A \in F \wedge x \in A) \wedge \forall{A}\neg(A \in G \wedge x \in A) && \\
        &\equiv \exists{A}(A \in F \wedge x \in A) \wedge \forall{A}(A \not \in G \vee x \not \in A)
    \end{align*}
\textbf{Solution 2(b)} \\
    \begin{align*}
        \{x \in \mathcal{B} | x \not \in C\} \in \mathcal{P}(A)
            &\equiv \forall{y}(y \in \{x \in \mathcal{B} | x \not \in C\} \rightarrow y \in A) && \\
        &\equiv \forall{y}((y \in\mathcal{B} \wedge \not\in C) \rightarrow y \in A) &&
    \end{align*}
\textbf{Solution 2(c)} \\
    \begin{align*}
        x \in \bigcap_{i \in I}(A_i \cup B_i) 
            &\equiv \forall{i \in I}(x \in (A_i \cup B_i)) && \\
        &\equiv \forall{i \in I}(x \in A_i \vee x \in B_i) && \\
    \end{align*}
\textbf{Solution 2(d)} \\
    \begin{align*}
        x \in \left(\bigcap_{i \in I} A_i\right) \cup \left(\bigcap_{i \in I} B_i\right)
            &\equiv x \in \left(\bigcap_{i \in I} A_i\right) \vee x \in \left(\bigcap_{i \in I} B_i\right) && \\
        &\equiv \forall{i \in I}(x \in A_i) \vee \forall{i \in I}(x \in B_i) && \\
    \end{align*}

\begin{tcolorbox}[title=Problem 3, breakable]
    We've seen that $\mathcal{P}(\emptyset) = {\emptyset}$, and $\{\emptyset\} \not = \emptyset$.
    What is $\mathcal{P}(\{\emptyset\})$?
\end{tcolorbox}

\textbf{Solution}
$\mathcal{P}(\{\emptyset\}) \equiv \{\emptyset\text{, } \{\emptyset\}\}$

\begin{tcolorbox}[title=Problem 6, breakable]
    Let $I = \{2, 3, 4, 5\}$, and for each $i \in I$ let $A_i = \{i, i + 1, i - 1, 2i\}$. \\
    (a) List the elements of all the sets $A_i$, for each $i \in I$. \\
    (b) Find $\bigcap_{i \in I} A_i$ and $\bigcup_{i \in I} A_i$.
\end{tcolorbox}

\textbf{Solution 6 (a)}
    \begin{align*}
        A_2 &= \{2\text{, }3\text{, }1\text{, }4\} && \\
        A_3 &= \{3\text{, }4\text{, }2\text{, }6\} && \\
        A_4 &= \{4\text{, }5\text{, }3\text{, }8\} && \\
        A_5 &= \{5\text{, }6\text{, }4\text{, }10\} && \\
    \end{align*}
\textbf{Solution 6 (b)}
    \begin{align*}
        A_2 \cap A_3 &= \{2\text{, }3\text{, }4\} && \\
        A_4 \cap A_5 &= \{4\text{, }5\} && \\
        \bigcap_{i \in I} A_i &= \{2\text{, }3\text{, }4\} \cap \{4\text{, }5\} = \{4\}
    \end{align*}
    \begin{align*}
        A_2 \cup A_3 &= \{1\text{, }2\text{, }3\text{, }4\text{, }6\} && \\
        A_4 \cup A_5 &= \{3\text{, }4\text{, }5\text{, }6\text{, }8\text{, }10\} && \\
        \bigcup_{i \in I} A_i &= \{1\text{, }2\text{, }3\text{, }4\text{, }6\} 
            \cup \{3\text{, }4\text{, }5\text{, }6\text{, }8\text{, }10\}
            = \{1\text{, }2\text{, }3\text{, }4\text{, }5\text{, }6\text{, }8\text{, }10\}
    \end{align*}

\begin{tcolorbox}[title=Problem 8, breakable]
    Let $I = \{2, 3\}$, and for each $i \in I$ let $A_i = \{i, 2i\}$ and $B_i = \{i, i + 1\}$. \\
    (a) List the elements of the sets $A_i$ and $B_i$ for $i \in I$. \\
    (b) Find $\bigcap_{i \in I}(A_i \cup B_i)$ 
        and $(\bigcap_{i \in I} A_i)\cup(\bigcap_{i \in I} B_i)$. \\
    (c) In parts (c) and (d) of excersize $2$ you analyzed the statements
        $x \in \bigcap_{i \in I}(A_i \cup B_i)$ and 
        $x \in (\bigcap_{i \in I} A_i) \cup (\bigcap_{i \in I} B_i)$.
        What can you conclude from your answer to part (b) about whether or not these statements
        are equivalent?
\end{tcolorbox}

\textbf{Solution 8 (a)} \\
$A_2 = \{2, 4\}$ and $A_3 = \{3, 6\}$ \\
$B_2 = \{2, 3\}$ and $B_3 = \{3, 4\}$ \\
\textbf{Solution 8 (b)} \\
Let $i = 2$, then $A_2 \cup B_2 = \{2, 3, 4\}$ \\
Let $i = 3$, then $A_3 \cup B_3 = \{3, 4, 6\}$ \\
$$\bigcap_{i \in I}(A_i \cup B_i) = \{2, 3, 4\} \cap \{3, 4, 6\} = \{3, 4\}$$ \\
$\bigcap_{i \in I}A_i = A_2 \cap A_3 = \{\}$ \\
$\bigcap_{i \in I}B_i = B_2 \cap B_3 = \{3\}$ \\
$$\left(\bigcap_{i \in I} A_i\right) \cup \left(\bigcap_{i \in I} B_i\right) = \{\} \cup \{3\} = \{3\}$$ \\
\textbf{Solution 8 (c)} \\
They are not equivalent.

\begin{tcolorbox}[title=Problem 9, breakable]
    (a) Analyze the logical forms of the statements $x \in \bigcup_{i \in I}(A_i \setminus B_i)$,
    $x \in (\bigcup_{i \in I}A_i) \setminus (\bigcup_{i \in I} B_i)$, and
    $x \in (\bigcup_{i \in I}A_i) \setminus (\bigcap_{i \in I} B_i)$. Do you think
    that any of these statements are equivalent to each other? \\
    (b) Let $I$, $A_i$, and $B_i$ be defined as in excersize $8$. Find $\bigcup_{i \in I}(A_i \setminus B_i)$,
    $(\bigcup_{i \in I}A_i) \setminus (\bigcup_{i \in I}B_i)$, and
    $(\bigcup_{i \in I}A_i) \setminus (\bigcap_{i \in I}B_i)$. Now do you think any of the statements 
    in part (a) are equivalent?
\end{tcolorbox}

\textbf{Solution 9 (a)}
\begin{align*}
    x \in \left(\bigcup_{i \in I}(A_i \setminus B_i)\right)
        &\equiv \exists{i \in I}(x \in (A_i \setminus B_i)) && \\
        &\equiv \exists{i \in I}(x \in A_i \wedge x \not \in B_i)
\end{align*}
\begin{align*}
    x\ in \left(\bigcup_{i \in I}A_i\right) \setminus \left(\bigcup_{i \in I} B_i\right)
        &\equiv x \in \left(\bigcup_{i \in I}A_i\right) \wedge \neg\left(x \in \left(\bigcup_{i \in I} B_i\right)\right) && \\
        &\equiv \exists{i \in I}(x \in A_i) \wedge \neg(\exists{i \in I}(x \in B_i)) && \\
        &\equiv \exists{i \in I}(x \in A_i) \wedge \forall{i \in I}(x \not \in B_i) 
\end{align*}
\begin{align*}
    x \in \left(\bigcup_{i \in I}A_i\right) \setminus \left(\bigcap_{i \in I} B_i\right) 
        &\equiv x \in \left(\bigcup_{i \in I} A_i\right) \wedge \neg\left(x \in \left(\bigcap_{i \in I} B_i\right)\right) && \\
        &\equiv \exists{i \in I}(x \in A_i) \wedge \neg\left(\forall{i \in I}(x \in B_i)\right) && \\
        &\equiv \exists{i \in I}(x \in A_i) \wedge \left(\exists{i \in I}(x \not \in B_i)\right) && \\
\end{align*}
I do not think any of the statements are equivalent.

\textbf{Solution 9 (b)}
\text{From prob. 8} \\
$I = \{2, 3\}$ \\
$A_2 = \{2, 4\}$ and $A_3 = \{3, 6\}$ \\
$B_2 = \{2, 3\}$ and $B_3 = \{3, 4\}$ \\
\[\bigcup_{i \in I}(A_i \setminus B_i) = \{4\} \cup \{6\} = \{4, 6\}\]
\[\left(\bigcup_{i \in I} A_i\right) \setminus \left(\bigcup_{i \in I} B_i\right)
    = \{2, 3, 4, 6\} \setminus \{2, 3, 4\} = \{6\}\]
\[\left(\bigcup_{i \in I} A_i\right) \setminus \left(\bigcap_{i \in I} B_i\right)
    = \{2, 3, 4, 6\} \setminus \{3\} = \{2, 4, 6\}\]
Still not equal.

\begin{tcolorbox}[title=Problem 10, breakable]
    Give an example of an index set $I$ and indexed families of sets $\{A_I | i \in I\}$
    and $\{B_i | i \in I\}$ such that 
    $\bigcup_{i \in I}(A_i \cap B_i) \not = (\bigcup_{i \in I} A_i) \cup (\bigcup_{i \in I} B_i)$.
\end{tcolorbox}

\textbf{Solution 10}
$I = \{1, 2\}$ \\
$A_1 = \{1\}$ \\
$A_2 = \{2\}$ \\
$B_1 = \{1\}$ \\
$B_2 = \emptyset$ \\
\[\bigcup_{i \in I}(A_i \cap B_i) = \{1\} \cap \emptyset = \emptyset\]
\[\left(U_{i \in I}A_i\right) = \{1, 2\}\]
\[\left(U_{i \in I}B_i\right) = \{1\}\]
\[\left(U_{i \in I}A_i\right) \cup \left(U_{i \in I}B_i\right) = \{1, 2\}\]
\[\therefore \bigcup_{i \in I}(A_i \cap B_i) \not = \left(U_{i \in I}A_i\right) \cup \left(U_{i \in I}B_i\right)\]

\begin{tcolorbox}[title=Problem 11, breakable]
    Show that for any sets $A$ and $B$, $\mathcal{P}(A \cap B) = \mathcal{P}(A) \cap \mathcal{P}(B)$,
    by showing that the statements $x \in \mathcal{P}(A \cap B)$ and 
    $x \in \mathcal{P}(A) \cap \mathcal{P}(B)$ are equivalent. (See Example 2.3.3)
\end{tcolorbox}

\begin{proof}
    \begin{align*}
        x \in \mathcal{P}(A \cap B)
            &\equiv \forall{y}(y \in x \rightarrow y \in A \cap B) && \\
            &\equiv \forall{y}(y \in x \rightarrow (y \in A \wedge y \in B)) 
    \end{align*}
    \begin{align*}
        x \in \mathcal{P}(A) \cap \mathcal{P}(B)
            &\equiv \forall{y}(y \in x \rightarrow y \in A) \wedge \forall{y}(y \in x \rightarrow y \in B) && \\
            &\equiv \forall{y}((\neg(y \in x) \vee y \in A) \wedge (\neg(y \in x) \vee y \in B)) && \\
            &\equiv \forall{y}(\neg(y \in x) \vee (y \in A \wedge y \in B)) && \\
            &\equiv \forall{y}(y \in x \rightarrow (y \in A \wedge y \in B)) && \\
    \end{align*}
    \[\therefore \mathcal{P}(A) \cap \mathcal{P}(B) = \mathcal{P}(A) \cap \mathcal{P}(B)\]
\end{proof}

\begin{tcolorbox}[title=Problem 12, breakable]
    Give examples of sets $A$ and $B$ for which 
    $\mathcal{P}(A \cup B) \not = \mathcal{P}(A) \cup \mathcal{P}(B)$.
\end{tcolorbox}

\textbf{Solution 12}
$A = \{1, 2\}$ \\
$B = \{1, 3\}$ \\
$A \cup B = \{1, 2, 3\}$
\[\mathcal{P}(A \cup B) = \{\{1, 2, 3\}, \{1, 2\}, \{1, 3\}, \{2, 3\}, \{1\}, \{2\}, \{3\}, \emptyset\}\]
\[\mathcal{P}(A) = \{\{1, 2\}, \{1\}, \{2\}, \emptyset\}\]
\[\mathcal{P}(B) = \{\{1, 3\}, \{1\}, \{3\}, \emptyset\}\]
\[\mathcal{P}(A) \cup \mathcal{P}(B) = \{\{1, 2\}, \{1, 3\}, \{2\}, \{3\}, \{1\}, \emptyset\}\]
\[\therefore \mathcal{P}(A \cup B) \not = \mathcal{P}(A) \cup \mathcal{P}(B)\]

\begin{tcolorbox}[title=Problem 13, breakable]
    Verify the following identities by writing out (using logical symbols) what it means
    for an object $x$ to be an element of each set and then using logical equivalences. \\
    (a) $\bigcup_{i \in I}(A_i \cup B_i) = (\bigcup_{i \in I}A_i) \cup (\bigcup_{i \in I} B_i)$ \\
    (b) $(\bigcap \mathcal{F}) \cap (\bigcap \mathcal{G}) = \bigcap(\mathcal{F} \cup \mathcal{G})$ \\
    (c) $\bigcap_{i \in I}(A_i \setminus B_i) = (\bigcap_{i \in I} A_i) \setminus (\bigcup_{i \in I} B_i)$
\end{tcolorbox}

\textbf{Solution 13 (a)}
\begin{align*}
    x \in \bigcup_{i \in I}(A_i \cup B_i)
        &\equiv \exists{i \in I}(x \in (A_i \vee B_i)) && \\
        &\equiv \exists{i \in I}(x \in A_i \vee x \in B_i) && \\
\end{align*}
\begin{align*}
    x \in \left(\bigcup_{i \in I}A_i\right) \cup \left(\bigcup_{i \in I} B_i\right)
        &\equiv x \in \left(\bigcup_{i \in I}A_i\right) \vee x \in \left(\bigcup_{i \in I} B_i\right) && \\
        &\equiv \exists{i \in I}(x \in A_i) \vee \exists{i \in I}(x \in B_i) && \\
        &\equiv \exists{i \in I}(x \in A_i \vee x \in B_i) && \\
\end{align*}
\[\therefore \bigcup_{i \in I}(A_i \cup B_i) = \left(\bigcup_{i \in I}A_i\right) \cup \left(\bigcup_{i \in I} B_i\right)\]
\textbf{Solution 13 (b)}
\begin{align*}
    x \in \left(\bigcap \mathcal{F}\right) \cap \left(\bigcap \mathcal{G}\right)
        &\equiv x \in \left(\bigcap \mathcal{F}\right) \wedge x \in \left(\bigcap \mathcal{G}\right) && \\
        &\equiv \forall{A \in F}(x \in A) \wedge \forall{A \in G}(x \in G) && \\
        &\equiv \forall{A \in (F \cup G)}(x \in A) \quad \text{prob. $8$ prev. sec.}
\end{align*}
\begin{align*}
    x \in \bigcap(\mathcal{F} \cup \mathcal{G})
        &\equiv \forall{A} \in (F \cup G)(x \in A)
\end{align*}
\[\therefore \left(\bigcap \mathcal{F}\right) \cap \left(\bigcap \mathcal{G}\right) = \bigcap(\mathcal{F} \cup \mathcal{G})\]
\textbf{Solution 13 (c)}
\begin{align*}
    x \in \bigcap_{i \in I}(A_i \setminus B_i)
        &\equiv \forall{i \in I}(x \in (A_i \setminus B_i)) && \\
        &\equiv \forall{i \in I}(x \in A_i \wedge x \not \in B_i)
\end{align*}
\begin{align*}
    x \in \left(\bigcap_{i \in I} A_i\right) \setminus \left(\bigcup_{i \in I} B_i\right) 
        &\equiv x \in \left(\bigcap_{i \in I} A_i\right) \wedge \neg(x \in \left(\bigcup_{i \in I} B_i\right)) && \\
        &\equiv \forall{i \in I}(x \in A_i) \wedge \neg(\exists{i \in I}(x \in B_i)) && \\
        &\equiv \forall{i \in I}(x \in A_i) \wedge \forall{i \in I}(x \not \in B_i) && \\
        &\equiv \forall{i \in I}(x \in A_i \wedge x \not \in B_i) 
\end{align*}
\[\therefore \bigcap_{i \in I}(A_i \setminus B_i) = \left(\bigcap_{i \in I} A_i\right) \setminus \left(\bigcup_{i \in I} B_i\right)\]

\begin{tcolorbox}[title=Problem 14, breakable]
    Sometimes each set in an indexed family of sets has two indices. For this problem,
    use the following definitions: $I = \{1, 2\}$, $J = \{3, 4\}$. For each
    $i \in I$ and $j \in J$, let $A_{i, j} = \{i, j, i + j\}$. Thus, for example,
    $A_{2, 3} = \{2, 3, 5\}$. \\
    (a) For each $j \in J$, let $B_j = \bigcup_{i \in I}A_{i, j} = A_{1, j} \cup A_{2, j}$. Find 
        $B_3$ and $B_4$. \\
    (b) Find $\bigcap_{j \in J} B_j$. (Note that, replacing $B_j$ with its definition, we could say that
        $\bigcap_{j \in J}B_j = \bigcap_{j \in J}(\bigcup_{i \in I} A_{i, j})$.)  \\
    (c) Find $\bigcup_{i \in I}(\bigcap_{j \in j} A_{i, j})$. (Hint: you may want to do this in two
        steps, coresponding to parts (a) and (b).) Are $\bigcap_{j \in J}(\bigcup_{i \in I} A_{i, j})$
        and $\bigcup_{i \in I}(\bigcap_{j \in J}) A_{i, j}$ equal? \\
    (d) Analyze the logical forms of the statements $x \in \bigcap_{j \in J}(\bigcup_{i \in I} A_{i, j})$ and 
        $x \in \bigcup_{i \in I}(\bigcap_{j \in J} A_{i, j})$. Are they equivalent?
\end{tcolorbox}

\textbf{Solution 14 (a)} \\
$B_j = \bigcup_{i \in I} A_{i, j} = A_{1, j} \cup A_{2, j}$ \\
$B_3 = A_{1, 3} \cup A_{2, 3}$ \\
$A_{1, 3} = \{1, 3, 4\}$ and
$A_{2, 3} = \{2, 3, 5\}$ \\
$B_3 = \{1, 2, 3, 4, 5\}$ \\ \\
$B_3 = A_{1, 4} \cup A_{2, 4}$ \\
$A_{1, 4} = \{1, 4, 5\}$ and
$A_{2, 4} = \{2, 4, 6\}$ \\
$B_4 = \{1, 2, 4, 5, 6\}$

\textbf{Solution 14 (b)} \\
$\bigcap_{j \in J} B_j = \bigcap_{j \in J}(\bigcup_{i \in I} A_{i, j})$ \\
$U_{i \in I} = A_{1, j} \cup A_{2, j}$ \\
$\bigcap_{j \in J}(A_{1, j} \cup A_{2, j}) = (A_{1, 3} \cup A_{2, 3}) \cap (A_{1, 4} \cup A_{2, 4})$ \\
$B_3 \cap B_4 = \{1, 2, 4, 5\}$

\textbf{Solution 14 (c)} \\
$\bigcup_{i \in I}(\bigcap_{j \in j} A_{i, j}) = \bigcup_{i \in I}(A_{i, 3} \cap A_{i, 4})$ \\
$\bigcup_{i \in I}(A_{i, 3} \cap A_{i, 4}) = (A_{1, 3} \cap A_{1, 4}) \cup (A_{2, 3} \cap A_{2, 4})$ \\
$(A_{1, 3} \cap A_{1, 4}) \cup (A_{2, 3} \cap A_{2, 4}) = \{1\} \cup \{2\} = \{1, 2\}$

\textbf{Solution 14 (d)} \\
\begin{align*}
    x \in \bigcap_{j \in J}\left(\bigcup_{i \in I} A_{i, j}\right)
        &\equiv \forall{j \in J}\left(x \in \left(\bigcup_{i \in I} A_{i, j}\right)\right) && \\
        &\equiv \forall{j \in J}(\exists{i \in I}(x \in A_{i, j}))
\end{align*}
\begin{align*}
    x \in \bigcup_{i \in I}\left(\bigcap_{j \in J} A_{i, j}\right)
        &\equiv \exists{i \in I}\left(x \in \left(\bigcap_{j \in J} A_{i, j}\right)\right) && \\
        &\equiv \exists{i \in I}(\forall{j \in J}(x \in A_{i, j}))
\end{align*}
\[\bigcap_{j \in J}\left(\bigcup_{i \in I} A_{i, j}\right) \not = \bigcup_{i \in I}\left(\bigcap_{j \in J} A_{i, j}\right)\]

\begin{tcolorbox}[title=Problem 15, breakable]
    (a) Show that if $\mathcal{F} = \emptyset$, then the statement $x \in \bigcup \mathcal{F}$ will be false
        no matter what $x$ is. It follows that $\bigcup \emptyset = \emptyset$. \\
    (b) Show that if $\mathcal{F} = \emptyset$, then the statement $x \in \bigcap\mathcal{F}$ will be true
        no matter what $x$ is. In a context in which it is clear what the universe of discourse $U$ is,
        we might therefore want to say that $\bigcap \emptyset = U$. However, this has the unfortunate 
        consequence that the notation $\bigcap \emptyset$ will mean different things in different contexts.
        Furthermore, when working with sets whose elements are sets, mathematicians  often do not use a 
        universe of discourse  at all. (For more on this, see the next excersize). For these reasons, some
        mathematicians consider the notation $\bigcap \emptyset$ to be meaningless. We will avoid this problem
        in this book by using the notation $\bigcap \mathcal{F}$ only in contexts in which we can be sure 
        that $\mathcal{F} \not = 0$.
\end{tcolorbox}

\begin{proof}
    Suppose for contradiction $x \in \bigcup F$ is true when $F = \emptyset$.
    Thus $\exists{A \in F}(x \in A)$, but for this to be true there must be at least
    one set in $F$ but $F = \emptyset$. \\ 
    $\therefore$ If $F = \emptyset$ then $x \in \bigcup F$ is always false.
\end{proof}
\begin{proof}
    Suppose for contradiction $x \in \bigcap F$ is false when $F = \emptyset$.
    Thus $\forall{A \in F}(x \in A)$, but for this to be false there must be a case in which $x$ 
    is not in a set within $F$ but $F = \emptyset$. \\
    $\therefore$ If $F = \emptyset$ then $x \in \bigcap F$ is always true.
\end{proof}

\begin{tcolorbox}[title=Problem 16, breakable]
    In Section $2.3$ we saw that a set can have other sets as elements. When discussing sets
    whose elements are sets, it might seem most natural to consider the universe of discourse
    to be the set of all sets. However, we will see in this problem, assuming there is such a
    set leads to contradictions. \\ \\
    Suppose $U$ were the set of all sets. Note that in particular $U$ is a set, so we should have 
    $U \in U$. This is not yet a contradictions; although most sets are not elements of themselves,
    perhaps some sets are elements of themselves, But it suggests that the sets in the universe 
    $U$ could be split into two categories: the unusual sets that, like $U$ itself, are elements of
    themselves, and the more typical sets that are not. Let $R$ be the set of sets in the second 
    category. In other words, $R = \{A \in U | A \not \in A\}$. This means for that for any set 
    $A$ in the universe of $U$, $A$ will be an element $R$ iff $A \not \in A$. In other words,
    we have $\forall{A} \in U (A \in R \leftrightarrow A \not \in A)$. \\
    (a) Show that applying this last fact to the set $R$ itself (in other words plugging in $R$ for $A$)
        leads to a contradiction. This contradiction was discovered by Bertrand Russell ($1872$ - $1970$)
        in $1901$, and is known as \emph{Russel's paradox}. \\
    (b) Think some more about the paradox in part (a). What do you think it tells us about sets?
\end{tcolorbox}

\begin{proof}
    Suppose for contradiction $\forall{A} \in U (A \in R \leftrightarrow A \not \in A)$.
    Then setting $R$ as $A$
    \begin{align*}
        R \in R \leftrightarrow R \not \in R
            &\equiv (R \in R \rightarrow R \not \in R) \wedge (R \not \in R \rightarrow R \in R) && \\
            &\equiv (R \not \in R \vee R \not \in R) \wedge (R \in R \vee R \in R) && \\
            &\equiv (R \not \in R) \wedge (R \in R)
    \end{align*}
    Thus we arrive at a contradiction. $\therefore$ You cannot construct $R = \{A \in U | A \not \in A\}$.
\end{proof}

\textbf{Solution 16 (b)}
With our current axioms not all\ set constuctions are valid. At the least not all self-referential
set constructions are valid because as we have seen it may lead to a contradiction.
