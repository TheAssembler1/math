\begin{tcolorbox}[title=Problem 2, breakable]
    (a) Why does every non-zero complex number have exactly 
        two square roots?

    (b) Given part a, check that the proof of the quadratic formula 
        obtained in Exercise 9.1 still holds in $\mathbb{C}[x]$.

    (c) Use the quadratic formula to compute the roots of the polynomials 
        $x^2 - (3 + 2i)x + (1 + 3i)$ and $x^2 - (1 + 3i)x + (-2 + 2i)$.
\end{tcolorbox} 

\textbf{Solution (a):}
The book states a non-zero complex number $\alpha = |\alpha|(\cos \theta + i \sin \theta) = |\alpha|e^{i \theta}$
has a square root $\beta = |\alpha|^{1/2}(\cos(\theta/2) + i \sin(\theta/2))$.
Any square root of $\alpha$ must have argument $\theta/2 \pmod{\pi}$, so the only square roots are $\beta$ and $-\beta$.

\textbf{Solution (b):} Checked, and it still holds.

\textbf{Solution (c):}
The roots of $x^2 - (3 + 2i)x + (1 + 3i)$ are
$\frac{-(-(3 + 2i)) \pm \sqrt{(-(3 + 2i))^2 - 4(1 + 3i)}}{2}$.
The roots of $x^2 - (1 + 3i)x + (-2 + 2i)$ are
$\frac{-(-(1 + 3i)) \pm \sqrt{(-(1 + 3i))^2 - 4(-2 + 2i)}}{2}$.

\begin{tcolorbox}[title=Problem 3, breakable]
    Give examples of two different polynomials in $\mathbb{Z}_5[x]$
    that are identical as functions over $\mathbb{Z}_5$.
    This shows that equality of polynomials in $F[x]$ cannot be 
    thought of as equality of the corresponding polynomial \emph{functions}.
    (See the Quick Exercise in Section 4.1 for $F = \mathbb{Z}_2$ case, and 
    Exercise 4.12 for the $F = \mathbb{Z}_3$.)
\end{tcolorbox} 

\textbf{Solution:} By Fermat's little Theorem in $\mathbb{Z}_5$
the polynomials $x^5$ and $x$ are equivalent for all $x$.

\begin{tcolorbox}[title=Problem 4, breakable]
    Consider the polynomial $f = x^3 + 3x^2 + 2x \in \mathbb{Z}_6[x]$.
    Show that this polynomial has more than three roots in $\mathbb{Z}_6$.
    Why doesn't this contradict the Root Theorem?
\end{tcolorbox} 

\textbf{Solution:} We can manually check that $x = 0, 1, 2, 3$ are roots.
The Root Theorem is about fields and $\mathbb{Z}_6$ is not a field.

\begin{tcolorbox}[title=Problem 8, breakable]
    Show that if $f$ is a polynomial with real coefficients and $\alpha = s + ti$
    is a root of $f$ in $\mathbb{C}$, then so is $\overline{\alpha} = s - ti$.
\end{tcolorbox} 

\begin{proof}
    Let $\alpha, \beta$ be complex numbers.
    We have the following two properties of the algebra of complex numbers 
    \begin{enumerate}
        \item $\overline{\alpha} + \overline{\beta} = \overline{\alpha + \beta}$.
        \item $\overline{\alpha\beta} = \overline{\alpha}\,\overline{\beta}$.
    \end{enumerate}
    Suppose $f$ has real coefficients. 
    Note that for a coefficient $x \in R$, $x = \overline{x}$.
    From this it clearly follows that
    $f(\overline{\alpha}) = \overline{f(\alpha)}$.
    But then suppose $f(\alpha) = 0$ and it follows that
    $f(\overline{\alpha}) = \overline{f(\alpha)} = \overline{0} = 0$.
\end{proof}

\begin{tcolorbox}[title=Problem 12, breakable]
    In this exercise, we describe the cubic formula for factoring an arbitrary polynomial
    of degree $3$ in $\mathbb{R}[x]$. This version of the formula is called the 
    \emph{Cardano-Tartaglia} formula, after two 16th-century Italian mathematicians 
    involved in its discovery. Consider the polynomial $f = x^3 + ax^2 + bx + c \in \mathbb{R}[x]$
    (by dividing by the leading coefficient if necessary, we have assumed without loss 
    of generality that it is $1$).

    (a) Show that the change in variables $x = y - \frac{1}{3}a$ changes $f$ into 
        a cubic polynomial that lacks a square term; that is, a 
        polynomial of the form $g = f(y - \frac{1}{3}a) = y^3 + py + q = 0$.
        \emph{Note: } This process is called \emph{depressing the conic}. Clearly 
        we can solve $f = 0$ for $x$ if and only if we can solve $g = 0$ for $y$.

    (b) Find explicit solutions $u, v$ to the pair of simultaneous equations 
        \textcircled{1} $v^3 - u^3 = q$ and \textcircled{2} $uv = \frac{1}{3}p$.

    (c) Prove the identity $(u - v)^3 + 3uv(u - v) + (v^3 - u^3) = 0$ and 
        use it to show that $y = u - v$ is a solution to the cubic equation 
        $y^3 + py + q = 0$.

    (d) Let $D = q^2 + \frac{3p^3}{27}$. (This is called the \emph{discriminant} of the conic.)
        Conclude that $y = \sqrt[3]{\frac{-q + \sqrt{D}}{2}} - \sqrt[3]{\frac{q + \sqrt{D}}{2}}$
        is a root for $g = 0$. (This is just $u - v$.)
\end{tcolorbox} 


\begin{proof}
\begin{align*}
    f\Bigl(y - \frac{1}{3}a\Bigr) &= \Bigl(y - \frac{1}{3}a\Bigr)^3 + a\Bigl(y - \frac{1}{3}a\Bigr)^2 + b\Bigl(y - \frac{1}{3}a\Bigr) + c \\
                                  &= y^3 + \left(b - \frac{a^2}{3}\right)y + \left(c - \frac{ab}{3} + \frac{2a^3}{27}\right).
\end{align*}
\end{proof}

\begin{proof}
    If $p = 0$, then $u = v = 0$. Suppose $p \ne 0$. From \textcircled{2} we know $u \ne 0$ and $v \ne 0$.
    From \textcircled{2}, $v = \frac{p}{3u}$. Plugging this into \textcircled{1} gives
        \[
            \left(\frac{p}{3u}\right)^3 - u^3 = q.
        \]
    Multiplying through by $u^3 \ne 0$ gives 
        \[
            \left(\frac{p}{3}\right)^3 - u^6 = u^3 q \quad \iff \quad u^6 + q u^3 - \left(\frac{p}{3}\right)^3 = 0.
        \]
    Letting $x = u^3$ we have the quadratic
        \[
            x^2 + q x - \left(\frac{p}{3}\right)^3 = 0.
        \]
    Applying the quadratic formula gives
        \[
            x = \frac{-q \pm \sqrt{q^2 + \frac{4p^3}{27}}}{2}.
        \]
    Taking cube roots shows
        \[
            u = \sqrt[3]{\frac{-q \pm \sqrt{q^2 + \frac{4p^3}{27}}}{2}}.
        \]
    Finally, from \textcircled{2} we get
        \[
            v = \frac{p}{3 \sqrt[3]{\frac{-q \pm \sqrt{q^2 + \frac{4p^3}{27}}}{2}}}.
        \]
\end{proof}

\begin{proof}
    Expanding $(u - v)^3 + 3uv(u - v)$, we have
    \begin{align*}
        (u - v)^3 + 3uv(u - v) 
        &= u^3 - 3u^2v + 3uv^2 - v^3 + 3uv(u - v) \\
        &= u^3 - 3u^2v + 3uv^2 - v^3 + 3u^2v - 3uv^2 \\
        &= u^3 - v^3.
    \end{align*}
    Therefore
        \[
            (u - v)^3 + 3uv(u - v) + (v^3 - u^3) = 0.
        \]
\end{proof}

\begin{proof}
    Since $uv = \frac{p}{3}$ and $v^3 - u^3 = q$, we have
    \[
        (u - v)^3 + 3uv(u - v) + (v^3 - u^3) = 0.
    \]
    Substituting $3uv = p$ and $v^3 - u^3 = q$ shows
    \[
        (u - v)^3 + p(u - v) + q = 0.
    \]
\end{proof}

\begin{proof}
    From part b we know
    \[
        u = \sqrt[3]{\frac{-q + \sqrt{D}}{2}}, \quad v = \sqrt[3]{\frac{q + \sqrt{D}}{2}}.
    \]
    Therefore,
    \[
        y = u - v = \sqrt[3]{\frac{-q + \sqrt{D}}{2}} - \sqrt[3]{\frac{q + \sqrt{D}}{2}}
    \]
    is a root of $g(y) = y^3 + py + q = 0$ as required.
\end{proof}

\begin{tcolorbox}[title=Problem 13, breakable]
    In Exercise 12, there is an apparent ambiguity arising from the plus or minus when extracting
        the square root of $D$ to obtain values for $u^3$ and $v^3$. However, show that we obtain 
        the same value value for the root $u - v$, regardless of which choice is made.
\end{tcolorbox} 

\begin{proof}
    We have the equation $v^3 - u^3 = q$, which any choice of roots satisfies.  
    Viewing our polynomial 
    \[
    (u - v)^3 + 3uv(u - v) + (v^3 - u^3) = 0,
    \] 
    we showed in Exercise $12$ that $(u - v)^3 + 3uv(u - v) = -(v^3 - u^3)$,
        which is independent of our choice of roots.
\end{proof}

\begin{tcolorbox}[title=Problem 15, breakable]
    Suppose as in Exercise 12 that $g = y^3 + px + q$ is a cubic polynomial
    with real coefficients, and $y = u - v$ is the root given by the 
    Cardano-Tartaglia formula. Suppose that $D > 0$. (Thus $u$ and $v$ are real numbers.)
    Let $\zeta = e^{\frac{2\pi}{3}}$ be a cube root of unity (called the primitive cube root of unity 
    in Exercise 25 below). Argue that the other two distinct roots of $g = 0$ are the complex 
    conjugates of $u \zeta - v \zeta^2$ and $u \zeta^2 - v \zeta$. \emph{Note: }
    be sure and check both that these are roots and that they are necessarily distinct.
\end{tcolorbox} 

\begin{proof}
    Let $y_1 = u - v, y_2 =  \zeta - v \zeta^2, y_3 = u \zeta^2 - v \zeta$.
    Notice
    \begin{align*}
        (u \zeta - v \zeta^2)^3 + 3uv(u \zeta - v \zeta^2) + (v^3 - u^3)
        &= u^3 \zeta^3 - v^3 (\zeta^2)^3 - 3 uv (u \zeta - v \zeta^2) + 3 uv (u \zeta - v \zeta^2) + (v^3 - u^3) \\
        &= u^3 - v^3 - 3uv(u \zeta - v \zeta^2) + 3 uv (u \zeta - v \zeta^2) + (v^3 - u^3) \\
        &= 0
    \end{align*}
    Similarly,
    \begin{align*}
        (u \zeta^2 - v \zeta)^3 + 3uv(u \zeta^2 - v \zeta) + (v^3 - u^3)
        &= u^3 (\zeta^2)^3 - v^3 \zeta^3 - 3 uv (u \zeta^2 - v \zeta) + 3 uv (u \zeta^2 - v \zeta) + (v^3 - u^3) \\
        &= u^3 - v^3 - 3 uv (u \zeta^2 - v \zeta) + 3 uv (u \zeta^2 - v \zeta) + (v^3 - u^3) \\
        &= 0
    \end{align*}
    We show that $y_2$ and $y_3$ are complex conjugates:
    \[
        \overline{y_2} = \overline{u \zeta - v \zeta^2} = u \overline{\zeta} - v \overline{\zeta^2} = u \zeta^2 - v \zeta = y_3.
    \]
    Clearly $y_1 \ne y_2$ and $y_1 \ne y_3$ since its imaginary part is $0$.
    Additionally the imaginary parts of $y_2$ and $y_3$ are opposite in sign, so $y_2 \neq y_3$. 
    Thus the roots are distinct.
\end{proof}

\begin{tcolorbox}[title=Problem 17, breakable]
    An interesting and suprising conclusion one can draw from example 15 is that 
    if the discriminant of $D > 0$, then the cubic polynomial $y^3 + py + q \in \mathbb{R}[x]$
        necessarily has exactly one real root, and a conjugate pair of complex roots. 
    In this exercise you will use elementary calculus to verify this fact again:
    \begin{enumerate}
        \item Consider the function $g(y) = y^3 + py + q$. Suppose that $p > 0$.
              Compute the derivative of $g'(y)$, and use it to argue that $g$ 
              has exactly one real root, and consequently two complex roots.
        \item Suppose now that $p = 0$. Then conclude that $q \ne 0$. In this simple case,
              what are the roots of $g$.
        \item Now suppose that $p < 0$, Compute the two roots of $g'(y) = 0$.
              Argue that the values of $g$ at these two roots are both positive (using the assumption that $D > 0$).
              Why does this mean that $g$ has exactly one root?
    \end{enumerate}
\end{tcolorbox} 

\begin{proof}
    $g'(y) = 3y^2 + p$. Since the deriviative is always $> 0$ the function is always increasing 
        thus can only cross the real x-axis once. Therefore there is a single real root.
\end{proof}

\begin{proof}
    If $q = 0$ then $D = 0$. The roots are $0$.
\end{proof}

\begin{proof}
    Notice
    \[
    3y^2 + p = 0 \quad \Rightarrow \quad y^2 = -\frac{p}{3} \quad \Rightarrow \quad y = \pm \sqrt{-\frac{p}{3}}.
    \]
    Let $y_1 = \sqrt{-p/3}$ and $y_2 = -\sqrt{-p/3}$ be the critical points.  
    Since the discriminant $D > 0$, the values $g(y_1)$ and $g(y_2)$ have the same sign.
    Thus the cubic can only cross the $y$-axis once so $g(y)$ as one real root.
\end{proof}

\begin{tcolorbox}[title=Problem 19, breakable]
    Exercise 18 is a particular example of what is called the \emph{irreducible}
    case for a real cubic. Show that in the case $D < 0$, we obtain real roots 
    for the polynomial $g = y^3 + px + q$ by an appropriate choice of $u$ and $v$.
\end{tcolorbox} 

\begin{proof}
    Choose $v$ to be the complex conjugate of $u$.
\end{proof}

\newpage
\begin{tcolorbox}[title=Problem 20, breakable]
    In this problem we will explore Ferrari's approach to solving the 
    general quartic equation. Consider the arbitrary quartic
    \[f = x^4 + a_3 x^3 + a_2 x^2 + a_1 x + a_0 \in \mathbb{R}[x].\]
    \begin{enumerate}
        \item Find a linear change of variables $y = x + m$ so as to depress the quartic 
                - that is, to eliminate the cubic term, as we eliminated the quadratic term in Exercise 12.
        \item We may by part a assume that our quartic equation is of the form $x^4 = px^2 + qx + r$, 
                where $p, q, r \in \mathbb{R}$. Add the term $2bx^2 + b^2$ to both sides of this equation.
                Clearly this makes the left-hand side of the equation a perfect square. We would like to 
                choose $b$ so that the right-hand side is also a perfect square. Obtain an equation for 
                b (in terms of $p, q, r$) that makes this true. The equation you obtain should be a cubic 
                equation in $b$. Explain why a (real) solution to this cubic will always lead to a solution to the 
                quartic equation. How do you then get all four solutions.
    \end{enumerate}
\end{tcolorbox} 

\begin{proof}
    Let $x = y - m$. Then plugging in, we have 
    \begin{align*}
        f(y - m) &= x^4 + a_3 x^3 + a_2 x^2 + a_1 x + a_0 \\
                 &= (y - m)^4 + a_3 (y - m)^3 + a_2 (y - m)^2 + a_1 (y - m) + a_0 \\
                 &= (y^4 - 4 y^3 m + 6 y^2 m^2 - 4 y m^3 + m^4) + a_3(y^3 - 3 y^2 m + 3 y m^2 - m^3) + a_2(y^2 - y m + m^2) + a_1 (y - m) + a_0 \\
                 &= y^4 + (a_3 - 4m) y^3 + (a_2 - 3 a_3 m + 6 m^2)y^2 + (a_1 - a_2 m + 3 a_3 m^2 - 4 m^3) y + (a_0 - a_1 m + a_2 m^2 - a_3 m^3 + m^4).
    \end{align*}
    To remove the cubic term, we set $a_3 - 4m = 0$ thus $m = \frac{a_3}{4}$.  
    Then plugging in $m$ we find
    \begin{align*}
        &y^4 + (a_3 - 4 \tfrac{a_3}{4}) y^3 + \big(a_2 - 3 a_3  \tfrac{a_3}{4} + 6 (\tfrac{a_3}{4})^2\big) y^2 \\ 
        &\quad + \big(a_1 - a_2 \tfrac{a_3}{4} + 3 a_3 (\tfrac{a_3}{4})^2 - 4 (\tfrac{a_3}{4})^3\big) y + \big(a_0 - a_1 \tfrac{a_3}{4} + a_2 (\tfrac{a_3}{4})^2 - a_3 (\tfrac{a_3}{4})^3 + (\tfrac{a_3}{4})^4\big) \\
        &= y^4 + \left(a_2 - \frac{3 a_3^2}{8}\right) y^2 + \left(a_1 - \frac{a_2 a_3}{4} + \frac{a_3^3}{16}\right) y + \left(a_0 - \frac{a_1 a_3}{4} + \frac{a_2 a_3^2}{16} - \frac{3 a_3^4}{256}\right).
    \end{align*}
\end{proof}

\begin{proof}
    Adding the term $2 b x^2 + b^2$ to both sides, we have
    \[
        x^4 + 2 b x^2 + b^2 = (x^2 + b)^2 = (p + 2b)x^2 + q x + r + b^2.
    \]
    We want 
    \[
        (p + 2b)x^2 + q x + (r + b^2) = (\sqrt{p + 2b}\, x + m)^2
    \]
    for some $m \in \mathbb{R}$. Expanding the right-hand side gives
    \[
        (\sqrt{p + 2b}\, x + m)^2 = (p + 2b)x^2 + 2 m \sqrt{p + 2b}\, x + m^2.
    \]
    Comparing coefficients, we must have
    \[
        q = 2 m \sqrt{p + 2b} \quad \text{and} \quad r + b^2 = m^2.
    \]
    Solving for $m$ gives $m = \frac{q}{2 \sqrt{p + 2b}}$. Plugging this into $r + b^2 = m^2$ shows
    \[
        4(r + b^2)(p + 2b) - q^2 = 0,
    \]
    a cubic equation in $b$. 
    Now note that 
    \[
        (x^2 + b)^2 - (\sqrt{p + 2b}\, x + m)^2 = (x^2 + b + \sqrt{p + 2b}\, x + m)(x^2 + b - \sqrt{p + 2b}\, x - m) = x^4 + p x^2 + q x + r.
    \]
    With a real solution $b$, we can then compute $m$ and factor the quartic as
    \[
        x^4 + p x^2 + q x + r = (x^2 + b + \sqrt{p + 2b}\, x + m)(x^2 + b - \sqrt{p + 2b}\, x - m).
    \]
    Each quadratic has two solutions, giving all four roots of the quartic.
\end{proof}

\newpage
\begin{tcolorbox}[title=Problem 23, breakable]
    Suppose that the complex number $\alpha = a + bi$ has been factored as 
    \[\alpha = \sqrt{a^2 + b^2}(\cos \theta + i \sin \theta) = |\alpha| e^{i \theta},\]
    as we did in the text when computing the square roots of $\alpha$.
    \begin{enumerate}
        \item Show that 
        \[\beta_k = \sqrt[2n]{a^2 + b^2}\left(\cos\left(\frac{\theta + 2 \pi k}{n}\right) + i \sin \left(\frac{\theta + 2 \pi k}{n}\right)\right)\]
        for $k = 0, 1, 2, \cdots, n - 1$, are all $n$th roots of $\alpha$.
        \item Show that these are all distinct roots.
        \item Why is this the \emph{complete} list of $n$th roots of $\alpha$.
    \end{enumerate}
\end{tcolorbox} 

\begin{proof}
    First note that 
    \[\beta_k 
        =  \sqrt[2n]{a^2 + b^2}\left(\cos\left(\frac{\theta + 2 \pi k}{n}\right) + i \sin \left(\frac{\theta + 2 \pi k}{n}\right)\right) 
        = \sqrt[2n]{a^2 + b^2}e^{i\frac{\theta + 2 \pi k}{n}}.\]
    Then 
    \[\left(\sqrt[2n]{a^2 + b^2}e^{i\frac{\theta + 2 \pi k}{n}}\right)^n 
        = \left(\sqrt[2n]{a^2 + b^2}\right)^n e^{i(\theta + 2\pi k)}
        = \sqrt{a^2 + b^2}e^{i\theta}
        = \alpha.\]
\end{proof}

\begin{proof}
    This is trivial since the arguments of the roots are clearly different.
\end{proof}

\begin{proof}
    Suppose there are more roots.
    Then the polynomial $x^n - \alpha = 0$ would have more than $n$ roots,
    violating the Fundamental Theorem of Algebra.
\end{proof}

\begin{tcolorbox}[title=Problem 26, breakable]
    A field $F$ is said to be \textbf{algebraically closed} if every polynomial 
    $f \in F[x]$ with $deg(F) \ge 1$ has a root in $F$; we can rephrase this 
    definition roughly by saying that a field is algebraically closed if it 
    satisfies the Fundamental Theorem of Algebra. Thus, $\mathbb{C}$ is 
    algebraically closed, while $\mathbb{R}$ and $\mathbb{Q}$ are not.
    Show that for every prime $p$, the field $\mathbb{Z}_p$ is not algebraically closed.
\end{tcolorbox} 

\begin{proof}
    From Fermat's little theorem $f(x) = x^p - x$ evaluates to $0$ for all $x \in \mathbb{Z}_p$.
    Then $g(x) = f(x) + 1$ evaluates to $1$ for all $x \in \mathbb{Z}_p$.
    Therefore $\mathbb{Z}_p$ is not algebraically closed.
\end{proof}

\begin{tcolorbox}[title=Problem 27, breakable]
    Show that the field in Exercise 8.12 is not algebraically closed.
    (See the previous exercise for a definition.)
    \begin{enumerate}
        \item The field $F = \{0, 1, \alpha, 1 + \alpha\}$.
        \item $0$ is the additive idenitty.
        \item $1$ is the multiplicative identity.
        \item $x + x = 0$ for all $x \in F$, and $\alpha^2 = \alpha + 1$.
    \end{enumerate}
\end{tcolorbox} 

\begin{proof}
    Consider $f(x) = x^4 - x$. Then 
    \begin{align*}
        f(0) &= 0
    \end{align*}
    \begin{align*}
        f(1) &= 1 - 1 = 0
    \end{align*}
    \begin{align*}
        f(\alpha) &= \alpha^4 - \alpha = (\alpha^2)^2 - \alpha = (\alpha + 1)^2 - \alpha = (\alpha^2 + 1) - \alpha = (\alpha + 1 + 1) - \alpha = \alpha + \alpha = 0
    \end{align*}
    \begin{align*}
        f(1+\alpha) &= (1+\alpha)^4 - (1+\alpha) = ((1+\alpha)^2)^2 - (1+\alpha) = (\alpha)^2 - (1+\alpha) = \alpha^2 - (1+\alpha) = (\alpha+1) - (1+\alpha) = 0
    \end{align*}
    Then $g(x) = f(x) + 1 = 1$ for all $x \in F$, thus $F$ is not algebraically closed.
\end{proof}

\begin{tcolorbox}[title=Problem 28, breakable]
    Show that, if $F$ is a field with infinitely many elements, then $f(x) = g(x)$
    for all $x \in F$ implies that $f = g$ as polynomials. (We have already seen 
    that this is not the case if $F$ is a finite field. For example, consider 
    $x^2 + x + 1$ and $1$ in $\mathbb{Z}_2[x]$.)
\end{tcolorbox} 

\begin{proof}
    Suppose $F$ is a field with infinitely many elements and $f(x) = g(x)$
        for all $x \in F$.
    Then $\psi(x) = f(x) - g(x) = 0$ for all $x \in F$ and thus has 
        infinitely many roots.
    But a nonzero polynomial over a field can only have a finite number of roots 
        thus $\psi$ must be the zero polynomial.
    Therefore $f = g$.
\end{proof}