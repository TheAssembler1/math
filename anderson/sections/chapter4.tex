\begin{tcolorbox}[title=Problem 2, breakable]
    Divide the polynomial $x^2 - 3x + 2$ by the polynomial $2x + 1$,
    to obtain a quotient and remainder as garunteed by the Division 
    Theorem $4.2$. Note that although $x^2 - 3x + 2$ and $2x + 1$ are 
    elements of $\mathbb{Z}[x]$, the quotient and remainder are not.
    Argue that this means that there is not Division Theorem for $\mathbb{Z}[x]$.
\end{tcolorbox}

\textbf{Solution:}

When $x^2 - 3x + 2$ is divided by $2x + 1$,
the quotient is $\frac{1}{2}x - \frac{7}{4}$ and the remainder is $\frac{15}{4}$.
To verify:
\begin{align*}
(2x + 1)\left(\frac{1}{2}x - \frac{7}{4}\right) + \frac{15}{4} 
&= x^2 - \frac{7}{2}x + \frac{1}{2}x - \frac{7}{4} + \frac{15}{4} \\
&= x^2 - 3x + 2
\end{align*}
Clearly, in this example the remainder is not in $\mathbb{Z}[x]$.
Therefore, the Division Theorem does not hold in $\mathbb{Z}[x]$.
Moreover, the quotient and remainder are unique as guaranteed by the Division Theorem in $\mathbb{Q}[x]$.

\begin{tcolorbox}[title=Problem 3, breakable]
    By Corollary $4.4$ we know that a third-degree polynomial in $\mathbb{Q}[x]$
    has at most three roots. Give four examples of third-degree polynomials in 
    $\mathbb{Q}[x]$ that have $0$, $1$, $2$, and $3$ roots, respectively; justify
    your assertions. (Recall that here a root must be a rational number!)
\end{tcolorbox}

\begin{enumerate}
     \item $0$ roots: $x^3 - 2 = 0$. $x$ would have to satisfy $x^3 = 2$.
           Clearly no integer cubed equals $2$ (so $n=1$ is impossible).
           Suppose there exists a rational number in lowest terms $x = \frac{m}{n}$ with $|n| > 1$.
           Then $(\frac{m}{n})^3 = 2 \iff m^3 = 2 n^3$.
           But this implies $n^3 \mid m^3$, so $n \mid m$, which contradicts that
           $\gcd(m,n)=1$. Thus there is no rational root.
     \item $1$ root:  $(x - 0)(x - 0)(x - 0) = 0$. Justification is obvious by root theorem.
     \item $2$ roots: $(x - 1)(x - 1)(x - 2) = 0$. Justification is obvious by root theorem.
     \item $3$ roots: $(x - 1)(x - 2)(x - 3) = 0$. Justification is obvious by root theorem.
\end{enumerate}

\begin{tcolorbox}[title=Problem 4, breakable]
    Your example in the previous exercise of a third-degree polynomial with 
    exactly $2$ roots had one repeated root; that is, a root $a$ where $(x - a)^2$
    is a factor of the polynomial. (Roots may have multiplicity greater than two 
    of course.) Why can't a third-degree polynomial in $\mathbb{Q}[x]$ have 
    exactly $2$ roots where neither is a multiple root.
\end{tcolorbox}

\begin{proof}
    Let $f$ be a degree $3$ polynomial with two roots, $a, b$
        such that $a \ne b$.
    We can express $f$ in the form 
    \[f = (x - a)(x - b)l\]
    where $l = (x - c)$ is another linear factor.
    If $c \ne a$ and $c \ne b$, then $f$ has three distinct roots $a, b, c$,
        contradicting that $f$ has only two roots.
    Therefore, $c = a$ or $c = b$, meaning one of the roots is repeated.
\end{proof}

\begin{tcolorbox}[title=Problem 6, breakable]
    Suppose that $f \in \mathbb{Q}[x]$, $q \in \mathbb{Q}$,
    and $deg(f) > 0$. Use the Root Theorem $4.3$ to prove that
    the equation $f(x) = q$ has at most finitely many solutions.
\end{tcolorbox}

\begin{proof}
    Solving $f(x) = q$ is equivalent to solving $l = f - q = 0$.
    Now $\deg(l) = \deg(f)$ since $\deg(f) > \deg(q)$.
    By the Root Theorem, every root divides $l$,
        reducing its degree by $1$.
    Since $\deg(l)$ is finite, the number of roots is 
        finite.
\end{proof}

\begin{tcolorbox}[title=Problem 8, breakable]
    Prove Theorem $4.7$: the GCD identity for $\mathbb{Q}[x]$.
    Use Euclid's Algorithm $4.5$, and the relationship we know 
    between the $\gcd$ produced by the algorithm and an arbitrary $\gcd$ (Theorem $4.6$).
\end{tcolorbox}

\begin{theorem}[GCD Identity $4.7$]
    If $d$ is a $gcd$  of polynomials $f$ and $g$,
        then there exists polynomials $a$ and $b$
        so that $d = af + bg$.
\end{theorem}

\begin{proof}
    Let $f, g \in \mathbb{Q}[x]$.
    By $4.5$, there exists a last nonzero remainder 
        $r_{n-1}$ such that $r_{n-2} = q_{n-1} r_{n-1}$ and $r_{n-1} \ne 0$.
    This remainder $r_{n-1}$ is the $\gcd(f, g)$.
    From the division steps in the algorithm, each remainder can be expressed 
        as a linear combination of $f$ and $g$.
    Thus there exist polynomials $a, b \in \mathbb{Q}[x]$ such that 
        $r_{n-1} = af + bg$.
    By Theorem $4.6$, any other $\gcd$ of $f$ and $g$ differs from $r_{n-1}$ 
        by a nonzero rational constant.
    Therefore, every $\gcd$ of $f$ and $g$ in $\mathbb{Q}[x]$ can be expressed 
        as a linear combination of $f$ and $g$.
\end{proof}

\begin{tcolorbox}[title=Problem 9, breakable]
    One can also prove the GCD identity for $\mathbb{Q}[x]$ with an argument 
    similar to the existential proof of the GCD identity for integers,
    found in Section $2.3$. Try this approach.
\end{tcolorbox}

\begin{proof}
    Consider the set of all linear combinations of the polynomials $f, g$:
    \[
    S = \{ f a + g b : a, b \in \mathbb{Q}[x] \}
    \]
    We must show the $\gcd(f, g)$ belongs to this set.
    By the Well-ordering Principle, $S$ contains an element $d$
    which has the smallest positive degree.
    Since $d \in S$:
    \[
    d = f a_0 + g b_0
    \]
    For some $a_0, b_0 \in \mathbb{Q}[x]$.
    Applying the Division Theorem to $d, f$
    we get $f = d q + r$.
    We now show $r = 0$.
    But:
    \[
    r = f - d q = f - (f a_0 + g b_0) q = f (1 - q a_0) + g(-q b_0)
    \]
    So $r \in S$. Because $\deg(r) < \deg(d)$, and $d$ has the smallest degree 
    of $S$, we must have $r = 0$.
    A similar argument shows $d \mid g$.

    Now suppose $c$ divides both $f$ and $g$.
    Then $f = n c$ and $g = m c$ for some $n, m \in \mathbb{Q}[x]$.
    Then any linear combination of $f$ and $g$ is also a multiple of $c$:
    \[
    f a + g b = n c a + m c b = c (n a + m b)
    \]
    So $c \mid d$.
    Thus $d$ is the $\gcd$ of $f$ and $g$.
\end{proof}

\begin{tcolorbox}[title=Problem 10, breakable]
    We say that $p \in \mathbb{Q}[x]$ has a multiplicative inverse 
    if there exists $q \in \mathbb{Q}[x]$ such that $pq = 1$.
    Prove that $p \in \mathbb{Q}[x]$ has a multiplicative inverse 
    if and only if $\deg(p) = 0$.
\end{tcolorbox}

\begin{proof}
    ($\rightarrow$) Suppose $p \in \mathbb{Q}[x]$ has a multiplicative inverse.
    Let $q$ be the multiplicative inverse of $p$ such that $p q = 1$.
    Clearly $q \ne 0$.
    If $\deg(p) > 0$ then $\deg(pq) > 0$, but $\deg(pq) = \deg(1) = 0$.
    Thus $\deg(p) = 0$.

    ($\leftarrow$) Suppose $\deg(p) = 0$. It follows that $p \in \mathbb{Q}$.
    Then let $a, b \in \mathbb{Z}$ such that $p = \frac{a}{b}$.
    We know $a \ne 0$ since $\deg(p) \ne -\infty$.
    Let $q = \frac{b}{a}$, then $p q = \frac{a}{b} \cdot \frac{b}{a} = 1$.
\end{proof}

\begin{tcolorbox}[title=Problem 11, breakable]
    Suppose that $g \in \mathbb{Q}[x]$, and $g$ divides all elements 
    of $\mathbb[Q][x]$. Prove that $g$ is a non-zero constant polynomial.
\end{tcolorbox}

\begin{proof}
    Clearly $g \ne 0$ since division by $0$ is undefined.  
    Suppose, for contradiction, that $g$ is non-constant. That is, $\deg(g) > 0$.  
    Consider $f \in \mathbb{Q}[x]$ such that $f \ne 0$ and $\deg(f) < \deg(g)$.
    Since $g \mid f$, there exists $c \in \mathbb{Q}[x]$ such that 
        $f = c \cdot g$.
    But $\deg(f) < \deg(g)$ and $\deg(f) = \deg(c) + \deg(g) \ge \deg(g)$,
        a contradiction.  
    Thus $g$ is a non-zero constant polynomial.
\end{proof}

\begin{tcolorbox}[title=Problem 12, breakable]
    Find two different polynomials in $\mathbb{Z}_3[x]$ that are equal 
    as functions from $\mathbb{Z}_3 = \mathbb{Z}_3$.
\end{tcolorbox}

\begin{proof}
    Let $f(x) = x^3$ and $g(x) = x$.
    Clearly as polynomials $f \ne g$.
    Now check each $a \in \mathbb{Z}_3$:
    \[
    0^3 \equiv 0, \quad 1^3 \equiv 1, \quad 2^3 \equiv 2 \pmod{3}.
    \]
    Thus $f(a) \equiv g(a) \pmod{3}$ for all $a \in \mathbb{Z}_3$, so $f$ and $g$ are equal as functions.
\end{proof}

\begin{tcolorbox}[title=Problem 13, breakable]
    Find a non-zero polynomial in $\mathbb{Z}_4[x]$ for which $f(a) = 0$,
    for all $a \in \mathbb{Z}_4$.
\end{tcolorbox}


\begin{proof}
    Let $f(x) = 2x^2 + 2x^4 \in \mathbb{Z}_4[x]$.  
    Now check each element in $\mathbb{Z}_4 = \{0,1,2,3\}$:
    \[
    \begin{aligned}
    f(0) &= 2 \cdot 0^2 + 2 \cdot 0^4 = 0, \\
    f(1) &= 2 \cdot 1^2 + 2 \cdot 1^4 = 2 + 2 = 4 \equiv 0 \pmod{4}, \\
    f(2) &= 2 \cdot 2^2 + 2 \cdot 2^4 = 8 + 32 = 40 \equiv 0 \pmod{4}, \\
    f(3) &= 2 \cdot 3^2 + 2 \cdot 3^4 = 18 + 162 = 180 \equiv 0 \pmod{4}.
    \end{aligned}
    \]
    Therefore $f(a) = 0$ for all $a \in \mathbb{Z}_4$, but $f(x)$ is not the zero polynomial.
\end{proof}