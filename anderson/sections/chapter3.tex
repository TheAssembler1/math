\begin{tcolorbox}[title=Problem 2, breakable]
    Determine the elements of $\mathbb{Z}_{15}$ that have multiplicative inverses.
    Give an example of an equation of the form $[a]X = [b]$ ($[a] \not = 0$)
    that has no solution in $\mathbb{Z}_{15}$.
\end{tcolorbox}

\textbf{Solution}

\begin{center}
\begin{tabular}{c | *{15}{c@{\hskip 2pt}}}
 & $[0]$ & $[1]$ & $[2]$ & $[3]$ & $[4]$ & $[5]$ & $[6]$ & $[7]$ & $[8]$ & $[9]$ & $[10]$ & $[11]$ & $[12]$ & $[13]$ & $[14]$ \\
\hline
$[0]$  & 0 & 0 & 0 & 0 & 0 & 0 & 0 & 0 & 0 & 0 & 0 & 0 & 0 & 0 & 0 \\
$[1]$  & 0 & 1 & 2 & 3 & 4 & 5 & 6 & 7 & 8 & 9 & 10 & 11 & 12 & 13 & 14 \\
$[2]$  & 0 & 2 & 4 & 6 & 8 & 10 & 12 & 14 & 1 & 3 & 5 & 7 & 9 & 11 & 13 \\
$[3]$  & 0 & 3 & 6 & 9 & 12 & 0 & 3 & 6 & 9 & 12 & 0 & 3 & 6 & 9 & 12 \\
$[4]$  & 0 & 4 & 8 & 12 & 1 & 5 & 9 & 13 & 2 & 6 & 10 & 14 & 3 & 7 & 11 \\
$[5]$  & 0 & 5 & 10 & 0 & 5 & 10 & 0 & 5 & 10 & 0 & 5 & 10 & 0 & 5 & 10 \\
$[6]$  & 0 & 6 & 12 & 3 & 9 & 0 & 6 & 12 & 3 & 9 & 0 & 6 & 12 & 3 & 9 \\
$[7]$  & 0 & 7 & 14 & 6 & 13 & 5 & 12 & 4 & 11 & 3 & 10 & 2 & 9 & 1 & 8 \\
$[8]$  & 0 & 8 & 1 & 9 & 2 & 10 & 3 & 11 & 4 & 12 & 5 & 13 & 6 & 14 & 7 \\
$[9]$  & 0 & 9 & 3 & 12 & 6 & 0 & 9 & 3 & 12 & 6 & 0 & 9 & 3 & 12 & 6 \\
$[10]$ & 0 & 10 & 5 & 0 & 10 & 5 & 0 & 10 & 5 & 0 & 10 & 5 & 0 & 10 & 5 \\
$[11]$ & 0 & 11 & 7 & 3 & 14 & 10 & 6 & 2 & 13 & 9 & 5 & 1 & 12 & 8 & 4 \\
$[12]$ & 0 & 12 & 9 & 6 & 3 & 0 & 12 & 9 & 6 & 3 & 0 & 12 & 9 & 6 & 3 \\
$[13]$ & 0 & 13 & 11 & 9 & 7 & 5 & 3 & 1 & 14 & 12 & 10 & 8 & 6 & 4 & 2 \\
$[14]$ & 0 & 14 & 13 & 12 & 11 & 10 & 9 & 8 & 7 & 6 & 5 & 4 & 3 & 2 & 1 \\
\end{tabular}
\end{center}

Elements with multiplicative inverses are $[1]$,$[2]$,$[4]$,$[7]$,$[8]$,$[11]$,$[13]$, and $[14]$.

Example of an equation of the form $[a]X = [b]$ ($[a] \not = 0$)
    that has no solution.
\[[3]X = [5]\]

\begin{tcolorbox}[title=Problem 4, breakable]
    Find an example in $\mathbb{Z}_6$ where $[a][b] = [a][c]$, but $[b] \not = [c]$.
    How is this related to the existence of  multiplicative inverses in $\mathbb{Z}_6$?
\end{tcolorbox}

Example where $[a][b] = [a][c]$, but $[b] \not = [c]$:
\[[2][2] = [2][5] = [4]\]

You cannot assume that if  $[a][b] = [a][c]$ then $[b] = [c]$.
This is only true if $[a]$ has a multiplicative inverse.

\begin{tcolorbox}[title=Problem 5, breakable]
    If $\gcd(a, b) = 1$ then the GCD identity $2.4$ garuntees that there exists 
    integers $u$ and $v$ such that $1 = au + mv$. Show that in this case, 
    $[u]_m$ is the multiplicate inverse of $[a]_m$ in $\mathbb{Z}_m$.
\end{tcolorbox}

\begin{proof}
    By Theorem $3.2$ if $x - y = km$ for some integer $k$ then $x$,$y$ are in the same residue (mod $m$).
    Now $1 = au + mv \iff au = -mv + 1$.
    Then $x - y = (-mv + 1) - 1 = (-v)m$.
    Thus ${[au]}_m = {[1]}_m$ and therefore $[a]_m \cdot [u]_m = [1]$.
\end{proof}

\begin{tcolorbox}[title=Problem 6, breakable]
    Now use essentially the reverse of the argument from Excersize $5$
    to show that if $[a]$ has a multiplicative inverse in $\mathbb{Z}_m$,
    then $\gcd(a, m) = 1$.
\end{tcolorbox}

\begin{proof}
    Suppose $[a]_m$ has a multiplicative inverse $[b]_m$ in $\mathbb{Z}_m$.
    Then $[a] \cdot [b] = [ab] = [1]$.
    By Theorem $3.2$, $ab - 1 = km$.
    But $ab - km = 1$ so $ab + m(-k) = 1$.
    Therefore $\gcd(a, m) = 1$.
\end{proof}

\begin{tcolorbox}[title=Problem 7, breakable]
    According to what you have shown in Excersize $5$ and $6$, which 
    elements of $\mathbb{Z}_{24}$ have multiplicative inverses?
    What are the inverses for each of those elements? (The answer is somewhat suprising.)
\end{tcolorbox}

\textbf{Solution:}

The following have multiplicate inverses in $\mathbb{Z}_{24}$.

\begin{enumerate}
    \item $[1]_{24}$
    \item $[5]_{24}$
    \item $[7]_{24}$
    \item $[11]_{24}$
    \item $[13]_{24}$
    \item $[17]_{24}$
    \item $[19]_{24}$
    \item $[23]_{24}$
\end{enumerate}

\begin{tcolorbox}[title=Problem 9, breakable]
    Prove that the multiplication on $\mathbb{Z}_m$ as defined in the text 
    is well defined, as claimed in Section $3.2$.
\end{tcolorbox}

\begin{proof}
    Consider $[a]$ and $[b]$.
    Let $x, y$ be elements in $[a]$ and $b, c$ be elements in $[b]$.
    We need to show $[xb] = [yc]$.
    But $x,y \in [a]$ implies $x - y = k_1 m$ for some integer $k_1$.
    Also $b, c \in [b]$ implies $b - c = k_2 m$ for some integer $k_2$.
    Then:
    \begin{align*}
        &xb - yc  \\
        =&(k_1 m + y)(k_2 m + c) - yc  \\
        =&k_1 k_2 m^2 + k_1 mc + k_2 my + yc - yc  \\
        =&k_1 k_2 m^2 + k_1 mc + k_2 my  \\
        =&m(k_1 k_2 m + k_1 c + k_2 y)
    \end{align*}
    Showing that $[xb]=[yc]$
\end{proof}

\begin{tcolorbox}[title=Problem 10, breakable]
    Prove that if all non-zero $\mathbb{Z}_m$ have multiplicative inverses,
    then multiplicative cancellation holds: that is, if $[a][b] = [a][c]$, 
    then $[b] = [c]$.
\end{tcolorbox}

\begin{proof}
    Suppose all non-zero $\mathbb{Z}_m$ have multiplicative inverses.
    Let $[t]$ be the multiplicative inverse of $[a]$.
    Then $[a][b] = [a][c] \iff [t][a][b] = [t][a][c] \iff [b] = [c]$.
\end{proof}

\begin{tcolorbox}[title=Problem 13, breakable]
    In the integers, the equation $x^2 = a$ has a solution only when $a$
    is a positive perfect square or zero. For which $[a]$ does the equation 
    $[X]^2 = [a]$ have a solution in $\mathbb{Z}_7$?
    What about in $\mathbb{Z}_8$?
    What about in $\mathbb{Z}_9$?
\end{tcolorbox}

\textbf{Solution:}

Elements with square roots in $\mathbb{Z}_7$: 
\[[0], [1], [2], [4]\]
Elements with square roots in $\mathbb{Z}_8$: 
\[[0], [1], [4]\]
Elements with square roots in $\mathbb{Z}_9$: 
\[[0], [1], [4], [7]\]

\begin{tcolorbox}[title=Problem 14, breakable]
    Explain what $a \equiv b (mod\ 1)$ means.
\end{tcolorbox}

\textbf{Solution:}

It means when elements in $[a]$ and $[b]$ are divided by $1$ the remainder is equivalent.
Of course this is true for any $a$ and $b$ since $\frac{a}{1} = a$ and $\frac{b}{1} = b$.
So $[a] = [b] = \mathbb{Z}$.