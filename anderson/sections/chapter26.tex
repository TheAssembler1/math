\begin{tcolorbox}[title=Problem 2, breakable]
    Prove that every subgroup of a cyclic group is cyclic.
\end{tcolorbox}

\begin{proof}
    Let $G$ be a cyclic group.
    Suppose $H$ is a subgroup of $G$.
    If $H$ only contains the identity then clearly it is cyclic.
    Thus, suppose $H$ has more than one element.
    Let $\langle g \rangle = G$ and $g^m = x \in H$ where $m$ is the smallest positive integer.
    Furthermore, let $h$ be an arbitrary element in $H$.
    Then $h = g^k$ for some $k \in \mathbb{Z}$.
    We can apply the Division Algorithm to $k$ and $m$ to write
    \[k = q m + r \text{ for some integer } 0 \le r < m.\]
    Then 
    \[h = g^k = g^{q m + r} = (g^m)^q \cdot g^r = x^q \cdot g^r.\]
    Since $h \in H$ and $x^q \in H$, we have $g^r \in H$.
    By minimality of $m$, $r = 0$, so $h = x^q \in \langle x \rangle$.
\end{proof}

\begin{tcolorbox}[title=Problem 3, breakable]
    Find an example to show that the converse of Exercise 2 is false.
    That is, give a non-cyclic group, each of whose proper subgroups is cylic.
\end{tcolorbox}

\textbf{Solution: }
\[D_3.\]

\begin{tcolorbox}[title=Problem 4, breakable]
    Suppose that $g$ is an element of an infinite order 
    in a group $G$. Prove that no two distinct powers of $g$
    (with any integer exponent) are equal.
\end{tcolorbox}

\begin{proof}
    For contradiction, suppose there exists $q, r$ such that $q \ne r$ and $g^q = g^r$.
    Furthermore, suppose wlog $r < q$. Then 
    \[g^q = g^r \iff g^q \cdot g^{-r} = 1 \iff g^{q - r} = 1 \iff q - r = 0 \iff q = r,\]
    which is a contradiction.
\end{proof}


\begin{tcolorbox}[title=Problem 5, breakable]
    If $a$ and $b$ are elements of a group that commute and $\langle a \rangle \cap \langle b \rangle = \{1\}$,
    what is the order of $ab$ if the order of $a$ is $m$ and the order of $b$ is $n$?
    Prove your assertion. Show by example that your assertion is false in general,
    in the case that $a$ and $b$ do not commute.
\end{tcolorbox}

\textbf{Solution: } The order of $ab$ is $mn$.

\begin{proof}
    Suppose $a$ and $b$ are elements of a group that commute and $\langle a \rangle \cap \langle b \rangle = \{1\}$.  
    Since $a$ and $b$ commute, we have
    \[
    (ab)^{mn} = a^{mn} b^{mn} = (a^m)^n (b^n)^m = 1^n \cdot 1^m = 1.
    \]
    Now suppose there exists $k < mn$ such that $(ab)^k = 1$.  
    Then, using commutativity,
    \[
    1 = (ab)^k = a^k b^k \implies a^k = (b^k)^{-1} \in \langle b \rangle.
    \]  
    But $a^k \in \langle a \rangle$, so $a^k \in \langle a \rangle \cap \langle b \rangle = \{1\}$.  
    Thus $a^k = 1$ and $b^k = 1$.  
    Now, $m \mid k$ and $n \mid k$.  
    The smallest positive $k$ divisible by both $m$ and $n$ is $mn$.  
    Therefore no $k < mn$ satisfies $(ab)^k = 1$, so the order of $ab$ is $mn$.
\end{proof}

\begin{tcolorbox}[title=Problem 6, breakable]
    If $a$ and $b$ are elements of a group whose orders are relatively prime,
    what can you say about $\langle a \rangle \cap \langle b \rangle$? Prove your assertion.
\end{tcolorbox}

\begin{proof}
    Suppose $o(a) = m$ and $o(g) = n$ where $\gcd(m, n) = 1$.
\end{proof}
