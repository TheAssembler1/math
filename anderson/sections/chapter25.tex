\begin{tcolorbox}[title=Problem 1, breakable]
    Consider the set 
    \[i \mathbb{R} = \{a_i \mid a \in \mathbb{R}\} \subseteq \mathbb{C}\]
    these are the \textbf{imaginary} numbers.
    Prove that this is a subgroup of the additive group $\mathbb{C}$.
    Is $I$ a subring of the ring $\mathbb{C}$?
    Similarly, show that $i\mathbb{Z} = \{ni \mid n \in \mathbb{Z}\}$
        is a subgroup of the additive group of the Gaussian
        integers $\mathbb{Z}[i]$.
\end{tcolorbox} 


\begin{proof}
    Clearly the set is nonempty.
    Let $xi, yi \in i\mathbb{R}$.
    Then $xi - yi = (x - y)i \in i\mathbb{R}$.
    By Theorem 35.2, $i\mathbb{R}$ is a subgroup of $\mathbb{C}$.
\end{proof}

\textbf{Solution:} No, $i\mathbb{R}$ is not a subring of $\mathbb{C}$ since it is 
not closed under multiplication.

\begin{proof}
    Clearly the set is nonempty.
    Let $xi, yi \in i\mathbb{Z}$.
    Then $xi - yi = (x - y)i \in i\mathbb{Z}$.
    By Theorem 35.2, $i\mathbb{Z}$ is a subgroup of $\mathbb{Z}[i]$.
\end{proof}

\begin{tcolorbox}[title=Problem 2, breakable]
    Prove Theorem 25.1c. That is, suppose that $G$ 
    is a group and $g, h \in G$. Prove that $gx = h$
    has a unique solution; likewise, prove that 
    $xg = h$ has a unique solution. (We have 
    written the equations multiplicatively.)
\end{tcolorbox} 

\begin{proof}
    Notice $gx = h \iff g^{-1}gx = g^{-1}h \iff x = g^{-1}h$.
    Similarly $xg = h \iff xgg^{-1} = hg^{-1} \iff x = h g^{-1}$.
\end{proof}

\begin{tcolorbox}[title=Problem 3, breakable]
    Prove Theorem 25.1d. That is, prove that in a group,
    every element has exactly one inverse.
\end{tcolorbox} 

\begin{proof}
    Suppose $G$ is a group and let $x \in G$.
    Futhermore, suppose $x$ has two inverses namely: $x^{-1}_1, x^{-1}_2$.
    Then $x x^{-1}_1 = 1 = x x^{-1}_2$.
    Then by the cancellation property $x^{-1}_1 = x^{-1}_2$.
\end{proof}

\begin{tcolorbox}[title=Problem 4, breakable]
    Prove the subgroup Theorem 25.2: A non-empty subset $H$ of a group $G$
    is a subgroup if and only if whenever $h, k \in H$, then $h k^{-1} \in H$.
\end{tcolorbox} 

\begin{proof}
    Suppose $H$ is a nonempty subset of a group $G$.

    ($\longrightarrow$) 
    Suppose $H$ is a subgroup of $G$.
    Let $h, k$ be arbitrary elements in $H$.
    Since $H$ is a subgroup, $k^{-1} \in H$.
    Also, since $H$ is closed under multiplication, $hk^{-1} \in H$ as required.

    ($\longleftarrow$)
    Suppose that if $h, k \in H$, then $hk^{-1} \in H$.

    (\textbf{Rule 1})
    Associativity holds in $G$, thus it holds in $H$.

    (\textbf{Rule 2})
    Since $H$ is nonempty, let $h \in H$.
    Then $hh^{-1} = 1 \in H$.
    Thus the multiplicative identity belongs to $H$.

    (\textbf{Rule 3})
    Let $h \in H$.
    Since $1, h \in H$, it follows that $1h^{-1} = h^{-1} \in H$.
\end{proof}

\begin{tcolorbox}[title=Problem 5, breakable]
    Show that if $H$ and $K$ are subgroups of the group $G$,
    then $H \cap K$ is also a subgroup of $G$. Show by example
    that $H \cup K$ need not be a subgroup. (This exercise 
    and should be compared to Exercises 7.9 and 7.10.)
\end{tcolorbox} 

\begin{proof}
    Suppose $H$ and $K$ are subroups of the group $G$.
    Now, $1 \in H$ and $1 \in K$ since they are subgroups 
        thus $1 \in H \cap K$.
    Therefore, $H \cap K$ is nonempty.
    Let $x, y \in H \cap K$.
    We know $x y^{-1} \in H$ and $x y^{-1} \in K$
        since $H, K$ are subgroups of $G$.
    Thus $x y^{-1} \in H \cap K$.
    By Theorem 35.2, $H \cap K$ is a subgroup of $G$.
\end{proof}

\begin{proof}
    Consider the group $\langle\mathbb{Z}, +\rangle$.
    Let $H = 2\mathbb{Z}$ and $K = 3\mathbb{Z}$.
    Then $H$ and $K$ are subgroups of $\mathbb{Z}$.

    But $2 \in H$ and $3 \in K$, and $2 + 3 = 5 \notin H \cup K$.
    Thus $H \cup K$ is not closed under addition and therefore is not a subgroup.
\end{proof}

\begin{tcolorbox}[title=Problem 6, breakable]
    Suppose that $G$ is a group, written multiplicatively.
    Let $g \in G$, and suppose that $g^2 = g$.
    Prove that $g$ is the identity.
\end{tcolorbox} 

\begin{proof}
    We have $g^2 = gg = g = g1$.
    Cancellation on the left shows $g = 1$, as required.
\end{proof}

\begin{tcolorbox}[title=Problem 7, breakable]
    Let $G$ be a group, and $a, b, c \in G$.
    Prove that the equation $axc = b$ has a unique solution in $G$.
\end{tcolorbox} 

\begin{proof}
    Notice $axc = b \iff axcc^{-1} = bc^{-1} \iff ax = bc^{-1} \iff a^{-1}ax = a^{-1}bc^{-1} \iff x = a^{-1}bc^{-1}$.
\end{proof}

\begin{tcolorbox}[title=Problem 8, breakable]
    \textcircled{1} Suppose that $G$ is equipped with an associative operation $*$.
    \textcircled{2} Suppose that $G$ has an element $e$ so that $g * e = g$,
        for all $g \in G$; \textcircled{3} futhermore, for all $g \in G$,
        there exists an element $g' \in G$, so that 
        $g * g' = e$. Why are these assumptions apparently weaker 
        than decreeing that $G$ be a group? Prove, however,
        that these assumptions are sufficient to force $G$ to 
        be a group.
\end{tcolorbox} 

\textbf{Solution: } A group requires the Rule 2 and 3 to commute.

\begin{proof}
    (\textbf{Rule 1}) It is given that $*$ is associative.

    (\textbf{Rule 2}) 
    Notice $g * g' = e \iff g * (g' * e) = e \textcircled{2} \iff (g * g') * e  = e \textcircled{1} \iff g * g' = e \textcircled{3}$.

    (\textbf{Rule 3}) 
\end{proof}

\begin{tcolorbox}[title=Problem 9, breakable]
    Show that if $(xy)^{-1} = x^{-1} y^{-1}$ for all $x$ and $y$
        in the group $G$. Then $G$ is abelian.
\end{tcolorbox} 

\begin{proof}
    Notice 
    \[(yx)(xy)^{-1} = (yx)(x^{-1} y^{-1}) = y(x x^{-1}) y^{-1} = y y^{-1} = 1,\]
    and 
    \[(xy)^{-1}(yx) = (x^{-1} y^{-1})(yx) = x^{-1} (y^{-1} y) x = x^{-1} x = 1.\]
    Thus $yx = \bigl((xy)^{-1}\bigr)^{-1} = xy$.
\end{proof}

\begin{tcolorbox}[title=Problem 10, breakable]
    Complete the following multiplicaiton table so that the following 
    will be a group.
    \[
    \begin{array}{c|cccc}
    & a & b & c & d \\ \hline
    a        &  &  &  &  \\
    b        &  &  &  & d \\
    c        &  &  & d &  \\
    d        &  &  &  &  \\
    \end{array}
    \]
\end{tcolorbox} 
\[
\begin{array}{c|cccc}
    & a & b & c & d \\ \hline
a & d & a & b & c \\
b & a & b & c & d \\
c & b & c & d & a \\
d & c & d & a & b
\end{array}
\]

\begin{tcolorbox}[title=Problem 12, breakable]
    Show that $n \mathbb{Z}$ is a subgroup of the additive group 
    of integers $\mathbb{Z}$, for all integers $n$.
\end{tcolorbox} 

\begin{proof}
    Clearly $n \mathbb{Z}$ is nonempty (consider $5n \in \mathbb{Z}$).
    Let $x, y$ be arbitrary elements in $n \mathbb{Z}$.
    Thus $x = k_1 n$ and $y = k_2 n$ for some $k_1, k_2 \in \mathbb{Z}$.
    Then $x - y = k_1 n - k_2 n = (k_1 - k_2)n \in n \mathbb{Z}$.
    Thus $n \mathbb{Z}$ is closed under subtraction.
    It follows from Theorem 35.2 that $n \mathbb{Z}$ is a subgroup of $\mathbb{Z}$.
\end{proof}

\begin{tcolorbox}[title=Problem 13, breakable]
    Find all finite subgroups of the additive group $\mathbb{C}$.
    What can you say about all finite subgroups of the multiplicative 
    group $\mathbb{C}^*$?
\end{tcolorbox} 

\textbf{Solution: } $\mathbb{C}$ has no nontrivial finite subgroups.
The elements in $\mathbb{C}^*$ formed using the roots of unity are finite subgroups.

\begin{tcolorbox}[title=Problem 14, breakable]
    Argue \emph{geometrically} that the dihedral group, $D_n$,
    has a subgroup of order $n$.
\end{tcolorbox} 

\begin{proof}
    There are $n$ rotations of the $n$-gon which form a subgroup of $D_n$.
\end{proof}

\begin{tcolorbox}[title=Problem 15, breakable]
    Let $G$ be a group and $a \in G$. Define the \textbf{centralizer}
    of $a$ to be 
    \[C(a) = \{g \in G \mid ga = ag\}\]
    That is, $C(a)$ consists of all elements that commute with $a$.

    (a) Find $C(\rho)$ in $D_3$.

    (b) Find $C(4)$ in $\mathbb{Z}_7$.

    (c) Show that $C(a)$ is a subgroup of $G$.

    (d) Let $H$ be a subgroup of $G$, and let 
        \[C(H) = \{g \in G \mid gh = hg \text{ for all } h \in H\}\]
        call $C(H)$ the \textbf{centralizer} of $H$. 
        Show that $C(H)$ is a subgroup of $G$.
\end{tcolorbox} 

\textbf{Solution (a):}
\[\{e,\rho,\rho^2\}\]

\textbf{Solution (b):}
\[0, 1, 2, 3, 4, 5, 6\]

\begin{proof}
    Let $h, k \in C(a)$.
    Let $g = a$.
    Then $hk^{-1}g = hgk^{-1} = gh k^{-1}$.
    Thus $h k^{-1} \in C(a)$.
    Therefore $C(a)$ is a subgroup of $G$.
\end{proof}

\begin{proof}
    Let $h, k \in C(H)$.
    Let $g$ be an arbitrary element in $H$.
    Then $hk^{-1}g = hgk^{-1} = gh k^{-1}$.
    Thus $h k^{-1} \in C(H)$.
    Therefore $C(H)$ is a subgroup of $G$.
\end{proof}

\begin{tcolorbox}[title=Problem 16, breakable]
    Let $Z(G)$, the \textbf{center of} $G$, be the set of elements of $G$
        that commute with all elements of $G$.

    (a) Find the center of the quaternions, defined in Example 24.15.

    (b) Find the center of $\mathbb{Z}_5$.

    (c) Show that $Z(G)$ is a subgroup of $G$.

    (d) If $Z(G) = G$, what can you say about the group $G$.
\end{tcolorbox}

\begin{tcolorbox}[title=Problem 17, breakable]
    If $H$ is a subgroup of $G$, then show that $Z(G) \cap H$
        is a subgroup of $Z(H)$.
\end{tcolorbox} 

\begin{tcolorbox}[title=Problem 23, breakable]
    Generalize the situtation in the previous two exercises,
        replacing $2$ and $3$ by some positive integer $m$.
\end{tcolorbox} 