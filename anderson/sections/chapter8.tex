\begin{tcolorbox}[title=Problem 1, breakable]
    Prove that if $R$ is a commutative ring 
    and $a \in R$ is a zero divisor, then $ax$
    is also a zero divisor or $0$, for all $x \in R$.
\end{tcolorbox} 

\begin{proof}
    Suppose $R$ is a commutative ring and $a \in R$ is a zero divisor.
    Then there exists $b \in R$ with $b \ne 0$ such that $ab = 0$.
    Now, let $x$ be an arbitrary element of $R$.
    If $ax = 0$, then we are done.
    Suppose $ax \ne 0$.
    Then $(ax)b = a(xb) = (ab)x = 0$.   
    Since $b \ne 0$, it follows that $ax$ is a zero divisor.
\end{proof}

\begin{tcolorbox}[title=Problem 6, breakable]
    Use Fermat's Little Theorem 8.7 to find 
    $[6]^{-1}$ in $\mathbb{Z}_{19}$.
\end{tcolorbox} 

\begin{proof}
    In $\mathbb{Z}_{19}$ the theorem asserts that 
        $[6]^{18} = [1]$.
    But then $[6][6]^{17} = [1]$ so $[6]^{-1} = [6]^{17} = [16]$.
\end{proof}

\begin{tcolorbox}[title=Problem 7, breakable]
    Use Euclid's Algorithm to find $[36]^{-1}$ in $\mathbb{Z}_{101}$.
\end{tcolorbox} 

\begin{proof}
    First notice $1 = 5 \cdot 101 - 14 \cdot 36$.
    Thus $-14 \cdot 36 = 1 \pmod 101$.
    Therefor $[36]^{-1} = [-14] = [87]$.
\end{proof}

\begin{tcolorbox}[title=Problem 9, breakable]
    Suppose that $b \in R$, a non-commutative ring 
    with unity. Suppose that $ab = bc = 1$;
    that is, $b$ has a \textbf{right inverse} $c$
    and a \textbf{left inverse} $a$. Prove that 
    $a = c$ and that $b$ is a unit.
\end{tcolorbox} 

\begin{proof}
    Now $a = a \cdot 1 = a(bc) = (ab)c = 1 \cdot c = c$ as required.
    It directly follows that $b$ is a unit.
\end{proof}

\begin{tcolorbox}[title=Problem 11, breakable]
    Let $R$ be a commutative ring with unity.
    Suppose that $n$ is the least postivive integer 
    for which we get $0$ when we add $1$ to itself $n$
    times; we then say $R$ has a \textbf{characteristic n}.
    If there exists no such $n$, we say that $R$ has 
    \textbf{characteristic 0}. For example, the 
    characteristic of $\mathbb{Z}_5$ is $5$ because 
    $1 + 1 + 1 + 1 + 1 = 0$, whereas 
    $1 + 1 + 1 + 1 \ne 0$. (Note that here we 
    have suppressed `[' and `]'.)

    (a) Show that, if the characteristic of a commutative ring with unity $R$ 
        is $n$ and $a$ is \emph{any} of $R$, then $na = 0$. (Recall that 
        $na = \underbrace{a + a + \cdots + a}_{n \text{ times}}$.)

    (b) What are the characteristics of $\mathbb{Q}, \mathbb{R}, \mathbb{Z}_{17}$?

    (c) Prove that if a field $F$ has characteristic $n$, where $n > 0$, 
        then $n$ is a prime integer.
\end{tcolorbox} 

\begin{proof}
    Notice
    \[
    na = \underbrace{a + a + \cdots + a}_{n \text{ times}}
       = a(\underbrace{1 + 1 + \cdots + 1}_{n \text{ times}})
       = a \cdot 0
       = 0.
    \]
\end{proof}

\textbf{Solution:} $\mathbb{Q}$ and $\mathbb{R}$ have characteristic $0$.
Finally, $\mathbb{Z}_{17}$ has characteristic $17$.

\begin{proof}
    Suppose $F$ is a field with characteristic $n$ where $n > 0$.
    For contradiction, suppose $n$ is not prime, so $n = ab$ for some
    $a,b$ with $1 < a,b < n$.
    Since $F$ is a field, it has no zero divisors, so
    $
    0 = n \cdot 1 = (a \cdot 1)(b \cdot 1)
    $.
    Now either $a \cdot 1 = 0$ or $b \cdot 1 = 0$ either of which is a contradiction.
\end{proof}

\begin{tcolorbox}[title=Problem 12, breakable]
    Consider the commutative ring $F = \{0, 1, \alpha, 1 + \alpha\}$,
    where $0$ is the additive identity, $1$ is the mutliplicative identity,
    $x + x = 0$, for all $x \in F$, and $\alpha^2 = 1 + \alpha$.

    (a) Write out explicitly the addition and mutliplication tables for $F$.

    (b) Prove that $F$ is a field.

    (c) Because $F$ has four elements, you might expect $F$ would be the ``same''
        as the ring $\mathbb{Z}_4$. Show this is false, by computing the characteristics 
        of $F$ and $\mathbb{Z}_4$ (see the previous exercise).
\end{tcolorbox} 

\textbf{Solution (a):}
\[
\begin{array}{c|cccc}
+ & 0 & 1 & \alpha & 1+\alpha \\
\hline
0 & 0 & 1 & \alpha & 1+\alpha \\
1 & 1 & 0 & 1+\alpha & \alpha \\
\alpha & \alpha & 1+\alpha & 0 & 1 \\
1+\alpha & 1+\alpha & \alpha & 1 & 0
\end{array}
\]
\[
\begin{array}{c|cccc}
\cdot & 0 & 1 & \alpha & 1+\alpha \\
\hline
0 & 0 & 0 & 0 & 0 \\
1 & 0 & 1 & \alpha & 1+\alpha \\
\alpha & 0 & \alpha & 1+\alpha & 1 \\
1+\alpha & 0 & 1+\alpha & 1 & \alpha
\end{array}
\]

\begin{proof}
    Notice $1 \cdot 1 = 1$ and
    \[
    \alpha \cdot (1+\alpha) = \alpha + \alpha^2 = \alpha + (\alpha+1) = 1.
    \]
    Since every nonzero element has a multiplicative inverse, $F$ is a field.
\end{proof}

\textbf{Solution (c):}
In $F$, $1 + 1 = 0$ but in $\mathbb{Z}_4$, $1 + 1 + 1 + 1 = 0$ and $1 + 1 \ne 0$. So they are not the same ring.

\begin{tcolorbox}[title=Problem 14, breakable]
    Prove that $\mathbb{Z}_m$ is the union of three mutually disjoint subsets: its 
    zero divisors, its units, and $\{0\}$. Show by example that this is false 
    for an arbitrary commutative ring.
\end{tcolorbox} 

\begin{proof}
    $\{0\}$ is clearly not a zero divisor or a unit in $\mathbb{Z}_m$.
    Now, suppose $a \in \mathbb{Z}_m$ is a zero divisor and a unit.
    Then there exists $b \ne 0$ such that $ab = 0$.
    But then $a^{-1} ab = a^{-1} 0 \implies 1 \cdot b = 0 \implies b = 0$,
        a contradiction.

    Now suppose $a \in \mathbb{Z}_m$ and $a \ne 0$.
    If $\gcd(a, m) = 1$ then $a$ is a unit.
    Suppose $\gcd(a, m) = d > 1$.
    Then, there exists $k_1, k_2 \in \mathbb{Z}$ such that $a = d k_1$ and $m = d k_2$.
    Then $a = d k_1 \iff a k_2 = d k_2 k_1 = m k_1 \equiv 0 \pmod m$.
    Thus $a$ is either a unit or a zero divisor.

    A counterexample is the trivial ring.
\end{proof}