\begin{tcolorbox}[title=Problem 2, breakable]
    Prove that all ideals in $\mathbb{Q}[x]$ are principal,
        using a similar proof to that for $\mathbb{Z}$ (Theorem 11.6).
\end{tcolorbox} 

\begin{proof}
    Suppose that $I$ is an ideal of $\mathbb{Q}[x]$.
    We wish to show that $I$ is principal.
    If $I$ is the zero ideal this is already obvious,
        so suppose $I$ has more elements than just the zero polynomial.
    Let $f$ be a non-zero polynomial with smallest degree in the ideal.
    We claim that $\langle f \rangle = I$.
    Because $I$ is an ideal, it is clear that $\langle f \rangle \subseteq I$.
    Suppose now that $g \in I$. 
    We apply the Division Theorem to obtain quotient $q$ and remainder $r$
        such that $g = q f + r$ with either $r = 0$ or $\deg(r) < \deg(f)$.
    Since $r \in I$ and $f$ has minimal degree, we must have $r = 0$.
    Therefore $g \in \langle f \rangle$.
    Thus for $\mathbb{Q}[x]$ all ideals are principal.
\end{proof}

\begin{tcolorbox}[title=Problem 3, breakable]
    Consider the set 
    \[I = \{f \in \mathbb{Q}[x] \mid f(i) = 0\}\]
    (Here, $i$ is the usual complex number.)
    \begin{enumerate}
        \item Prove that $I$ is ideal.
        \item We know that $I$ is a principal ideal (why?). 
              Find a generator for $I$, and prove that it works.
              [Hint: The generator should be an element in $I$ with the smallest degree greater than $0$.]
    \end{enumerate}
\end{tcolorbox} 

\begin{proof}
    Suppose $f, g \in I$.  
    Then $(f - g)(i) = f(i) - g(i) = 0 - 0 = 0$, so $f - g \in I$.  
    Similarly $(fg)(i) = f(i)g(i) = 0 g(i) = 0$. so $fg \in I$.
    Thus $I$ is an ideal.
\end{proof}

\begin{proof}
    Consider the polynomial $x^2 + 1$.  
    Now suppose $f \in I$. By the Division Theorem, there exist $q, r \in \mathbb{Q}[x]$ such that  
    \[
    f = q(x^2 + 1) + r \quad \text{with } \deg(r) < 2.
    \]  
    Since no degree $1$ polynomial with rational coefficients has $i$ as a root, we must have $r = 0$.  
    Thus $f = q(x^2 + 1)$, so $x^2 + 1$ divides every $f \in I$.  
    Thus $I = \langle x^2 + 1 \rangle$ is principal with generator $x^2 + 1$.
\end{proof}

\newpage
\begin{tcolorbox}[title=Problem 4, breakable]
    Consider the set 
    \[I = \{f \in \mathbb{C}[x] \mid f(i) = 0\}\]
    Repeat Exercise 3 for this set.
\end{tcolorbox} 

\begin{proof}
    Suppose $f, g \in I$.  
    Then $(f - g)(i) = f(i) - g(i) = 0 - 0 = 0$, so $f - g \in I$.  
    Similarly $(fg)(i) = f(i)g(i) = 0 \cdot g(i) = 0$, so $fg \in I$.  
    Thus $I$ is an ideal.
\end{proof}

\begin{proof}
    Consider the polynomial $x - i$.  
    Now suppose $f \in I$. By the Division Theorem, there exist $q, r \in \mathbb{C}[x]$ such that  
    \[
    f = q(x - i) + r \quad \text{with } \deg(r) < 1.
    \]  
    Since no degree $0$ polynomial (constant) with complex coefficients can have $i$ as a root unless it is $0$, we must have $r = 0$.  
    Thus $f = q(x - i)$, so $x - i$ divides every $f \in I$.  
    Thus $I = \langle x - i \rangle$ is principal with generator $x - i$.
\end{proof}

\begin{tcolorbox}[title=Problem 9, breakable]
    Consider 
    \[
    I = \left\{
    \begin{bmatrix}
        a & 0 \\[2mm] 
        b & 0
    \end{bmatrix} \in M_2(\mathbb{Z}) \;\middle|\; a, b \in \mathbb{Z} 
    \right\}.
    \]
    \begin{enumerate}
        \item Show that $I$ is a subring of $M_2(\mathbb{Z})$.
        \item Show that $I$ is \emph{not} an ideal of $M_2(\mathbb{Z})$ (using the stronger 
                definition of ideal used for non-commutative rings).
    \end{enumerate}
\end{tcolorbox} 

\begin{proof}
    Let 
    \[\begin{bmatrix}
        a & 0 \\[2mm] 
        b & 0
    \end{bmatrix}, \begin{bmatrix}
        c & 0 \\[2mm] 
        d & 0
    \end{bmatrix} \in I.\]
    Then 
    \[\begin{bmatrix}
        a & 0 \\[2mm] 
        b & 0
    \end{bmatrix} - \begin{bmatrix}
        c & 0 \\[2mm] 
        d & 0
    \end{bmatrix} = \begin{bmatrix}
        a - c & 0 \\[2mm] 
        b - d & 0
    \end{bmatrix} \in I.\]
    Similarly,
    \[\begin{bmatrix}
        a & 0 \\[2mm] 
        b & 0
    \end{bmatrix} \times \begin{bmatrix}
        c & 0 \\[2mm] 
        d & 0
    \end{bmatrix} = \begin{bmatrix}
        ac & 0 \\[2mm] 
        bc & 0
    \end{bmatrix} \in I.\]
    Thus $I$ is a subring of $M_2(\mathbb{Z})$.
\end{proof}

\textbf{Solution: }
\[\begin{bmatrix}
    a & 0 \\[2mm] 
    b & 0
\end{bmatrix} \times \begin{bmatrix}
    c & 1 \\[2mm] 
    d & 0
\end{bmatrix} = \begin{bmatrix}
    ac & a \\[2mm] 
    bd & b
\end{bmatrix} \notin I.\]

\newpage
\begin{tcolorbox}[title=Problem 11, breakable]
    Suppose $p, q$ are distinct prime integers in $\mathbb{Z}$.
    Prove that 
    \[\langle p \rangle \cap \langle q \rangle = \langle pq \rangle.\]
\end{tcolorbox} 

\begin{proof}
    Suppose $x \in \langle p \rangle \cap \langle q \rangle$.
    Then $x = p k_1$ and $x = q k_2$ for $k_1, k_2 \in \mathbb{Z}$.
    Since $p, q \mid x$ and $\gcd(p, q) = 1$ we have $pq \mid x$ thus $x \in \langle pq \rangle$.

    Suppose $x \in \langle pq \rangle$.
    Then $x = p q k$ for some $k \in \mathbb{Z}$.
    Since $x = p(qk)$ and $x = q(pk)$ we have $x \in \langle p \rangle \cap \langle q \rangle$.
\end{proof}

\begin{tcolorbox}[title=Problem 12, breakable]
    Let $R$ be a commutative ring, and suppose that $I$ and $J$ are 
        ideals. Prove that $I \cap J$ is an ideal.
    (Note the many examples of intersections of ideals provided in Exercise 11.
    Also compare this exercise to Exercise 7.9.) Now describe the ideal in Exercise 5 
    as an intersection of two proper ideals.
\end{tcolorbox} 

\textbf{Solution: }
\[\langle x^2 - 3 \rangle \cap \langle x - 3 \rangle = \langle (x^2 - 3)(x - 3) \rangle\]

\begin{proof}
    Let $x, y$ be arbitrary elements in $I \cap J$.
    Let $t$ be an arbitrary element in $R$.
    Since $x,y \in I \cap J$ both of which are closed under subtraction,
        $x - y \in I \cap J$.
    Similarly, $xt \in I \cap J$.
    Thus $I \cap J$ is an ideal.
\end{proof}

\begin{tcolorbox}[title=Problem 13, breakable]
    Suppose that $R$ is a commutative ring, $I$ and $J$ are ideal of $R$,
    and $I \subseteq J$. Prove that $I$ is an ideal of the ring $J$.
\end{tcolorbox} 

\begin{proof}
    Let $i$ be an arbitrary element in $I$ and $j$ an arbitrary element in $J$.
    Since $I$ is an ideal it is closed under subtraction.
    Now $i \cdot j \in I$ since $I$ is an ideal of $R$ and $j \in R$.
    Thus $I$ is an ideal of the ring $J$.
\end{proof}

\begin{tcolorbox}[title=Problem 14, breakable]
    Let $R$ be a commutative ring, with ideals $I$ and $J$.
    Let 
    \[I + J = \{a + b \mid a \in I, b \in J\}\]
    Prove that $I + J$ is an ideal of $R$.
    For obvious reasons, we call the ideal $I + J$ the \textbf{sum}
    of $I$ and $J$.
\end{tcolorbox} 

\begin{proof}
    Let $x, y$ be arbitrary elements in $I + J$.
    Let $t$ be an arbitrary element in $R$.
    Let $x = a_1 + b_1, y = a_2 + b_2$ for some $a_1, a_2 \in I$ and $b_1, b_2 \in J$.
    Then 
    \[x - y = (a_1 - a_2) + (b_1 - b_2) \in I + J \quad \text{I, J \text{ are closed under subtraction.}}\]
    Similarly 
    \[tx = t(a_1 + b_1) = ta_1 + tb_1 \in I + J \quad \text{I, J \text{ are closed under multiplication for $x \in R$.}}\]
    Thus $I + J$ is an ideal of $R$.
\end{proof}

\begin{tcolorbox}[title=Problem 15, breakable]
    Suppose that $R$ is a commutative ring with unity, and $I$ and $J$
    are ideals. Define 
    \[I \cdot J = \left\{\sum_{k = 1}^{n} a_k b_k \mid a_k \in I, b_k \in J, n \in \mathbb{N}\right\}.\]
    (That is, $I \cdot J$ consists of all possible finite sums of products of elements from $I$ and $J$.)
    \begin{enumerate}
        \item Prove that $I \cdot J$ is an ideal. We call the ideal $I \cdot J$ the \textbf{product} of the ideals $I$ and $J$.
        \item Prove that $I \cdot J \subseteq I \cap J$.
        \item Prove that if $a, b \in R$, we have that $\langle a \rangle \cdot \langle b \rangle = \langle ab \rangle$.
        \item Show by example in $R = \mathbb{Z}$ that we can have $I \cdot J \subseteq I \cap J$.
    \end{enumerate}
\end{tcolorbox} 

\begin{proof}
    Let $c \in R$.  
    Suppose $x, y \in I \cdot J$ such that 
    \[
    x = \sum_{k = 1}^{n_1} a_k b_k \quad \text{and} \quad 
    y = \sum_{j = 1}^{n_2} a'_j b'_j, 
    \quad a_k, a'_j \in I, \ b_k, b'_j \in J, \ n_1, n_2 \in \mathbb{N}.
    \]
    Let
    \[
    c_i =
    \begin{cases}
    a_i, & 1 \le i \le n_1, \\
    - a'_{\,i-n_1}, & n_1 < i \le n_1 + n_2,
    \end{cases}
    \qquad
    d_i =
    \begin{cases}
    b_i, & 1 \le i \le n_1, \\
    b'_{\,i-n_1}, & n_1 < i \le n_1 + n_2.
    \end{cases}
    \]
    Then 
    \[
    x - y
    = \sum_{k = 1}^{n_1} a_k b_k - \sum_{j = 1}^{n_2} a'_j b'_j
    = \sum_{i = 1}^{n_1 + n_2} c_i d_i \in I \cdot J.
    \]
    Similarly,
    \[
    cx = c \sum_{k = 1}^{n_1} a_k b_k = \sum_{k = 1}^{n_1} (c a_k) b_k \in I \cdot J,
    \]
    since $c a_k \in I$ as $I$ is closed under multiplication by elements of $R$.
    Thus $I \cdot J$ is an ideal.
\end{proof}

\begin{proof}
    Suppose $x \in I \cdot J$ such that 
    \[
        x = \sum_{i = 1}^{n} a_i b_i, \quad a_i \in I,\ b_i \in J.
    \]
    Viewing each term in the sum, we see $a_i b_i \in I$ since $I$ is closed under multiplication by elements of $R$.
    Since $I$ is closed under subtraction, the sum is in $I$, thus $x \in I$.
    Similarly, $x \in J$.
    Therefore $x \in I \cap J$, and thus $I \cdot J \subset I \cap J$.
\end{proof}

\begin{proof}
    Let $x$ be an arbitrary element in $\langle a \rangle \cdot \langle b \rangle$ such that 
    \[
        x = \sum_{i = 1}^{n} a_i b_i = \sum_{i = 1}^{n} k_i a k'_i b = ab \sum_{i = 1}^{n} k_i k'_i, 
        \quad a_i \in \langle a \rangle, \ b_i \in \langle b \rangle, \ n \in \mathbb{N}, k_i, k'_i \in R.
    \]
    Thus $x \in \langle ab \rangle$.

    Let $x$ be an arbitrary element in $\langle ab \rangle$.
    Then $x = k ab$ for some $k \in R$.
    Then $x = k a \cdot b = \sum_{i = 1}^{1} (k a) b \in \langle a \rangle \cdot \langle b \rangle$ by Part 1.
\end{proof}

\textbf{Solution (4): }
Take $I = 2 \mathbb{Z}$ and $J = 3 \mathbb{Z}$ then $I \cdot J = 6 \mathbb{Z}$
    and $I \cap J = 6 \mathbb{Z}$ thus $I \cdot J \subseteq 6 \mathbb{Z}$.

\begin{tcolorbox}[title=Problem 16, breakable]
    Suppose that $I, J, K$ are ideals in $R$, a commutative ring with unity.
    Prove that $I \cdot (J + K) = I \cdot J + I \cdot K$.
\end{tcolorbox} 

\begin{proof}
    Let $x$ be an arbitrary element in $I \cdot (J + K)$ such that 
    \[
        x = \sum_{i = 1}^{n} a_i b_i, \quad a_i \in I, \ b_i \in J + K, \ n \in \mathbb{N}.
    \]
    Then each $b_i = j_i + k_i$ such that $j_i \in J$ and $k_i \in K$. Then
    \[
        x = \sum_{i = 1}^{n} a_i (j_i + k_i) = \sum_{i = 1}^{n} a_i j_i + \sum_{i = 1}^{n} a_i k_i \in I \cdot J + I \cdot K.
    \]
    Thus $I \cdot (J + K) \subseteq I \cdot J + I \cdot K$.
    Conversely, let $x$ be an arbitrary element in $I \cdot J + I \cdot K$ such that 
    \[
        x = \sum_{i = 1}^{n_1} a'_i j'_i + \sum_{i = 1}^{n_2} a''_i k'_i, \quad 
        a'_i, a''_i \in I, \ j'_i \in J, \ k'_i \in K.
    \]
    Then $a'_i j'_i = a'_i (j'_i + 0) \in I \cdot (J + K)$ and $a''_i k'_i = a''_i (0 + k'_i) \in I \cdot (J + K)$.  
    Therefore $x \in I \cdot (J + K)$.
\end{proof}

\begin{tcolorbox}[title=Problem 17, breakable]
    Let $X$ be a set; consider the power set ring $\mathcal{P}(X)$, described in
        Exercise 6.20. Pick a fixed subset $a$ of $X$, and let 
        \[I = \{b \in \mathcal{P}(X) \mid b \subseteq a\}.\]
    \begin{enumerate}
        \item Prove that $I$ is an ideal of $\mathcal{P}(X)$.
        \item Prove that $I$ is a principal ideal (find the generator!).
    \end{enumerate}
\end{tcolorbox} 

\begin{proof}
    Let $x, y$ be arbitrary elements in $I$.
    Let $t$ be an arbitrary element in $\mathcal{P}(X)$.
    Then $x + y = (x \cup y) \setminus (x \cap y)$.
    Since $x \subset a$ and $y \subset a$, $x \cup y \subset a$ then $-$
        only removes elements thus $x + y \in I$.
    Then $xt = x \cap t \subset x \subset a$ thus $xt \in I$.
    It follows that $I$ is an ideal of $\mathcal{P}(X)$.
\end{proof}

\begin{proof}
    The ideal $I$ is principal, generated by $a$
    \[
        I = \langle a \rangle = \{ b \subseteq X \mid b \subseteq a \}.
    \]  
    Every subset of $a$ can be written as $b \cap a = b$ for some $b \in \mathcal{P}(X)$, so $\langle a \rangle = I$.
\end{proof}

\newpage
\begin{tcolorbox}[title=Problem 18, breakable]
    Prove that the nilradical of a commutative ring with unity is an ideal. (See Exercise 7.15, where you proved it is a subring.)
\end{tcolorbox} 

\begin{proof}
    Let $R$ be a commutative ring with unity.
    Suppose $\mathcal{N}(P)$ be the nilpotent elements in $R$.
    Let $x, y$ be arbitrary elements in $\mathcal{N}(P)$.
    Suppose $x^n = 0$ and $y^m = 0$.
    Let $t$ be an arbitrary element in $R$.
    Then
    Consider $x - y$. Using the binomial theorem in a commutative ring
    \[
        (x - y)^{n + m} = \sum_{k=0}^{n+m} \binom{n+m}{k} x^k (-y)^{\,n+m-k} = 0.
    \]
    Thus $x - y \in \mathcal{N}(P)$.
    Then $(cx)^n = c^n x^n = c^n \cdot 0 = 0$.
    Thus $cx \in \mathcal{N}(P)$.
    It follows that $\mathcal{N}(P)$ is an ideal.
\end{proof}

\begin{tcolorbox}[title=Problem 21, breakable]
    Generalize Exercise 19: Suppose that $R$ is a commutative ring with unity, and $I$ is an ideal.
    Let 
    \[A(I) = \{s \in R \mid rs = 0, \text{ for all } r \in I\}.\]
    we call $A(I)$ the \textbf{annihilator} of $I$. Prove that $A(I)$ is an ideal.
\end{tcolorbox} 

\begin{proof}
    Let $x, y \in A(I)$ and $r \in R$ be arbitrary.  
    For any $s \in I$, we have
    \[
        s(x - y) = sx - sy = 0 - 0 = 0,
    \]
    so $x - y \in A(I)$.  
    Now, for any $r \in R$ and $x \in A(I)$, we have
    \[
        s(rx) = (sr)x = r(sx) = r \cdot 0 = 0
    \]
    for all $s \in I$, since $R$ is commutative.  
    Thus $rx \in A(I)$.  
    Therefore, $A(I)$ is an ideal of $R$.
\end{proof}

\newpage
\begin{tcolorbox}[title=Problem 23, breakable]
    Generalize Theorem 11.5: Suppose that $R$ is a commutative ring with unity and $r, s \in R$,
    with $r$ not a zero divisor. Prove that $\langle r \rangle = \langle s \rangle$ if and only if $r, s$ are associates.
\end{tcolorbox} 

\begin{proof}
    Suppose $\langle r \rangle = \langle s \rangle$.  
    Then $s \in \langle r \rangle$ so there exists $u \in R$ such that $s = u r$.
    Similarly, $r \in \langle s \rangle$ so there exists $v \in R$ such that 
    \[
        r = v s = v (u r) = (v u) r.
    \]  
    Since $r$ is not a zero divisor, we can cancel $r$ to get $v u = 1$.  
    Thus $u$ is a unit so $r$ and $s$ are associates.  
    Conversely, suppose $r$ and $s$ are associates thus $s = u r$ for some unit $u \in R$.  
    Then 
    \[
        \langle s \rangle = \langle u r \rangle = \langle r \rangle
    \]  
    because multiplying by a unit does not change the principal ideal.  
\end{proof}
