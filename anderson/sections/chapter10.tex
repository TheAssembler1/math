\begin{tcolorbox}[title=Problem 1, breakable]
    Find all simultaneous solutions to the Diophantine
    equations $ac - bd = 1$, $ad + bc = 0$ directly,
    by eliminating variables, interpret your solutions 
    as determining all units in the Gaussian integers.
\end{tcolorbox} 

\begin{proof}
    Now, if $c = 0$ then $ad = 0$ implies $a = 0$ or $d = 0$.
    Suppose $a = 0$ then $-bd = 1$ so $b = 1,\ d = -1$
    or $b = -1,\ d = 1$.
    Suppose $d = 0$ then $ac = 1$ so $a = 1,\ c = 1$
    or $a = -1,\ c = -1$.

    Now, suppose $c \ne 0$.
    Then $ad + bc = 0 \iff bc = -ad \iff b = -\frac{ad}{c}$.
    Plugging in we find
    $$
    ac - \left(-\frac{ad}{c}\right)d = 1
    \iff ac + \frac{ad^2}{c} = 1
    \iff a(c^2 + d^2) = c
    \iff a = \frac{c}{c^2 + d^2}.
    $$
    Which only has solutions $c^2 + d^2 = 1$.

    Thus the solutions are $a = \pm 1,\ b = 0,\ c = \pm 1,\ d = 0$
    or $a = 0,\ b = \pm 1,\ c = 0,\ d = \pm 1$.
\end{proof}

\begin{tcolorbox}[title=Problem 2, breakable]
    Prove Theorem 10.1. That is, let $n$, be a square-free integer.
    As in the text, define $N(a + b\sqrt{n}) = |a^2 - nb^2|$.
    Prove that $N$ preserves multiplication, that is,
    $N(\alpha \beta) = N(\alpha) N(\beta)$.
\end{tcolorbox} 

\begin{proof}
    Let $\alpha = a + b\sqrt{n}$ and $\beta = c + d\sqrt{n}$.
    Then
    \[
    N(\alpha \beta)
        = N((a + b\sqrt{n})(c + d\sqrt{n}))
        = N(ac + ad\sqrt{n} + bc\sqrt{n} + bdn)
    \]
    \[
        = N((ac + bdn) + (ad + bc)\sqrt{n})
        = |(ac + bdn)^2 - n(ad + bc)^2|.
    \]
    Similarly
    \[
    N(\alpha)N(\beta)
        = N(a + b\sqrt{n})N(c + d\sqrt{n})
        = |a^2 - nb^2||c^2 - nd^2|.
    \]
    Then
    \[
    (ac + bdn)^2 - n(ad + bc)^2
    = a^2c^2 + 2abcd\,n + b^2d^2n^2
    - n(a^2d^2 + 2abcd + b^2c^2),
    \]
    which simplifies to
    \[
    a^2c^2 - na^2d^2 - nb^2c^2 + n^2b^2d^2.
    \]
    Then
    \[
    (a^2 - nb^2)(c^2 - nd^2)
    = a^2c^2 - na^2d^2 - nb^2c^2 + n^2b^2d^2.
    \]
    Thus $N(\alpha\beta) = N(\alpha)N(\beta)$.
\end{proof}

\begin{tcolorbox}[title=Problem 3, breakable]
    Suppose that $n, m$ are distinct square-free integers.
    Prove that $\mathbb{Z}[\sqrt{n}] \cap \mathbb{Z}[\sqrt{m}] = \mathbb{Z}$.
    This is not true if at least one of the integers $n$ and $m$ is not 
    square-free. Give an example to show this.
\end{tcolorbox} 


\begin{theorem}
    Suppose $n, m$ are square free integers such that $n \ne m$ and $a, b \in \mathbb{Z}$.
    Futhermore, suppose $a \ne 0, b \ne 0$ and $x = a\sqrt{n} - b\sqrt{m}$.
    Then $x$ is an irrational number.
\end{theorem}

\begin{proof}
    For contradiction, suppose $x$ is rational.
    Then $x = a\sqrt{n} - b \sqrt{m} \iff x + b\sqrt{m} = a \sqrt{n} \iff \frac{x + b\sqrt{m}}{a} = \sqrt{n}$.
    Now squaring both sides we find $\frac{x^2 + 2xb\sqrt{m} + b^2 m}{a^2} = n \iff x^2 + 2xb\sqrt{m} + b^2 m = a^2 n \iff 2xb\sqrt{m} = a^2 n - x^2 - b^2 m$.
    Now the rhs is an integer but the lhs is irrational which is a contradiction.
\end{proof}

\begin{proof}
    Let $x$ be an arbitrary element in $\mathbb{Z}[\sqrt{n}] \cap \mathbb{Z}[\sqrt{m}]$.
    Then $x = a + b\sqrt{n}$ and $x = c + d\sqrt{m}$ for some $a, b, c, d \in \mathbb{Z}$.
    Then $a + b\sqrt{n} = c + d\sqrt{m} \iff b\sqrt{n} - d\sqrt{m} = c - a$. 
    Now the lhs is irrational unless $b = d = 0$.
    Thus $a = c$ and $x \in \mathbb{Z}$.
    Conversely, any integer $z \in \mathbb{Z}$ is in both $\mathbb{Z}[\sqrt{n}]$ and $\mathbb{Z}[\sqrt{m}]$,
    so $\mathbb{Z} \subseteq \mathbb{Z}[\sqrt{n}] \cap \mathbb{Z}[\sqrt{m}]$.
    Thus, $\mathbb{Z}[\sqrt{n}] \cap \mathbb{Z}[\sqrt{m}] = \mathbb{Z}$.
\end{proof}

\begin{proof}
    Let $n = 2$ and $m = 8$.
    Then $\sqrt{8} = 2\sqrt{2}$ and thus $\mathbb{Z}[\sqrt{2}] = \mathbb{Z}[\sqrt{8}]$.
    Therefore,
    \[
        \mathbb{Z}[\sqrt{2}] \cap \mathbb{Z}[\sqrt{8}] = \mathbb{Z}[\sqrt{2}] \neq \mathbb{Z}.
    \]
\end{proof}

\begin{tcolorbox}[title=Problem 4, breakable]
    Prove that the Diophantine equation $2b^2 = a^2 + 3$ has no 
        integer solutions, proceeding similarly as the problem 
        $a^2 = 2b^2 + 3$ is handled in the text in Example 10.16.
\end{tcolorbox} 

\begin{proof}
    Now $2b^2 = a^2 + 3 \iff a^2 = 2b^2 - 3$, thus $a$ is odd
        and can be written as $2k_1 + 1$ with $k_1 \in \mathbb{Z}$.
    Then $(2k_1 + 1)^2 = 2b^2 - 3 \iff 4 k_1^2 + 4k_1 + 1 = 2b^2 - 3 \iff 2(2k_1^2 + 2k_1 + 2) = 2b^2$.
    Thus $b^2 = 2k_1^2 + 2k_1 + 2$, so $b$ is even and can be written as $2k_2$ for some $k_2 \in \mathbb{Z}$.
    Substituting, we obtain $2(2k_2)^2 = a^2 + 3 \iff 2k_2^2 = k_1^2 + 1$.
    But this equation has the same form as the original with smaller positive integers.
    Repeating this process contradicts the well-ordering principle for positive integers,
     and thus there are no integer solutions.
\end{proof}

\begin{tcolorbox}[title=Problem 5, breakable]
    Find infinitely many distinct units in $\mathbb{Z}[\sqrt{7}]$. 
    Then list infinitely many associates of $\sqrt{7}$ in $\mathbb{Z}[\sqrt{7}]$.
\end{tcolorbox} 

\begin{proof}
    Let $\alpha = 8 + 3\sqrt{7}$ and note that $N(\alpha) = 1$,
        thus $\alpha$ is a unit by Theorem 10.2.
    To find infinitely many additional units simply consider $\alpha^n$
        where $n \ge 1$.
    To find infinitely many associates of $\sqrt{7}$,
        consider $\alpha^n \sqrt{7}$ where $n \ge 1$.
\end{proof}

\begin{tcolorbox}[title=Problem 6, breakable]
    Suppose that $n$ is a square-free integer and $n > 0$.
    Prove that $\mathbb{Z}[\sqrt{-n}]$ has only finitely many units.
\end{tcolorbox} 

\begin{proof}
    Since $-n < 0$, for $\alpha \in \mathbb{Z}[\sqrt{-n}]$ with $\alpha = a + b\sqrt{-n}$, we have
        $N(\alpha) = a^2 + n b^2$.
    Suppose there are infinitely many units.
    Then there would be infinitely many integer solutions 
        to the equation $N(\alpha) = a^2 + n b^2 = 1$ with $a, b \in \mathbb{Z}$.
    Clearly, this is not possible.
\end{proof}

\begin{tcolorbox}[title=Problem 7, breakable]
    Show that there are no irreducible elements in $\mathbb{Z}_6$.
\end{tcolorbox} 

\begin{proof}
    The units in $\mathbb{Z}_6$ are $1$ and $5$.
    But then 
    \begin{enumerate}
        \item $2 = 2 \cdot 1 = 4 \cdot 2$.
        \item $3 = 3 \cdot 3$.
        \item $4 = 2 \cdot 2$.
    \end{enumerate}
    Thus there are no irreducible elements in $\mathbb{Z}_6$.
\end{proof}

\begin{tcolorbox}[title=Problem 8, breakable]
    Show that $2$ is irreducible in $\mathbb{Z}_8$, and that 
    every non-unit in $\mathbb{Z}_8$ is irreducible or a product of irreducibles.
\end{tcolorbox} 

\begin{proof}
    We can view the multiplication table and see $2$ is irreducible in $\mathbb{Z}_8$.
    Similarly we can do the same for non-units and see they are either irreducible or a product of irreducibles.
\end{proof}

\begin{tcolorbox}[title=Problem 9, breakable]
    Determine all irreducible elements in $\mathbb{Z} \times \mathbb{Z}$.
\end{tcolorbox} 

\begin{proof}
    $(a, b) \in \mathbb{Z} \times \mathbb{Z}$ where $\pm a$ or $\pm b$ is prime
    and the other element is $\pm 1$.
\end{proof}

\begin{tcolorbox}[title=Problem 10, breakable]
    Prove that if $p$ is prime in $\mathbb{Z}$ and $p$ is congruent to $3 \pmod{4}$,
    then $p$ is irreducible in $\mathbb{Z}[i]$.
\end{tcolorbox}

\begin{proof}
    Suppose $p$ is prime in $\mathbb{Z}$ and $p \equiv 3 \pmod{4}$.
    Then $p = 4k + 3$ for some $k \in \mathbb{Z}$.
    Consider the norm $N(p) = (4k+3)^2$. 
    Clearly $N(p) \ne 1$, so $p$ is not a unit. 
    Suppose $p = \alpha \beta$ in $\mathbb{Z}[i]$.
    Then
    \[
        N(p) = p^2 = N(\alpha) N(\beta) > 1.
    \]
    Since $p$ is prime in $\mathbb{Z}$ we must have $N(\alpha) = N(\beta) = p$,
    which is impossible because a prime $p \equiv 3 \pmod 4$ cannot be written as a sum of two squares.
\end{proof}

\begin{tcolorbox}[title=Problem 11, breakable]
    Prove $2$ is irreducible in $\mathbb{Z}[n]$ for all square-free $n < -2$.
\end{tcolorbox} 

\begin{proof}
    Suppose $n < -2$ is square free.
    Furthermore, suppose $2 = \alpha \beta$ where $\alpha, \beta \in \mathbb{Z}[\sqrt{n}]$ and neither $\alpha$ nor $\beta$ is a unit.
    Now 
    \[
        N(2) = 2^2 = N(\alpha) N(\beta) = 4.
    \]
    Let $p > 0$ such that $n = -p$. 
    Suppose $\alpha = a + b\sqrt{-p}$ and $\beta = c + d\sqrt{-p}$.
    Then 
    \[
        N(\alpha) = a^2 + p b^2, \qquad N(\beta) = c^2 + p d^2,
    \]
    so
    \[
        (a^2 + p b^2)(c^2 + p d^2) = 4.
    \]
    Ignoring signs and wlog, we must have either
    \[
        a^2 + p b^2 = 4, \quad c^2 + p d^2 = 1,
    \]
    or
    \[
        a^2 + p b^2 = c^2 + p d^2 = 2.
    \]
    The first case is ruled out because we assumed neither factor is a unit.  
    In the second case, since $p > 2$, we cannot have $a^2 + p b^2 = 2$ with integers $a,b$.
    Thus $2$ is irreducible.
\end{proof}

\begin{tcolorbox}[title=Problem 12, breakable]
    Find two distinct square-free integers $n$ (with $n > 1$) for which $2$
        is not irreducible in $\mathbb{Z}[\sqrt{n}]$.
\end{tcolorbox} 

\textbf{Solution: }
\[n = 2, 3\]

\begin{tcolorbox}[title=Problem 13, breakable]
    Suppose that $p$ is a postive prime integer.
    Prove that $\sqrt{p}$ is irreducible in $\mathbb{Z}[\sqrt{p}]$.
    Show by example that this is false if $p$ is not prime; in particular, consider $p = 6$.
\end{tcolorbox} 

\begin{proof}
    For contradiction, suppose $\sqrt{p} = \alpha \beta$ for some $\alpha, \beta \in \mathbb{Z}[\sqrt{p}]$ that are not units.  
    Suppose $\alpha = a + b \sqrt{p}$ and $\beta = c + d \sqrt{p}$ where $a,b,c,d \in \mathbb{Z}$.  
    Then 
    \[
        N(\sqrt{p}) = p = N(\alpha)N(\beta)
    \]
    Since $p$ is prime either $N(\alpha) = 1$ or $N(\beta) = 1$
    In either case, one is a unit which is a contradiction.
\end{proof}

\textbf{Solution: }
Note that $\sqrt{2}, \sqrt{3}$ are not units in $\mathbb{Z}[\sqrt{6}]$.
\[\sqrt{6} = \sqrt{2}\sqrt{3} \in \mathbb{Z}[\sqrt{6}].\]

\begin{tcolorbox}[title=Problem 14, breakable]
    Show that $11 + 6 \sqrt{-5}$ and $16 + 3 \sqrt{-5}$ are irreducible elements in $\mathbb{Z}[\sqrt{-5}]$,
        with the same norm. Show that these elements are not associates.
\end{tcolorbox} 

\begin{proof}
    Suppose $11 + 6 \sqrt{-5} = \alpha \beta$ for some non-units $\alpha, \beta \in \mathbb{Z}[\sqrt{-5}]$.
    Then $N(11 + 6 \sqrt{-5}) = 11^2 + 5 \cdot 6^2 = 301 = N(\alpha) N(\beta)$.
    Since $\alpha, \beta$ are not units, we can assume wlog $N(\alpha) = 7$ and $N(\beta) = 43$.
    Now, suppose $\alpha = a + b \sqrt{-5}$ for integers $a,b$. Then $N(\alpha) = a^2 + 5 b^2 = 7$.
    Checking all possibilities there are no integer solutions.
    Similarly, no $\beta$ with norm $43$ exists.
    Therefore, $11 + 6 \sqrt{-5}$ cannot be factored into non-units, so it is irreducible.
    A similar argument shows $16 + 3 \sqrt{-5}$ is irreducible.
    Neither are associates because the only units in $\mathbb{Z}[\sqrt{-5}]$ are $\pm 1$.
\end{proof}

\begin{tcolorbox}[title=Problem 16, breakable]
    Suppose that $n$ is a square-free integer.
    Prove that $\sqrt{n}$ is irrational.
\end{tcolorbox} 

\begin{proof}
    Suppose $n$ is square-free and $\sqrt{n}$ is rational.
    Let $\sqrt{n} = \frac{a}{b}$ where $a,b \in \mathbb{Z}$ and $\gcd(a, b) = 1$.
    Then $n = \frac{a^2}{b^2}$, thus $n b^2 = a^2$.
    Thus $b^2 \mid a^2 \implies b \mid a$, so $a = b k$ for some $k \in \mathbb{Z}$.
    Then $n = \frac{b^2 k^2}{b^2} = k^2$, contradicting that $n$ is square-free.
\end{proof}