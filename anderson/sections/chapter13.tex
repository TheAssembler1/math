\begin{tcolorbox}[title=Warmup (a), breakable]
    Give examples of the following (or explain why no example exists).
    You should specify both a domain and an element of it:
    \begin{enumerate}
        \item A prime element that isn't irreducible.
        \item An irreducible element that isn't prime.
        \item A non-unit that is neither prime nor irreducible.
    \end{enumerate}
\end{tcolorbox} 

\begin{tcolorbox}[title=Warmup (b), breakable]
    Give examples of the following (that is,
    specify both a domain and an ideal of it):
    \begin{enumerate}
        \item A principal ideal that is maximal.
        \item A principle ideal that is not maximal.
        \item A maximal ideal that is not principal.
        \item An ideal that is neither maximal nor principal.
    \end{enumerate}
\end{tcolorbox} 

\begin{tcolorbox}[title=Warmup (c), breakable]
    Translate the following statements into statements about principal ideals:
    \begin{enumerate}
        \item $3$ divides $12$ in $\mathbb{Z}$.
        \item $2 + \sqrt{3}$ is a unit in $\mathbb{Z}[\sqrt{3}]$.
        \item $2 + i$ and $-1 + 2i$ are assciates in $\mathbb{Z}[i]$.
        \item $2x + 1$ is irreducible in $\mathbb{Q}[x]$.
        \item $2x + 1$ is irreducible in $\mathbb{Z}[x]$.
    \end{enumerate}
\end{tcolorbox} 

\begin{tcolorbox}[title=Warmup (d), breakable]
    Why is the following statementn silly?
    Every commutative ring has exactly one maximal ideal because the whole ring is an ideal,
        which is obviously as large as possible.
\end{tcolorbox} 

\begin{tcolorbox}[title=Problem 1, breakable]
    Prove Theorem 13.1: In any domain, prime elements are irreducible.
\end{tcolorbox} 

\begin{tcolorbox}[title=Problem 2, breakable]
    Complete the proof of Theorem 13.2.
    That is, suppose we have a domain in which all non-zero non-units are irreducible,
        or a product of irreducibles. Furthermore, suppose all irreducible elements 
        are prime. Prove that the factorization of any non-unit into irreducibles is unique
        (up to order and unit factors).
\end{tcolorbox} 

\begin{tcolorbox}[title=Problem 3, breakable]
    Exhibit in the ring $\mathbb{Z}_6$ a non-unit $a$ for which $a^n = a$,
        for all positive integers $n$. Why does this mean that there is 
        no unique factorization into irreducibles for this ring? 
    Now repeat this exercise for $\mathbb{Z} \times \mathbb{Z}$.
\end{tcolorbox} 

\begin{tcolorbox}[title=Problem 4, breakable]
    Prove that $x$ is a prime element of $mathbb{Z}[x]$.
\end{tcolorbox} 

\begin{tcolorbox}[title=Problem 8, breakable]
    Describe the elements of the ideal $\langle 3, x \rangle$ in $\mathbb{Z}[x]$.
    Show that $\langle 3, x \rangle$ is not a principal ideal. (See Exercise 12.2.)
\end{tcolorbox} 

\begin{tcolorbox}[title=Problem 9, breakable]
    Show that $\langle 4, x \rangle$ is a maximal ideal in $\mathbb{Z}[x]$.
\end{tcolorbox} 

\begin{tcolorbox}[title=Problem 10, breakable]
    Show that $\langle 9, x \rangle$ is an ideal of $\mathbb{Z}[x]$,
        which is neither principal nor maximal.
\end{tcolorbox} 

\begin{tcolorbox}[title=Problem 14, breakable]
    \begin{enumerate}
        \item Why is $x^2 + 1$ irreducible in $\mathbb{Q}[x]$.
        \item By part a we now know that $\langle x^2 + 1 \rangle$ is a maximal ideal 
        in $\mathbb{Q}[x]$. Provide a direct argument for this, similar to the corresponding
        argument for $\langle x \rangle$ in Example 13.8.
    \end{enumerate}
\end{tcolorbox} 

\begin{tcolorbox}[title=Problem 15, breakable]
    Let $n$ be a square-free integer and suppose that $n 1 \pmod{4}$.
    Prove that $\mathbb{Z}[\sqrt{n}]$ is not a UFD.
\end{tcolorbox} 

\begin{tcolorbox}[title=Problem 16, breakable]
    Consider the ideal $\langle x \rangle$ in $\mathbb{Q}[x, y]$; see 
    Exercise 12.12 for more about this ring and this ideal.
    Show that this ideal is not maximal.
\end{tcolorbox} 

\begin{tcolorbox}[title=Problem 17, breakable]
    Consider the element $x + y$ in $\mathbb{Q}[x, y]$; see Exercise 12.12
    for more about this ring. Argue that $x + y$ is irreducible, and so 
    $\langle x + y \rangle$ is maximal among all principal ideals in $\mathbb{Q}[x, y]$.
    Show that $\langle x + y \rangle$ is not maximal among all ideals in $\mathbb{Q}[x, y]$.
\end{tcolorbox} 

\begin{tcolorbox}[title=Problem 18, breakable]
    In Exercise c part e, your translation should have been that 
    \[\langle 2x + 1 \rangle,\]
    is maximal among all principal ideals in $\mathbb{Z}[x]$
    Show that this ideal is \emph{not} maximal among all ideals,
    by considering the ideal 
    \[\langle 2x + 1, 3 \rangle.\]
\end{tcolorbox} 