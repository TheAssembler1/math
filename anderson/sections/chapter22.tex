\begin{tcolorbox}[title=Problem 1, breakable]
    Complete the analysis of the symmetries 
    of the square, which we begain in the text.
    Some will be rotations, and some will be flips.
    Determine the matrix and permutation representations
    for them, draw a table of correspondence, and compute 
    the group table for your symmetries.
\end{tcolorbox} 

\textbf{Group Table:}
\[
\begin{array}{c|cccccccc}
\circ 
& \iota & p & p^2 & p^3 & \varphi & p\varphi & \varphi p & \varphi p^2 \\ \hline
\iota 
& \iota & p & p^2 & p^3 & \varphi & p\varphi & \varphi p & \varphi p^2 \\
p 
& p & p^2 & p^3 & \iota & p\varphi & \varphi p^2 & \varphi & \varphi p \\
p^2 
& p^2 & p^3 & \iota & p & \varphi p & \varphi p^2 & p\varphi & \varphi \\
p^3 
& p^3 & \iota & p & p^2 & \varphi p^2 & \varphi & \varphi p & p\varphi \\
\varphi 
& \varphi & \varphi p & \varphi p^2 & p\varphi & \iota & p & p^2 & p^3 \\
p\varphi 
& p\varphi & \varphi p^2 & \varphi & \varphi p & p & p^2 & p^3 & \iota \\
\varphi p 
& \varphi p & p\varphi & \varphi p^2 & \varphi & p^2 & p^3 & \iota & p \\
\varphi p^2 
& \varphi p^2 & \varphi & p\varphi & \varphi p & p^3 & \iota & p & p^2
\end{array}
\]
\textbf{Permutations:}
\[\iota \longleftrightarrow \begin{bmatrix}
    1 & 2 & 3 & 4 \\
    1 & 2 & 3 & 4
\end{bmatrix}, p \longleftrightarrow \begin{bmatrix}
    1 & 2 & 3 & 4 \\
    4 & 1 & 2 & 3
\end{bmatrix}, p^2 \longleftrightarrow \begin{bmatrix}
    1 & 2 & 3 & 4 \\
    3 & 4 & 1 & 2
\end{bmatrix}, p^3 \longleftrightarrow \begin{bmatrix}
    1 & 2 & 3 & 4 \\
    2 & 3 & 4 & 1
\end{bmatrix}\]
\[\varphi \longleftrightarrow \begin{bmatrix}
    1 & 2 & 3 & 4 \\
    2 & 1 & 4 & 3
\end{bmatrix}, p \varphi \longleftrightarrow  \begin{bmatrix}
    1 & 2 & 3 & 4 \\
    3 & 2 & 1 & 4
\end{bmatrix}, \varphi p \longleftrightarrow \begin{bmatrix}
    1 & 2 & 3 & 4 \\
    1 & 4 & 3 & 2
\end{bmatrix}, \varphi p^2 \longleftrightarrow \begin{bmatrix}
    1 & 2 & 3 & 4 \\
    4 & 3 & 2 & 1
\end{bmatrix}\]
\textbf{Table of Correspondence:}
\[
\begin{array}{c|cccc}
 & 1 & 2 & 3 & 4 \\ \hline
\iota        & 1 & 2 & 3 & 4 \\
p            & 4 & 1 & 2 & 3 \\
p^2          & 3 & 4 & 1 & 2 \\
p^3          & 2 & 3 & 4 & 1 \\
\varphi      & 2 & 1 & 4 & 3 \\
p\varphi     & 3 & 2 & 1 & 4 \\
\varphi p    & 1 & 4 & 3 & 2 \\
\varphi p^2  & 4 & 3 & 2 & 1
\end{array}
\]
\textbf{Matrix Correspondence:}
\[
\iota \longleftrightarrow 
\begin{bmatrix} 1 & 0 \\ 0 & 1 \end{bmatrix}, 
p \longleftrightarrow 
\begin{bmatrix} 0 & -1 \\ 1 & 0 \end{bmatrix}, 
p^2 \longleftrightarrow 
\begin{bmatrix} -1 & 0 \\ 0 & -1 \end{bmatrix}, 
p^3 \longleftrightarrow 
\begin{bmatrix} 0 & 1 \\ -1 & 0 \end{bmatrix}
\]
\[
\varphi \longleftrightarrow 
\begin{bmatrix} 1 & 0 \\ 0 & -1 \end{bmatrix}, 
p\varphi \longleftrightarrow 
\begin{bmatrix} 0 & 1 \\ 1 & 0 \end{bmatrix}, 
\varphi p \longleftrightarrow 
\begin{bmatrix} 0 & -1 \\ -1 & 0 \end{bmatrix}, 
\varphi p^2 \longleftrightarrow 
\begin{bmatrix} -1 & 0 \\ 0 & 1 \end{bmatrix}
\]

\begin{tcolorbox}[title=Problem 3, breakable]
    Determine all symmetries of a non-square rectangle,
    and represent them with matrices and permutations.
    How many rotations, and how many are flips.
\end{tcolorbox} 

\begin{proof}
    We simply remove the square symmetries constructed from $1$ or $3$ rotations.
    Thus there are $4$ symmetries of a non-square rectangle.
\end{proof}

\begin{tcolorbox}[title=Problem 5, breakable]
    Show algebraically that the rotation transformation 
    preserves distance: Consider the points 
    $P_1(x_1, y_1)$ and $P_2(x_2, y_2)$.

    (a) What is the square of the distance between $P_1$
        and $P_2$.

    (b) Now rotate through the angle $\theta$, by multiplying 
        by the appropriate matrix, to obtain the points 
        $P_1'(x_1', y_1')$ and $P_2'(x_2', y_2')$.
        Compute the square of the distance of these points.
        Use trig identities to show that this is the same 
        as in part a.
\end{tcolorbox} 

\textbf{Solution (a):}
The square of the distance between $P_1$ and $P_2$ is  
    $(x_2 - x_1)^2 + (y_2 - y_1)^2$.

\textbf{Solution (b):}
Notice 
\[
    \begin{bmatrix}x_1 & y_1\end{bmatrix}
    \begin{bmatrix}
                   \cos \theta & -\sin \theta \\
                   \sin \theta & \cos \theta
    \end{bmatrix} = 
    \begin{bmatrix}
     x_1 \cos\theta + y_1 \sin\theta, &
    -x_1 \sin\theta + y_1 \cos\theta
    \end{bmatrix}
\]
\[
    \begin{bmatrix}x_2 & y_2\end{bmatrix}
    \begin{bmatrix}
                   \cos \theta & -\sin \theta \\
                   \sin \theta & \cos \theta
    \end{bmatrix} = 
    \begin{bmatrix}
     x_2 \cos\theta + y_2 \sin\theta, &
    -x_2 \sin\theta + y_2 \cos\theta
    \end{bmatrix}
\]
Thus 
\[P_1'(x_1', y_1') = P_1'(x_1 \cos\theta + y_1 \sin\theta, -x_1 \sin\theta + y_1 \cos\theta),\]
and 
\[P_2'(x_2', y_2') = P_2'(x_2 \cos\theta + y_2 \sin\theta, -x_2 \sin\theta + y_2 \cos\theta).\]
Then the square of the distance between these points is 
\[((x_2 \cos\theta + y_2 \sin\theta) - (x_1 \cos\theta + y_1 \sin\theta))^2 
    + ((-x_2 \sin\theta + y_2 \cos\theta) - (-x_1 \sin\theta + y_1 \cos\theta))^2.\]
Some basic algebra show this simplifies to $(x_2 - x_1)^2 + (y_2 - y_1)^2$.

\begin{tcolorbox}[title=Problem 6, breakable]
    Verify by mutliplying two matrices together that a 
    rotation through angle $\theta$, followed by a 
    rotation through angle $\varphi$, gives a rotation through 
    angle $\theta + \varphi$.
\end{tcolorbox} 

\textbf{Solution:}
\[
\begin{bmatrix}
\cos \theta & -\sin \theta \\
\sin \theta & \cos \theta
\end{bmatrix}
\begin{bmatrix}
\cos \phi & -\sin \phi \\
\sin \phi & \cos \phi
\end{bmatrix}
=
\begin{bmatrix}
\cos\theta \cos\phi + (-\sin\theta)(\sin\phi) & \cos\theta(-\sin\phi) + (-\sin\theta)\cos\phi \\
\sin\theta \cos\phi + \cos\theta \sin\phi & \sin\theta(-\sin\phi) + \cos\theta \cos\phi
\end{bmatrix}
\]

\[
=
\begin{bmatrix}
\cos\theta \cos\phi - \sin\theta \sin\phi & -\cos\theta \sin\phi - \sin\theta \cos\phi \\
\sin\theta \cos\phi + \cos\theta \sin\phi & -\sin\theta \sin\phi + \cos\theta \cos\phi
\end{bmatrix}
\]

\[
=
\begin{bmatrix}
\cos(\theta+\phi) & -\sin(\theta+\phi) \\
\sin(\theta+\phi) & \cos(\theta+\phi)
\end{bmatrix}
\]

\begin{tcolorbox}[title=Problem 7, breakable]
    How many symmetries can you find for the unit circle?
    Which rotations are possible?
    Which flips?
\end{tcolorbox} 

\textbf{Solution:} There are an infinite number of 
symmetries for both flips and rotations on the unit circle.

\begin{tcolorbox}[title=Problem 8, breakable]
    Find out how many elements there are in $D_n$,
    the group of symmetries of a regular $n$-sided 
    polygon.
\end{tcolorbox} 

\textbf{Solution:}
There are $n$ rotations including the identity.
For each of these rotations we can flip granting another $n$ symmetries.
Thus there are $2n$ total symmetries.

\begin{tcolorbox}[title=Problem 9, breakable]
    You can check that all of the matrices of the 
    symmetries of the equilateral triangle and the 
    square have the property that their determinants 
    are always $\pm 1$ (See Exercise 8.2 for a 
    definition of the determinant of a $2 \times 2$
    matrix.) In this exercise you will show thaSt if a 
    matrix preserves distance, then its determinant 
    must be $1$.

    (a) Suppose that $A \in M_2(\mathbb{R})$, and let 
        $det(A) = 0$. Show that mutliplication by $A$ 
        cannot preserve distance. Do this by showing that 
        multiplication by $A$ takes some point in the plane 
        to the origin and hence cannot preserve distance.

    (b) Suppose next that 
        \[\begin{bmatrix}
            a & b \\
            c & d
        \end{bmatrix} = A \in M_2(\mathbb{R}),\]
        but $det(A) \ne 0$. Suppose that multiplication 
        by $A$ does preserve distance, and consider 
        successively what happens to 
        \[\begin{bmatrix}
            1 \\ 0
        \end{bmatrix},
        \begin{bmatrix}
            0 \\ 1
        \end{bmatrix}, 
        \begin{bmatrix}
            d \\ -c
        \end{bmatrix},
        \begin{bmatrix}
            -b \\ a
        \end{bmatrix}\]
        You will be able to infer that $det(A) = \pm 1$.
\end{tcolorbox} 

\begin{proof}
    Let $A = \begin{bmatrix}a & b \\ c & d\end{bmatrix}$ be an arbitrary matrix in $M_2(\mathbb{R})$ 
    such that $\det(A) = 0$. Then $\det(A) = ad - bc = 0$, so $ad = bc$.
    Let $\begin{bmatrix}x & y\end{bmatrix}$ be a point with $x, y \in \mathbb{R}$.  
    Applying the transformation $A$ gives 
    \[
    \begin{bmatrix}x & y\end{bmatrix}\begin{bmatrix}a & b \\ c & d\end{bmatrix} =
    \begin{bmatrix}ax + by & cx + dy\end{bmatrix}.
    \]
    If $a = b = c = d = 0$, then all points are mapped to the origin.  
    Suppose w.l.o.g. that $a \ne 0$. From $ax + by = 0$, we get $x = -\frac{b}{a}y$.  
    Substituting into $cx + dy = 0$ gives 
    \[
    c\left(-\frac{b}{a}y\right) + dy = \left(d - \frac{bc}{a}\right)y = 0.
    \]
    Since $\det(A) = ad - bc = 0$, we have $d - \frac{bc}{a} = 0$.  
    Thus, any point on the line $x = -\frac{b}{a}y$ is mapped to the origin by $A$.
\end{proof}

\[
A \begin{bmatrix}1 \\ 0\end{bmatrix} = 
\begin{bmatrix}a & b \\ c & d\end{bmatrix} \begin{bmatrix}1 \\ 0\end{bmatrix} = 
\begin{bmatrix}a \\ c\end{bmatrix}, \quad
A \begin{bmatrix}0 \\ 1\end{bmatrix} = 
\begin{bmatrix}a & b \\ c & d\end{bmatrix} \begin{bmatrix}0 \\ 1\end{bmatrix} = 
\begin{bmatrix}b \\ d\end{bmatrix},
\]
From this it follows that $a^2 + c^2 = 1$ and $b^2 + d^2 = 1$.
\[
A \begin{bmatrix}d \\ -c\end{bmatrix} = 
\begin{bmatrix}a & b \\ c & d\end{bmatrix} \begin{bmatrix}d \\ -c\end{bmatrix} = 
\begin{bmatrix}ad - bc \\ cd - cd\end{bmatrix} = 
\begin{bmatrix}ad - bc \\ 0\end{bmatrix}, \quad
A \begin{bmatrix}-b \\ a\end{bmatrix} = 
\begin{bmatrix}a & b \\ c & d\end{bmatrix} \begin{bmatrix}-b \\ a\end{bmatrix} = 
\begin{bmatrix}-ab + ab \\ -bc + ad\end{bmatrix} = 
\begin{bmatrix}0 \\ ad - bc\end{bmatrix}.
\]
We see that $|det(A)| = |ad - bc| = 1$ if distance is to be preserved.

\begin{tcolorbox}[title=Problem 10, breakable]
    Our description of the symmetries of the equilateral 
    triangle can be elegantly rephrased using the 
    arithmetic of the complex numbers $\mathbb{C}$,
    described in Chapter $8$.

    (a) Argue that the three vertices of the triangle can be thought 
        of as numbers of the form $e^{i \alpha_i}$ in the 
        complex plane, for appropriate angles $\alpha_i$.

    (b) Show that you can represent the rotations of the triangle 
        in the symmetry group by complex mutliplication by a 
        number of the form $e^{i \theta}$, for an 
        appropriate choice of $\theta$.

    (c) What operation on the complex numbers performs the flip 
        $\phi$?
\end{tcolorbox} 

\textbf{Solution (a):}The three vertices can be represented as
\[
e^{i \alpha_1}, \quad e^{i \alpha_2}, \quad e^{i \alpha_3},
\]
where $\alpha_1 = 0$, $\alpha_2 = 2\pi/3$, and $\alpha_3 = 4\pi/3$.  
The modulus is $1$ because the vertices are on the unit circle.

\textbf{Solution (b):}
Rotations of the triangle correspond to multiplication by 
\[
e^{i \theta}, \quad \theta \in \{0, 2\pi/3, 4\pi/3\}.
\] 

\textbf{Solution (c):}
A flip across a line through the origin is represented by complex conjugation $e^{-i \alpha_i}$, 
possibly combined with multiplication by $e^{i \phi}$.