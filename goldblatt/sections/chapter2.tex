\section{Exercises on Filters}

\begin{tcolorbox}[title=Problem 1, breakable]
    If $\emptyset \ne A \subseteq I$, there is an ultrafilter $\mathcal{F}$
    on $I$ with $A \in \mathcal{F}$.
\end{tcolorbox}

\begin{proof}
    Consider the set $\{A\}$, which has the finite intersection property since $A \ne \emptyset$.
    Thus, by Theorem 2.6.1, there exists an ultrafilter $\mathcal{F}$ on $I$ such that
    \[
        \{A\} \subseteq \mathcal{F}.
    \]
    Therefore, $A \in \mathcal{F}$.
\end{proof}

\newpage
\begin{tcolorbox}[title=Problem 2, breakable]
    There exists a nonprincipal ultrafilter on $\mathbb{N}$
    containing the set of even numbers, and another containing the 
    set of odd numbers.
\end{tcolorbox}

\begin{proof}
    Let
    \[
        \mathcal{E} = \{x \in \mathbb{N} \mid x = 2k \text{ for some } k \in \mathbb{N}\}
    \]
    and
    \[
        \mathcal{O} = \{x \in \mathbb{N} \mid x = 2k+1 \text{ for some } k \in \mathbb{N}\}.
    \]
    Since $\mathcal{E}$ is infinite, by Corollary 2.6.2 there exists
    a nonprincipal ultrafilter $\mathcal{U}_E$ on $\mathcal{E}$.
    Let
    \[
        \mathcal{F}_E
        =
        \{ A \subseteq \mathbb{N} \mid A \cap \mathcal{E} \in \mathcal{U}_E \}.
    \]
    We first show $\mathcal{F}_E$ is a filter. Since $\mathcal{E} \in \mathcal{U}_E$,
    we have $\mathbb{N} \cap \mathcal{E} = \mathcal{E} \in \mathcal{U}_E$, so
    $\mathbb{N} \in \mathcal{F}_E$. Also $\emptyset \notin \mathcal{F}_E$ since
    $\emptyset \cap \mathcal{E} = \emptyset \notin \mathcal{U}_E$.
    If $A,B \in \mathcal{F}_E$, then $A \cap \mathcal{E} \in \mathcal{U}_E$
    and $B \cap \mathcal{E} \in \mathcal{U}_E$. Since $\mathcal{U}_E$ is a filter,
    \[
        (A \cap \mathcal{E}) \cap (B \cap \mathcal{E})
        =
        (A \cap B) \cap \mathcal{E}
        \in \mathcal{U}_E,
    \]
    so $A \cap B \in \mathcal{F}_E$.
    If $A \in \mathcal{F}_E$ and $A \subseteq B \subseteq \mathbb{N}$, then
    $A \cap \mathcal{E} \subseteq B \cap \mathcal{E}$. Since $\mathcal{U}_E$
    is upward closed, $B \cap \mathcal{E} \in \mathcal{U}_E$, so
    $B \in \mathcal{F}_E$. Thus $\mathcal{F}_E$ is a filter.
    Now, let $X \subseteq \mathbb{N}$. Since $\mathcal{U}_E$ is an ultrafilter
    on $\mathcal{E}$, either $X \cap \mathcal{E} \in \mathcal{U}_E$ or
    \[
        \mathcal{E} \setminus (X \cap \mathcal{E})
        =
        X^C \cap \mathcal{E}
        \in \mathcal{U}_E.
    \]
    Thus either $X \in \mathcal{F}_E$ or $X^C \in \mathcal{F}_E$.
    Thus $\mathcal{F}_E$ is an ultrafilter on $\mathbb{N}$.
    Notice
    \[
        \mathcal{E} \cap \mathcal{E} = \mathcal{E} \in \mathcal{U}_E,
    \]
    so $\mathcal{E} \in \mathcal{F}_E$.
    Since $\mathcal{U}_E$ is nonprincipal, $\mathcal{F}_E$ is also nonprincipal.
    The same argument applied to $\mathcal{O}$ produces a
    nonprincipal ultrafilter on $\mathbb{N}$ containing $\mathcal{O}$.
\end{proof}

\begin{tcolorbox}[title=Problem 3, breakable]
    An ultrafilter on a finite set must be principle.
\end{tcolorbox}

\begin{proof}
    Suppose $\mathcal{F}$ is an ultrafilter on a finite set $I$.
    Since $I$ is finite, there exists a smallest set $X \in \mathcal{F}$.
    If $|X| = 1$ then obviously $\mathcal{F}$ is principal.
    Therefore, suppose, $|X| \ge 2$.
    Take $p \in X$ and consider $X - \{p\}$. 
    Since $\mathcal{F}$ is an ultrafilter
    \[
        X - \{p\} \in \mathcal{F} \quad \text{or} \quad I - (X - \{p\}) \in \mathcal{F}.
    \]
    But $X$ has the smallest size among sets in $\mathcal{F}$, so $X - \{p\} \notin \mathcal{F}$. 
    Therefore
    \[
        I - (X - \{p\}) = \{p\} \cup (I - X) \in \mathcal{F}.
    \]
    Then
    \[
        X \cap (\{p\} \cup (I - X)) = \{p\} \in \mathcal{F}.
    \]
    Thus $\mathcal{F}$ contains a singleton and is a principal ultrafilter generated by $p$.
\end{proof}

\begin{tcolorbox}[title=Problem 4, breakable]
    For $\mathcal{H} \subseteq \mathcal{P}(I)$, let $\mathcal{F}^{\mathcal{H}}$
    be defined as in Example 2.4(3).
    \begin{enumerate}
        \item Show that $\mathcal{F}^{\mathcal{H}}$ is a filter that includes 
        $\mathcal{H}$, i.e., $\mathcal{H} \subseteq \mathcal{F}^{\mathcal{H}}$.
        \item Show that $\mathcal{F}^{\mathcal{H}}$ is included in any other filter 
        that includes $\mathcal{H}$.
    \end{enumerate}
\end{tcolorbox}

\begin{proof}
    Suppose $T, T' \in \mathcal{F}^{\mathcal{H}}$.
    Then $T, T' \subseteq I$ where
    \[T \supseteq B_1 \cap B_2 \cap \cdots \cap B_n \text{ and } 
      T' \supseteq B_1' \cap B_2' \cap \cdots \cap B_{n'}' \]
    where $n, n' \in \mathbb{N}$ and $B_i, B_i' \in \mathcal{H}$.
    Notice
    \[
    I \supseteq T \cap T' \supseteq
    B_1 \cap B_2 \cap \cdots \cap B_n \cap
    B_1' \cap B_2' \cap \cdots \cap B_{n'}'.
    \]
    Since each $B_i$ and $B_i'$ is in $\mathcal{H}$,  we have $T \cap T' \in \mathcal{F}^{\mathcal{H}}$.
    If $T \in \mathcal{F}^{\mathcal{H}}$ and $T \subseteq S \subseteq I$,
    then since $T \supseteq B_1 \cap \cdots \cap B_n$ for some $B_i \in \mathcal{H}$,
    we also have $S \supseteq B_1 \cap \cdots \cap B_n$, so $S \in \mathcal{F}^{\mathcal{H}}$.
    For any $B \in \mathcal{H}$ we have $B \supseteq B$, which is a finite intersection of elements of $\mathcal{H}$.
    Thus $B \in \mathcal{F}^{\mathcal{H}}$ and it follows that $\mathcal{H} \subseteq \mathcal{F}^{\mathcal{H}}$.
    Therefore, $\mathcal{F}^{\mathcal{H}}$ is a filter that contains $\mathcal{H}$.
\end{proof}

\begin{proof}
    Let $\mathcal G$ be a filter such that $\mathcal H \subseteq \mathcal G$.
    Suppose $T \in \mathcal F^{\mathcal H}$.
    Then $T \supseteq B_1 \cap \cdots \cap B_n$ for some $B_i \in \mathcal H$.
    Since $\mathcal H \subseteq \mathcal G$ every $B_i \in \mathcal G$.
    Since $\mathcal G$ is a filter $B_1 \cap \cdots \cap B_n \in \mathcal G$.
    Since $I \supseteq T \supseteq B_1 \cap \cdots \cap B_n$ we find $T \in \mathcal G$.
    Therefore $\mathcal F^{\mathcal H} \subseteq \mathcal G$.
\end{proof}

\begin{tcolorbox}[title=Problem 5, breakable]
    Let $\mathcal{F}$ be a proper filter on $I$.
    \begin{enumerate}
        \item Show that $\mathcal{F} \cup \{A^c\}$ has the finite intersection property iff $A^c \not \in \mathcal{F}$.
        \item Use (i) to deduce that $\mathcal{F}$ is an ultrafilter iff it is a maximal proper filter on $I$.
    \end{enumerate}
\end{tcolorbox}