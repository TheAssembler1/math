\subsection{Introduction}

\begin{tcolorbox}[title=Problem 4, breakable]
    Let $x_1, x_2, x_3, \ldots$  be an infinite collection of independent
    variables indexed by the natural numbers. A  polynomial with coefficients 
    in a field $k$ in the $x_i$  is a finite linear combination of (finite)
    monomials $x_{i_1}^{e_1} \ldots x_{i_n}^{e_n}$. Let $R$ detote the set of all 
    polynomials in $x_i$. Note that we an add and multiply elements of $R$ in the usual 
    way. Thus, $R$ is the polynomial ring, $k[x_1, x_2, \ldots]$ in infinitely many variables.
    \begin{enumerate}
        \item Let $I = \langle x_1, x_2, \ldots \rangle$ be the set of $f = \sum_{i = 1}^{\infty} A_i x_i$,
        where $A_i = 0$ for all $i$ sufficiently large. Show that $I$ is an ideal in the ring $R$.
        \item Show, arguing by contradiction, that $I$ has no finite generating set. Hint: note that 
        if $I = \langle g_1, \ldots, g_m \rangle$, then there must be some variable $x_l$ that is not contained
        in any of the $g_j$.
    \end{enumerate}
\end{tcolorbox}

\begin{proof}
    Let $p \in k[x_1, x_2, \ldots]$.
    Let $f, g \in I$ with $j, j'$ being sufficiently large such that $A_j = 0$ and $A_{j'} = 0$.
    Then $f + g = \sum_{i = 1}^{\infty} A_i x_i + \sum_{i = 1}^{\infty} B_i x_i = \sum_{i = 1}^{\infty} (A_i + B_i) x_i \in I$.
    Also, $p \cdot f = p \cdot \sum_{i = 1}^{\infty} A_i x_i \in I$ because it is a finite sum of monomials in the $x_i$.
    Thus $I$ is an ideal.
\end{proof}

\begin{proof}
    Suppose, for contradiction, that $I$ has a finite generating set $\langle g_1, \ldots, g_m \rangle$
        for some $m \in \mathbb{N}$.
    Then there exists some variable $x_l$ that does not appear in any of the $g_j$.
    Clearly, $x_l$ cannot be expressed as a linear combination of the other generators.
    Thus $x_l$ is not in the ideal generated by $g_1, \ldots, g_m$, contradicting that they generate $I$.
\end{proof}

\newpage
\begin{tcolorbox}[title=Problem 5, breakable]
    In this problem you will show that all polynomial parametric curves in $k^2$ are contained
    in affine varieties.
    \begin{enumerate}
        \item Show that the number of distinct monomials $x^a y^b$ of total degree $\le m$
        in $k[x, y]$ is equal to $(m + 1)(m + 2) / 2$. [Note: This is the binomial coefficients 
        $\binom{m + 2}{2}$.]
        \item Show that if $f(t)$ and $g(t)$ are polynomials of degree $\le n$ in $t$, then for $m$ large enough,
        the ``monomials''
        \[[f(t)]^a [g(t)]^b,\]
        with $a + b \le m$ are linearly \emph{dependent}.
        \item Deduce form part (b) that if $C$ is a curve in $k^2$ given parametrically by $x = f(t), y = g(t)$
        for $f(t), g(t) \in k[t],$ then $C$ is contained in $\textbf{V}(F)$ for some nonzero $F \in k[x, y]$
        \item Generalize parts (a), (b), and (c) to show that any polynomial parametric surface 
        \[x = f(t, u), \quad y = g(t, u), \quad z = h(t, u),\]
        is contained in an algebraic surface $\textbf{V}(f)$, where $F \in k[x, y, z]$ is nonzero.
    \end{enumerate}
\end{tcolorbox}

\begin{proof}[Part 1]
    The number of monomials $x^a y^b$ of total degree $\le m$ is the number of $(a,b)$ such that $a+b \le m$.
    For each $a = 0, 1, \dots, m$, $b$ can range from $0$ to $m-a$, giving $m-a+1$ choices.
    Thus
    \[
    \sum_{a=0}^{m} (m-a+1) = \sum_{a=0}^{m} (m+1) - \sum_{a=0}^{m} a = (m+1)(m+1) - \frac{m(m+1)}{2} = \frac{(m+1)(m+2)}{2}.
    \]
\end{proof}

\begin{proof}[Part 2]
    Suppose $f(t)$ and $g(t)$ are polynomials of degree $\le n$ in $t$.
    Then $\deg(f(t)^a) \le an$ and $\deg(g(t)^b) \le bn$, so
    \[
    \deg([f(t)]^a [g(t)]^b) = \deg(f(t)^a) + \deg(g(t)^b) \le an + bn = n(a+b) \le nm.
    \]
    There are $\frac{(m+1)(m+2)}{2}$ such polynomials. 
    For $m$ large enough, 
    \[
    \frac{(m+1)(m+2)}{2} > nm+1,
    \]
    so these polynomials are linearly dependent.
\end{proof}

\begin{proof}[Part 3]
    By part 2, for sufficiently large $m$ there is some linear combination
    \[
    a_1 x^0 y^0 + a_2 x^1 y^0 + a_3 x^0 y^1 + \cdots = 0,
    \]
    with at least one $a_i \ne 0$.
    Then taking $x = f(t)$ and $y = g(t)$, gives
    \[
    a_1 [f(t)]^0 [g(t)]^0 + a_2 [f(t)]^1 [g(t)]^0 + a_3 [f(t)]^0 [g(t)]^1 + \cdots = 0
    \]
    for all $t$. Let
    \[
    F(x,y) = a_1 x^0 y^0 + a_2 x^1 y^0 + a_3 x^0 y^1 + \cdots \in k[x,y].
    \]
    Then $F(f(t),g(t)) = 0$ for all $t$ thus $C \subset \textbf{V}(F)$.
\end{proof}

\begin{proof}[Part 4]
    Consider all monomials
    \[
    [f(t,u)]^a [g(t,u)]^b [h(t,u)]^c
    \]
    of total degree $\le m$.
    For $m$ large enough,
    the number of such monomials exceeds the dimension of the vector space of polynomials in $t,u$ of that degree. 
    Thus there exists a linear combination
    \[
    F(x,y,z) = \sum a_{a,b,c} x^a y^b z^c \in k[x,y,z]
    \]
    that vanishes when evaluated on the surface
    \[
    F(f(t,u), g(t,u), h(t,u)) = 0.
    \]
    Therefore, the surface is contained in $\textbf{V}(F)$.
\end{proof}

\subsection{Orderings on the Monomials in $k[x_1, \ldots, x_n]$}

\newpage
\begin{tcolorbox}[title=Problem 1, breakable]
    Rewrite each of the following polynomials,
    ordering the terms using lex order,
    grlex order, and grevlex order,
    giving $LM(f), LT(f)$, and $multideg(f)$
    in each case.
    \begin{enumerate}
        \item $f_1(x, y, z) = 2x + 3y + z + x^2 - z^2 + x^3$.
        \item $f_2(x, y, z) = 2x^2 y^8 - 3x^5 y z^4 + x y z^3 - x y^4$.
    \end{enumerate}
\end{tcolorbox}

\textbf{Solution (lex order): }
For $f_1$, we have the following exponent vectors 
\[(1, 0, 0),\ (0, 1, 0),\ (0, 0, 1),\ (2, 0, 0),\ (0, 0, 2),\ (3, 0, 0).\]
Reordering in lex order ($x > y > z$) gives 
\[
x^3 + x^2 + 2x + 3y - z^2 + z.
\]
So
\[
LM(f_1) = x^3, \quad LT(f_1) = x^3, \quad multideg(f_1) = (3,0,0).
\]
For $f_2$, the exponent vectors are
\[(2, 8, 0),\ (5, 1, 4),\ (1, 1, 3),\ (1, 4, 0).\]
Reordering in lex order gives
\[
-3x^5 y z^4 + 2x^2 y^8 - x y^4 + x y z^3.
\]
So:
\[
LM(f_2) = x^5 y z^4, \quad LT(f_2) = -3x^5 y z^4, \quad multideg(f_2) = (5,1,4).
\]
\textbf{Solution (grlex order): }
For $f_1$, compute total degrees
\[
|2x| = 1,\ |3y| = 1,\ |z| = 1,\ |x^2| = 2,\ |-z^2| = 2,\ |x^3| = 3.
\]
Sort by total degree, breaking ties with lex order
\[
x^3 - x^2 - z^2 + 2x + 3y + z
\]
\[
LM(f_1) = x^3, \quad LT(f_1) = x^3, \quad multideg(f_1) = (3,0,0)
\]
For $f_2$, total degrees
\[
2x^2 y^8 = 10, \quad -3x^5 y z^4 = 10, \quad x y z^3 = 5, \quad -x y^4 = 5
\]
Sorting by total degree, then lex to break ties
\[
-3x^5 y z^4 + 2x^2 y^8 - x y^4 + x y z^3
\]
\[
LM(f_2) = x^5 y z^4, \quad LT(f_2) = -3x^5 y z^4, \quad multideg(f_2) = (5,1,4)
\]
\textbf{Solution (grevlex order): }
For $f_1$, total degrees as before: 1,1,1,2,2,3  
\[
x^3 + x^2 + 2x - z^2 + z + 3y
\]
\[
LM(f_1) = x^3, \quad LT(f_1) = x^3, \quad multideg(f_1) = (3,0,0)
\]
For $f_2$, total degrees as before: 10,10,5,5  
\[
 - 3x^5 y z^4 + 2x^2 y^8 + x y z^3 - x y^4
\]
\[
LM(f_2) = x^2 y^8, \quad LT(f_2) = 2x^2 y^8, \quad multideg(f_2) = (2,8,0)
\]

\begin{tcolorbox}[title=Problem 4, breakable]
    Show that grlex is a monomial order according to Definition 1.
\end{tcolorbox}

\begin{proof}
    Grlex compares monomials first by total degree, then by lex order to break ties.  
    Since total degree is a total order on nonnegative integers, and lex order is a total order on exponent vectors,  
    the composition of these two total orders is again a total order.  
    Thus grlex is a monomial order.
\end{proof}

\begin{tcolorbox}[title=Problem 5, breakable]
    Show that grevlex is a monomial order according to Definition 1.
\end{tcolorbox}

\begin{proof}
    Grevle compares monomials first by total degree, then by reverse lex order to break ties.  
    Again, total degree is a total order and reverse lex is a total order on exponent vectors, so their composition is a total order.  
    Therefore, grevlex is a monomial order, just as in Problem 4.
\end{proof}

\begin{tcolorbox}[title=Problem 7, breakable]
    Let $>$ be any monomial order.
    \begin{enumerate}
        \item Show that $\alpha \ge 0$ for all $\alpha \in \mathbb{Z}_{\ge 0}^n$. Hint: Proof by contradiction.
        \item Show that if $x^\alpha$ divides $x^\beta$, then $\alpha \le \beta$. Is the converse true?
        \item Show that if $\alpha \in \mathbb{Z}_{\ge 0}^n$, then $\alpha$ is the smallest element of $\alpha + \mathbb{Z}_{\ge 0}^n = \{\alpha + \beta \mid \beta \in \mathbb{Z}_{\ge 0}^n\}$.
    \end{enumerate}
\end{tcolorbox}

\begin{tcolorbox}[title=Problem 10, breakable]
    In $\mathbb{Z}_{\ge 0}$ with the usual order, between any two integers,
    there are only a finite number of other integers.
    Is this necessarily true in $\mathbb{Z}_{\ge 0}^n$ for a monomial order? Is it true for grlex?
\end{tcolorbox}

\begin{tcolorbox}[title=Problem 11, breakable]
    Let $>$ be a monomial order on $k[x_1, \ldots, x_n]$.
    \begin{enumerate}
        \item Let $f \in k[x_1, \ldots, x_n]$ and let $m$ be a monomial. Show that $LT(m \cdot f) = m \cdot LT(f)$.
        \item Let $f, g \in k[x_1, \ldots, x_n]$. Is $LT(f \cdot g)$ necessarily the same as $LT(f) \cdot LT(g)$?
        \item If $f_i, g_i \in k[x_1, \ldots, x_n]$, $1 \le i \le x$, is $LM(\sum_{i=1}^{s} f_i g_i)$ ncecessarily equal to $LM(f_i) \cdot LM(g_i)$ for some $i$?
    \end{enumerate}
\end{tcolorbox}

\begin{tcolorbox}[title=Problem 12, breakable]
    Lemma 8 gives two properties of the multidegree.
    \begin{enumerate}
        \item Prove Lemma 8. Hint: The arguments used in Exercise 11 may be relevant.
        \item Suppose that $multideg(f) = multideg(g)$ and $f + g \ne 0$.
        Give examples to show that $multideg(f + g)$ may or may not equal $max(multideg(f), multideg(g))$.
    \end{enumerate}
\end{tcolorbox}

\begin{tcolorbox}[title=Problem 13, breakable]
    Prove that $1 < x < x^2 < x^3 < \cdots$ is the unique monimial order on $k[x]$.
\end{tcolorbox}
