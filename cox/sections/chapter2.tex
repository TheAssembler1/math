\subsection{Introduction}

\begin{tcolorbox}[title=Problem 4, breakable]
    Let $x_1, x_2, x_3, \ldots$  be an infinite collection of independent
    variables indexed by the natural numbers. A  polynomial with coefficients 
    in a field $k$ in the $x_i$  is a finite linear combination of (finite)
    monomials $x_{i_1}^{e_1} \ldots x_{i_n}^{e_n}$. Let $R$ detote the set of all 
    polynomials in $x_i$. Note that we an add and multiply elements of $R$ in the usual 
    way. Thus, $R$ is the polynomial ring, $k[x_1, x_2, \ldots]$ in infinitely many variables.
    \begin{enumerate}
        \item Let $I = \langle x_1, x_2, \ldots \rangle$ be the set of $f = \sum_{i = 1}^{\infty} A_i x_i$,
        where $A_i = 0$ for all $i$ sufficiently large. Show that $I$ is an ideal in the ring $R$.
        \item Show, arguing by contradiction, that $I$ has no finite generating set. Hint: note that 
        if $I = \langle g_1, \ldots, g_m \rangle$, then there must be some variable $x_l$ that is not contained
        in any of the $g_j$.
    \end{enumerate}
\end{tcolorbox}

\begin{proof}
    Let $p \in k[x_1, x_2, \ldots]$.
    Let $f, g \in I$ with $j, j'$ being sufficiently large such that $A_j = 0$ and $A_{j'} = 0$.
    Then $f + g = \sum_{i = 1}^{\infty} A_i x_i + \sum_{i = 1}^{\infty} B_i x_i = \sum_{i = 1}^{\infty} (A_i + B_i) x_i \in I$.
    Also, $p \cdot f = p \cdot \sum_{i = 1}^{\infty} A_i x_i \in I$ because it is a finite sum of monomials in the $x_i$.
    Thus $I$ is an ideal.
\end{proof}

\begin{proof}
    Suppose, for contradiction, that $I$ has a finite generating set $\langle g_1, \ldots, g_m \rangle$
        for some $m \in \mathbb{N}$.
    Then there exists some variable $x_l$ that does not appear in any of the $g_j$.
    Clearly, $x_l$ cannot be expressed as a linear combination of the other generators.
    Thus $x_l$ is not in the ideal generated by $g_1, \ldots, g_m$, contradicting that they generate $I$.
\end{proof}

\newpage
\begin{tcolorbox}[title=Problem 5, breakable]
    In this problem you will show that all polynomial parametric curves in $k^2$ are contained
    in affine varieties.
    \begin{enumerate}
        \item Show that the number of distinct monomials $x^a y^b$ of total degree $\le m$
        in $k[x, y]$ is equal to $(m + 1)(m + 2) / 2$. [Note: This is the binomial coefficients 
        $\binom{m + 2}{2}$.]
        \item Show that if $f(t)$ and $g(t)$ are polynomials of degree $\le n$ in $t$, then for $m$ large enough,
        the ``monomials''
        \[[f(t)]^a [g(t)]^b,\]
        with $a + b \le m$ are linearly \emph{dependent}.
        \item Deduce form part (b) that if $C$ is a curve in $k^2$ given parametrically by $x = f(t), y = g(t)$
        for $f(t), g(t) \in k[t],$ then $C$ is contained in $\textbf{V}(f)$ for some nonzero $F \in k[x, y]$
        \item Generalize parts (a), (b), and (c) to show that any polynomial parametric surface 
        \[x = f(t, u), \quad y = g(t, u), \quad z = h(t, u),\]
        is contained in an algebraic surface $\textbf{V}(f)$, where $F \in k[x, y, z]$ is nonzero.
    \end{enumerate}
\end{tcolorbox}

\begin{proof}
    We must find all combinations of $(a, b)$ where $a + b \le m$.
    Now, we first select $a$ and see there are $m - a + 1$ choices remaining for $b$
        since $b = 0, 1, \dots, m - a$.
    Thus, we have the series 
    \[(0 + (m - 0 + 1) + (1 + (m - 1 + 1)) + \cdots + ((m - 1) + (m - (m - 1) + 1)) + (m + (m - m + 1)) ).\]
    Then rearranging terms we see 
    \[(1 + 2 + \cdots + (m+1)) = \frac{(m+1)(m+2)}{2}.\]
\end{proof}

\subsection{Orderings on the Monomials in $k[x_1, \ldots, x_n]$}

\begin{tcolorbox}[title=Problem 1, breakable]
\end{tcolorbox}

\begin{tcolorbox}[title=Problem 4, breakable]
\end{tcolorbox}

\begin{tcolorbox}[title=Problem 5, breakable]
\end{tcolorbox}

\begin{tcolorbox}[title=Problem 7, breakable]
\end{tcolorbox}

\begin{tcolorbox}[title=Problem 10, breakable]
\end{tcolorbox}

\begin{tcolorbox}[title=Problem 11, breakable]
\end{tcolorbox}

\begin{tcolorbox}[title=Problem 12, breakable]
\end{tcolorbox}

\begin{tcolorbox}[title=Problem 13, breakable]
\end{tcolorbox}
