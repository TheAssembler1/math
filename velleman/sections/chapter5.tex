\subsection{Functions}

\begin{tcolorbox}[title=Problem 1, breakable]
    (a) Let $A = \{1, 2, 3\}, B = \{4\}$, and 
        $f = \{(1, 4), (2, 4), (3, 4)\}$.
        Is $f$ a function from $A$ to $B$?

    (b) Let $A = \{1\}, B = \{2, 3, 4\}$ and
        $f = \{(1, 2), (1, 3), (1, 4)\}$.
        Is $f$ a function from $A$ to $B$.

    (c) Let $C$ be the set of all cars registered
        in your state, and let $S$ be the set of all
        finite sequences of letters and digits.
        Let $L = \{(c, s) \in C \times S \mid 
            \text{the license plate number of the car $c$ is $s$}\}$.
        Is $L$ a function from $C$ to $S$.
\end{tcolorbox}

\textbf{Solution (a):}

Yes.

\textbf{Solution (b):}

No.

\textbf{Solution (c):}

Yes.

\begin{tcolorbox}[title=Problem 2, breakable]
    (a) Let $f$ be the relation represented by the graph in Figure $5.3$.
        Is $f$ a function from $A$ to $B$.

    (b) Let $W$ be the set of all words of English, and let $A$ be the set 
        of all letters of the alphabet. 
        Let 
        \[f = \{(w, a) \in W \times A \mid \text{the letter $a$ occurs in the word $w$}\}\]
        and let 
        \[g = \{(w, a) \in W \times A \mid \text{the letter $a$ is the first letter of the word $w$}\}\]
        Is $f$ a function from $W$ to $A$?
        How about $g$?

    (c) John, Mary, Susan, and Fred go out to dinner and sit at a round table.
        Let 
        \[P = \{\text{John, Mary, Susan, Fred}\}\]
        and let 
        \[R = \{(p, q) \in P \times P \mid \text{the person $p$ is sitting immediately to the right of person $q$}\}\]
        Is $R$ a function from $P$ to $P$.
\end{tcolorbox}

\begin{figure}[h]
    \centering
    \includegraphics[width=0.4\textwidth]{images/func.png}
\end{figure}

\textbf{Solution (a):}

No.

\textbf{Solution (b):}

$f$: No.
$g$: Yes.

\textbf{Solution (c):}

Yes.

\begin{tcolorbox}[title=Problem 7, breakable]
    Suppose $f : A \rightarrow B$ and $C \subseteq A$.
    The set $f \cap (C \times B)$, which is a relation from $C$ to $B$,
        is called the \emph{restriction} of $f$ to $C$, and is sometimes
        denoted $f | C$. 
    In other words.\
    \[f | C = f \cap (C \times B)\]
    (a) Prove that $f | C$ is a function from $C$ to $B$ and that for all $c \in C$,
        $f(c) = (f | C)(c)$.

    (b) Suppose $g : C \rightarrow B$. Prove that $g = f | C$ iff $g \subseteq f$.

    (c) Let $g$ and $h$ be the functions defined in parts $2$ and $3$ of Example $5.1.3$.
        Show that $g = h | \mathbb{Z}$.
        \[g : \mathbb{Z} \rightarrow \mathbb{R} \text{ and } g(x) = 2x + 3\]
        \[h : \mathbb{R} \rightarrow \mathbb{R} \text{ and } h(x) = 2x + 3\]
\end{tcolorbox}

\begin{proof}
    We must show that $f|C$ is a function.  
    For all $x \in C$, there exists $y$ such that $(x, y) \in f|C$,  
    and there is exactly one $y \in B$ such that $(x, y) \in f|C$.

    Let $x$ be an arbitrary element in $C$.
    Since $C \subseteq A$, it follows that $x \in A$.
    Since $f$ is a function from $A$ to $B$, 
        there exists $y \in B$ such that $(x, y) \in f$.
    Clearly, $(x, y) \in A \times B$.
    Since $(x, y) \in f$ and $(x, y) \in C \times B$,
        it follows that $(x, y) \in f \cap (C \times B) = f|C$.
    Thus $f|C$ maps all elements in $C$ to $B$.

    Notice that $f \cap (C \times B) \subseteq f$.
    Then, since $(x, y) \in f|C$, it follows that $(x, y) \in f$.
    It follows that, since $f$ is a function, 
        $x$ maps to exactly one element in $B$, namely $y$.
\end{proof}

\begin{proof}
    Suppose $g : C \rightarrow B$.

    ($\rightarrow$) Suppose $g = f|C$.
    Then $g = f|C = f \cap (C \times B) \subseteq f$.

    ($\leftarrow$) Suppose $g \subseteq f$.
    Let $(x, y)$ be an arbitrary element in $g$.
    Since $g \subseteq f$, it follows that $(x, y) \in f$.
    Since $(x, y) \in g$, it follows that $(x, y) \in C \times B$.
    Thus $(x, y) \in f$ and $(x, y) \in C \times B$.
    Therefore $(x, y) \in f \cap (C \times B) = f|C$.
    It follows that $g \subseteq f|C$.

    Let $(x, y)$ be an arbitrary element in $f|C$.
    It follows that $(x, y) \in f$ and $(x, y) \in C \times B$.
    Since $g \subseteq f$ and $x \in C$, it follows that $(x, y) \in g$.
    Thus $f|C \subseteq g$.
\end{proof}

\begin{proof}
    We need to show that $g = h|\mathbb{Z} = h \cap (\mathbb{Z} \times \mathbb{R})$.
    
    Suppose $(x, y)$ is an arbitrary element in $g$.
    We know that $(x, y) \in \mathbb{Z} \times \mathbb{R}$.
    It follows that $x \in \mathbb{Z}$ and $y \in \mathbb{R}$.
    Since $\mathbb{Z} \subseteq \mathbb{R}$, it follows that $x \in \mathbb{R}$.
    Therefore $(x, y) \in h$.
    Since $(x, y) \in h$ and $(x, y) \in \mathbb{Z} \times \mathbb{R}$,
        it follows that $(x, y) \in h|\mathbb{Z}$.
    Thus $g \subseteq h|\mathbb{Z}$.

    Suppose $(x, y)$ is an arbitrary element in $h|\mathbb{Z}$.
    It follows that $(x, y) \in h$ and $(x, y) \in \mathbb{Z} \times \mathbb{R}$.
    Since $x \in \mathbb{Z}$, it follows that $(x, y) \in g$.
    Thus $h|\mathbb{Z} \subseteq g$.
\end{proof}

\begin{tcolorbox}[title=Problem 8, breakable]
    Suppose $f : A \rightarrow B$ and $g \subseteq f$.
    Prove that there is a set $A' \subseteq A$ such that 
        $g : A' \rightarrow B$.
\end{tcolorbox}

\begin{proof}
    Let $A' = dom(g)$.
    It is clear that $A' \subseteq A$,
        since $dom(g) \subseteq dom(f)$.
    Let $x \in A'$. Then there exists $y$ such that $(x, y) \in g$.
    Since $g \subseteq f$, it follows that $(x, y) \in f$.
    There is only a single element that $x$ maps to,
        since $g \subseteq f$ and $f$ is a function.
    Thus $g : A' \rightarrow B$.
\end{proof}

\begin{tcolorbox}[title=Problem 9, breakable]
    Suppose $f : A \rightarrow B$, $B \ne \emptyset$, and $A \subseteq A'$.
    Prove that there is a function $g : A' \rightarrow B$ such that $f \subseteq g$.
\end{tcolorbox}

\begin{proof}
    Let $A'' = A' \setminus A$.
    Since $B \ne \emptyset$, let $y$ be an element in $B$.
    Let $g = f \cup \{(x, y) \mid x \in A''\}$.
    Then for each $x \in A$ there is a unique $y$ such that $(x, y) \in f$,
    and for each $x \in A''$ there is exactly one pair $(x, y)$ in $g$.
    Thus $g$ is a function from $A'$ to $B$ and $f \subseteq g$.
\end{proof}

\begin{tcolorbox}[title=Problem 11, breakable]\
    Suppose $A$ is a set. 
    Show that $i_A$ is the only relation on $A$ that 
        is both an equivalence relation and also a function from $A$ to $A$.
\end{tcolorbox}

\begin{proof}
    For contradiction, suppose $R \ne i_A$ is an equivalence relation on $A$
        and $R : A \rightarrow A$.
    Then there exists a pair $(x, y) \in R$ such that $x \ne y$.
    Since $R$ is reflexive, $(x, x) \in R$.
    Thus $R$ contains both $(x, x)$ and $(x, y)$ with $y \ne x$,
        contradicting that $R$ is a function.
    Therefore $R = i_A$.
\end{proof}

\begin{tcolorbox}[title=Problem 12, breakable]
    Suppose $f : A \rightarrow C$ and $g : B \rightarrow C$.

    (a) Prove that if $A$ and $B$ are disjoint, then $f \cup g : A \cup B \rightarrow C$.

    (b) Prove that $f \cup g : A \cup B \rightarrow C$ iff $f | (A \cap B) = g | (A \cap B)$.
        (See excersize $7$ for the meaning of the notation used here.)
\end{tcolorbox}

\begin{proof}
    Suppose $A$ and $B$ are disjoint.
    Let $x$ be an arbitrary element in $A \cup B$.
    Either $x \in A$ or $x \in B$.
    If $x \in A$, then under $f$ there exists a unique $y \in C$
        such that $(x, y) \in f$.
    If $x \in B$, then under $g$ there exists a unique $y \in C$
        such that $(x, y) \in g$.
    Since $A$ and $B$ are disjoint, $f \cup g$ assigns exactly one $y \in C$
        to each $x \in A \cup B$.
    Therefore $f \cup g : A \cup B \rightarrow C$.
\end{proof}

\begin{proof}
    ($\rightarrow$) Suppose $f \cup g : A \cup B \rightarrow C$.
    For contradiction assume $f | (A \cap B) \ne g | (A \cap B)$.
    Suppose w.l.o.g. since $f | (A \cap B) \ne g | (A \cap B)$
        there is a pair $(x, y) \in f | (A \cap B)$
        such that $(x, y) \notin g | (A \cap B)$.
    Now clearly $(x, y) \in f$, but since $g$ is a function on $B$
        and $x \in A \cap B$, there is a pair $(x, y') \in g$.
    Thus $f \cup g$ results in two mappings from $x$ to $y$ and $y'$,
    which contradicts that $f \cup g$ is a function.
    Therefore, $f | (A \cap B) = g | (A \cap B)$.

    ($\leftarrow$) Suppose $f | (A \cap B) = g | (A \cap B)$.
    We must show that $f \cup g : A \cup B \rightarrow C$.
    Let $(x, y)$ and $(x, y')$ be elements of $f \cup g$.
    We need to show $y = y'$.
    If both pairs are in $f$ or both in $g$, this follows since
    $f$ and $g$ are functions.
    If one pair is from $f$ and the other from $g$, then
    $x \in A \cap B$ and hence $(x, y) \in f | (A \cap B)$
    and $(x, y') \in g | (A \cap B)$.
    By assumption $f | (A \cap B) = g | (A \cap B)$, so $y = y'$.
    Thus $f \cup g$ is a function from $A \cup B$ to $C$.
\end{proof}

\begin{tcolorbox}[title=Problem 13, breakable]
    Suppose $R$ is a relation from $A$ to $B$, $S$ is a relation 
    from $B$ to $C$, $Ran(R) = Dom(S) = B$, and $S \circ R : A \rightarrow C$.

    (a) Prove that $S : B \rightarrow C$.

    (b) Give an example to show that it need not be the case 
        that $R : A \rightarrow B$.
\end{tcolorbox}

\begin{proof}
    Let $b$ be an arbitrary element in $B$.
    Since $Ran(R) = Dom(S) = B$, there exists $a \in A$ such that $(a, b) \in R$.
    Because $S \circ R$ is a function, there is exactly one $c \in C$
        such that $(a, c) \in S \circ R$.
    By the definition of composition, this means $(a, b) \in R$ and $(b, c) \in S$.
    Thus $S$ maps $b$ to a unique $c \in C$.
    Therefore, $S : B \rightarrow C$.
\end{proof}

\textbf{Solution (b):}
\[R = \{(1, 2), (1, 3)\}, S = \{(2, 4), (3, 4)\}, S \circ R = \{(1, 4)\}\]

\begin{tcolorbox}[title=Problem 17, breakable]
    Suppose $A$ is a nonempty set and $f : A \rightarrow A$.

    (a) Suppose there is some $a \in A$ such that 
        $\forall{x} \in A(f(x) = a)$. (In this case, $f$ is called 
        a constant function.) Prove that for all $g : A \rightarrow A$,
        $f \circ g  = f$.

    (b) Suppose that for all $g : A \rightarrow A$, $f \circ g = f$.
        Prove that $f$ is a constant function. (Hint: what happens
        if $g$ is a constant function?)
\end{tcolorbox}

\begin{proof}
    Let $(x, z)$ be an arbitrary pair in $f \circ g$.
    There exists $y$ such that $(x, y) \in g$ and $(y, z) \in f$.
    It follows, since $f$ is a constant function, that $(x, z) \in f$.
    Thus $f \circ g \subseteq f$.

    Let $(x, z)$ be an arbitrary pair in $f$.
    Now, since $g : A \rightarrow A$ is a function, for this $x \in A$
        there exists a unique $y \in A$ such that $(x, y) \in g$.
    Since $f$ is a constant function and $y \in A$,
        it follows that $(y, z) \in f$.
    Since $(x, y) \in g$ and $(y, z) \in f$, 
        it follows that $(x, z) \in f \circ g$.
    Thus $f \subseteq f \circ g$.

    Therefore $f \circ g = f$.
\end{proof}

\begin{proof}
    Let $x_0 \in A$ and define 
        $g : A \to A$ such that $g(x) = x_0$ for all $x \in A$.
    Then for every $x \in A$, $(f \circ g)(x) = f(g(x)) = f(x_0)$.
    By assumption, $f \circ g = f$, so for every $x \in A$, $f(x) = (f \circ g)(x) = f(x_0)$.
    Therefore, $f$ maps every element of $A$ to the same value.
    It follows that $f$ is a constant function.
\end{proof}

\begin{tcolorbox}[title=Problem 19, breakable]
    Let $\mathcal{F} = \{f \mid f : \mathbb{Z}^+ \rightarrow \mathbb{R}\}$.
    For $g \in \mathcal{F}$, we define the set $O(g)$ as follows:
    \[O(g) = \{f \in \mathcal{F} \mid \exists{a} \in \mathbb{Z}^+
                \exists{c} \in \mathbb{R}^+ \forall{x} > a (|f(x)| \le c|g(x)|)\}\]
    (If $f \in O(g)$, then mathematicians say ``$f$ is big-oh of $g$''.)

    (a) Let $f : \mathbb{Z}^+ \rightarrow \mathbb{R}$ and $g : \mathbb{Z}^+ \rightarrow \mathbb{R}$
        be defined by the formulas $f(x) = 7x + 3$ and $g(x) = x^2$. 
        Prove that $f \in O(g)$, but $g \notin O(f)$.

    (b) Let $S = \{(f, g) \in \mathcal{F} \times \mathcal{F} \mid f \in O(g)\}$. Prove that 
        $S$ is a preorder, but not a partial order. (See excersize $25$  of Section $4.5$
        for the definition of \emph{preorder}.)

    (c) Suppose $f_1 \in O(g)$ and $f_2 \in O(g)$, and $s$ and $t$ are real numbers.
        Define a function $f : \mathbb{Z}^+ \rightarrow \mathbb{R}$ by the formula 
        $f(x) = sf_1(x) + tf_2(x)$. Prove that $f \in O(g)$.
        (Hint: You may find the triangle inequality helpful. See excersize $13$(c)
         of Section $3.5$.)
\end{tcolorbox}

\begin{proof}
    Let $a = 500$ and $c = 1$.
    Then $|f(x)| \le c|g(x)| \iff |7x + 3| \le 1|x^2|$.
    With $x \ge a = 500$ we have $7x + 3 \le x^2$.
    At the point $x = 500$, we have $7(500) + 3 = 3503 \le 500^2 = 250000$.
    To show that $|f(x)| \le c|g(x)|$ for all $x > a$, we can look at the derivatives:
        $f'(x) = 7$ and $g'(x) = 2x$, and for $x > 500$, $g'(x) > f'(x)$, so $g(x)$ grows faster than $f(x)$.
    Thus $f \in O(g)$.
\end{proof}

\begin{proof}
    Suppose $g \in O(f)$ then it follows that 
        there exists $c \in \mathbb{R}^+$ and $a \in \mathbb{Z}^+$
        such that for all $x > a$, $|g(x)| \le c|f(x)|$.
    Now plugging in gives $|x^2| \le c|7x + 3|$.
    Since $x^2$ grows faster than $7x + 3$, for any fixed $c$
        there exists $x > a$ such that $x^2 > c(7x + 3)$,
        which is a contradiction.
    Therefore, $g \notin O(f)$.
\end{proof}

\begin{proof}
    We must show $S$ is reflexive and transitive on $\mathcal{F}$.
    Suppose $f$ is an arbitrary element in $\mathcal{F}$.
    Clearly if we let $a = 1, c = 1$ then $|f(x)| \le |f(x)|$
        thus $(f, f) \in S$.

    Suppose $(f, g), (g, t)$ are arbitrary pairs in $S$.
    Now there exists $a_1 \in \mathbb{Z}^+$ and $c_1 \in \mathbb{R}^+$
        such that for all $x > a_1$,
        $|f(x)| \le c_1 |g(x)|$.
    Similarly there exists $a_2 \in \mathbb{Z}^+$ and $c_2 \in \mathbb{R}^+$
        such that for all $x > a_2$,
        $|g(x)| \le c_2 |t(x)|$.
    Let $a = \max(a_1, a_2)$. Then for all $x > a$,
        $|f(x)| \le c_1 |g(x)| \le c_1 c_2 |t(x)|$.
    Thus $(f, t) \in S$.
\end{proof}

\begin{proof}
    Since $f_1 \in O(g)$ and $f_2 \in O(g)$ it follows 
        there exists $c_1, c_2 \in \mathbb{R}^+$ and $a_1, a_2 \in \mathbb{Z}^+$
        such that for all $x > a_1$ and $x > a_2$,
        $|f_1(x)| \le c_1 |g(x)|$ and $|f_2(x)| \le c_2 |g(x)|$.
    Then since $|f_1(x)| \le c_1 |g(x)|$ it follows that 
        $|s f_1(x)| \le |s| c_1 |g(x)|$.
    Similarly $|t f_2(x)| \le |t| c_2 |g(x)|$.
    Taking sums gives 
        $|s f_1(x)| + |t f_2(x)| \le (|s| c_1 + |t| c_2)|g(x)|$.
    By the triangle inequality 
        $|s f_1(x) + t f_2(x)| \le |s f_1(x)| + |t f_2(x)|$.
    Let $a = \max(a_1, a_2)$. Then for all $x > a$
        it follows that $|s f_1(x) + t f_2(x)| \le (|s| c_1 + |t| c_2)|g(x)|$.
    Let $c = |s| c_1 + |t| c_2$. Then $|f(x)| \le c |g(x)|$ for all $x > a$,
    and therefore $f \in O(g)$.
\end{proof}

\begin{tcolorbox}[title=Problem 21, breakable]
    Suppose $f : A \rightarrow B$ and $R$ is an equivalence relation on $A$.
    We will say that $f$ is compatible with $R$ if 
    $\forall{x} \in A \forall{y} \in A (xRy \rightarrow f(x) = f(y))$.

    (a) Suppose $f$ is compatible with $R$ prove that there is a unique function 
        $h : A / R \rightarrow B$ such that for all $x \in A, h([x]_R) = f(x)$.

    (b) Suppose $h : A / R \rightarrow B$ and for all $x \in A, h([x]_R) = f(x)$.
        Prove that $f$ is compatible with $R$.
\end{tcolorbox}

\begin{proof}
    First note that since $R$ is an equivalence relation, 
        for all $x \in A$ there exists $y$ such that $xRy$.
    This follows since $R$ is reflexive, so $xRx$.
    Define a function $h : A / R \rightarrow B$ by setting 
        $h([x]_R) = f(x)$ for each $x \in A$.

    Suppose $x, y \in A$ such that $[x]_R = [y]_R$.
    Then $xRy$ and it follows since $f$ is compatible with $R$ that $f(x) = f(y)$.

    Uniqueness of $h$ follows because such a function must satisfy 
        $h([x]_R) = f(x)$ for all $x \in A$.
\end{proof}

\begin{proof}
    Let $x, y$ be arbitrary elements in $A$
        such that $(x, y) \in R$.
    Since $(x, y) \in R$ it follows that 
        $f(x) = h([x]_R) = h([y]_R) = f(y)$.
    Thus $f(x) = f(y)$.
    Therefore $f$ is compatible with $R$.
\end{proof}

\begin{tcolorbox}[title=Problem 22, breakable]
    Let $R = \{(x, y) \in \mathbb{N} \times \mathbb{N} \mid x \equiv y \pmod{5}\}$.
    Note that by Theorem $4.5.10$ and excersize $14$ in Section $4.5$, $R$ is an 
    equivalence relation on $\mathbb{N}$.

    (a) Show that there is a unique function $h : \mathbb{N} / R \rightarrow \mathbb{N} / R$
        such that for every natural number $x, h([x]_R) = [x^2]_R$. (Hint: Use excersize $21$.)

    (b) Show that there is no function $h : \mathbb{N} / R \rightarrow \mathbb{N} / R$
        such that for every natural number $x, h([x]_r) = [2^x]_R$.
\end{tcolorbox}

\begin{proof}
    Suppose $x, y \in \mathbb{N}$ such that $[x]_R = [y]_R$.
    Then $x = 5k_1 + r$ and $y = 5k_2 + r$ for some $k_1, k_2 \in \mathbb{Z}$.
    Now $x^2 = 25k_1^2 + 10k_1 r + r^2$ and $y^2 = 25k_2^2 + 10k_2 r + r^2$.
    Taking differences gives $x^2 - y^2 = 25(k_1^2 - k_2^2) + 10r(k_1 - k_2)$.
    It follows that $x^2 - y^2$ is divisible by $5$, so $[x^2]_R = [y^2]_R$.
    From problem $21$, let $A = \mathbb{N}$ and $R' = R$,
        where $R'$ refers to $R$ from the previous problem.
    Then this follows immediately.
\end{proof}

\begin{proof}
    For contradiction, suppose there exists a function $h : \mathbb{N} / R \rightarrow \mathbb{N} / R$
    such that for every natural number $x, h([x]_R) = [2^x]_R$.
    Let $x, y \in \mathbb{N}$ such that $x \ne y$ and $y \in [x]_R$.
    It follows that $h([x]_R) = h([y]_R)$, so $[2^x]_R = [2^y]_R$.
    Now let $x = 1$ and $y = 6$. Then $y \in [x]_R$ because $6 \equiv 1 \pmod{5}$.
    But $2^1 = 2 \equiv 2 \pmod{5}$ and $2^6 = 64 \equiv 4 \pmod{5}$, 
    so $[2^1]_R \ne [2^6]_R$. This is a contradiction.
\end{proof}

\subsection{One-to-One and Onto}

\begin{tcolorbox}[title=Problem 1, breakable]
    Which of the functions in exercize $1$ of Section $5.1$
        are one-to-one? Which are onto?

    (a) Let $A = \{1, 2, 3\}, B = \{4\}$, and 
        $f = \{(1, 4), (2, 4), (3, 4)\}$.
        Is $f$ a function from $A$ to $B$?

    (b) Let $A = \{1\}, B = \{2, 3, 4\}$ and
        $f = \{(1, 2), (1, 3), (1, 4)\}$.
        Is $f$ a function from $A$ to $B$.

    (c) Let $C$ be the set of all cars registered
        in your state, and let $S$ be the set of all
        finite sequences of letters and digits.
        Let $L = \{(c, s) \in C \times S \mid 
            \text{the license plate number of the car $c$ is $s$}\}$.
        Is $L$ a function from $C$ to $S$.
\end{tcolorbox}

\textbf{Solution (a):}
\begin{center}
    One-to-one: No, Onto: Yes.
\end{center}
\textbf{Solution (b):}
\begin{center}
    Was not a function.
\end{center}
\textbf{Solution (c):}
\begin{center}
    One-to-one: Yes, Onto: No (Some license plates haven't been assigned to vehicles.)
\end{center}

\begin{tcolorbox}[title=Problem 2, breakable]
    Which of the functions in exercize $2$ of Section $5.1$
        are one-to-one? Which are onto?

    (a) Let $f$ be the relation represented by the graph in Figure $5.3$.
        Is $f$ a function from $A$ to $B$.

    (b) Let $W$ be the set of all words of English, and let $A$ be the set 
        of all letters of the alphabet. 
        Let 
        \[f = \{(w, a) \in W \times A \mid \text{the letter $a$ occurs in the word $w$}\}\]
        and let 
        \[g = \{(w, a) \in W \times A \mid \text{the letter $a$ is the first letter of the word $w$}\}\]
        Is $f$ a function from $W$ to $A$?
        How about $g$?

    (c) John, Mary, Susan, and Fred go out to dinner and sit at a round table.
        Let 
        \[P = \{\text{John, Mary, Susan, Fred}\}\]
        and let 
        \[R = \{(p, q) \in P \times P \mid \text{the person $p$ is sitting immediately to the right of person $q$}\}\]
        Is $R$ a function from $P$ to $P$.
\end{tcolorbox}

\begin{figure}[h]
    \centering
    \includegraphics[width=0.4\textwidth]{images/func.png}
\end{figure}

\textbf{Solution (a):}
\begin{center}
    Was not a function.
\end{center}
\textbf{Solution (b):}
\begin{center}
    $f$ was not a function.
    $g$ One-to-one: no, Onto: yes.
\end{center}
\textbf{Solution (c):}
\begin{center}
    One-to-one: yes, Onto: yes.
\end{center}

\begin{tcolorbox}[title=Problem 3, breakable]
    Which of the functions in exercize $3$ of Section $5.1$
        are one-to-one? Which are onto?

    (a) Let $A = \{a, b, c\}, B = \{a, b\}$, and $f = \{(a, b), (b, b), (c, a)\}$.

    (b) Let $f : \mathbb{R} \rightarrow \mathbb{R}$ be the function defined by the
        formula $f(x) = x^2 - 2x$.

    (c) Let $f = \{(x, n) \in \mathbb{R} \times \mathbb{Z} \mid n \le x < n + 1\}$. 
        Then $f : \mathbb{R} \rightarrow \mathbb{Z}$.
\end{tcolorbox}

\textbf{Solution (a):}
\begin{center}
    One-to-one: no, Onto: yes.
\end{center}
\textbf{Solution (b):}
\begin{center}
    One-to-one: no, Onto: no.
\end{center}
\textbf{Solution (c):}
\begin{center}
    One-to-one: no, Onto: yes.
\end{center}

\begin{tcolorbox}[title=Problem 4, breakable]
    Which of the functions in exercize $4$ of Section $5.1$
        are one-to-one? Which are onto?

    (a) Let $N$ be the set of all countries and $C$ the set of all 
        cities. Let $H : N \rightarrow C$ be the function defined by the rule 
        that for every country $n$, $H(n) =$ the capital of the country $n$.

    (b) Let $A = \{1, 2, 3\}$ and $B = \mathcal{P}(A)$.
        Let $F : B \rightarrow B$ be the function defined by the 
            formula $F(X) = A \setminus X$.

    (c) Let $f : \mathcal{R} \rightarrow \mathcal{R}$ be the function 
        defined by the formula $f(x) = (x + 1, x - 1)$.
\end{tcolorbox}

\textbf{Solution (a):}
\begin{center}
    One-to-one: no, Onto: yes.
\end{center}
\textbf{Solution (b):}
\begin{center}
    One-to-one: yes, Onto: yes.
\end{center}
\textbf{Solution (c):}
\begin{center}
    One-to-one: yes, Onto: no.
\end{center}

\begin{tcolorbox}[title=Problem 6, breakable]
    Suppose $a$ and $b$ are real numbers and $a \ne 0$. 
    Define $f : \mathbb{R} \rightarrow \mathbb{R}$ by the 
        formula $f(x) = ax + b$.
    Show that $f$ is one-to-one and onto.
\end{tcolorbox}

\begin{proof}
    Let $x_1, x_2$ be arbitrary numbers in $\mathbb{R}$.
    Then 
    \begin{align*}
        f(x_1) &= f(x_2) \\
        \iff a(x_1) + b &= a(x_2) + b \\
        \iff a(x_1) &= a(x_2) \\
        \iff x_1 &= x_2 \quad \text{Note: $a \ne 0$}
    \end{align*}
    Thus $f$ is one-to-one.
\end{proof}

\begin{proof}
    Let $y$ be an arbitrary number in $\mathbb{R}$.
    Let $x = \frac{y - b}{a}$ which is defined since $a \ne 0$.
    Then 
    \begin{align*}
        f\left(\frac{y - b}{a}\right)
            &= a\left(\frac{y - b}{a}\right) + b \\
            &= y - b + b \\
            &= y
    \end{align*}
\end{proof}

\begin{tcolorbox}[title=Problem 8, breakable]
    Let $A = \mathcal{P}(\mathbb{R})$. 
    Define $f : \mathbb{R} \rightarrow A$
        by the formula $f(x) = \{y \in \mathbb{R} \mid y^2 < x\}$.
    
    (a) Find $f(2)$.

    (b) Is $f$ one-to-one? Is it onto?
\end{tcolorbox}

\textbf{Solution (a):}
\[f(2) = \{x \mid -\sqrt{2} < x < \sqrt{2}\}\]

\textbf{Solution (b):}
\begin{center}
    One-to-one: no, Onto: no.
\end{center}

\begin{tcolorbox}[title=Problem 10, breakable]
    Suppose $f : A \rightarrow B$ and $g : B \rightarrow C$.

    (a) Prove that if $g \circ f$ is onto then $g$ is onto.

    (b) Prove that if $g \circ f$ is one-to-one then $f$ is one-to-one.
\end{tcolorbox}

\begin{proof}
    Suppose $g \circ f$ is onto.
    Let $y$ be an arbitrary element in $C$.
    Since $g \circ f$ is onto it follows that 
        there exists $x$ such that $(g \circ f)(x) = y$.
    Then there exists $c$ such that 
        $f(x) = c$ and $g(c) = y$.
    Thus $g$ is onto.
\end{proof}

\begin{proof}
    Suppose $x_1, x_2$ are arbitrary elements in $A$
        such that $f(x_1) = f(x_2)$.
    Then applying $g$ to both sides gives 
        $(g \circ f)(x_1) = (g \circ f)(x_2)$. 
    Then since $g \circ f$ is one-to-one it follows that 
        $x_1 = x_2$.
    Thus $f$ is one-to-one.
\end{proof}

\begin{tcolorbox}[title=Problem 11, breakable]
    Suppose $f : A \rightarrow B$ and $g : B \rightarrow C$.

    (a) Prove that if $f$ is onto and $g$ is not one-to-one,
        then $g \circ f$ is not one-to-one.

    (b) Prove that if $f$ is not onto and $g$ is one-to-one, 
        then $g \circ f$ is not onto.
\end{tcolorbox}

\begin{proof}
    Suppose $f$ is onto and $g$ is not one-to-one.
    Since $g$ is not one-to-one there exist 
        $x_1, x_2 \in B$ such that $x_1 \ne x_2$ and $g(x_1) = g(x_2)$.
    Now since $f$ is onto there exist $c_1, c_2 \in A$
        such that $f(c_1) = x_1$ and $f(c_2) = x_2$.
    If $c_1 = c_2$, then $x_1 = f(c_1) = f(c_2) = x_2$, a contradiction.
    Thus $c_1 \ne c_2$.
    Then $(g \circ f)(c_1) = g(f(c_1)) = g(x_1) = g(x_2) = g(f(c_2)) = (g \circ f)(c_2)$.
    Thus $g \circ f$ is not one-to-one.
\end{proof}

\begin{proof}
    Since $f$ is not onto there exists $y \in B$
        such that for all $x \in A$, $f(x) \ne y$.
    Let $c = g(y) \in C$.
    Now if for some $x \in A$, $(g \circ f)(x) = g(y)$,
        then since $g$ is one-to-one, $f(x) = y$, a contradiction.
    Thus $g \circ f$ is not onto.
\end{proof}


\begin{tcolorbox}[title=Problem 13, breakable]
    Suppose $f : A \rightarrow B$ and $C \subseteq A$.
    In exercize $7$ of Section $5.1$ we defined 
        $f | C$ (the restriction of $f$ to $C$),
        and you showed that $f | C : C \rightarrow B$

    (a) Prove that if $f$ is one-to-one, then so is $f | C$.

    (b) Prove that if $f | C$ is onto, then so is $f$.

    (c) Give examples to show that the converses of parts (a)
        and (b) are not always true.
\end{tcolorbox}

\begin{proof}
    Suppose $f$ is one-to-one.
    Let $x_1, x_2$ be two arbitrary elements in $C$ such that $(f | C)(x_1) = (f | C)(x_2)$.
    It follows that $(x_1, (f | C)(x_1)) \in f$ and $(x_2, (f | C)(x_2)) \in f$.
    Then since $f$ is one-to-one it follows that $x_1 = x_2$.
\end{proof}

\begin{proof}
    Suppose $f | C$ is onto.
    Let $y$ be an arbitrary element in $B$.
    Since $f | C$ is onto there exists $x \in C$ such that $(f | C)(x) = y$.
    It follows that $(x, y) \in f$ thus $f(x) = y$.
    Therefore $f$ is onto.
\end{proof}

\textbf{Solution:}

Counter example: If $f | C$ is one-to-one then $f$ is one-to-one.
\[A = \{1, 2\}, B = \{1, 2\}, C = \{1\}, f = \{(1, 1), (2, 1)\}, f | C = \{(1, 1)\}\]
Counter example: If $f$ is onto then $f | C$ is onto.
\[A = \{1, 2\}, B = \{1, 2\}, C = \{1\}, f = \{(1, 1), (2, 2)\}, f | C = \{(1, 1)\}\]

\begin{tcolorbox}[title=Problem 14, breakable]
    Suppose $f : A \rightarrow B$, and there is some $b \in B$
        such that $\forall{x} \in A (f(x) = b)$.
        (Thus, $f$ is a constant function.)

    (a) Prove that if $A$ has more than one element then $f$ is 
        not one-to-one.

    (b) Prove that if $B$ has more than one element then $f$ is 
        not onto.
\end{tcolorbox}

\begin{proof}
    Suppose that $A$ has more than one element.
    Since $A$ has more than one element there exists $x_1, x_2 \in A$
        such that $x_1 \ne x_2$ and $f(x_1) = b$ and $f(x_2) = b$.
    Thus $f$ is not one-to-one.
\end{proof}

\begin{proof}
    Suppose that $B$ has more than one element.
    Then there exists $y \in B$ such that $y \ne b$.
    Since $f(x) = b$ for all $x \in A$, there does not exist $x \in A$ such that $f(x) = y$.
    Therefore, $f$ is not onto.
\end{proof}

\begin{tcolorbox}[title=Problem 15, breakable]
    Suppose $f : A \rightarrow C$, $g : B \rightarrow C$,
        and $A$ and $B$ are disjoint. In excersize 
        $12(a)$ of Section $5.1$ you proved that 
        $f \cup g : A \cup B \rightarrow C$.
    Now suppose that $f$ and $g$ are both one-to-one.
    Prove that $f \cup g$ is one-to-one iff $Ran(f)$ and $Ran(g)$
        are disjoint.
\end{tcolorbox}

\begin{proof}
    ($\rightarrow$) Suppose $f \cup g$ is one-to-one.
    For contradiction, suppose $Ran(f) \cap Ran(g) \ne \emptyset$.
    Let $y$ be an element in $Ran(f) \cap Ran(g)$.
    It follows there exists $x_1 \in A$ and $x_2 \in B$
        such that $f(x_1) = y$ and $g(x_2) = y$.
    Since $A$ and $B$ are disjoint, $x_1 \ne x_2$,
    contradicting that $f \cup g$ is one-to-one.
    Thus $Ran(f)$ and $Ran(g)$ are disjoint.

    ($\leftarrow$) Suppose $Ran(f)$ and $Ran(g)$ are disjoint.
    Suppose $x_1, x_2 \in A \cup B$ such that $(f \cup g)(x_1) = (f \cup g)(x_2)$.
    Since $A$ and $B$ are disjoint, either $x_1, x_2 \in A$ or $x_1, x_2 \in B$.
    Suppose w.l.o.g.  $x_1 \in A$ and $x_2 \in B$ but
    then $(f \cup g)(x_1) \in Ran(f)$ and $(f \cup g)(x_2) \in Ran(g)$,
        which are disjoint.
    Suppose w.l.o.g. $x_1, x_2 \in A$. Since $f$ is one-to-one, it follows that 
        $x_1 = x_2$.
    Thus $f \cup g$ is one-to-one.
\end{proof}

\begin{tcolorbox}[title=Problem 16, breakable]
    Suppose $R$ is a relation from $A$ to $B$, $S$ is a relation 
        from $B$ to $C$, $Ran(R) = Dom(S) = B$,
        and $S \circ R : A \rightarrow C$.
    In exercize $13(a)$ of Section $5.1$ you proved that 
        $S : B \rightarrow C$.
    Now prove that if $S$ is one-to-one then $R : A \rightarrow B$.
\end{tcolorbox}

\begin{proof}
    Suppose $S$ is one-to-one.
    Let $x$ be an arbitrary element in $A$.
    Since $S \circ R$ is a function, there exists $y \in C$
        such that $(S \circ R)(x) = y$.
    It follows that there exists $c \in B$ such that 
        $R(x) = c$ and $S(c) = y$.
    Thus $R$ maps each element in $A$ to some element in $B$.

    Suppose there exists $x \in A$ and $y_1, y_2 \in B$
        such that $y_1 \ne y_2$ and $(x, y_1), (x, y_2) \in R$.
    Since $Ran(R) = Dom(S)$, there exist $z_1, z_2 \in C$
        such that $(y_1, z_1) \in S$ and $(y_2, z_2) \in S$.
    Then $(x, z_1), (x, z_2) \in (S \circ R)$.
    Since $S$ is one-to-one and $y_1 \ne y_2$, it follows that $z_1 \ne z_2$,
        contradicting that $S \circ R$ is a function.
    Thus for every $x \in A$ there exists a single $y \in B$
        such that $(x, y) \in R$.
\end{proof}

\begin{tcolorbox}[title=Problem 18, breakable]
    Suppose $R$ is an equivalence relation on $A$,
        and let $g : A \rightarrow A / R$ be defined 
        by the formula $g(x) = [x]_R$, as in exersize $20(b)$
        in Section $5.1$.

    (a) Show that $g$ is onto.

    (b) Show that $g$ is one-to-one iff $R = i_A$.
\end{tcolorbox}

\begin{proof}
    Let $T$ be an arbitrary element in $A / R$.
    Since $T$ is not empty, there exists $x$ such that $T = [x]_R$.
    It follows that $g(x) = [x]_R = T$.
    Thus $g$ is onto.
\end{proof}

\begin{proof}
    ($\rightarrow$) Suppose $g$ is one-to-one.
    Let $x$ be an arbitrary element in $A$.
    Since $R$ is reflexive, $(x, x) \in R$,
        thus $x \in [x]_R$.
    Since $g$ is one-to-one, no other element in $A$
        maps to $[x]_R$.
    It follows that each equivalence class contains exactly one element,
        so $R = i_A$.

    ($\leftarrow$) Suppose $R = i_A$.
    Let $x_1, x_2$ be arbitrary elements in $A$ such that $g(x_1) = g(x_2)$.
    Then $[x_1]_R = [x_2]_R$.
    Since each equivalence class contains exactly one element, it follows that $x_1 = x_2$.
    Therefore, $g$ is one-to-one.
\end{proof}

\begin{tcolorbox}[title=Problem 19, breakable]
    Suppose $f : A \rightarrow B$, $R$ is an equivalence relation on $A$,
        and $f$ is compatible with $R$. (See exersize $21$ of Section $5.1$
        for the definition of \emph{compatible}.)
    In exercize $21(a)$ of Section $5.1$ you proved that there is a unique 
        function $h : A / R \rightarrow B$ such that for all $x \in A$,
        $h([x]_R) = f(x)$. Now prove that $h$ is one-to-one iff
        $\forall{x} \in A \forall{y} \in A (f(x) = f(y) \rightarrow xRy)$.
\end{tcolorbox}

\begin{proof}
    ($\rightarrow$) Suppose $h$ is one-to-one.
    Let $x, y$ be arbitrary elements in $A$.
    Furthermore, suppose $f(x) = f(y)$.
    It follows that $h([x]_R) = h([y]_R)$.
    Then, since $h$ is one-to-one, $[x]_R = [y]_R$.
    It follows that $(x, y) \in R$.

    ($\leftarrow$) Suppose $\forall x, y \in A, (f(x) = f(y) \rightarrow xRy)$.
    Let $x_1, x_2$ be arbitrary elements in $A$
        such that $h([x_1]_R) = h([x_2]_R)$.
    Thus $f(x_1) = f(x_2)$, and it follows that $(x_1, x_2) \in R$.
    Therefore $[x_1]_R = [x_2]_R$.
    It follows that $h$ is one-to-one.
\end{proof}

\begin{tcolorbox}[title=Problem 20, breakable]
    Suppose $A, B$, and $C$ are sets and $f : A \rightarrow B$.

    (a) Prove that if $f$ is onto, $g : B \rightarrow C$, 
        $h : B \rightarrow C$, and $g \circ f = h \circ f$,
        then $g = h$.

    (b) Suppose that $C$ has at least two elements,
        and for all functions $g$ and $h$ from $B$ to $C$,
        if $g \circ f = h \circ f$ then $g = h$.
        Prove that $f$ is onto.
\end{tcolorbox}

\begin{proof}
    Suppose $f$ is onto, $g : B \rightarrow C$, 
        $h : B \rightarrow C$, and $g \circ f = h \circ f$.
    Let $b$ be an arbitrary element in $B$.
    Since $f$ is onto, there exists $a \in A$ such that $f(a) = b$.
    Now, since $g \circ f = h \circ f$, it follows that
        $(g \circ f)(a) = (h \circ f)(a)$.
    Thus $g(f(a)) = h(f(a))$, and hence $g(b) = h(b)$.
    Since $b$ was arbitrary, $g = h$.
\end{proof}

\begin{proof}
    For contradiction, suppose $f$ is not onto.
    There exists $y \in B$ such that for all $x \in A$,
        $f(x) \ne y$.
    Let $y_1, y_2 \in C$ with $y_1 \ne y_2$.
    For all $x \in B$ let 
        $g(x) = y_1$, $h(x) = y_2$ if $x = y$; otherwise $g(x) = h(x) = y_1$.
    Clearly $g \ne h$.
    However, $g \circ f = h \circ f$ since for all $x \in A$
        $f(x) \ne y$, which is a contradiction.
    Thus $f$ is onto.
\end{proof}

\begin{tcolorbox}[title=Problem 21, breakable]
     Suppose $A, B$, and $C$ are sets and $f : B \rightarrow C$.

     (a) Prove that if $f$ is one-to-one, $g : A \rightarrow B$,
         $h : A \rightarrow B$, and $f \circ g = f \circ h$, then $g = h$.

    (b) Suppose that $A \ne \emptyset$, and for all functions $g$ and $h$
        from $A$ to $B$, if $f \circ g = f \circ h$ then $g = h$.
        Prove that $f$ is one-to-one.
\end{tcolorbox}

\begin{proof}
    Suppose $f$ is one-to-one, $g : A \rightarrow B$,
         $h : A \rightarrow B$, and $f \circ g = f \circ h$.
    Let $x$ be an arbitrary element in $A$.
    There exists $y_1, y_2$ such that $h(x) = y_1$ 
        and $g(x) = y_2$.
    Since $f \circ g = f \circ h$ it 
        follows that $f(y_1) = f(y_2)$.
    Then since $f$ is one-to-one it follows that $y_1 = y_2$.
    Therefore $g = h$. 
\end{proof}

\begin{proof}
    For contradiction, suppose $f$ is not one-to-one.
    There exist $x_1, x_2 \in B$ such that $f(x_1) = f(x_2)$
        and $x_1 \ne x_2$.
    For all $x \in A$ let $g(x) = x_1$ and $h(x) = x_2$.
    Clearly $g \ne h$.
    However, for all $x \in A$, $(f \circ g)(x) = (f \circ h)(x)$,
        which is a contradiction.
    Thus $f$ is one-to-one.
\end{proof}

\begin{tcolorbox}[title=Problem 22, breakable]
    Let $\mathcal{F} = \{f \mid f : \mathbb{R} \rightarrow \mathbb{R}\}$,
        and define a relation $R$ on $\mathcal{F}$ as follows:
    \[R = \{(f, g) \in \mathcal{F} \times \mathcal{F} \mid \exists{h} \in \mathcal{F}(f = h \circ g)\}\]

    (a) Let $f, g$, and $h$ be the functions from $\mathbb{R}$ to $\mathbb{R}$ defined by the 
        formulas $f(x) = x^2 + 1$, $g(x) = x^3 + 1$, and $h(x) = x^4 + 1$. Prove that $hRf$,
        but it is not the case $gRf$.

    (b) Prove that $R$ is a preorder. (See exercize $25$ of Section $4.5$ for the
        definition of \emph{preorder}.)

    (c) Prove that for all $f \in \mathcal{F}$, $f R i_{\mathbb{R}}$.

    (d) Prove that for all $f \in \mathcal{F}$, $i_{\mathbb{R}} R f$ iff $f$ is one-to-one.
        (Hint for right-to-left direction: Suppose $f$ is one-to-one. Let $A = Ran(f)$, and 
         let $h = f^{-1} \cup ((\mathbb{R} \setminus A) \times \{0\}))$. Now prove that 
         $h : \mathbb{R} \rightarrow \mathbb{R}$ and $i_{\mathbb{R}} = h \circ f$.

    (e) Suppose $g \in \mathcal{F}$ is a constant function; in other words, there is some 
        real number $c$ such that $\forall{x} \in \mathbb{R}(g(x) = c)$. Prove that for all 
        $f \in \mathcal{F}$, $g R f$. (Hint: See exercize $17$ of Section $5.1$.)

    (f) Suppose that $g \in \mathcal{F}$ is a constant function.
        Prove that for all $f \in \mathcal{F}$, $f R g$ iff $f$ is a constant function.

    (g) As in exercize $25$ of Section $4.5$, if we let $S = R \cap R^{-1}$, then $S$ 
        is an equivalence relation on $\mathcal{F}$. Also, there is a unique relation 
        $T$ on $\mathcal{F} / S$ such that for all $f$ and $g$ in $\mathcal{F}$,
        $[f]_S T [g]_S$ iff $f R g$, and $T$ is a partial order on $\mathcal{F} / S$.
        Prove that the set of all one-to-one functions from $\mathbb{R}$ to $\mathbb{R}$
        is the largest element of $\mathcal{F} / S$ in the partial order $T$, and the set 
        of all constant functions from $\mathbb{R}$ to $\mathbb{R}$ is the smallest element.
\end{tcolorbox}

\textbf{Solution (a):}

\begin{proof}
    Let $l(x) = x^2 - 2x + 2$. Then 
    \begin{align*}
    (l \circ f)(x) 
        &= l(f(x)) \\
        &= (x^2 + 1)^2 - 2(x^2 + 1) + 2 \\
        &= x^4 + 2x^2 + 1 - 2x^2 - 2 + 2 \\
        &= x^4 + 1 \\
        &= h
    \end{align*}
    Thus $h = l \circ f$, which shows that $(h, f) \in R$.
\end{proof}

\begin{proof}
    For contradiction, suppose $(g, f) \in R$.  
    There exists a function $h$ such that $g = h \circ f$. Then
    \[
    x^3 + 1 = h(x^2 + 1) \text{ for all } x \in \mathbb{R}.
    \]  
    Suppose $t_1, t_2 \in \mathbb{R}$ with $t_1 = -t_2 \ne 0$.  
    Then $f(t_1) = t_1^2 + 1 = f(t_2)$, but $g(t_1) \ne g(t_2)$.  
    Thus $g \ne h \circ f$ for any function $h$.
\end{proof}

\textbf{Solution (b):}

\begin{proof}
    We must show $R$ is reflexive and transitive.
    Suppose $f \in \mathcal{F}$.
    Let $h : \mathbb{R} \rightarrow \mathbb{R}$ be defined be the formula $h(y) = y$.
    Suppose $x$ is an arbitrary real number.
    Then $(h \circ f)(x) = h(f(x)) = f(x)$.
    Thus $(f, f) \in R$.
    It follows that $R$ is reflexive.
    Suppose $(f, g), (g, l)$ are arbitrary elements in $R$.
    It follows that there exists $h_1, h_2$ such that 
        $f = h_1 \circ g$ and $g = h_2 \circ l$.
    Then $f = h_1 \circ g = (h_1 \circ h_2) \circ l$.
    Thus $(f, l) \in R$.
    It follows that $R$ is transitive.
    Therefore $R$ is a  preorder.
\end{proof}

\textbf{Solution (c):}

\begin{proof}
    Let $f$ be an arbitrary element in $\mathcal{F}$.
    Furthermore, let $x$ be an arbitrary real number.
    Then $(i_{\mathbb{R}} \circ f)(x) = i_{\mathbb{R}}(f(x)) = f(x)$.
    Thus $(f, i_{\mathbb{R}}) \in R$.
\end{proof}

\textbf{Solution (d):}

\begin{proof}
    ($\rightarrow$) Let $f$ be an arbitrary element in $\mathcal{F}$.
    Suppose $(i_{\mathbb{R}}, f) \in R$.
    There exists $h$ such that $i_{\mathbb{R}} = h \circ f$.
    Let $x_1, x_2$ be arbitrary real numbers such that 
        $f(x_1) = f(x_2)$.
    Applying $h$ to both sides gives 
        $(h \circ f)(x_1) = (h \circ f)(x_2)$.
    Then $x_1 = (h \circ f)(x_1) = (h \circ f)(x_2) = x_2$.
    Thus $f$ is one-to-one.

    ($\leftarrow$) Suppose $f$ is one-to-one.
    Let $A = \mathrm{Ran}(f)$.
    Define 
    \[
        h = f^{-1} \cup ((\mathbb{R} \setminus A) \times \{0\}).
    \]
    Then for each $x \in \mathbb{R}$, either $x \in A$ or $x \in \mathbb{R} \setminus A$.
    If $x \in A$, then $h(x) = f^{-1}(x)$.
    If $x \in \mathbb{R} \setminus A$, then $h(x) = 0$.
    Thus $h$ is a function from $\mathbb{R}$ to $\mathbb{R}$.
    Furthermore, for all $x \in \mathbb{R}$,
    \[
        (h \circ f)(x) = h(f(x)) = f^{-1}(f(x)) = x.
    \]
    Thus $i_{\mathbb{R}} = h \circ f$, so $(i_{\mathbb{R}}, f) \in R$.
\end{proof}

\textbf{Solution (e):}

\begin{proof}
    Let $h$ be an arbitrary function in $\mathcal{F}$.
    By Section $5.1$ Problem $17(a)$,
        $g \circ h = g$
    Thus $(g, h) \in R$.
\end{proof}

\textbf{Solution (f):}

\begin{proof}
    Let $f$ be an arbitrary element in $\mathcal{F}$.

    ($\rightarrow$) Suppose $(f, g) \in R$.
    Then there exists $h$ such that $f = h \circ g$.
    Since $g$ is a constant function, there exists $y \in \mathbb{R}$ such that $g(x) = y$ for all $x \in \mathbb{R}$.
    Then for all $x \in \mathbb{R}$,
        $f(x) = (h \circ g)(x) = h(g(x)) = h(y)$,
    which shows that $f$ is constant.

    ($\leftarrow$) Suppose $f$ is a constant function.
    Let $c$ be a real number such that $f(x) = c$ for all $x \in \mathbb{R}$.
    Furthermore, let $h$ be a function
        from $\mathbb{R}$ to $\mathbb{R}$ such that $h(y) = c$ for all $y \in \mathbb{R}$.
    Then for all $x \in \mathbb{R}$,
        $(h \circ g)(x) = h(g(x)) = c = f(x)$.
    Thus $f = h \circ g$, so $(f, g) \in R$.
\end{proof}

\textbf{Solution (g):}

\begin{proof}
    Let $f$ be an arbitrary one-to-one function in $\mathcal{F}$.
    Let $g$ be an arbitrary function in $\mathcal{F}$.
    Since $f$ is one-to-one, $f^{-1}$ exists on $\mathrm{Ran}(f)$.
    Let $h$ be a function such that 
    \[
        h = g \circ f^{-1} \;\; \cup \;\; ((\mathbb{R} \setminus \mathrm{Ran}(f)) \times \{0\}).
    \]
    Then for all $x \in \mathbb{R}$, $(h \circ f)(x) = h(f(x)) = g(x)$.
    Thus $(g, f) \in R$.
\end{proof}


\begin{proof}
    Let $f$ be an arbitrary constant function in $\mathcal{F}$.
    Let $g$ be an arbitrary function in $\mathcal{F}$.
    From part (f) it follows that there exists $h$ 
        such that $f = h \circ g$.
    Thus $(f, g) \in R$.
\end{proof}

\begin{tcolorbox}[title=Problem 23, breakable]
    Let $f : \mathbb{N} \rightarrow \mathbb{N}$ be defined by the formula $f(n) = n$.
    Note that we could also say that $f : \mathbb{N} \rightarrow \mathbb{Z}$.
    This exersize will illustrate why,
        in Definition $5.2.1$, we defined the phrase ``$f$ maps onto $B$,''
        rather than simply ``$f$ is onto.''

    (a) Does $f$ map onto $\mathbb{N}$.

    (b) Does $f$ map onto $\mathbb{Z}$.
\end{tcolorbox}

\begin{proof}
    Yes. Let $y$ be an arbitrary natural number. Then let $x = y$.
    Clearly $f(x) = x = y$. Thus $f$ maps onto $\mathbb{N}$.
\end{proof}

\begin{proof}
    No. Suppose $f$ maps onto $\mathbb{Z}$.
    Let $y$ be an arbitrary integer such that $y < 0$.
    Since $f$ is onto, exists a natural number $x$ such that $f(x) = y$.
    Then $x = y$, but $y < 0$, contradicting
    that $x$ is a natural number. Thus $f$ does not map onto $\mathbb{Z}$.
\end{proof}

\subsection{Inverses of Functions}

\begin{tcolorbox}[title=Problem 1, breakable]
    Let $R$ be the function defined in exercise $2(c)$ of Section $5.1$.
    In exercise $2$ of Section $5.2$, you showed that $R$ is one-to-one
    and onto, so $R^{-1} : P \rightarrow P$. If $p \in P$,
        what is $R^{-1}(p)$?

    Problem $2(c)$

    John, Mary, Susan, and Fred go out to dinner and sit at a round table.
    Let 
    \[P = \{\text{John, Mary, Susan, Fred}\}\]
    and let 
    \[R = \{(p, q) \in P \times P \mid \text{the person $p$ is sitting immediately to the right of person $q$}\}\]
    Is $R$ a function from $P$ to $P$.
\end{tcolorbox}

\textbf{Solution:}
\[R^{-1}(p) = \text{The person sitting immediately to the left of person $p$}\]

\begin{tcolorbox}[title=Problem 2, breakable]
    Let $F$ be the function defined in exercise $4(b)$ of Section $5.1$.
    In exercise $4$ of Section $5.2$, you showed that $F$ is one-to-one 
    and onto, so $F^{-1} : B \rightarrow B$. If $X \in B$,
    what is $F^{-1}(X)$.

    Problem $4(b)$

    Let $A = \{1, 2, 3\}$ and $B = \mathcal{P}(A)$.
    Let $F : B \rightarrow B$ be the function defined by the 
        formula $F(X) = A \setminus X$.
\end{tcolorbox}

\textbf{Solution}
\[F^{-1}(X) = A \setminus X\]

\begin{tcolorbox}[title=Problem 3, breakable]
    Let $f : \mathbb{R} \rightarrow \mathbb{R}$ be defined by the formula 
    \[f(x) = \frac{2x + 5}{3}\]
    Show that $f$ is one-to-one and onto, and find a formula for $f^{-1}(x)$.
    (You may want to imitate the method used in the example after Theorem $5.3.2$,)
\end{tcolorbox}

\begin{proof}
    Define
    \[g(x) = \frac{3x - 5}{2}\]
    Then 
    \[(f \circ g)(x) = f(g(x)) = f(\frac{3x - 5}{2}) = \frac{2\left(\frac{3x - 5}{2}\right) + 5}{3} = \frac{3x}{3} = x\]
    Also 
    \[(g \circ f)(x) = g(f(x)) = g(\frac{2x + 5}{3}) = \frac{3\left(\frac{2x + 5}{3}\right) - 5}{2} = \frac{2x}{2} = x\]
    By Theorem $5.3.5$ it follows that $g = f^{-1}$.
    Futhermore, by Theorem $5.3.4$ it follows that $f$ is one-to-one and onto.
\end{proof}

\begin{tcolorbox}[title=Problem 8, breakable]
    (a) Prove the second half of Theorem $5.3.2$ by imitating the proof of the first half.

    (b) Give an alternative proof of the second half of Theorem $5.3.2$ by appying the 
        first half to $f^{-1}$.
\end{tcolorbox}

\begin{theorem}[$5.3.2$]
    Suppose $f$ is a function from $A$ to $B$, and suppose that $f^{-1}$
        is a function from $B$ to $A$. Then $f^{-1} \circ f = i_A$ and $f \circ f^{-1} = i_B$.
\end{theorem}

\begin{proof}
    Let $b$ be an arbitrary element in $B$.
    Let $a = f^{-1}(b) \in A$.
    Then $(b, a) \in f^{-1}$ and therefore $(a, b) \in f$
    Thus,
    \[(f \circ f^{-1})(b) = f(f^{-1}(b)) = f(a) = b = i_B(b)\]
    Since $b$ was arbitrary, it follows that $\forall{b \in B}((f \circ f^{-1})(b) = i_B(b))$,
        so $f \circ f^{-1} = i_B$. 
\end{proof}

\begin{proof}
    Let $b$ be an arbitrary element in $B$.
    Let $a = f^{-1}(b) \in A$.
    Then composing with $f^{-1} \circ f$ 
        gives 
    \[((f^{-1} \circ f) \circ f^{-1})(b) = (f^{-1} \circ f)(a)
                \implies ((f^{-1} \circ f) \circ f^{-1})(b) = a\]
    Applying $f$ gives 
    \[(f \circ (f^{-1} \circ f) \circ f^{-1})(b) = f(a)
        \implies (f \circ (f^{-1} \circ f) \circ f^{-1})(b) = b = i_B\]
    Then 
    \[(f \circ (f^{-1} \circ f) \circ f^{-1})(b) = (f \circ i_A \circ f^{-1})(b) = (f \circ f^{-1})(b)\]
    Thus $f \circ f^{-1} = i_B$.
\end{proof}

\begin{tcolorbox}[title=Problem 9, breakable]
    Prove part $2$ of Theorem $5.3.3$.
\end{tcolorbox}

\begin{theorem}[Part $2$, $5.3.3$]
    If there is a function $g : B \rightarrow A$ such that $f \circ g = i_B$
        then $f$ is onto.
\end{theorem}

\begin{proof}
    Let $b$ be an arbitrary element in $B$.
    There exists $a \in A$ such that $g(b) = a$.
    Composing with $f$ gives $(f \circ g)(b) = f(a) = i_B(b) = b$.
    Thus $f$ is onto.
\end{proof}

\begin{tcolorbox}[title=Problem 10, breakable]
    Use the following strategy to give an alternative proof  of Theorem $5.3.5$:
    
    Let $(b, a)$ be an arbitrary element of $B \times A$.
    Assume $(b, a) \in g$ and prove $(b, a) \in f^{-1}$.
    Then assume $(b, a) \in f^{-1}$ and prove $(b, a) \in g$.
\end{tcolorbox}

\begin{theorem}
    Suppose $f : A \rightarrow B$, $g : B \rightarrow A$, $g \circ f = i_A$,
        and $f \circ g = i_B$. Then $g = f^{-1}$.
\end{theorem}

\begin{proof}
    Let $(b, a)$ be an arbitrary element of $B \times A$.
    Assume $(b, a) \in g$ and therefore $g(b) = a$.
    Applying $f$ shows 
    \[(f \circ g)(b) = f(a) = i_B(b) = b\]
    Thus $(a, b) \in f$ and it follows that $(b, a) \in f^{-1}$.
    Therefore $g \subseteq f^{-1}$.
    Now, assume $(b, a) \in f^{-1}$ and therefore $(a, b) \in f$.
    It follows that $f(a) = b$.
    Applying $g$ shows 
    \[(g \circ f)(a) = g(b) = i_A(a) = a\]
    Thus $(b, a) \in g$.
    Therefore $f^{-1} \subseteq g$.
    Thus $g = f^{-1}$.
\end{proof}

\begin{tcolorbox}[title=Problem 11, breakable]
    Suppose $f : A \rightarrow B$ and $g : B \rightarrow A$.

    (a) Prove that if $f$ is one-to-one and $f \circ g = i_B$, then $g = f^{-1}$.

    (b) Prove that if $f$ is onto and $g \circ f = i_A$, then $g = f^{-1}$.

    (c) Prove that if $f \circ g = i_B$ but $g \circ f \ne i_A$, then $f$ is onto 
        but not one-to-one, and $g$ is one-to-one but not onto.
\end{tcolorbox}

\begin{proof}
    Suppose $f$ is one-to-one and $f \circ g = i_B$.
    Let $b$ be an arbitrary element in $B$, and let $a = g(b) \in A$.
    Then 
    \[
        f(a) = f(g(b)) = (f \circ g)(b) = i_B(b) = b.
    \]
    Since $f$ is one-to-one, this $a$ is the unique element such that $f(a)=b$.  
    Thus $g(b) = a = f^{-1}(b)$.
    Therefore $g = f^{-1}$.
\end{proof}

\begin{proof}
    Suppose $f$ is onto and $g \circ f = i_A$.
    Let $a$ be an arbitrary element in $A$, and let $b = f(a) \in B$.
    Then
    \[
        g(b) = g(f(a)) = (g \circ f)(a) = i_A(a) = a.
    \]
    Thus $g(b) = a$. Since $b = f(a)$, it follows that $g(b) = f^{-1}(b)$.
    Therefore $g = f^{-1}$.
\end{proof}

\begin{proof}
    Suppose $f \circ g = i_B$ but $g \circ f \ne i_A$.  
    Then $f$ is onto and $g$ is one-to-one.  

    For contradiction, suppose $f$ is one-to-one.  
    Then by part (a), $g = f^{-1}$, contradicting $g \circ f \ne i_A$.  
    Hence $f$ is not one-to-one.  

    For contradiction, suppose $g$ is onto.  
    Then by part (b), $g = f^{-1}$, contradicting $g \circ f \ne i_A$.  
    Hence $g$ is not onto.  
\end{proof}

\begin{tcolorbox}[title=Problem 12, breakable]
    Suppose $f : A \rightarrow B$ and $f$ is one-to-one.
    Prove that there is some set $B' \subseteq B$ 
        such that $f^{-1} : B' \rightarrow A$.
\end{tcolorbox}

\begin{proof}
    Let $B' = Ran(f) \subseteq B$.
    Since $f$ is one-to-one, for each $y \in Ran(f)$ there exists a unique $x \in A$
        such that $f(x) = y$.
    Let 
    \[
        f^{-1} : B' \rightarrow A, \quad f^{-1}(y) = x \text{ such that } f(x) = y
    \]
\end{proof}

\begin{tcolorbox}[title=Problem 13, breakable]
    Suppose $f : A \rightarrow B$ and $f$ is onto.
    Let $R = \{(x, y) \in A \times A \mid f(x) = f(y)\}$.
    By exercise $20(a)$ of Section $5.1$, $R$ is an equivalence
        relation on $A$.

    (a) Prove that there is a function $h : A / R \rightarrow B$ such that 
        for all $x \in A$, $h([x]_R) = f(x)$.
        (Hint: See exercise $21$ of Section $5.1.$)

    (b) Prove that $h$ is one-to-one and onto. (Hint: See exercise $19$ of 
        Section $5.2.$)

    (c) It follows from part (b) that $h^{-1} : B \rightarrow A / R$.
        Prove that for all $b \in B$, $h^{-1}(b) = \{x \in A \mid f(x) = b\}$.

    (d) Suppose $g : B \rightarrow A$. Prove that $f \circ g = i_B$
        iff $\forall{b} \in B(g(b) \in h^{-1}(b))$.
\end{tcolorbox}

\begin{proof}
    Since for all $x, y \in A \times A$
        if $(x, y) \in R$ then $f(x) = f(y)$
        it follows that $R$ is compatible with $f$.
    Then from Section $5.1$ Problem $21(a)$
        it follows that there is a unique function $h : A / R \rightarrow B$ such that 
        for all $x \in A$, $h([x]_R) = f(x)$.
\end{proof}

\begin{proof}
    Since for all $x, y \in A \times A$
        if $(x, y) \in R$ then $f(x) = f(y)$,
        it follows immediately from Section $5.2$ 
        problem $19$ that $h$ is one-to-one.
    Let $b$ be an arbitrary element in $B$.
    Since $f$ is onto, there exists $a \in A$
        such that $f(a) = b$.
    It follows that $h([a]_R) = f(a) = b$.
    Thus $h$ is onto.
\end{proof}

\begin{proof}
    Let $b$ be an arbitrary element in $B$.

    Let $A = h^{-1}(b)$, so $A \in A / R$ and $h(A) = b$.
    By the definition of $h$, there exists $x \in A$ such that $h(A) = f(x) = b$.
    Let $y$ be an arbitrary element in $A$.
    Since $(x, y) \in R$, it follows that $f(y) = f(x) = b$.
    Thus $y \in \{x \in A \mid f(x) = b\}$.
    Therefore $h^{-1}(b) = A \subseteq \{x \in A \mid f(x) = b\}$.

    Let $x$ be an arbitrary element in $\{x \in A \mid f(x) = b\}$.
    It follows that $f(x) = b$.
    There exists $X \in A / R$ such that $x \in X$.
    Then $h(X) = f(x) = b$.
    Thus $X = h^{-1}(b)$ and it follows that $x \in  h^{-1}(b)$.
    Therefore $\{x \in A \mid f(x) = b\} \subseteq h^{-1}(b)$.
\end{proof}

\begin{proof}
    ($\rightarrow$) Suppose $f \circ g = i_B$.
    Let $b$ be an arbitrary element in $B$.
    Let $a = g(b) \in A$.
    Applying $f$ shows 
    \[(f \circ g)(b) = f(a) = i_B(b) = b.\]
    Thus $f(a) = b$, so $g(b) \in \{x \in A \mid f(x) = b\} = h^{-1}(b)$.
    Therefore $\forall b \in B,\ g(b) \in h^{-1}(b)$.

    ($\leftarrow$) Suppose $\forall b \in B,\ g(b) \in h^{-1}(b)$.
    Let $b$ be an arbitrary element in $B$.
    It follows that $g(b) \in h^{-1}(b) = \{x \in A \mid f(x) = b\}$.
    Thus $f(g(b)) = b$.
    Therefore $(f \circ g)(b) = b$, and since $b$ was arbitrary, $f \circ g = i_B$.
\end{proof}

\begin{tcolorbox}[title=Problem 14, breakable]
    Suppose $f : A \rightarrow B$, $g : B \rightarrow A$, and $f \circ g = i_B$.
    Let $A' = Ran(g) \subseteq A$.

    (a) Prove that for all $x \in A'$, $(g \circ f)(x) = x$.

    (b) Prove that $f | A'$ is a one-to-one, onto function from $A'$ to $B$
        and $g = (f | A')^{-1}$. (See exercise $7$ of Section $5.1$ for the 
        meaning of the notation here.)
\end{tcolorbox}

\begin{proof}
    Let $x$ be an arbitrary element in $Ran(g) = A' \subseteq A$.
    There exists $y$ such that $g(y) = x$.
    Then applying $g \circ $f:
    \begin{align*}
        (g \circ f \circ g)(y) &= (g \circ f)(x) \\
        \iff (g \circ i_B)(y) &= (g \circ f)(x) \\
        \iff g(y) &= (g \circ f)(x)
    \end{align*}
    Since $g(y) = x$ it follows that $(g \circ f)(x) = x$.
\end{proof}

\begin{proof}
    For all $b \in B$, $(f \circ g)(b) = i_B(b)$.
    By part (a), for all $a \in A'$, $(g \circ f)(a) = i_A(a)$.
    It follows that $g = (f | A')^{-1}$.
\end{proof}

\begin{tcolorbox}[title=Problem 15, breakable]
    Let $B = \{x \in \mathbb{R} \mid x \ge 0\}$.
    Let $f : \mathbb{R} \rightarrow B$ and $g : B \rightarrow \mathbb{R}$
    be defined by the formulas $f(x) = x^2$ and $g(x) = \sqrt{x}$.
    As we saw in part $2$ of Example $5.3.6$, $g \ne f^{-1}$.
    Show that $g = (f | B)^{-1}$.
    (Hint: See exercise $14$.)
\end{tcolorbox}

\begin{proof}
    Let $x$ be positive real number such  that $x \ge 0$.
    Note that $(f \circ g)(x) = f(\sqrt{x}) = (\sqrt{x})^2 = x$.
    Since $B = Ran(g)$ it follows immediately from exercise $14$
        that $g = (f | B)^{-1}$.
\end{proof}

\begin{tcolorbox}[title=Problem 17, breakable]
    Suppose $A$ is a set, and let $\mathcal{F} = \{f \mid f : A \rightarrow A\}$
    and $\mathcal{P} = \{f \in \mathcal{F} \mid f \text{ is one-to-one and onto}\}$.
    Define a relation $R$ on $\mathcal{F}$ as follows:
    \[R = \{(f, g) \in \mathcal{F} \times \mathcal{F} \mid \exists{h} \in \mathcal{P}(f = h^{-1} \circ g \circ h)\}\]
    (a) Prove that $R$ is an equivalence relation.

    (b) Prove that if $f R g$ then $(f \circ f) R (g \circ g)$.

    (c) For any $f \in \mathcal{F}$ and $a \in A$, if $f(a) = a$ then we say that $a$
        is a \emph{fixed point} of $f$. Prove that if $f$ has a fixed point and $f R g$,
        then $g$ also has a fixed point.
\end{tcolorbox}

\begin{proof}
    We must show $R$ is reflexive, transitive, 
        and symmetric.

    Let $f$ be an arbitrary element in $\mathcal{F}$.
    Let $h = i_A$ then 
    \[h^{-1} \circ f \circ h = i_A \circ f \circ i_A = f\]
    Thus $(f, f) \in R$ and therefore $R$ is reflexive.

    Let $(f, g), (g, l)$ be arbitrary elements in $R$.
    There exists $h_1, h_2 \in \mathcal{P}$ such that 
    \[f = h_1^{-1} \circ g \circ h_1 \text{ and } g = h_2^{-1} \circ l \circ h_2\]
    It follows that 
    \[f = h_1^{-1} \circ h_2^{-1} \circ l \circ h_2 \circ h_1\]
    Therefore 
    \[f = (h_1^{-1} \circ h_2^{-1}) \circ l \circ (h_2 \circ h_1) \iff f = (h_1 \circ h_2)^{-1} \circ l \circ  (h_2 \circ h_1)\]
    Thus $(f, l) \in R$ and therefore $R$ is transitive.

    Let $(f, g)$ be an arbitrary element in $R$.
    There exists $h \in \mathcal{P}$ such that 
    \[f = h^{-1} \circ g \circ h\]
    Applying $h$ shows 
    \[h \circ f = h \circ h^{-1} \circ g \circ h = i_A \circ g \circ h = g \circ h\]
    Let $a_1$ be an arbitrary element in $A$.
    There exists $a_2$ such that $h^{-1}(a_1) = a_2$.
    It follows that $a_1 = h(a_2)$.
    Then appying $h \circ f$ shows
    \[(h \circ f \circ h^{-1})(a_1) = (h \circ f)(a_2) = (g \circ h)(a_2) = g(h(a_2)) = g(a_1)\]
    Thus $g = h \circ f \circ h^{-1}$
    It follows that $(g, f) \in R$ and therefore $R$ is symmetric.
\end{proof}

\begin{proof}
    Let $(f, g)$ be an arbitrary element in $R$.
    It follows that there exists $h \in \mathcal{P}$
        such that $f = h^{-1} \circ g \circ h$.
    \[f \circ f = h^{-1} \circ g \circ h \circ h^{-1} \circ g \circ h\]
    Then since $h \circ h^{-1} = i_A$ it follows that 
    \[f \circ f = h^{-1} \circ g  \circ g \circ h\]
    Thus $(f \circ f, g \circ g) \in R$.
\end{proof}

\begin{proof}
    Suppose $f \in \mathcal{F}$ has a fixed point $a \in A$
        such that $f(a) = a$.
    Furthermore suppose there exists $g \in \mathcal{F}$
        such that $(f, g) \in R$.
    It follows that there exists $h \in \mathcal{P}$ 
        such that $f = h^{-1} \circ g \circ h$.
    Then 
    \[f(a) = (h^{-1} \circ g \circ h)(a) =  a\]
    Then applying $h$ shows
    \[(h \circ f)(a) = (g \circ h)(a) =  h(a)\]
    Now there exists $b \in A$ such that $h(a) = b$.
    Therefore
    \[(g \circ h)(a) =  h(a) \iff g(b) = b\]
    Thus $g$ has a fixed point.
\end{proof}

\begin{tcolorbox}[title=Problem 18, breakable]
    Suppose $f : A \rightarrow C$, $g : B \rightarrow C$, and $g$ is one-to-one and onto.
    Prove that there is a function $h : A \rightarrow B$ such that $g \circ h = f$.
\end{tcolorbox}

\begin{proof}
    Let 
    \[
        h : A \rightarrow B, \quad h(a) = b \text{ such that } g(b) = f(a).
    \]
    Such $b$ exists because $g$ is onto, and is unique because $g$ is one-to-one.
\end{proof}

\subsection{Closures}

\begin{tcolorbox}[title=Problem 1, breakable]
    Let $f : \mathbb{R} \rightarrow \mathbb{R}$
    be defined by the formula $f(x) = (x + 1)/2$.
    Are the following sets closed under $f$?

    (a) $\mathbb{Z}$.

    (b) $\mathbb{Q}$.

    (c) $\{x \in \mathbb{R} \mid 0 \le x \le 4\}$.

    (d) $\{x \in \mathbb{R} \mid 2 \le x < 4\}$.
\end{tcolorbox}

\textbf{Solution (a):}

No.

\textbf{Solution (b):}

Yes.

\textbf{Solution (c):}

Yes.

\textbf{Solution (d):}

No.

\begin{tcolorbox}[title=Problem 2, breakable]
    Let $f : \mathcal{P}(\mathbb{N}) \rightarrow \mathcal{P}(\mathbb{N})$
    be defined by the formula $f(X) = X \cup \{17\}$.
    Are the following sets closed under $f$.

    (a) $\{X \subseteq \mathbb{N} \mid X \text{ is infinite}\}$.

    (b) $\{X \subseteq \mathbb{N} \mid X \text{ is finite}\}$.

    (c) $\{X \subseteq \mathbb{N} \mid X \text{ has at most $100$ elements}\}$.

    (d) $\{X \subseteq \mathbb{N} \mid 16 \in X\}$.
\end{tcolorbox}

\textbf{Solution (a):}

Yes.

\textbf{Solution (b):}

Yes.

\textbf{Solution (c):}

If $17 \in X$ then yes otherewise no.

\textbf{Solution (d):}

Yes

\begin{tcolorbox}[title=Problem 3, breakable]
    Let $f : \mathbb{Z} \rightarrow \mathbb{Z}$ be defined by the 
        formula $f(n) = n^2 - n$. Find the closure of
        $\{-1, 1\}$ under $f$.
\end{tcolorbox}

We see $f(1) = 0$ and $f(-1) = 2$. 
New set is $\{-1, 0, 1, 2\}$.
We see $f(0) = 0$ and $f(2) = 2$.
Thus the closure is $\{-1, 0, 1, 2\}$.

\begin{tcolorbox}[title=Problem 4, breakable]
    For any set $A$, the set of all relations on $A$ is 
        $\mathcal{P}(A \times A)$.
    Let $f : \mathcal{P}(A \times A) \rightarrow \mathcal{P}(A \times A)$
    be defined by the formula $f(R) = R^{-1}$.
    Is the set of reflexive relations on $A$ closed under $f$?
    What about the set of symmetric relations
        and the set of transitive relations?
    (Hint: See exercise of $12$ of Section $4.3$.)
\end{tcolorbox}

\textbf{Solution:}

The set of all reflexive operations: Yes.

The set of all symmetric operations: Yes.

The set of all transitive operations: Yes.

These all follow immediately from $12$ of Section $4.3$.

\begin{tcolorbox}[title=Problem 5, breakable]
    Suppose $f : A \rightarrow A$. Is $\emptyset$ closed under $f$.
\end{tcolorbox}

\textbf{Solution:}

Yes.

\begin{tcolorbox}[title=Problem 6, breakable]
    Suppose $f : A \rightarrow A$.

    (a) Prove that if $Ran(f) \subseteq C \subseteq A$ then $C$ is closed under $f$.

    (b) Prove that for every set $B \subseteq A$, the closure of $B$ under $f$ is 
        a subset of $B \cup Ran(f)$.
\end{tcolorbox}

\begin{proof}
    Suppose $Ran(f) \subseteq C \subseteq A$. 
    Let $c$ be an arbitrary element in $C$.
    Then $f(c) \in Ran(f) \subseteq C$.
    Therefore $C$ is closed under $f$.
\end{proof}

\begin{proof}
    Let $C$ be the closure of $B$ under $f$. 
    Then $C$ is the smallest set containing $B$ that is closed under $f$.
    Every element $c \in C$ is either 
    \begin{enumerate}
        \item an element of $B$
        \item an element obtained by applying $f$ to some element of $A$
    \end{enumerate}
    Thus $c \in B \cup Ran(f)$.
    It follows that $C \subseteq B \cup Ran(f)$.
\end{proof}

\begin{tcolorbox}[title=Problem 9, breakable]
    Suppose $f : A \rightarrow A$ and $C_1$ and $C_2$
    are subsets of $A$ that are closed under $f$.

    (a) Prove that $C_1 \cup C_2$ is closed under $f$.

    (b) Must $C_1 \cap C_2$ be closed under $f$? Justify your answer.

    (c) Must $C_1 \setminus C_2$ be closed under $f$? Justify your answer.
\end{tcolorbox}

\begin{proof}
    Let $x$ be an arbitrary element in $C_1 \cup C_2$.
    Either $x \in C_1$ or $x \in C_2$.
    Suppose $x \in C_1$.
    Since $C_1$ is closed under $f$ 
        it follows that $f(x) \in C_1$.
    Clearly $f(x) \in C_1 \cup C_2$.
    Suppose $x \in C_2$.
    Since $C_2$ is closed under $f$ 
        it follows that $f(x) \in C_2$.
    Clearly $f(x) \in C_1 \cup C_2$.
    Therefore $C_1 \cup C_2$ is closed under $f$.
\end{proof}

\begin{proof}
    Let $x$ be an arbitrary element in $C_1 \cap C_2$.
    It follows that $x \in C_1$ and $x \in C_2$.
    Since $C_1$ is closed under $f$, $f(x) \in C_1$.
    Since $C_2$ is closed under $f$, $f(x) \in C_2$.
    Therefore $C_1 \cap C_2$ is closed under $f$.
\end{proof}

\textbf{Solution (c):}

Let $A = \{1,2\}$, $C_1 = \{1,2\}$, $C_2 = \{2\}$, and define $f : A \to A$ by $f(x) = 2$ for all $x \in A$.  
Then $C_1 \setminus C_2 = \{1\}$, but $f(1) = 2 \notin C_1 \setminus C_2$.  
Therefore $C_1 \setminus C_2$ is not closed under $f$.

\begin{tcolorbox}[title=Problem 11, breakable]
    Prove Theorem $5.4.9$
\end{tcolorbox}

\begin{theorem}
    Suppose that $f : A \times A \rightarrow A$ and $B \subseteq A$.
    Then $B$ has a closure under $f$.
\end{theorem}

\begin{proof}
    Let $\mathcal{F} = \{C \subseteq A \mid B \subseteq C \text{ and } C \text{ is closed under } f\}$.
    Clearly $A \in \mathcal{F}$ and therefore $\mathcal{F} \ne \emptyset$.
    Thus, we can let $C = \bigcap \mathcal{F}$.
    By exercise $9$ of Section $3.3$, $C \subseteq A$.
    We will show that $C$ is the closure of $B$ under $f$
        by proving the three properties of Definition $5.4.8$.
    
    To prove the first property, suppose $x \in B$.
    Let $D$ be an arbitrary element in $\mathcal{F}$.
    Then by the definition of $\mathcal{F}$,
        $B \subseteq D$, so $x \in D$.
    Since $D$ was arbitrary,
        $\forall{D} \in \mathcal{F}(x \in D)$,
    so $x \in \bigcap \mathcal{F} = C$.
    Thus, $B \subseteq C$.

    Next, suppose $x, y \in C$ and again let $D$ be an 
        arbitrary element of $\mathcal{F}$.
    Then since $x, y \in C = \bigcap \mathcal{F}$, $x, y \in D$.
    But since $D \in \mathcal{F}$, $D$ is closed under $f$,
        so $f(x, y) \in D$.
    Since $D$ was arbitrary, we can conclude that 
        $\forall{D} \in \mathcal{F}(f(x, y) \in D)$,
        so $f(x, y) \in \bigcap \mathcal{F} = C$.
    Thus, we have shown that $C$ is closed under $f$,
        which is the second property of Definition $5.4.8$.

    Finally, to prove the third property,
        suppose $B \subseteq D \subseteq A$
        and $D$ is closed under $f$.
    Then $D \in \mathcal{F}$, and applying 
        exercise $9$ of Section $3.3$ again we 
        can conclude that $C = \bigcap \mathcal{F} \subseteq D$.
\end{proof}

\begin{tcolorbox}[title=Problem 12, breakable]
    If $\mathcal{F}$ is a set of functions from $A$ to $A$ and $C \subseteq A$,
        then we will say that $C$ is closed under $\mathcal{F}$
        if $\forall{f} \in \mathcal{F}\forall{x} \in C(f(x) \in C)$.
    In other words $C$ is closed under $\mathcal{F}$ iff
        for all $f \in \mathcal{F}$, $C$ is closed under $f$.
    If $B \subseteq A$, the closure of $B$ under $\mathcal{F}$
        is the smallest set $C \subseteq A$ such that $B \subseteq C$
        and $C$ is closed under $\mathcal{F}$.
    (You are asked to prove in the next exercise that the closure always exist.)

    (a) Let $f$ and $g$ be functions from $\mathbb{R}$ to $\mathbb{R}$ defined 
        by the formulas $f(x) = x + 1$ and $g(x) = x - 1$. Find the closure 
        of $\{0\}$ under $\{f, g\}$.

    (b) For each natural number $n$, let $f_n : \mathcal{P}(\mathbb{N}) \rightarrow \mathcal{P}(\mathbb{N})$
        be defined by the formula $f_n(X) = X \cup \{n\}$, and let $\mathcal{F} = \{f_n \mid n \in \mathbb{N}\}$.
        Find the closure of $\{\emptyset\}$ under $\mathcal{F}$.
\end{tcolorbox}

\textbf{Solution (a):}

The closure is $\mathbb{Z}$.

\textbf{Solution (b):}

The closure is $\{X \mid X \subset \mathbb{N} \wedge X \text{ is finite}\}$.

\begin{tcolorbox}[title=Problem 13, breakable]
    Suppose $\mathcal{F}$ is a set of functions from $A$ to $A$ and $B \subseteq A$.
    See the previous exercise for the definition of the closure of $B$ under $\mathcal{F}$.

    (a) Prove that $B$ has a closure under $\mathcal{F}$.
    For each $f \in \mathcal{F}$ let $C_f$ be the closure
    of $B$ under $f$, and let $C$ be the closure of $B$ under $F$.

    (b) Prove that $\bigcup_{f \in \mathcal{F}} C_f \subseteq C$.

    (c) Must $\bigcup_{f \in \mathcal{F}} C_f$ be closed under $\mathcal{F}$.
        Justify your answer with either a proof or counterexample.

    (d) Must $\bigcup_{f \in \mathcal{F}} C_f = C$? Justify your answer 
        with either a proof or counterexample.
\end{tcolorbox}

\begin{proof}
    Let 
    \[\mathcal{I} = \{C \mid B \subseteq C, C \subseteq A, C \text{ is closed under $\mathcal{F}$}\}\]
    Note that $A \in \mathcal{I}$, since $B \subseteq A$ and $A$ is closed under $\mathcal{F}$. 
    Thus $\mathcal{I} \ne \emptyset$.
    Now, let $C = \bigcap {\mathcal{I}}$
        and by exercise $9$ of Section $3.3$, $C \subseteq A$.

    Suppose $x \in B$. Let $D$ be an arbitrary element in $\mathcal{I}$.
    Then by the definition of $\mathcal{I}$, $B \subseteq D$, so $x \in D$.
    Since $D$ was arbitrary this shows that $\forall{D \in \mathcal{I}}(x \in D)$
        thus $x \in \bigcap {\mathcal{I}} = C$. Thus $B \subseteq C$. 

    Next, suppose $x \in C$.
    Let $f \in \mathcal{F}$ be arbitrary and again let $D$ be an arbitrary element of $\mathcal{I}$.
    Since $x \in C = \bigcap\mathcal{I}$, $x \in D$, and since $D$ is closed under $\mathcal{F}$,
       it follows that $f(x) \in D$.
    Since this holds for every $D \in \mathcal{I}$ we conclude that $f(x) \in C$.
    Thus $C$ is closed under $\mathcal{F}$.

    Suppose $B \subseteq D \subseteq A$ and $D$ is closed under $\mathcal{F}$.
    Then $D \in \mathcal{I}$ and by applying exercise $9$ of Section $3.3$
        again we can conclude that $C = \bigcap \mathcal{I} \subseteq D$.
\end{proof}

\begin{proof}
    Let $x$ be an arbitrary element in $\bigcup_{f \in \mathcal{F}} C_f$.
    There exists $f \in \mathcal{F}$ such that $C_f$ is the closure of $B$ under $f$
        and $x \in C_f$.
    Since for all $f \in \mathcal{F}$, $C$ is closed under $f$
        it follows that $C_f \subseteq C$.
    Thus $x \in C$.
    Therefore $\bigcup_{f \in \mathcal{F}} C_f \subseteq C$.
\end{proof}

\textbf{Solution (c):}
\[A = \{0,1,2\}, \quad B = \{0\}, \quad \mathcal{F} = \{f,g\}\] 
\[f = \{(0,1), (1,1), (2,2)\}, \quad g = \{(0,0), (1,2), (2,2)\}\]
\[C_f = \{0,1\}, C_g = \{0\}\]
\[\bigcup_{h \in \mathcal{F}} C_h = C_f \cup C_g = \{0,1\}\]
\[g(1) = 2 \notin \{0,1\}\]

\begin{proof}
    Clearly not since the closure exists by part (a) and 
        is not necesarily equivalent to $\bigcup_{h \in \mathcal{F}} C_h$
        by part (c).
\end{proof}

\begin{tcolorbox}[title=Problem 14, breakable]    
    Let $f : \mathbb{R} \times \mathbb{R} \rightarrow \mathbb{R}$ be defined 
        by the formula $f(x, y) = x - y$. What is the closure of $\mathbb{N}$
        under $f$.
\end{tcolorbox}

\textbf{Solution:}
\[\mathbb{Z}\]

\begin{tcolorbox}[title=Problem 15, breakable]
    Let $f : \mathbb{R}^+ \times \mathbb{R}^+ \rightarrow \mathbb{R}^+$
    be defined by the formula $f(x, y) = x / y$.
    What is the closure of $\mathbb{Z}^+$ under $f$.
\end{tcolorbox}

\textbf{Solution:}
\[\mathbb{Q}^+\]

\begin{tcolorbox}[title=Problem 16, breakable]
    As in part $2$ of Example $5.4.7$, let $\mathcal{I} = \{X \in \mathcal{P}(\mathbb{N}) \mid X \text{ is infinite}\}$.

    (a) Prove that for ever set $X \subseteq \mathbb{N}$ there are sets $Y, Z \in \mathcal{I}$  
        such that $Y \cap Z = X$.

    (b) What is the closure of $\mathcal{I}$ under the binary operation $\cap$.
\end{tcolorbox}

\begin{proof}
    Let $X$ be an arbitrary element in $\mathcal{P}(\mathbb{N})$.
    Define
    \[E = \{x \mid x = 2k, k \in \mathbb{N}\}, \quad O = \{x \mid x = 2k + 1, k \in \mathbb{N}\}\]
    and 
    \[Y = E \cup X, \quad Z = O \cup X\]
    Now clearly $Y$ and $Z$ are infinite subsets of $\mathbb{N}$.
    Let $x$ be an arbitrary element in $Y \cap Z$.
    It follows that $x \in E \cup X$ and $x \in O \cup X$.
    Since $E \cap O = \emptyset$, we must have $x \in X$.
    Therefore $Y \cap Z \subseteq X$.
    Let $x$ be an arbitrary element in $X$.
    It follows that $x \in E \cup X = Y$ and $x \in O \cup X = Z$.
    Therefore $X \subseteq Y \cap Z$.
    Since $Y \cap Z \subseteq X$ and $X \subseteq Y \cap Z$,
    it follows that $Y \cap Z = X$.
\end{proof}

\textbf{Solution (b):}
\[\mathcal{P}(\mathbb{N})\]

\begin{tcolorbox}[title=Problem 17, breakable]
    Let $\mathcal{F}=  \{f \mid f : \mathbb{R} \rightarrow \mathbb{R}\}$.
    Then for any $f, g \in \mathcal{F}$, $f \circ g \in \mathcal{F}$,
    so $\circ$ is a binary operation on $\mathcal{F}$. Are the following 
    sets closed under $\circ$?

    (a) $\{f \in \mathcal{F} \mid $f$ \text{ is one-to-one}\}$. (Hint: See Theorem $5.2.5.$)

    (b) $\{f \in \mathcal{F} \mid $f$ \text{ is onto}\}$.

    (c) $\{f \in \mathcal{F} \mid $f$ \text{ is strictly increasing}\}$.
        (A function $f : \mathbb{R} \rightarrow \mathbb{R}$ is \emph{strictly
        increasing} if $\forall{x} \in \mathbb{R} \forall{y} \in \mathbb{R} (x < y \rightarrow f(x) < f(y))$.)

    (d) $\{f \in \mathcal{F} \mid $f$ \text{ is strictly decreasing}\}$.
        (A function $f : \mathbb{R} \rightarrow \mathbb{R}$ is \emph{strictly
        decreasing} if $\forall{x} \in \mathbb{R} \forall{y} \in \mathbb{R} (x < y \rightarrow f(x) > f(y))$.)
\end{tcolorbox}

\textbf{Solution (a):}
\begin{center}
    Yes by Theorem $5.2.5.$ part $1$.
\end{center}
\textbf{Solution (b):}
\begin{center}
    Yes by Theorem $5.2.5.$ part $2$.
\end{center}
\textbf{Solution (c):}
Yes. Suppose $x, y \in \mathbb{R}$ and $x < y$.
Let $f, g \in \mathcal{F}$ such that $f,g$ are \emph{strictly increasing}.
Now $g(x) <  g(y)$ and composing with $f$ shows 
\[(f \circ g)(x) < (f \circ g)(y)\]

\textbf{Solution (d):}
No. Suppose $x, y \in \mathbb{R}$ and $x < y$.
Let $f, g \in \mathcal{F}$ such that $f,g$ are \emph{strictly decreasing}.
Now $g(x) > g(y)$ and composing with $f$ shows 
\[(f \circ g)(x) < (f \circ g)(y)\]

\begin{tcolorbox}[title=Problem 18, breakable]
    Let $\mathcal{F} = \{f \mid f : \mathbb{R} \rightarrow \mathbb{R}\}$.
    If $f, g \in \mathcal{F}$, then we define
    the function $f + g : \mathbb{R} \rightarrow \mathbb{R}$
    by the formula $(f + g)(x) = f(x) + g(x)$.
    Note that $+$ is a binary operation on $\mathcal{F}$.
    Are the following sets closed under $+$?

    (a) $\{f \in \mathcal{F} \mid $f$ \text{ is one-to-one}\}$.

    (b) $\{f \in \mathcal{F} \mid $f$ \text{ is onto}\}$.

    (c) $\{f \in \mathcal{F} \mid $f$ \text{ is is strictly increasing}\}$.
        (See the previous exercise for the definition of strictly increasing.)

    (d) $\{f \in \mathcal{F} \mid $f$ \text{ is is strictly decreasing}\}$.
        (See the previous exercise for the definition of strictly decreasing.)
\end{tcolorbox}

\textbf{Solution (a):}
No. Let $f, g \in \mathcal{F}$ be defined as 
\[f(x) = x, \quad g(x) = -x\]
Now $f, g$ are clearly one-to-one. 
Let $x, y$ be an arbitrary real numbers such that $x \ne y$ then 
\[(f + g)(x) = x - x = 0 = y - y = (f + g)(y)\]
Therefore $(f + g)$ is not one-to-one.

\textbf{Solution (b):}
No. Let $f, g \in \mathcal{F}$ be defined as 
\[f(x) = x, \quad g(x) = -x\]
Now $f, g$ are clearly onto. 
Let $x$ be an arbitrary real number.
\[(f + g)(x) = x - x = 0\]
Therefore $(f + g)$ is a constant function and is not onto.

\textbf{Solution (c):}
Yes. Let $f, g \in \mathcal{F}$.
Let $x, y$ be arbitrary real numbers such that $x < y$.
Now $f(x) < f(y)$ and $g(x) < g(y)$.
Consider $(g(y) + f(y)) - (g(x) + f(x))$.
Now $g(y) - g(x) > 0$ and $f(y) - f(x) > 0$
    thus $(g(y) + f(y)) - (g(x) + f(x)) > 0$, so
    $(g + f)(y) > (g + f)(x)$.

\textbf{Solution (d):}
Yes. Let $f, g \in \mathcal{F}$.
Let $x, y$ be arbitrary real numbers such that $x < y$.
Now $f(x) > f(y)$ and $g(x) > g(y)$.
Consider $(g(y) + f(y)) - (g(x) + f(x))$.
Now $g(y) - g(x) < 0$ and $f(y) - f(x) < 0$
    thus $(g(y) + f(y)) - (g(x) + f(x)) < 0$, so
    $(g + f)(y) < (g + f)(x)$.

\begin{tcolorbox}[title=Problem 19, breakable]
    For any set $A$, the set of all relations is $\mathcal{P}(A \times A)$,
        and $\circ$ is a binary operation on $\mathcal{P}(A \times A)$.
    Is the set of reflexive relations on $A$ closed under $\circ$?
    What about the set of symmetric relations and the set of transitive relations.
\end{tcolorbox}

Reflexive: Yes. Let $R_1, R_2$ be arbitrary reflexive relations on $A$.
Let $x$ be an arbitrary element in $A$. It follows that $(x, x) \in R_1$
and $(x, x) \in R_2$ thus $(x, x) \in R_2 \circ R_1$.

Symmetric: No. Counterexample 
\[A = \{1, 2, 3\}, R_1 = \{(1, 2), (2, 1)\}, R_2 = \{(2, 3), (3, 2)\}, R_2 \circ R_1 = \{(1, 3)\}\]
Clearly $(3, 1) \notin R_2 \circ R_1$ so it is not symmetric.

Transitive: No. Counterexample
\[A = \{1, 2, 3\}, R_1 = \{(1, 2), (2, 3), (1, 3)\}, R_2 = \{(2, 3), (3, 1), (2, 1)\}, R_2 \circ R_1 = \{(1, 3), (1, 1), (2, 1)\}\]
Clearly $(2, 1), (1, 3) \in R_2 \circ R_1$ but $(2, 3) \notin R_2 \circ R_1$ so it is not transitive.

\begin{tcolorbox}[title=Problem 20, breakable]
    Division is not a binary operation on $\mathbb{R}$, because you can't divide by $0$.
    But we can fix this problem. We begin by adding a new element to $\mathbb{R}$.
    We will call the new element ``Nan'' (for ``Not a Number''). Let $\bar{\mathbb{R}} = \mathbb{R} \cup \{\text{NaN}\}$,
    and define $f : \bar{\mathbb{R}} \times \bar{\mathbb{R}} \rightarrow \bar{\mathbb{R}}$ as follows:
    \[
        f(x, y) = 
            \begin{cases}
            x/y & \text{if } x, y \in \mathbb{R} \text{ and } y \ne 0,\\
            \text{NaN}  & \text{otherwise}.
        \end{cases}
    \]
    (a) $\mathbb{R}$.

    (b) $\mathbb{R}^+$.

    (c) $\mathbb{R}^-$.

    (d) $\mathbb{Q}$.

    (e) $\mathbb{Q} \cup \{\text{NaN}\}$.

    This notation means that if $x, y \in \mathbb{R}$ and $y \ne 0$
    then $f(x, y) = x/y$, and otherwise $f(x, y) = \text{NaN}$.
    Thus, for example, $f(3, 7) = 3/7$, $f(3, 0) = \text{NaN}$,
    and $f(\text{NaN}, 7) = \text{NaN}$. Which of the following sets 
    are closed under $f$.
\end{tcolorbox}

\textbf{Solution (a):}
\begin{center}
    No.
\end{center}
\textbf{Solution (b):}
\begin{center}
    Yes.
\end{center}
\textbf{Solution (c):}
\begin{center}
    No.
\end{center}
\textbf{Solution (d):}
\begin{center}
    No.
\end{center}
\textbf{Solution (e):}
\begin{center}
    Yes.
\end{center}

\begin{tcolorbox}[title=Problem 21, breakable]
    If $\mathcal{F}$ is a set of functions from $A \times A$ to $A$
        and $C \subseteq A$, then we will say that $C$ is closed 
        under $\mathcal{F}$ iff for all $f \in \mathcal{F}$, $C$ 
        is closed under $f$.
    If $B \subseteq A$, then the \emph{closure} of $B$ under $\mathcal{F}$
        is the smallest set $C \subseteq A$ such that $B \subseteq C$
        and $C$ is closed under $\mathcal{F}$, if there is such a smallest set.
    (Compare these definitions to the definitions in exercise $12$.)

    (a) Prove that the closure of $B$ under $\mathcal{F}$ exists.

    (b) Let $f : \mathbb{R} \times \mathbb{R} \rightarrow \mathbb{R}$
        and $g : \mathbb{R} \times \mathbb{R} \rightarrow \mathbb{R}$
        be defined by the formulas $f(x, y) = x + y$ and $g(x, y) = xy$.
        Prove that the closure of $\mathbb{Q} \cup \{\sqrt{2}\}$
        under $\{f, g\}$ is the set $\{a + b\sqrt{2} \mid a, b \in \mathbb{Q}\}$.
        (This set is called $\mathbb{Q}$ \emph{with $\sqrt{2}$ adjoined}, and is 
        denoted $\mathbb{Q}(\sqrt{2})$.)

    (c) With $f$ and $g$ defined as in part (b), what is the closure of $\mathbb{Q} \cup \{\sqrt[3]{2}\}$
        under $\{f, g\}$?
\end{tcolorbox}

\begin{proof}
    Let 
    \[\mathcal{I} = \{C \mid B \subseteq C, C \subseteq A, C \text{ is closed under $\mathcal{F}$}\}\]
    Note that $A \in \mathcal{I}$, since $B \subseteq A$ and $A$ is closed under $\mathcal{F}$. 
    Thus $\mathcal{I} \ne \emptyset$.
    Now, let $C = \bigcap {\mathcal{I}}$
        and by exercise $9$ of Section $3.3$, $C \subseteq A$.

    Suppose $x, y \in B$. Let $D$ be an arbitrary element in $\mathcal{I}$.
    Then by the definition of $\mathcal{I}$, $B \subseteq D$, so $x \in D$ and $y \in D$.
    Since $D$ was arbitrary, we have $x, y \in \bigcap {\mathcal{I}} = C$. 
    Thus $B \subseteq C$. 

    Next, suppose $x, y \in C$.
    Let $f \in \mathcal{F}$ be arbitrary and again let $D$ be an arbitrary element of $\mathcal{I}$.
    Since $x, y \in C = \bigcap\mathcal{I}$, $x, y \in D$, and since $D$ is closed under $\mathcal{F}$,
       it follows that $f(x, y) \in D$.
    Since this holds for every $D \in \mathcal{I}$ we conclude that $f(x, y) \in C$.
    Thus $C$ is closed under $\mathcal{F}$.

    Suppose $B \subseteq D \subseteq A$ and $D$ is closed under $\mathcal{F}$.
    Then $D \in \mathcal{I}$ and by applying exercise $9$ of Section $3.3$
        again we can conclude that $C = \bigcap \mathcal{I} \subseteq D$.
\end{proof}

\begin{proof}
    Let $x, y$ be arbitrary elements in $\mathbb{Q}(\sqrt{2})$.
    So $x = a_1 + b_1\sqrt{2}$ and $y = a_2 + b_2\sqrt{2}$
        where $a_1, a_2, b_1, b_2 \in \mathbb{Q}$.
    Then \[f(x, y) 
        = (a_1 + b_1\sqrt{2}) + (a_2 + b_2\sqrt{2}) 
        = (a_1 + a_2) + (b_1 + b_2)\sqrt{2}.\]
    Since $\mathbb{Q}$ is closed under addition, 
        $a_1 + a_2 \in \mathbb{Q}$ and $b_1 + b_2 \in \mathbb{Q}$.
        It follows that $f(x, y) \in \mathbb{Q}(\sqrt{2})$.
    Also 
    \begin{align*}
        g(x, y)
        &= (a_1 + b_1\sqrt{2})(a_2 + b_2\sqrt{2}) \\
        &= a_1 a_2 + a_1 b_2\sqrt{2} + a_2 b_1\sqrt{2} + 2 b_1 b_2 \\
        &= (a_1 a_2 + 2 b_1 b_2) + (a_1 b_2 + a_2 b_1)\sqrt{2}.
    \end{align*}
    Since $\mathbb{Q}$ is closed under multiplication, 
        it follows that $g(x, y) \in \mathbb{Q}(\sqrt{2})$.
    Thus $\mathbb{Q}(\sqrt{2})$ is closed under $\{f, g\}$.

    Since $\mathbb{Q}(\sqrt{2})$ 
        is closed under $\{f,g\}$, and any closed set containing 
        $\mathbb{Q}(\sqrt{2})$ must contain all such $a+b\sqrt{2}$, 
        it follows that $\mathbb{Q}(\sqrt{2})$ is the closure.
\end{proof}

\textbf{Solution (c):}
\[\{a + b \sqrt[3]{2} + c(\sqrt[3]{2})^2 \mid a, b, c \in \mathbb{Q}\}\]

\subsection{Images and Inverse Images: A Research Project}

\newpage
\begin{tcolorbox}[title=Problem 1, breakable]
    Suppose $f : A \rightarrow B$ and $W$ and $X$ are subsets of $A$.

    (a) Will it always be true that $f(W \cup X) = f(W) \cup f(X)$?

    (b) Will it always be true that $f(W \setminus X) = f(W) \setminus f(X)$?

    (c) Will it always be true that $W \subseteq X \iff f(W) \subseteq f(X)$?
\end{tcolorbox}


\begin{proof}
    Let $y$ be an arbitrary element in $f(W \cup X)$.
    There exists $x \in W \cup X$ such that $f(x) = y$.
    Either $x \in W$ or $x \in X$.
    Suppose $x \in W$ then $y \in f(W)$ and clearly $y \in f(W) \cup f(X)$.
    Suppose $x \in X$ then $y \in f(X)$ and clearly $y \in f(W) \cup f(X)$.
    Thus $f(W \cup X) \subseteq f(W) \cup f(X)$.

    Let $y$ be an arbitrary element in $f(W) \cup f(X)$.
    Either $y \in f(W)$ or $y \in f(X)$.
    Suppose $y \in f(W)$.
    It follows there exists $x \in W$ such that $f(x) = y$
        and $x \in X \cup W$.
    Thus $y \in f(X \cup W)$.
    Suppose $y \in f(X)$.
    It follows there exists $x \in X$ such that $f(x) = y$
        and $x \in X \cup W$.
    Therefore $y \in f(X \cup W)$.
    Thus $f(W) \cup f(X) \subseteq f(W \cup X)$.

    Since $f(W \cup X) \subseteq f(W) \cup f(X)$
        and $f(W) \cup f(X) \subseteq f(W \cup X)$
        it follows that $f(W \cup X) = f(W) \cup f(X)$.
\end{proof}

\textbf{Solution (b):}
\[A = \{1, 2\}, B = \{3\}, X = \{1, 2\}, W = \{1\}, f = \{(1, 3), (2, 3)\}\]
So 
\[f(X \setminus W) = f(\{2\}) = \{3\}\]
\[f(X) \setminus f(W) = \{3\} \setminus \{3\} = \emptyset\]

\textbf{Solution (c):}
\[A = \{1, 2\}, B = \{3\}, W = \{1\}, X = \{2\}, f = \{(1, 3), (2, 3)\}\]
\[f(W) = \{3\}, \quad f(X) = \{3\}, f(W) \subseteq f(X)\]

\newpage
\begin{tcolorbox}[title=Problem 2, breakable]
    Suppose $f : A \rightarrow B$ and $Y$ and $Z$ are subsets of $B$.
    
    (a) Will it always be true that $f^{-1}(Y \cap Z) = f^{-1}(Y) \cap f^{-1}(Z)$?

    (b) Will it always be true that $f^{-1}(Y \cup Z) = f^{-1}(Y) \cup f^{-1}(Z)$?

    (c) Will it always be true that $f^{-1}(Y \setminus Z) = f^{-1}(Y) \setminus f^{-1}(Z)$?

    (d) Will it always be true that $Y \subseteq Z \iff f^{-1}(Y) \subseteq f^{-1}(Z)$?
\end{tcolorbox}

\begin{proof}
    If $Y$ or $Z$ is the emptyset.
    Note that $f^{-1}(Y) \cap f^{-1}(Z) = \emptyset$
        if $f^{-1}(Y) = \emptyset$ or $f^{-1}(Z) = \emptyset$.
    Then
    \[f^{-1}(Y \cap Z) = f^{-1}(\emptyset) = \emptyset = f^{-1}(Y) \cap f^{-1}(Z)\]
    
    Now, suppose $Y$ and $Z$ are not the emptyset.
    Let $x$ be an arbitrary element in $f^{-1}(Y \cap Z)$.
    There exists $y \in Y \cap Z$ such that $f(x) = y$.
    It follows that $y \in Y$ and $y \in Z$.
    Since $y \in Y$ it follows that $x \in f^{-1}(Y)$.
    Since $y \in Z$ it follows that $x \in f^{-1}(Z)$.
    Thus $f^{-1}(Y \cap Z) \subseteq f^{-1}(Y) \cap f^{-1}(Z)$.

    Let $x$ be an arbitrary element in $f^{-1}(Y) \cap f^{-1}(Z)$.
    It follows that $x \in f^{-1}(Y)$ and $x \in f^{-1}(Z)$.
    There exists $y_1 \in Y$ and $y_2 \in Z$ such that 
        $f(x) = y_1$ and $f(x) = y_2$.
    Since $f$ is a function it follows that $y_1 = y_2$.
    Let $y = y_1 = y_2$.
    Thus $y \in Y$ and $y \in Z$ and it follows that 
        $y \in Y \cap Z$.
    It then follows that $x \in f^{-1}(Y \cap Z)$.
    Thus $f^{-1}(Y) \cap f^{-1}(Z) \subseteq f^{-1}(Y \cap Z)$.

    Therefore $f^{-1}(Y \cap Z) = f^{-1}(Y) \cap f^{-1}(Z)$.
\end{proof}

\begin{proof}
    If $Y = \emptyset$ or $Z = \emptyset$ it is clear 
        that $f^{-1}(Y \cup Z) = f^{-1}(Y) \cup f^{-1}(Z)$.

    Suppose $Y$ and $Z$ are not emptysets.
    Let $x$ be an arbitrary element in $f^{-1}(Y \cup Z)$.
    There exists $y \in Y \cup Z$ such that $f(x) = y$.
    Either $y \in Y$ or $y \in Z$.
    Suppose $y \in Y$ then $x \in f^{-1}(Y)$
        and certainly $x \in f^{-1}(Y) \cup f^{-1}(Z)$.
    Suppose $y \in Z$ then $x \in f^{-1}(Z)$
        and certainly $x \in f^{-1}(Y) \cup f^{-1}(Z)$.
    Thus $f^{-1}(Y \cup Z) \subseteq f^{-1}(Y) \cup f^{-1}(Z)$.

    Let $x$ be an arbitrary element in $f^{-1}(Y) \cup f^{-1}(Z)$.
    Either $x \in f^{-1}(Y)$ or $x \in f^{-1}(Z)$.
    Suppose $x \in f^{-1}(Y)$.
    There exists $y \in Y$ such that $f(x) = y$.
    Certainly $y \in Y \cup Z$ so $x \in f^{-1}(Y \cup Z)$.
    Suppose $x \in f^{-1}(Z)$.
    There exists $y \in Z$ such that $f(x) = y$.
    Certainly $y \in Y \cup Z$ so $x \in f^{-1}(Y \cup Z)$.
    Thus $f^{-1}(Y) \cup f^{-1}(Z) \subseteq f^{-1}(Y \cup Z)$.

    Therefore $f^{-1}(Y \cup Z) = f^{-1}(Y) \cup f^{-1}(Z)$.
\end{proof}

\begin{proof}
    If $Y = \emptyset$ or $Z = \emptyset$ it is clear 
        that $f^{-1}(Y \setminus Z) = f^{-1}(Y) \setminus f^{-1}(Z)$.

    Suppose $Y$ and $Z$ are not emptysets.
    Let $x$ be an arbitrary element in $f^{-1}(Y \setminus Z)$.
    It follows that there exists $y \in Y \setminus Z$
        such that $f(x) = y$.
    Then $y \in Y$ and $y \notin Z$.
    Thus $x \in f^{-1}(Y)$.
    Since $f$ is a function and $y \notin Z$ it follows that $x \notin f^{-1}(Z)$.
    Thus $x \in f^{-1}(Y) \setminus f^{-1}(Z)$.
    Thus $f^{-1}(Y \setminus Z) \subseteq f^{-1}(Y) \setminus f^{-1}(Z)$

    Let $x$ be an arbitrary element in $f^{-1}(Y) \setminus f^{-1}(Z)$.
    It follows that $x \in f^{-1}(Y)$ and $x \notin f^{-1}(Z)$.
    There exists $y \in Y$ such that $f(x) = y$.
    Now $y \notin Z$ since $x \notin f^{-1}(Z)$.
    Since $y \in Y$ and $y \notin Z$ it follows that $y \in Y \setminus Z$.
    Thus $x \in f^{-1}(Y \setminus Z)$.
    Thus $f^{-1}(Y) \setminus f^{-1}(Z) \subseteq f^{-1}(Y \setminus Z)$.
\end{proof}

\textbf{Solution (c):}
\[A = \{1, 2\}, \quad B = \{a, b, c\}, \quad Y = \{a, c\}, \quad Z = \{a, b\}, \quad f = \{(1, a), (2, b)\}\]
\[f^{-1}(Y) = \{1\}, \quad f^{-1}(Z) = \{1, 2\}, \quad f^{-1}(Y) \subseteq f^{-1}(Z)\]

\begin{tcolorbox}[title=Problem 3, breakable]
    Suppose $f : A \rightarrow B$ and $X \subseteq A$.
    Will it always be true that $f^{-1}(f(X)) = X$?
\end{tcolorbox}

\textbf{Solution:}
\[A = \{1, 5\}, B = \{2\}, X = \{1\} f = \{(1, 2), (5, 2)\}\]
\[f(X) = \{2\}, f^{-1}(\{2\}) = \{1, 5\}\]

\begin{tcolorbox}[title=Problem 4, breakable]
    Suppose $f : A \rightarrow B$ and $Y \subseteq B$.
    Will it always be true that $f(f^{-1}(Y)) = Y$?
\end{tcolorbox}

Counterexample:
\[A = \{1\}, B = \{1, 2\}, f = \{a, 1\}, Y = \{1, 2\}\]
\[(f \circ f^{-1})(Y) 
    = (f \circ f^{-1})(\{1, 2\})
    = f(\{a\})
    = \{1\}\]

\begin{tcolorbox}[title=Problem 5, breakable]S
    Suppose $f : A \rightarrow A$ and $C \subseteq A$.
    Prove that the following statements are equivalent:

    (a) $C$ is closed under $f$

    (b) $f(C) \subseteq C$.

    (c) $C \subseteq f^{-1}(C)$.
\end{tcolorbox}

\begin{proof}
    We prove $(a) \implies (b) \implies (c) \implies (a)$.

    Suppose $C$ is closed under $f$.
    Let $c$ be an arbitrary element in $C$.
    Since $C$ is closed under $f$
        it follows that $f(c) \in C$.
    Thus $f(C) \subseteq C$.

    Suppose $f(C) \subseteq C$.
    Let $c$ be an arbitrary element in $C$.
    Since $f(C) \subseteq C$ it follows that $f(c) \in C$.
    Thus $c \in f^{-1}(C)$.
    Therefore $C \subseteq f^{-1}(C)$.

    Suppose $C \subseteq f^{-1}(C)$.
    Let $c$ be an arbitrary element in $C$.
    Then $c \in f^{-1}(C)$.
    It follows that $f(c) \in C$.
    Therefore $C$ is closed under $f$.
\end{proof}

\begin{tcolorbox}[title=Problem 6, breakable]
    Suppose $f : A \rightarrow B$ and $g : B \rightarrow C$.
    Can you prove any interesting theorems about images and inverse 
    images of sets under $g \circ f$?
\end{tcolorbox}

\begin{theorem}
Let $X \subseteq A$ and $Y \subseteq C$. Then
\[(g \circ f)(X) = g(f(X)).\]
\[(g \circ f)^{-1}(Y) = f^{-1}(g^{-1}(Y)).\]
\end{theorem}

\begin{proof}
    Let $y \in (g \circ f)(X)$. 
    There exists $x \in X$ 
        such that $(g \circ f)(x) = y = g(f(x)) = y$.  
    Thus $f(x) \in f(X)$ and $y = g(f(x)) \in g(f(X))$. 
    Therefore $(g \circ f)(X) \subseteq g(f(X))$.  

    Let $y \in g(f(X))$. 
    Then there exists $b \in f(X)$ 
        such that $g(b) = y$. 
    Since $b \in f(X)$, there exists $x \in X$ with $f(x) = b$. 
    Then $(g \circ f)(x) = g(f(x)) = g(b) = y$.  
    Thus $y \in (g \circ f)(X)$
        and it follows that $(g \circ f)(X) = g(f(X))$.

    Let $x \in (g \circ f)^{-1}(Y)$. Then $(g \circ f)(x) = g(f(x)) \in Y$, 
    It follows that $f(x) \in g^{-1}(Y)$.
    Thus $x \in f^{-1}(g^{-1}(Y))$. 
    Therefore, $(g \circ f)^{-1}(Y) \subseteq f^{-1}(g^{-1}(Y))$.  

    If $x \in f^{-1}(g^{-1}(Y))$, then $f(x) \in g^{-1}(Y)$, 
        so $g(f(x)) \in Y$.
    It follows that $x \in (g \circ f)^{-1}(Y)$.  
    Thus, $(g \circ f)^{-1}(Y) = f^{-1}(g^{-1}(Y))$.
\end{proof}

\begin{tcolorbox}[title=Problem 7, breakable]
    Suppose $f : A \rightarrow B$, $f$ is one-to-one 
    and onto, and $Y \subseteq B$. Show that the inverse 
    image of $Y$ under $f$ and the image of $Y$ under $f^{-1}$  
    are equal. (Hint: First write out the definitions of 
    the two sets carefully!)
\end{tcolorbox}

\begin{proof}
    We must show that the inverse image of $Y$ under $f$, which is
    \[
        f^{-1}(Y) = \{ a \in A \mid f(a) \in Y \},
    \]
    is equal to the set
    \[
        \{ a \in A \mid \exists b \in Y \text{ such that } f(a) = b \}.
    \]
    Let $a$ be an arbitrary element in $\{ a \in A \mid f(a) \in Y \}$. 
    Then by definition $f(a) \in Y$, 
        there exists $b \in Y$ with $f(a) = b$. 
    Thus $a \in \{ a \in A \mid \exists b \in Y \text{ such that } f(a) = b \}$.

    Let $a$ be an arbitrary element in 
        $\{ a \in A \mid \exists b \in Y \text{ such that } f(a) = b \}$. 
    There exists $b \in Y$ such that $f(a) = b$, so $f(a) \in Y$. 
    Thus $a \in \{ a \in A \mid f(a) \in Y \}$.
\end{proof}