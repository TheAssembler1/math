\subsection{Proof by Mathematical Induction}

\begin{tcolorbox}[title=Problem 1, breakable]
    Prove that for all $n \in \mathbb{N}$,
        $0 + 1 + 2 + \cdots + n = \frac{n(n + 1)}{2}$
\end{tcolorbox}

\begin{proof}
    (\textbf{Base Case}) 
    \[\frac{1(1 + 1)}{2} = \frac{2}{2} = 1\]
    (\textbf{Induction Step}) Suppose $0 + 1 + 2 + \cdots + n = \frac{n(n + 1)}{2}$.
    Then 
    \begin{align*}
        0 + 1 + 2 + \cdots + n + (n + 1) 
            &= \frac{n(n + 1)}{2} + (n + 1) \\
            &= \frac{n(n + 1)}{2} + \frac{2n}{2} + \frac{2}{2} \\
            &= \frac{n(n + 1) + 2n + 2}{2} \\
            &= \frac{n(n + 1) + 2(n + 1)}{2} \\
            &= \frac{(n + 2)(n + 1)}{2} \\
            &= \frac{(n + 1)((n + 1) + 1)}{2}
    \end{align*}
\end{proof}

\begin{tcolorbox}[title=Problem 4, breakable]
    Find a formula for $1 + 3 + 5 + \cdots + (2n - 1)$,
        for $n \ge 1$, and prove your formula is correct.
    (Hint: First try some particular values of $n$ and 
           look for a pattern.)
\end{tcolorbox}
\[1 + 3 + 5 + \cdots + (2n - 1) = n^2\]
\begin{proof}
    (\textbf{Base Case})
    \[n^2 = 1^2 = 1\]
    (\textbf{Induction Step})
    Suppose $1 + 3 + 5 + \cdots + (2n - 1) = n^2$
    Then 
    \begin{align*}
        1 + 3 + 5 + \cdots + (2n - 1) + 2n + 1 
            &= n^2 + 2n + 1 \\
            &= (n + 1)^2
    \end{align*}
\end{proof}

\begin{tcolorbox}[title=Problem 9, breakable]
    (a) Prove that for all $n \in \mathbb{N}$, $2 \mid (n^2 + n)$.

    (b) Prove that for all $n \in \mathbb{N}$, $6 \mid (n^3 - n)$.
\end{tcolorbox}

\begin{proof}
    (\textbf{Base Case}) 
    \[n^2 + n = 1^2 + 1 = 2\]
    Clearly $2 \mid 2$.

    (\textbf{Induction Step})
    Suppose $n \in \mathbb{N}$, $2 \mid (n^2 + n)$.
    Let $k$ be the integer such that $2k = n^2 + n$.
    Then
    \begin{align*}
        (n + 1)^2 + (n + 1) &= n^2 + 2n + 1 + n + 1 \\
        &= n^2 + n + 2n + 2 \\
        &= 2k + 2n + 2 \\
        &= 2(k + n + 1)
    \end{align*}
\end{proof}

\begin{proof}
    (\textbf{Base Case})
    \[1^3 - 1 = 0\]
    Clearly $6 \mid 0$.

    (\textbf{Induction Step})
    Suppose $6 \mid (n^3 - n)$.
    Let $k$ be the integer such that $6k = n^3 - n$.
    It follows that $6k + n = n^3$.
    Then 
    \begin{align*}
        (n + 1)^3 - (n + 1) &= n^3 + 3n^2 + 3n + 1 - n - 1 \\
        &= n^3 - n + 3n^2 + 3n \\
        &= 6k + 3(n^2 + n)
    \end{align*}
    Now, $n^2 + n = n(n + 1)$ which is the product of two consecutive 
        natural numbers. One of which is even thus $n^2 + n$ is even.
    Let $j$ be the integer such that $2j = n^2 + n$.
    Then $6k + 3(n^2 + n) = 6k + 3(2j) = 6k + 6j = 6(k + j)$.
\end{proof}

\begin{tcolorbox}[title=Problem 13, breakable]
    Prove that for all integers $a$ and $b$ and all $n \in \mathbb{N}$,
        $(a - b) \mid (a^n - b^n)$.
    (Hint: Let $a$ and $b$ be arbitrary integers and then prove by induction 
        that $\forall{n} \in \mathbb{N}[(a - b) \mid (a^n - b^n)])$.
    For the induction step, you must relate $a^{n + 1} - b^{n + 1}$
        to $a^n - b^n$. 
    You might find it useful to start by completing the following equation:
    $a^{n + 1} - b^{n + 1} = a(a^n - b^n) + ?$.
    The $?$ equals $b^n(a - b)$.
\end{tcolorbox}

\begin{proof}
    (\textbf{Base Case}) 
    \[a^n - b^n = a^1 - b^1 = a - b\]
    Clearly $(a - b) \mid (a - b)$
    (\textbf{Induction Step}) Suppose for all integers $a$ and $b$ and all $n \in \mathbb{N}$,
        $(a - b) \mid (a^n - b^n)$.
        Let $k$ be the integer such that $k(a - b) = a^n - b^n$.
    Then 
    \begin{align*}
        a^{n + 1} - b^{n + 1} &= a(a^n - b^n) + b^n(a - b) \\
        &= a k(a - b) + b^n(a - b) \\
        &= (a - b)(ak + b^n)
    \end{align*}
\end{proof}

\begin{tcolorbox}[title=Problem 15, breakable]
    Prove that for all $n \ge 10$, $2^n > n^3$.
\end{tcolorbox}

\begin{proof}
    (\textbf{Base Case})
    \[2^{10} > 10^3 \iff 1024 > 1000\]
    (\textbf{Induction Step}) Suppose for some fixed $n \ge 10$, $2^n > n^3$.
    Then $2^{n + 1} = 2 \cdot 2^n > 2n^3$.
    Now $(n + 1)^3 = n^3 + 3n^2 + 3n + 1$ and we need to show 
        $2n^3 > n^3 + 3n^2 + 3n + 1$ which is equivalent to 
        $n^3 > 3n^2 + 3n + 1$ for all $n \ge 10$.
    We can transform the problem into
    \[n^3 - 3n^2 - 3n - 1 > 0 \quad \text{for } n \ge 10\]
    Then $n^3 - 3n^2 - 3n - 1 > 10^3 - 3(10^2) - 3(10) - 1 = 669 > 0$.
    Since \(f(n) = n^3 - 3n^2 - 3n - 1\) has derivative 
    \(f'(n) = 3n^2 - 6n - 3 = 3(n^2 - 2n - 1) > 0\) for \(n \ge 3\), \(f\) is increasing, 
    so \(f(n) \ge f(10) = 669 > 0\) for all \(n \ge 10\).
\end{proof}

\begin{tcolorbox}[title=Problem 16, breakable]
    (a) Prove that for all $n \in \mathbb{N}$, either $n$ 
        is even or $n$ is odd, but not both.

    (b) Prove that, as claimed in Section $3.4$, every integer 
        is either even or odd, but not both. (Hint: 
        To prove that a negative integer $n$ is even or odd,
        but not both, apply part (a) to $-n$.)
\end{tcolorbox}

\begin{proof}
    For contradiction, suppose $n$ is even and odd.
    Let $k_1, k_2$ be the integers such that $n = 2k_1$ and $n = 2 k_2 + 1$.
    Then 
    \[
        2k_1 = 2k_2 + 1 \iff 2(k_1 - k_2) = 1
    \]
    There is clearly no integer solution to $2x = 1$, so $n$ is not both even and odd.

    (\textbf{Base Case}) If $n = 1$, it is clearly odd since $1 = 2(0) + 1$,
        and if $n = 2$, then it is clearly even since $2 = 2(1)$.
    
    (\textbf{Induction Step}) Suppose $n$ is even or odd but not both.
    Now consider $n + 1$. Either $n$ is even or odd. 
    Suppose $n$ is even; let $k_1$ be the integer such that $2k_1 = n$.
    Then $2k_1 + 1$, thus $n + 1$ is odd.
    Suppose $n$ is odd; let $k_1$ be the integer such that $2k_1 + 1 = n$.
    Then $2k_1 + 2 = 2(k_1 + 1)$, thus $n + 1$ is even.
\end{proof}

\begin{proof}
    The case where $n \ge 0$ is clear.
    Now suppose $n < 0$.
    Let $b$ be the positive number such that $-b = n$.
    From part (a) $b$ is either even or odd, but not both.
    Thus either there exists $k_1$ such that $2 k_1 = b$,
        in which case $- 2 k_1 = n$ is even,
    or there exists $k_2$ such that $2 k_2 + 1 = b$,
        in which case $- (2 k_2 + 1) = n$ is odd.
\end{proof}

\begin{tcolorbox}[title=Problem 18, breakable]
    (a) What's wrong with the following proof that every $n \in \mathbb{N}$,
        $1 \cdot 3^0 + 3 \cdot 3^1 + 5 \cdot 3^2 + \cdots + (2n + 1)3^n = n 3^{n + 1}$?
    \begin{proof}
        We use mathematical induction. Let $n$ be an arbitrary natural number,
        and suppose $1 \cdot 3^0 + 3 \cdot 3^1 + 5 \cdot 3^2 + \cdots + (2n + 1)3^n = n 3^{n + 1}$.
        Then 
        \begin{align*}
            1 \cdot 3^0 + 3 \cdot 3^1 + 5 \cdot 3^2 + \cdots + (2n + 1)3^n + (2n + 3)3^{n + 1}
            &= n 3^{n + 1} + (2n + 3) 3^{n + 1} \\
            &= (3n + 3) 3^{n + 1} \\
            &= (n + 1)3^{n + 2}
        \end{align*}
        as required.
    \end{proof} 
    (b) Find a formula for $1 \cdot 3^0 + 3 \cdot 3^1 + 5 \cdot 3^2 + \cdots + (2n + 1)3^n = n 3^{n + 1}$, and 
        prove that your forumla is correct.
\end{tcolorbox}

\textbf{Solution:}

It never establishes the base case (which fails).
The formula shows the power of $3$ should be $1$ more than the coefficient.
But on the l.h.s. the power has $1$ fewer than the coefficient.
The formula should be 
\[1 \cdot 3^0 + 3 \cdot 3^1 + 5 \cdot 3^2 + \cdots + (2n + 1)3^n = 3^n \cdot (n - 1) + 1\]

\begin{proof}
    (\textbf{Base Case}) 
    \[
    3^n \cdot (n - 1) + 1 = 3^1 \cdot (1 - 1) + 1 = 0 + 1 = 1
    \]

    (\textbf{Induction Step})  
    Suppose 
    \[
    1 \cdot 3^0 + 3 \cdot 3^1 + 5 \cdot 3^2 + \cdots + (2n - 1)3^{n-1} = 3^n \cdot (n - 1) + 1.
    \]  
    Then 
    \begin{align*}
        1 \cdot 3^0 + 3 \cdot 3^1 + 5 \cdot 3^2 + \cdots + (2n - 1)3^{n-1} + (2n + 1)3^n 
        &= 3^n \cdot (n - 1) + 1 + (2n + 1)3^n \\
        &= 3^n(n - 1 + 2n + 1) + 1 \\
        &= 3^n (3n) + 1 \\
        &= 3^{n+1} \cdot n + 1 
    \end{align*}
\end{proof}

\begin{tcolorbox}[title=Problem 19, breakable]
    Suppose $a$ is a real number and $a < 0$. Prove that for all $n \in \mathbb{N}$,
    if $n$ is even then $a^n > 0$, and if $n$ is odd then $a^n < 0$.
\end{tcolorbox}

\begin{proof}
    (\textbf{Base Case}) First consider $n = 1$. Let $a = -b$ where $b > 0$.
    Then $a^1 = (-b)^{1} = -b$ which is negative.
    Now consider $n = 2$. Again, let $a = -b$ where $b > 0$.
    Then $a^2 = a \cdot a = -b \cdot -b = (-1 \cdot -1) b^2 = 1 b^2 = b^2$ which is positive.

    (\textbf{Induction Step}) Suppose the relationship holds up to some $n \in \mathbb{N}$.
    Consider $a^{n + 1} = a^n \cdot a$.
    Suppose $n$ is even by the hypothesis $a^n$ is positive thus $a^n \cdot a$ is negative.
    Thefore $a^{n + 1}$ is negative.
    Similarly, suppose $n$ is odd by the hypothesis $a^n$ is negative thus $a^n \cdot a$ is positive.
    Thefore $a^{n + 1}$ is positive.
\end{proof}

\begin{tcolorbox}[title=Problem 20, breakable]
    Suppose $a$ and $b$ are real numbers and $0 < a < b$.

    (a) Prove that for all $n \ge 1$, $0 < a^n < b^n$. (Notice that this 
        generalizes Example $3.1.2.$)

    (b) Prove that for all $n \ge 2$, $0 < \sqrt[n]{a} < \sqrt[n]{b}$.

    (c) Prove that for all $n \ge 1$, $a b^n + b a^n < a^{n + 1} + b^{n + 1}$.

    (d) Prove that for all $n \ge 2$.
    \[\left(\frac{a + b}{2}\right)^n < \frac{a^n + b^n}{2}\]
\end{tcolorbox}

\begin{proof}
    (\textbf{Base Case}) $n = 1$ is clear and $n = 2$ was shown 
        by Example $3.1.2$.
    
    (\textbf{Induction Step}) Suppose the relationship holds up to some $n \in \mathbb{N}$.
    Consider the $n + 1$ case
    \[0 < a^{n + 1} < b^{n + 1} \iff a^{n + 1} - b^{n + 1} < 0 \iff a(a^n - b^n) + b^n(a - b) < 0\]
    By the hypothesis $a^n - b^n < 0$ and $a > 0$ thus $a(a^n - b^n) < 0$.
    Also $b^n > 0$ and since $a < b$ it follows that $a - b < 0$
        thus $b^n(a - b) < 0$.
    The sum of two numbers $< 0$ is clearly $< 0$ thus the relationship holds for $n + 1$.
\end{proof}

\begin{proof}
    Suppose $0 < a < b$.
    Note $a = a^{1} = a^{\frac{n}{n}} = \left(a^{\frac{1}{n}}\right)^n$.
    Similarly $b = \left(b^{\frac{1}{n}}\right)^n$.
    Then 
    \begin{align*}
        a < b
        \iff &{\left(a^{\frac{1}{n}}\right)}^{n} < {\left(b^{\frac{1}{n}}\right)}^{n} \\
        \iff &a^{\frac{1}{n}} < b^{\frac{1}{n}} \quad \text{Part (a)}
    \end{align*}
\end{proof}

\begin{proof}
    Consider
    \begin{align*}
        (a b^n + b a^n) - (a^{n + 1} + b^{n + 1})
        &= (a b^n - a^{n + 1}) + (b a^n - b^{n + 1}) \\
        &= a(b^n - a^n) + b(a^n - b^n)
    \end{align*}
    Now let $t = b^n - a^n > 0$ by part (a).
    Clearly $-t = a^n - b^n$.
    Thus $a(b^n - a^n) + b(a^n - b^n) = at - bt = t(a - b)$.
    Then since $t > 0$ and $(a - b) < 0$
        it follows that $t(a - b) < 0$.
    Thus $(a b^n + b a^n) - (a^{n + 1} + b^{n + 1}) < 0$
        and it follows that $a b^n + b a^n < a^{n + 1} + b^{n + 1}$.
\end{proof}

\begin{proof}
    ($\textbf{Base Case}$) Let $n = 2$ then consider 
    \begin{align*}
        \left(\frac{a + b}{2}\right)^n - \frac{a^n + b^n}{2}
        &= \frac{(a + b)^n}{2^n} - \frac{a^n + b^n}{2} \\
        &= \frac{(a + b)^2}{2^2} - \frac{a^2 + b^2}{2} \\
        &= \frac{(a + b)^2}{4} - \frac{2a^2 + 2b^2}{4} \\
        &= \frac{a^2 + 2ab + b^2}{4} - \frac{2 a^2 + 2b^2}{4} \\
        &= \frac{a^2 + 2ab + b^2 - (2 a^2 + 2b^2)}{4} \\
        &= \frac{-a^2 - b^2 + 2ab}{4} \\
        &= \frac{(ab + ab) - (a^2 + b^2)}{4} < 0 \quad \text{Apply part (c) to numerator}
    \end{align*}
    It follows that $\left(\frac{a + b}{2}\right)^n < \frac{a^n + b^n}{2}$.
    Thus the relationship holds for $n = 2$.

    (\textbf{Induction Step})  
    Suppose the inequality holds for some $n \ge 2$.
    Multiplying both sides by $\frac{a+b}{2} > 0$ shows
    \[
    \left(\frac{a+b}{2}\right)^{n+1} < \frac{a^n+b^n}{2} \cdot \frac{a+b}{2} = \frac{(a^n+b^n)(a+b)}{4}
    \]
    Then
    \[
    \frac{(a^n+b^n)(a+b)}{4} = \frac{a^{n+1} + b a^n + a b^n + b^{n+1}}{4}
    \]
    By part (c), $a b^n + b a^n < a^{n+1} + b^{n+1}$, so
    \[
    a^{n+1} + b a^n + a b^n + b^{n+1} < 2(a^{n+1} + b^{n+1})
    \]
    Then dividing both sides by $4$ shows
    \[
    \frac{a^{n+1} + b a^n + a b^n + b^{n+1}}{4} < \frac{a^{n+1} + b^{n+1}}{2}
    \]
    It follows that 
    \[\left(\frac{a+b}{2}\right)^{n+1} < \frac{a^{n+1} + b^{n+1}}{2}\]
\end{proof}

\subsection{More Examples}

\begin{tcolorbox}[title=Problem 4, breakable]
    (a) Suppose $R$ is a relation on $A$, 
        and $\forall{x} \in A \forall{y} \in A (x R y \vee y R x)$.
        (Note this implies $R$ is reflexive.) Prove that for every 
        finite, nonempty set $B \subseteq A$ there is some $x \in B$
        such that $\forall{y} \in B ((x, y) \in R \circ R)$.
        (Hint: Imitate Example $6.2.1$)

    (b) Consider a tournament in which each contestant plays every every other 
        contestant exactly once, and one of them wins. We'll say that a 
        contestant $x$ is \emph{excellent} if, for every other contestant $y$,
        either $x$ beats $y$ or there is a third contestant $z$ such that
        $x$ beats $z$ and $z$ beats $y$.
        Prove that there is at least one excellent contestant.
\end{tcolorbox}

\begin{proof}
    We will show by induction that for ever natural number $n \ge 1$,
        for every set $B \subseteq A$ with $n$ elements
        there is some element $x \in B$ such that
        for all $y \in B$, $(x, y) \in R \circ R$.

    (\textbf{Base Case}) 
    Suppose $n = 1$ and $B \subseteq A$ where $B$ has $1$ element.
    Then let $B = \{b\}$ for some $b \in A$.
    Since $R$ is reflexive it follows that $(b, b) \in R$. 
    Clearly $(b, b) \in R \circ R$.

    (\textbf{Induction Step})
    Suppose $n \ge 1$ and $B \subseteq A$ where $B$ has more than $1$ element.
    Let $b$ be any element in $B$, and let $B' = B \setminus \{b\}$.
    By our hypothesis there exists $x \in B'$ such that 
        for all $y \in B'$, $(x, y) \in R \circ R$.
    It follows that there exists $c$ such that 
        $(x, c) \in R$ and $(c, y) \in R$.
    Now either $(c, b) \in R$ or $(b, c) \in R$.

    Suppose $(c, b) \in R$.
    Since $(x, c) \in R$ and $(c, b) \in R$
        it follows that $(x, b) \in R \circ R$.

    Suppose $(b, c) \in R$.
    For every $y \in B'$, there exists some $c$ such that $(x, c) \in R$ and $(c, y) \in R$.
    Since $(b, c) \in R$, it follows that $(b, y) \in R \circ R$ for each $y \in B'$.
    Also, because $R$ is reflexive, $(b,b) \in R \circ R$.
    Thus $b$ is the element in $B$ such that for all $y \in B$, $(b,y) \in R \circ R$.
\end{proof}

\begin{proof}
    (\textbf{Base Case}) Suppose there is $1$ contestant.
    Clearly this single contestant is excellent.

    (\textbf{Inductive Hypothesis}) 
    Suppose that for $n$ contestants there
        is one who is excellent.
    Now consider a tournament between $n + 1$ contestants.
    By our hypothesis there was an excellent contestant from the first $n$ contestants
        who we will label $a$.
    Let the $n + 1$ contestant be $b$, who plays against $a$.
    Either $a$ beats $b$ or $b$ beats $a$.
    If $a$ beats $b$, then $a$ remains excellent.
    If $b$ beats $a$, then $b$ can reach $y$ by a chain through $a$, who is excellent.
    Thus $b$ is excellent.
\end{proof}

\begin{tcolorbox}[title=Problem 6, breakable]
    Prove that if $n \ge 1$ and $a_1, a_2, \ldots, a_n$
    are any real numbers, then $|a_1, a_2, \ldots, a_n|
        \le |a_1| + |a_2| + \cdots  + |a_n|$.
    (Note that this generalizes the triangle inequality;
        see exercise $13(c)$ of Section $3.5$)
\end{tcolorbox}

\begin{proof}
    For $n = 1$, trivially $|x_1| \le |x_1|$.
    For $n = 2$, $|x_1 + x_2| \le |x_1| + |x_2|$ by the triangle inequality.
    Now assume the formula holds for $k = n - 1$, thus:
    \begin{align*}
        |x_1 + x_2 + \cdots + x_{n - 1}| \le |x_1| + |x_2| + \cdots |x_{n - 1}|
    \end{align*}
    Thus:
    \begin{align*}
         & |x_1 + x_2 + \cdots + x_{n - 1} + x_n|        &  &                                  \\
         & \le  |(x_1 + x_2 + \cdots + x_{n - 1}) + x_n| &  &                                  \\
         & \le  |x_1 + x_2 + \cdots + x_{n - 1}| + |x_n| &  & \quad \text{triangle inequality} \\
         & \le  |x_1| + |x_2| + \cdots + |x_n|           &  & 
    \end{align*}
\end{proof}

\begin{tcolorbox}[title=Problem 8, breakable]
    If $n \ge 2$ and $a_1, a_2, \ldots, a_n$ is a list
        of positive real numbers, then the number 
        $(a_1 + a_2 + \cdots + a_n) / n$ is called 
        the \emph{arithmetic mean} of the numbers 
        $a_1, a_2, \ldots, a_n$, and the number 
        $\sqrt[n]{a_1 a_2 \ldots a_n}$ is called 
        their \emph{geometric mean}. 
    In this exercise you will prove the \emph{arithmetic mean-geometric mean inequality},
    which says that the arithmetic mean is always at least as large as the geometric mean.

    (a) Prove that the arithmetic mean-geometric mean inequality holds for lists 
        of numbers of length $2$. In other words, prove that for all positive real 
        real numbers $a$ and $b$,$(a + b)/2 \ge \sqrt{ab}$.

    (b) Prove that the arithmetic mean-geometric mean inequality holds for
        any list of numbers whose length is a power of $2$.
        In other words, prove that for all $n \ge 1$,
        if $a_1, a_2, \ldots, a_{2^n}$ is a list of positive
        real numbers, then 
        \[\frac{a_1 + a_2 + \cdots + a_{2^n}}{2^n} \ge \sqrt[2^n]{a_1 a_2 \ldots a_{2^n}}\]

    (c) Suppose that $n_0 \ge 2$ and the arithmetic mean-geometric mean 
        inequality fails for some list of length $n_0$. In other words,
        there are positive real number $a_1, a_2, \ldots, a_{n0}$ such that 
        \[\frac{a_1 + a_2 + \cdots + a_{n_0}}{n_0} < \sqrt[n_0]{a_1 a_2 \ldots a_{n_0}}\]
        Prove that for all $n \ge n_0$, the arithmetic mean-geometric mean 
        inequality fails for some list of length $n$.

    (d) Prove that the arithmetic mean-geometric mean inequality always 
        holds.
\end{tcolorbox}

\begin{proof}
    Suppose $a$ and $b$ are two positive real numbers.
    Then
    \begin{align*}
    \frac{a + b}{2} - \sqrt{ab} 
        &= \frac{a + b}{2} - \frac{2\sqrt{ab}}{2} 
        = \frac{a + b - 2\sqrt{ab}}{2} \\
        &= \frac{a + b - 2\sqrt{ab}}{2} \cdot \frac{a + b + 2\sqrt{ab}}{a + b + 2\sqrt{ab}} 
        = \frac{(a + b)^2 - (2\sqrt{ab})^2}{2(a + b + 2\sqrt{ab})} \\
        &= \frac{a^2 + 2ab + b^2 - 4ab}{2(a + b + 2\sqrt{ab})} 
        = \frac{a^2 - 2ab + b^2}{2(a + b + 2\sqrt{ab})} 
        = \frac{(a - b)^2}{2(a + b + 2\sqrt{ab})}
    \end{align*}
    Consider the numerator $(a - b)^2$.
    If $a \ge b$ then $(a - b)^2 \ge 0$.
    Similarly, if $a < b$ then $(a - b)^2 > 0$.
    It follows that the numerator is $\ge 0$ thus $\frac{a + b}{2} \ge \sqrt{ab}$.
\end{proof}

\begin{proof}
    (\textbf{Base Case:}) If $n = 1$ then $A$ has length $2$
        and by part (a) the inequality holds.

    (\textbf{Induction Step:}) Suppose the inequality holds for 
    lists of length $2^n$ where $n \in \mathbb{N}$.
    Let 
    \[a_1, a_2, \ldots, a_{2^n}, b_1, b_2, \ldots, b_{2^n}\]
    be a list of $2^{n+1}$ elements. Define
    \[A = a_1 + \cdots + a_{2^n}, \quad A' = a_1 a_2 \cdots a_{2^n}, \quad
        B = b_1 + \cdots + b_{2^n}, \quad B' = b_1 b_2 \cdots b_{2^n}\]

    By the induction hypothesis,
    \[\frac{A}{2^n} \ge \sqrt[2^n]{A'} \quad \text{and} \quad
        \frac{B}{2^n} \ge \sqrt[2^n]{B'}\]
    It follows that 
    \[A \ge 2^n \cdot \sqrt[2^n]{A'} \quad \text{and} \quad B \ge 2^n \cdot \sqrt[2^n]{B'}.\]

    Applying part (a) to $A$ and $B$,
    \[\frac{A + B}{2} \ge \sqrt{A B} \ge \sqrt{2^n \cdot \sqrt[2^n]{A'} \cdot 2^n \cdot \sqrt[2^n]{B'}}
        = 2^n \cdot \sqrt[2^{n+1}]{A' B'}\]

    Dividing both sides by $2^n$ gives
    \[\frac{A + B}{2^{n+1}} \ge \sqrt[2^{n+1}]{A' B'},\]
    which is equivalent to
    \[\frac{a_1 + \cdots + a_{2^n} + b_1 + \cdots + b_{2^n}}{2^{n+1}}
        \ge \sqrt[2^{n+1}]{a_1 \cdots a_{2^n} b_1 \cdots b_{2^n}}\]
\end{proof}

\begin{tcolorbox}[title=Problem 9, breakable]
    Prove that if $n \ge 2$ and $a_1, a_2, \ldots, a_n$
        is a list of positive real numbers,
        then 
    \[\frac{n}{\frac{1}{a_1}} + \frac{1}{a_2} + \cdots + \frac{1}{a_n} \le \sqrt[n]{a_1 a_2 \cdots a_n}\]
    (Hint: Apply exercise $8$. The number on the left side of the 
    inequality above is called the \emph{harmonic mean} of the 
    numbers $a_1, a_2, \ldots a_n$.)
\end{tcolorbox}

\begin{tcolorbox}[title=Problem 11, breakable]
    Prove that for every set $A$, if $A$ has $n$ elements 
        then $\mathcal{P}(A)$ has $2^n$ elements.
\end{tcolorbox}

\begin{proof}
    (\textbf{Base Case}) Consider a set $S$ with $1$ element $a$.
    Then there are two subsets of $S$, namely $\{a\}$ and the $\emptyset$.
    Thus the number of elements in $\mathcal{P}(S)$ is $2$.
    Then $2^n = 2^1 = 2$ so the formula holds for $n = 1$.

    (\textbf{Induction Step}) Suppose the formula holds for 
        sets with $n$ elements.
    Now consider a set $S$ with $n + 1$ elements.
    Let $a$ be an arbitrary element in $S$
        and let $S' = S \setminus \{a\}$.
    By our hypothesis the number of subsets of $S'$ is $2^n$.
    Now consider the union of $S'$ and $\{a\}$.
    The union of each subset of $S'$ with $\{a\}$
        gives a new subset of $S$.
    There are $2^n$ additional subsets of $S$,
        one for each subset of $S'$.
    Then summing the subsets of $\mathcal{P}(S')$
        with the additional subsets when 
        considering $S' \cup \{a\}$ shows
    \[2^n + 2^n = 2 \cdot 2^n = 2^{n + 1}\]
\end{proof}

\begin{tcolorbox}[title=Problem 12, breakable]
    If $A$ is a set, let $\mathcal{P}_2(A)$
    be the set of all subsets of $A$ that have exactly
    two elements. Prove that for every set $A$,
    if $A$ has $n$ elements then $\mathcal{P}_2(A)$
    has $n(n - 1) / 2$ elements.
    (Hint: See the solution for exercise $11$.)
\end{tcolorbox}

\begin{proof}
    (\textbf{Base Case}) Suppose $A$ is a set with $1$ element.
    There is clearly no subsets of $A$ with $2$ elements.
    Then $n(n - 1)/2 = 1(1 - 1)/2 = 0/2 = 0$.
    Thus the formula holds fore $n = 1$.

    (\textbf{Induction Step}) Suppose the formula holds for sets 
        with $n$ elements.
    Now consider a set $A$ with $n + 1$ elements.
    Let $a$ be an arbitrary element in $A$
        and let $A' = A \setminus \{a\}$.
    By our hypothesis the number of subsets of $A'$
        with $2$ elements is $n(n - 1)/2$.
    Now consider the union of $A'$ and $\{a\}$.
    Each subset of $A'$ containng a single element 
        can be unioned with $\{a\}$ to create 
        a $2$ element subset of $A$.
    Thus there are an additional $n$ subsets.
    Then summing the subsets of $\mathcal{P}(A')$
        with $2$ elements with the additional subsets with $2$ elements when 
        considering $A' \cup \{a\}$ shows
    \[n(n - 1)/2 + n = \frac{n(n - 1)}{2} + \frac{2n}{2} = \frac{n^2 + n}{2} = \frac{n(n + 1)}{2}\]
\end{proof}

\begin{tcolorbox}[title=Problem 16, breakable]
    Prove that for every finite set $A$ and every function
    $f : A \rightarrow A$, if $f$ is one-to-one 
    then $f$ is onto. (Hint: Use induction on the number 
    of elements in $A$ with $n$ elements, and suppose that 
    $A$ has $n + 1$ elements and $f : A \rightarrow A$.
    Suppose $f$ is one-to-one but not onto. Then there is some 
    $a \in A$ such that $a \notin Ran(f)$.
    Let $A' = A \setminus \{a\}$ and $f'= f \cap (A' \times A')$.
    Show that $f' : A' \rightarrow A'$, $f'$ is one-to-one,
    and $f'$ is not onto, which contradicts the inductive 
    hypothesis.)
\end{tcolorbox}

\begin{proof}
    (\textbf{Base Case}) Suppose $A$ is a set with a single element.
    Let $f : A \rightarrow A$ be an arbitrary one-to-one function.
    Futhermore, let $a$ be an arbitrary element in $A$.
    It follows that there exists $b \in A$ such that $f(a) = b$.
    But since $A$ has a single element, $a = b$ must be this element 
        and therefore $f$ is clearly onto.

    (\textbf{Induction Step}) Suppose the theorem holds for sets with $n$ elements.
    Suppose $A$ is a set with $n + 1$ elements and $f : A \rightarrow A$.
    Suppose $f$ is one-to-one but not onto.
    There exists some $a' \in A$ such that $a' \notin Ran(f)$.
    Let $A' = A \setminus \{a'\}$ and $f' = f \cap (A' \times A')$.

    Let $a$ be an arbitrary element in $A'$. Since $A' \subseteq A$
        it follows that $a \in A$.
    It follows that there exists $b \in A$ such that $(a, b) \in f$.
    Since $b \in Ran(f)$ it follows that $b \ne a'$ thus $b \in A'$.
    Since $(a, b) \in f$ and $(a, b) \in A' \times A'$
        it follows that $(a, b) \in f \cap (A' \times A')$.
    Thus $f'$ maps each element in $A'$ to an element in $A'$.
    Now suppose $(a, b), (a, b') \in f'$.
    Then $(a, b), (a, b') \in f$ and since $f$ is a function $b = b'$.
    Thus $f'$ maps each element in $A'$ to exactly one element in $A'$.
    Thus $f' : A' \rightarrow A'$.

    Now suppose $(a, c), (b, c) \in f'$.
    Then $(a, c), (b, c) \in f$ and since $f$ is one-to-one $a = b$.
    Thus $f'$ is one-to-one.

    Suppose $f'$ is onto.
    Since $f'$ is one-to-one it maps each of the $n$ elements in $A'$
        to a single unique element in $A'$.
    Thus $f'$ maps the $n$ elements of $A'$ to $n$ distinct elements in $A'$.
    Now consider the element $a'$ which was removed from $A$ to construct $A'$.
    Since $f'$ already maps to all of $A'$, $f' \subseteq f$, and $f$ is one-to-one,
        $a'$ cannot be mapped to any element of $A'$ under $f$.
    Thus $f$ cannot be onto $A$, contradicting our assumption that $f$ is not onto.
    It follows that $f'$ is not onto which contradicts our induction hypothesis.
\end{proof}

\begin{tcolorbox}[title=Problem 17, breakable]
    What's wrong with the following proof that if $A \subseteq \mathbb{N}$
        and $0 \in A$ then $A = \mathbb{N}$.
    \begin{proof}
        We will prove by induction that $\forall n \in \mathbb{N}(n \in A)$.

        Base Case: If $n = 0$, then $n \in A$ by assumption.

        Induction Step: Let $n \in \mathbb{N}$ be arbitrary,
            and suppose $n \in A$.
        Since $n$ was arbitrary, it follows that every 
            natural number is an element of $A$,
            and therefore in particular $n + 1 \in A$.
    \end{proof}
\end{tcolorbox}

\textbf{Solution:}

Base case doesn't establish $A = \mathbb{N}$.
Induction step says $n$ is an arbitrary element of $A$
    then claims $n + 1 \in A$ because $n$ is arbitrary.
That doesn't logically follow.

\newpage
\begin{tcolorbox}[title=Problem 18, breakable]
    Suppose $f : \mathbb{R} \rightarrow \mathbb{R}$.
    What's wrong with the following proof that for 
        every finite, nonempty set $A \subseteq \mathbb{R}$
        there is a real number $c$ such that 
        $\forall{x \in A}(f(x) = c)$?
    \begin{proof}
        We will prove by induction that for every $n \ge 1$, if $A$
        is any subset of $\mathbb{R}$ with $n$ elements then 
        $\exists{c \in \mathbb{R}}\forall{x \in A}(f(x) = c)$.

        Base Case: $n = 1$. Suppose $A \subseteq R$ and $A$ has one element.
        Then $A = \{a\}$, for some $a \in \mathbb{R}$.
        Let $c = f(a)$. Then clearly $\forall{x \in A}(f(x) = c)$.

        Induction Step: Suppose $n \ge 1$, and for all $A \subseteq \mathbb{R}$,
        if $A$ has $n$ elements then $\exists{c \in \mathbb{R}}\forall{x \in \mathbb{A}}(f(x) = c)$.
        Now supppose $A \subseteq \mathbb{R}$ and $A$ has $n + 1$ elements.
        Let $a_1$ be any element of $A$, and let $A_1 = A \setminus \{a_1\}$.
        Then $A_1$ has $n$ elements, so by the inductive hypothesis there 
            is some $c_1 \in \mathbb{R}$ such that $\forall{x \in A_1}(f(x) = c_1)$.
        If we can show that $f(a_1) = c_1$ then we will be done,
            since then it will follow that $\forall{x \in A}(f(x) = c_1)$.

        Let $a_2$ be an element of $A$ that is different from $a_1$,
            and let $A_2 = A \setminus \{a_2\}$.
        Applying the inductive hypothesis again, we can choose 
        a number $c_2 \in \mathbb{R}$ such that 
        $\forall{x \in A_2}(f(x) = c_2)$. 
        Notice since $a_1 \ne a_2$, $a_2 \in A_2$,
            so $f(a_2) = c_2$.
        Now let $a_3$ be an element in $A$ that is different 
        from both $a_1$ and $a_2$. 
        Then $a_3 \in A_1$ and $a_3 \in A_2$,
            so $f(a_3) = c_1$ and $f(a_3) = c_2$.
        Therefore $c_1 = c_2$, so $f(a_1) = c_1$
            as required.
    \end{proof}
\end{tcolorbox}

\textbf{Solution:}
If $A$ has $n > 1$ elements then $A$ may have $2$ elements.
In this case there is no way to choose $a_3$ as the proof suggests.

\subsection{Recursion}

\begin{tcolorbox}[title=Problem 5, breakable]
    Suppose $r$ is a real number and $r \ne 1$.
    Prove that for all $n \in \mathbb{N}$,
    \[\sum_{i = 0}^{n} r^i = \frac{r^{n + 1} - 1}{r - 1}\]
    (Note that this exercise generalizes Example $6.1.1$
    and exercise $7$ of Section $6.1$)
\end{tcolorbox}

\begin{proof}
    (\textbf{Base Case})
    Let $n = 0$.
    Then $\sum_{i = 0}^n = \sum_{i = 0}^0 = r^0 = 1 = \frac{r^{1} - 1}{r - 1} = \frac{r - 1}{r - 1} = 1$.
    Thus, for $n = 0$ the theorem holds.

    (\textbf{Induction Step}) Suppose for some $n \in \mathbb{N}$ such that $n > 0$ the theorem holds.
    Consider 
    \[\sum_{i = 0}^{n + 1} r^i = \sum_{i = 0}^{n} r^i + r^{n + 1}\]
    By the induction hypothesis $\sum_{i = 0}^{n} r^i = \frac{r^{n + 1} - 1}{r - 1}$. Then 
    \[\sum_{i = 0}^{n} r^i + r^{n + 1} 
        = \frac{r^{n + 1} - 1}{r - 1} + r^{n + 1} 
        = \frac{r^{n + 1} - 1}{r - 1} + \frac{r^{n + 1}(r - 1)}{r - 1} 
        = \frac{r^{n + 1}(1 + r - 1) - 1}{r - 1} 
        = \frac{r^{(n + 1) + 1} - 1}{r - 1}\]
    So the equation holds for $n + 1$ proving our theorem.
\end{proof}

\begin{tcolorbox}[title=Problem 6, breakable]
    Prove that for all $n \ge 1$,
    \[\sum_{i = 1}^n \frac{1}{i^2} \le 2 - \frac{1}{n}\]
\end{tcolorbox}

\begin{tcolorbox}[title=Problem 7, breakable]
    (a) Suppose $a_0, a_1, a_2, \ldots$ and $b_0, b_1, b_2, \ldots, b_n$
        are two sequences of real numbers. Prove that 
        \[\sum_{i = 0}^n(a_i + b_i) = \sum_{i = 0}^n a_i + \sum_{i = 0}^n b_i\]

    (b) Suppose $c$ is a real number fand $a_0, a_1, \ldots, a_n$
        is a sequence of real numbers. Prove that 
        \[c \cdot \sum_{i = 0}^n a_i = \sum_{i = 0}^{n} (c \cdot a_i)\]
\end{tcolorbox}

\begin{tcolorbox}[title=Problem 8, breakable]
    The \emph{harmonic numbers} are numbers $H_n$ for $n \ge 1$ defined by 
    the formula 
    \[H_n = \sum_{i = 1}^n \frac{1}{i}\]
    (a) Prove that for all natural numbers $n$ and $m$,
        if $n \ge m \ge 1$ then $H_n - H_m \ge (n - m)/n$
        (Hint: Let $m$ be an arbitrary natural number with $m \ge 1$
        and then proceed by induction on $n$, with $n = m$ as the 
        base case of induction.)

    (b) Prove that for all $n \ge 0$, $H_{2^n} \ge 1 + n/2$.

    (c) (For those who have studied calculus.) Show that $\lim_{n \rightarrow \infty} H_n = \infty$,
        so $\sum_{i = 1}^{\infty} (1/i)$ diverges.
\end{tcolorbox}

\begin{tcolorbox}[title=Problem 12, breakable]
    (a) Prove that for all $n \in \mathbb{N}$, $2^n > n$.

    (b) Prove that for all $n \ge 9$, $n! \ge (2^n)^2$.

    (c) Prove that for all $n \in \mathbb{N}$, $n! \le 2^{(2^n)^2}$.
\end{tcolorbox}

\begin{tcolorbox}[title=Problem 13, breakable]
\end{tcolorbox}

\begin{tcolorbox}[title=Problem 14, breakable]
\end{tcolorbox}

\begin{tcolorbox}[title=Problem 18, breakable]
\end{tcolorbox}

\begin{tcolorbox}[title=Problem 18, breakable]
\end{tcolorbox}

\begin{tcolorbox}[title=Problem 21, breakable]
\end{tcolorbox}