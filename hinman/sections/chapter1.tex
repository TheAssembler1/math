\subsection{The Propositional Language}

\begin{tcolorbox}[title=Problem 10, breakable]
    Give a precise definition of the set of sentences using infix
    notation and prove unique readability for your definition.
\end{tcolorbox}

\textbf{Solution:}

\textbf{Definition of infix sentences:}
\begin{enumerate}
    \item $S_0 :=$ the set of $L$-atomic sentences.
    \item For each $n \in \omega$,
          \begin{align*}
            S_{n + 1} :=\text{ }&S_n \cup \{\neg \phi : \phi \in S_0\} 
                              \cup \{(\neg \phi) : \phi \in S_n \sim S_0\} \\
                         &\cup \{(\phi) \bullet (\psi): \phi, \psi \in S_n \sim S_0, \bullet \in \{\wedge, \vee, \rightarrow, \leftrightarrow\}\} \\
                         &\cup \{(\phi) \bullet \psi: \phi \in S_n \sim S_0, \psi \in S_0, \bullet \in \{\wedge, \vee, \rightarrow, \leftrightarrow\}\} \\
                         &\cup \{\phi \bullet (\psi): \phi \in S_0, \psi \in S_n \sim S_0, \bullet \in \{\wedge, \vee, \rightarrow, \leftrightarrow\}\} \\
                         &\cup \{\phi \bullet \psi: \phi, \psi \in S_0, \bullet \in \{\wedge, \vee, \rightarrow, \leftrightarrow\}\}
          \end{align*}
    \item Let $S_f := \bigcup_{n \in \omega} S_n$.
\end{enumerate}

\newpage
\begin{definition}[Parenthesization of binary connectives]\label{def:parenth}
Let $\phi$ and $\psi$ be sentences, and let $\bullet$ be any binary connective. Then the four possible parenthesizations of $\phi \bullet \psi$ are:
\begin{enumerate}
    \item $(\phi) \bullet (\psi)$
    \item $(\phi) \bullet \psi$
    \item $\phi \bullet (\psi)$
    \item $\phi \bullet \psi$
\end{enumerate}
\end{definition}

\begin{proposition}[Unique readability for propositional sentences]
    For any sentence $\theta \in S_f$, exactly one of the following holds:
    \begin{enumerate}
        \item $\theta$ is atomic;
        \item for some sentence $\phi$, $\theta = \neg \phi$;
        \item $\theta$ is a \textbf{conjunction} $\phi \wedge \psi$ for some sentences $\phi$ and $\psi$, with one of the four parenthesizations from Definition~\ref{def:parenth};
        \item $\theta$ is a \textbf{disjunction} $\phi \vee \psi$ for some sentences $\phi$ and $\psi$, with one of the four parenthesizations from Definition~\ref{def:parenth};
        \item $\theta$ is an \textbf{implication} $\phi \rightarrow \psi$ for some sentences $\phi$ and $\psi$, with one of the four parenthesizations from Definition~\ref{def:parenth};
        \item $\theta$ is a \textbf{bi-implication} $\phi \leftrightarrow \psi$ for some sentences $\phi$ and $\psi$, with one of the four parenthesizations from Definition~\ref{def:parenth}.
    \end{enumerate}
\end{proposition}

\textbf{Proof of unique readability:}
\begin{proof}
    
\end{proof}

\begin{tcolorbox}[title=Problem 11, breakable]
    For any expression $\phi  = s_0 \ldots s_k$, a \textbf{proper initial segment}
    of $\phi$ is any sequence of symbols $s_0 \ldots s_l$ with $l < k$.
    Prove that no proper initial segment of a sentence is a sentence,
        and show that this can be used as an alternative to ($4$) as 
        a technical lemma for the proof of Proposition $1.1.5$.
\end{tcolorbox}

\begin{tcolorbox}[title=Problem 12, breakable]
    Give a careful proof of Proposition $1.1.9$, and show how Theorem $1.1.7$
        is an application of it.
\end{tcolorbox}