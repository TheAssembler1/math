% Compile with XeLaTeX or LuaLaTeX
\documentclass[10pt]{article}  % Spivak uses ~10pt

% -----------------------------
% Fonts
% -----------------------------
\usepackage{fontspec}
\setmainfont{TeX Gyre Pagella}
\usepackage{unicode-math}
\setmathfont{Libertinus Math}

% -----------------------------
% Page layout
% -----------------------------
\usepackage[margin=2.5cm]{geometry}
\usepackage[parfill]{parskip}

% -----------------------------
% Theorems and QED
% -----------------------------
\usepackage{amsthm}
\usepackage{tcolorbox}
\tcbuselibrary{breakable}

% QED symbol like Spivak (tall gray rectangle)
\renewcommand{\qedsymbol}{\textcolor{black}{\rule{1ex}{2.2ex}}}

\newtheorem{theorem}{Theorem}
\newtheorem{axiom}{Axiom}
\newtheorem{definition}{Definition}
\newtheorem{proposition}{Proposition}

% -----------------------------
% Misc packages
% -----------------------------
\usepackage{graphicx, subfig}
\usepackage{booktabs, array}
\usepackage{paralist, verbatim}
\usepackage{xcolor, pagecolor}
\usepackage{fancyhdr}
\pagestyle{fancy}
\renewcommand{\headrulewidth}{0pt}
\lhead{}\chead{}\rhead{}
\lfoot{}\cfoot{\thepage}\rfoot{}
\usepackage{sectsty}
\allsectionsfont{\sffamily\mdseries\upshape}
\usepackage[nottoc,notlof,notlot]{tocbibind}
\usepackage[titles,subfigure]{tocloft}
\renewcommand{\cftsecfont}{\rmfamily\mdseries\upshape}
\renewcommand{\cftsecpagefont}{\rmfamily\mdseries\upshape}
\usepackage{changepage, comment}
% Define a light gray
\definecolor{lightgraypaper}{RGB}{240,240,240}
% Set the page background
\pagecolor{lightgraypaper}
\color{black} % keep text black

\title{Fundamentals of Mathematical Logic by Hinman}
\author{Noah Lewis}
\begin{document}
\maketitle

\tableofcontents

\section{Propositional Logic and Other Fundamentals}
\begin{tcolorbox}[title=Problem 1, breakable]
    Prove using mathematical induction that for all positive integers $n$,
    \[1 + 2 + 3 + \cdots + n = \frac{n(n + 1)}{2}\]
\end{tcolorbox}

\begin{proof}
    Let $n = 1$ then $\frac{n(n + 1)}{2} = \frac{1(1 + 1)}{2} = \frac{1(2)}{2} = 1$.
    Assume the formula is true for some integer $k = n - 1$, thus:
    \begin{align*}
        1 + 2 + 3 + \cdots + (n - 1) = \frac{(n - 1)((n - 1) + 1)}{2}
    \end{align*}
    Thus:
    \begin{align*}
         & 1 + 2 + 3 + \cdots + (n - 1) + n               \\
         & = \frac{(n - 1)((n - 1) + 1)}{2} + n           \\
         & = \frac{{(n - 1)}^2 + n - 1}{2} + \frac{2n}{2} \\
         & = \frac{{(n - 1)}^2 + 3n - 1}{2}               \\
         & = \frac{n^2 - 2n + 1 + 3n - 1}{2}              \\
         & = \frac{n(n + 1)}{2}
    \end{align*}
\end{proof}

\begin{tcolorbox}[title=Problem 3, breakable]
    You probably recall from your previous mathematical work the \emph{triangle inequality:}
    for any real numbers $x$ and $y$,
    \[|x + y| \le |x| + |y|\]
    Accepts this as given (or see a calculus text to recall how it is proved).
    Generalize the triangle inequality, by proving that 
    \[|x_1 + x_2 + \cdots + x_n \le |x_1| + |x_2| + \cdots |x_n|,\]
    for any positive integer $n$.
\end{tcolorbox}

\begin{proof}
    For $n = 1$, trivially $|x_1| \le |x_1|$.
    For $n = 2$, $|x_1 + x_2| \le |x_1| + |x_2|$ by the triangle inequality.
    Now assume the formula holds for $k = n - 1$, thus:
    \begin{align*}
        |x_1 + x_2 + \cdots + x_{n - 1}| \le |x_1| + |x_2| + \cdots |x_{n - 1}|
    \end{align*}
    Thus:
    \begin{align*}
         & |x_1 + x_2 + \cdots + x_{n - 1} + x_n|        &  &                                  \\
         & \le  |(x_1 + x_2 + \cdots + x_{n - 1}) + x_n| &  &                                  \\
         & \le  |x_1 + x_2 + \cdots + x_{n - 1}| + |x_n| &  & \quad \text{triangle inequality} \\
         & \le  |x_1| + |x_2| + \cdots + |x_n|           &  & 
    \end{align*}
\end{proof}

\begin{tcolorbox}[title=Problem 4, breakable]
    Given a positive integer $n$, recall that $n! = 1 \cdot 2 \cdot 3 \cdots$ (this is read as 
    $n$ factorial). Provide an inductive definition for $n!$. (It  is customary to actually 
    start this defintion at $n = 0$, setting $0! = 1$) 
\end{tcolorbox}

\textbf{Solution}

We can define $n!$ as follows. If $n <=1$, then $n! = 1$. If $n > 1$, then $n!
    = n(n - 1)!$.

\begin{tcolorbox}[title=Problem 5, breakable]
    Prove that $2^n < n!$ for all $n \ge 4$.
\end{tcolorbox}

\begin{proof}
    Let $n = 4$, then $2^4 = 16 < 4! = 24$.
    Assume the inequality holds for $k = n - 1$, thus:
    \begin{align*}
        2^{n - 1} < (n - 1)!
    \end{align*}
    Thus:
    \begin{align*}
         & 2^{n - 1} \cdot 2 < (n - 1)! \cdot n \quad \text{Note: $2 < 4 \le n$} \\
         & 2^n < n! 
    \end{align*}
\end{proof}

\begin{tcolorbox}[title=Problem 7, breakable]
    Prove the familiar geometric progression formula.
    Namely, suppose that $a$ and $r$ are real numbers with $r \not = 1$.
    Then show that:
    \[a + ar  + ar^2 + \cdots + ar^{n - 1} = \frac{a - ar^n}{1 - r}\]
\end{tcolorbox}

\begin{proof}
    Let $n = 1$, then $a = \frac{a - ar^n}{1 - r} = \frac{a - ar}{1 - r} = \frac{a(1 - r)}{1 - r} = a$.
    Assume the formula holds for $k = n - 1$, thus:
    \[a + ar + ar^2 + \cdots + ar^{n - 2} = \frac{a - ar^{n - 1}}{1 - r}\]
    Thus
    \begin{align*}
         & a + ar + ar^2 + \cdots + ar^{n - 2} + ar^{n - 1}                \\
         & =\frac{a - ar^{n - 1}}{1 - r} + ar^{n - 1}                      \\
         & =\frac{a - ar^{n - 1}}{1 - r} + \frac{(1 - r)ar^{n - 1}}{1 - r} \\
         & =\frac{a - ar^{n - 1} + (1 - r)(ar^{n - 1})}{1 - r}             \\
         & =\frac{a - ar^{n - 1} + ar^{n - 1} - ar^n}{1 - r}               \\
         & =\frac{a - ar^n}{1 - r}
    \end{align*}
\end{proof}

\begin{tcolorbox}[title=Problem 12, breakable]
    Consider the sequence ${a_n}$ defined inductively as follows:
    \[a_1 = 5, a_2  = 7, a_{n + 2} = 3a_{n + 1} - 2a_n\]
\end{tcolorbox}

\begin{proof}
    Let $n = 1$, then $a_1 = 5 = 3 + 2^n =  3 + 2^1 = 5$.
    Let $n = 2$, then $a_2 = 7 = 3 + 2^n = 3 + 2^2 = 7$
    Assume the formula holds for $k < n$ thus:
    \[a_{n - 1} = 3 + 2^{n - 1}\]
    and 
    \[a_{n - 2} = 3 + 2^{n - 2}\]
    So $k = n$ is:
    \[a_{n} = 3a_{n - 1} - 2a_{n - 2} = 3(3 + 2^{n - 1}) - 2(3 + 2^{n - 2})\]
    Then:
    \begin{align*}
         & 3(3 + 2^{n - 1}) - 2(3 + 2^{n - 2})             \\
         & = 9 + 3 \cdot 2^{n - 1} - 6 - 2 \cdot 2^{n - 2} \\
         & = 3 + 3 \cdot 2^{n - 1} - 2^{n - 1}             \\
         & = 3 + 2 \cdot 2^{n - 1}                         \\
         & = 3 + 2^n                            
    \end{align*}
\end{proof}

\begin{tcolorbox}[title=Problem 14, breakable]
    In this problem you will prove some results about the binomial coefficients, using induction.
    Recall that:
    \[\binom{n}{k} = \frac{n!}{(n - k)!k!}\]
    where $n$ is a positive integer, and $0 \le k \le n$.

    (a) Prove that 
    \[\binom{n}{k} = \binom{n - 1}{k} + \binom{n - 1}{k - 1}\]
    $n \ge 2$ and $k < n$. Hint: You do not need induction to prove this.
    Bear in mind that $0! = 1$.

    (b) Verify that $\binom{n}{0} = 1$ and $\binom{n}{n} = 1$. Use these facts,
    together with part a, to prove by induction on $n$ that $\binom{n}{k}$ is an integer,
    for all $k$ with $0 \le k \le n$. (Note: You may have encountered $\binom{n}{k}$ as the 
    count of the number of $k$ element subsets of a set of $n$ objects; it follows that from this 
    $\binom{n}{k}$ is an integer. What we are asking for here is an inductive proof based on algebra.)

    (c) Use part a and induction to prove the Binomical Theorem: For non-negative $n$ and variables $x$, $y$,
    \[{(x + y)}^n = \sum_{k = 0}^{n} \binom{n}{k}x^{n - k}y^k\]
\end{tcolorbox}

\begin{proof}
    \begin{align*}
         & \binom{n - 1}{k} + \binom{n - 1}{k - 1}                                                             \\
         & = \frac{(n - 1)!}{((n - 1) - k)!k!} + \frac{(n - 1)!}{((n - 1) - (k - 1))!(k - 1)!}                 \\
         & =  (n - 1)!\left(\frac{1}{((n - 1) - k)!k!} + \frac{1}{((n - 1) - (k - 1))!(k - 1)!}\right)         \\
         & =  (n - 1)!\left(\frac{1}{((n - 1) - k)!k(k - 1)!} + \frac{1}{((n - 1) - (k - 1))!(k - 1)!}\right)  \\
         & =  \frac{(n - 1)!}{(k - 1)!}\left(\frac{1}{((n - 1) - k)!k} + \frac{1}{((n - 1) - (k - 1))!}\right) \\
         & =  \frac{(n - 1)!}{(k - 1)!}\left(\frac{1}{(n - k - 1)!k} + \frac{1}{(n - k)!}\right)               \\
         & =  \frac{(n - 1)!}{(k - 1)!}\left(\frac{1}{(n - k - 1)!k} + \frac{1}{(n - k)(n - k - 1)!}\right)    \\
         & =  \frac{(n - 1)!}{(k - 1)!(n - k - 1)!}\left(\frac{1}{k} + \frac{1}{(n - k)}\right)                \\
         & =  \frac{(n - 1)!}{(k - 1)!(n - k - 1)!}\left(\frac{n - k}{k(n - k)} + \frac{k}{k(n - k)}\right)    \\
         & =  \frac{(n - 1)!}{(k - 1)!(n - k - 1)!}\left(\frac{n}{k(n - k)}\right)                             \\
         & =  \frac{n!}{k!(n - k)!}
    \end{align*}
\end{proof}

\begin{proof}
    Let $k = 0$ then, $\binom{n}{0} = \frac{n!}{(n - 0)!(0!)} = \frac{n!}{n!} = 1 \in \mathbb{Z}$.
    Let $k = n$ then, $\binom{n}{n} = \frac{n!}{(n - n)!(n!)} = \frac{n!}{n!} = 1 \in \mathbb{Z}$.
    Assume this holds for $n - 1$, thus for all $k$ where $0 \le k \le n - 1$:
    \[\binom{n - 1}{k} \in \mathbb{Z}\]
    Then:
    \[\binom{n}{k} =  \binom{n - 1}{k} + \binom{n - 1}{k - 1}\]
    Since each of these terms exist in $\mathbb{Z}$ their sum $\binom{n}{k}$ is in
    $\mathbb{Z}$ since the integers are closed over addition.
\end{proof}

\begin{proof}
    Let $n = 0$. Then:
    \begin{align*}
        {(x + y)}^0 = 1 = \sum_{k = 0}^{0} \binom{0}{k} x^k y^{0-k} = \binom{0}{0} x^0 y^0 = 1 \cdot 1 \cdot 1 = 1
    \end{align*}

    Assume the formula holds for $n - 1$, thus:
    \begin{align*}
        \sum_{k = 0}^{n - 1} \binom{n - 1}{k} x^k y^{(n-1)-k} = (x+y)^{n - 1}
    \end{align*}
    Then:
    \begin{align*}
        {(x+y)}^n & = {(x+y)}^{n - 1} \cdot (x + y)  s                                                                                              \\
                  & = \left(\sum_{k = 0}^{n - 1} \binom{n - 1}{k} x^k y^{(n-1)-k}\right) \cdot (x + y)                                              \\
                  & = x \cdot \sum_{k = 0}^{n - 1} \binom{n - 1}{k} x^k y^{(n-1)-k} + y \cdot \sum_{k = 0}^{n - 1} \binom{n - 1}{k} x^k y^{(n-1)-k} \\
                  & = \sum_{k = 1}^{n} \binom{n - 1}{k - 1} x^k y^{n - k} + \sum_{k = 0}^{n - 1} \binom{n - 1}{k} x^k y^{n - k}                     \\
                  & = \sum_{k = 0}^{n} \left( \binom{n - 1}{k - 1} + \binom{n - 1}{k} \right) x^k y^{n - k}                                         \\
                  & = \sum_{k = 0}^{n} \binom{n}{k} x^k y^{n - k}
    \end{align*}
\end{proof}

\begin{tcolorbox}[title=Problem 15, breakable]
    Criticize the following ``proof'' showing that all cows are the same
    color.

    It suffices to show that any herd of $n$ cows has the same color. If the herd
    has but one cow, then trivially all the cows in the herd have the same color.
    Now suppose that we have a herd of n cows and $n > 1$. Pick out a cow and
    remove it from the herd, leaving $n - 1$ cows; by the induction hypothesis
    these cows all have the same color. Now put the cow back and remove another
    cow. (We can do so because $n > 1$.) The remaining $n - 1$ again must all be
    the same color. Hence, the first cow selected and the second cow selected have
    the same color as those not selected, and so the entire herd of n cows has the
    same color.
\end{tcolorbox}

\textbf{Solution}

The proof selects a different set of $n - 1$ cows each time.

\begin{tcolorbox}[title=Problem 16, breakable]
    Prove the converse of Theorem 1.1; that is, prove that the Principle 
    of Mathematical Induction implies the Well-ordering Princple. 
    (This shows that these two principles are logically equivalent,
    and so from an axiomatic point of view it doesn't matter which
    we assume is an axiom for the natural numbers.)
\end{tcolorbox}

\begin{proof}
    Assume that the principle of mathematical induction holds.
    Let $G \subseteq \mathbb{N}$ be nonempty. For contradiction, suppose $G$ has no least element. 
    Define $P(n)$ to be the statement: ``Nothing $\le n$ is in $G$.''

    If $1 \in G$, then $1$ would be the least element of $G$, a contradiction. So
    $1 \notin G$ and $P(1)$ is true.

    Assume $P(n)$ holds meaning no element of $G$ is $\le n$. If $n+1 \in G$, then
    $n+1$ would be the least element of $G$, a contradiction. Therefore $n+1 \notin
        G$, and hence $P(n+1)$ holds.

    By induction, $P(n)$ holds for all $n \in \mathbb{N}$. So no element of
    $\mathbb{N}$ is in $G$, so $G = \emptyset$, contradicting the assumption that
    $G$ is nonempty. 
\end{proof}

\end{document}
