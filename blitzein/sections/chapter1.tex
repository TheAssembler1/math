\subsection{Probability and Counting}

\begin{tcolorbox}[title=Problem 1, breakable]
    How many ways are there to permute the letters 
        in the word MISSISSIPPI?
\end{tcolorbox}

\begin{proof}
    There are $\binom{11}{4}$, $\binom{7}{4}$, $\binom{3}{2}$ places to put the S's,
    I's, then P's respectively. Then the M is fixed.
    Thus there are $\binom{11}{4}\binom{7}{4}\binom{3}{2}$ ways 
        to permute the word MISSISSIPPI.
\end{proof}

\begin{tcolorbox}[title=Problem 4, breakable]
    A \emph{round-robin tournament} is being held with $n$ tennis players;
    this means that every player will play against every other player exactly once.
    \begin{enumerate}
        \item How many possible outcomes are there for the tournament
              (the outcome lists out who won and who lost for each game)?
        \item How many games are played in total?
    \end{enumerate}
\end{tcolorbox}

\begin{proof}
    By the next part there are $\binom{n}{2}$ total games.  
    Now, for each game there are two possibilities.
    Either player 1 or player 2 wins.
    Thus there are $2^{\binom{n}{2}}$ possible outcomes for the tournament.
\end{proof}

\begin{proof}
    This is equivalent to finding how many ways to select $2$ players from $n$ players
        where order does not matter.
    Thus the total numbers of games is $\binom{n}{2}$.
\end{proof}

\begin{tcolorbox}[title=Problem 5, breakable]
    A \emph{knock-out tournament} is being held with $2^n$ tennis players.
    This means that for each round, the winners move on to the next round 
    and the losers are elminated, until only one person remains.
    For example, if initially there are $2^4 = 16$ players, then there 
    are $8$ games in the first round, then the $8$ winners move on to round $2$,
    then the $4$ winners move on to round $3$, then the $2$ winnners move on to round $4$,
    the winner of which is declared teh winner of the tournament. (There are various systems 
    for determining who plays whom within a round, but these do not matter for this problem.)
    \begin{enumerate}
        \item How many rounds are there?
        \item Count how many games in total are played, by adding up the numbers of games 
              played in each round.
        \item Count how many games in total are played, this time by directly thinking about it without 
              doing almost any calculation.
    \end{enumerate}
    Hint: How many players need to be eliminated?
\end{tcolorbox}

\begin{proof}
    We can count the number of rounds starting from the last round until we reach the total number of players over two: 
        $\frac{2^n}{2} = 2^{n - 1}$
        because two people play each each game.
    Thus we see the following pattern of games within each round 
    \[1, 2, 4, 8, \cdots, 2^{n - 1} = 2^0, 2^1, 2^2, 2^3, \cdots 2^{n - 1}\]
    Thus the number of rounds can be found by first solving $2^r =\frac{2^n}{2} = 2^{n - 1}$.
    It follows that $r = n - 1$. Then adjusting by $1$ to reach $n$ rounds.
\end{proof}

\begin{proof}
    We can simply sum to find $\sum_{i = 0}^{n - 1} 2^i = 2^n - 1$ games.
\end{proof}

\begin{proof}
    Since the number of games doubles each round it is clear that
        the number of games is $2^n - 1$.
\end{proof}

\begin{tcolorbox}[title=Problem 10, breakable]
    To fullfill the requirements for a certain degree, a student can 
    choose to take any $7$ out of a list of $20$ courses,
    with the constraint that at least $1$ of the $7$ courses 
    must be a statistics course. Suppose that $5$ of the $20$ courses are 
    statistics courses. 
    \begin{enumerate}
        \item How many choices are there for which $7$ courses to take?
        \item Explain intuitively why the answer to (a) is \emph{not} $\binom{5}{1} \cdot \binom{19}{6}$.
    \end{enumerate}
\end{tcolorbox}

\begin{proof}
    We can begin by counting the total number of ways to choose $7$ courses from the 
        available $20$ which is $\binom{20}{7}$.
    Then we have to subtract choices which do not include one statistics course.
    There are $\binom{15}{7}$ choices which do not include a statistics course.
    Thus there are $\binom{20}{7} - \binom{15}{7}$ choices.
\end{proof}

\begin{proof}
    Consider a fixed choice of $7$ courses that contains two statistics courses
    which we will label \textbf{class 1} and \textbf{class 2}. This is a single set of courses.
    However, in the computation $\binom{5}{1}\binom{19}{6}$, this same set is counted
    more than once. One time it is counted when \textbf{class 1} is chosen in the
    $\binom{5}{1}$ factor and \textbf{class 2} is chosen among the remaining $6$
    courses in the $\binom{19}{6}$. Another time it is counted when \textbf{class 2}
    is chosen in the $\binom{5}{1}$ factor and \textbf{class 1} is chosen among the
    remaining $6$ courses in the $\binom{19}{6}$.
    Thus the computation distinguishes between different ways of forming the same
    set of courses, and therefore overcounts.
\end{proof}

\begin{tcolorbox}[title=Problem 12, breakable]
    Four players named $A$, $B$, $C$, and $D$, are playing a card game.
    A standard, well-shuffled deck of cards is dealt to the players (so each 
    player receives a $13$-card hand).
    \begin{enumerate}
        \item How many possibilities are there for the hand that player A will get?
        (Within a hand, the order in which cards were received doesn't matter.)
        \item How many possibilities are there overall for what hands everyone will get,
              assuming that it matters which player gets which hand, but not the order 
              of cards within a hand?
        \item Explain intuitively why the answer to Part (b) is not the fourth power 
               of the answer to Part (a).
    \end{enumerate}
\end{tcolorbox}

\begin{proof}
    There are $\binom{52}{13}$ hands player $A$ might get.
\end{proof}

\begin{proof}
    There are $\binom{52}{13}, \binom{52 - (13 \cdot 1)}{13}, \binom {52 - (13 \cdot 2)}{13}, \binom{52 - (13 \cdot 3)}{13}$
        ways to choose cards for players $1, 2, 3, 4$.
    Thus there are  
        $\binom{52}{13} \cdot \binom{52 - (13 \cdot 1)}{13} \cdot \binom {52 - (13 \cdot 2)}{13} \cdot \binom{52 - (13 \cdot 3)}{13}$
        possibilities overall.
\end{proof}

\begin{proof}
    There are fewer cards to choose from after selecting the cards for a given person.
    Thus the choices are not independent.
\end{proof}

\begin{tcolorbox}[title=Problem 13, breakable]
    A certain casino uses $10$ standard decks of cards mixed toegether into one 
    big deck, which we will call a \emph{superdeck}. Thus, the superdeck has 
    $52 \cdot 10 = 520$ cards, with $10$ copies of each card. How many different 
    $10$-card hands can be dealth from the superdeck? The order of the cards does not matter,
    nor does it matter which of the original $10$ decks the cards came from. Express 
    your answer as a binomial coefficient.

    Hint: Bose-Einstein.
\end{tcolorbox}

\begin{proof}
    We must find the number of ways of selecting $10$ items from $52$ categories with replacement 
    where order does not matter.
    Thus the number of choices is $\binom{52 + 10 - 1}{10}$.
\end{proof}

\subsection{Story Proofs}

\begin{tcolorbox}[title=Problem 15, breakable]
    Give a story proof that $\sum_{k = 0}^{n} \binom{n}{k} = 2^n$.
\end{tcolorbox}

\begin{proof}
    Consider a set of $n$ items.
    We want to count all subsets that can be constructed from this set.
    One method is to count subsets by their size.
    For each $k = 0, 1, \ldots, n$, there are $\binom{n}{k}$ subsets of size $k$,
    so the total number of subsets is $\sum_{k=0}^{n} \binom{n}{k}$.
    Another method is to go one by one through each element and either include
    it in a subset or exclude it.
    Since each of the $n$ elements has two choices, there are $2^n$ subsets.
    Both methods count the same collection of subsets,thus
        $\sum_{k=0}^{n} \binom{n}{k} = 2^n$.
\end{proof}

\begin{tcolorbox}[title=Problem 16, breakable]
    Show that for all positive integers $n$ and $k$ with $n \ge k$.
    \[\binom{n}{k} + \binom{n}{k - 1} = \binom{n + 1}{k}\]
    doing this in two ways: (a) algebraically and (b) with a story, giving an 
    interpretation for why both sides count the same thing.
    Hint for the story proof: Imagine an organization consisting of $n + 1$ people,
    with one of them pre-designated as the president of the organization.
\end{tcolorbox}

\begin{proof}
    \begin{align*}
        \binom{n}{k} + \binom{n}{k - 1} 
            &= \frac{n!}{(n - k)!k!} + \frac{n!}{(n - (k - 1))!(k - 1)!} \\
            &= \frac{n!}{(n - k)!k!} + \frac{n!}{(n - k + 1)!(k - 1)!} \\
            &= \frac{n!(n - k + 1)}{(n - k + 1)! k!} + \frac{n! k}{(n - k + 1)! k!} \\
            &= \frac{n! (n - k + 1 + k)}{(n - k + 1)!k!} \\
            &= \frac{n!(n + 1)}{(n - k + 1)k!} \\
            &= \frac{(n + 1)!}{((n + 1) - k)! k!} \\
            &= \binom{n + 1}{k}
    \end{align*}
\end{proof}

\begin{proof}
    Let $S$ be a set of $n + 1$ people, and let $p \in S$ be an arbitrary person.
    Consider the subsets of size $k$ of $S$, which can be divided into two categories:
    those that contain $p$ and those that do not.
    For the subsets that contain $p$, remove $p$ from each subset.  
    These are exactly the subsets of size $k-1$ of the remaining $n$ people, 
    so there are $\binom{n}{k-1}$ of them.
    For the subsets that do not contain $p$, they are simply subsets of size $k$ 
    of the remaining $n$ people, giving $\binom{n}{k}$.
    Combining these two cases gives all subsets of size $k$ of $S$
        thus $\binom{n+1}{k} = \binom{n}{k-1} + \binom{n}{k}$.
\end{proof}

\begin{tcolorbox}[title=Problem 20, breakable]
    (a) Show using a story proof that 
    \[\binom{k}{k} + \binom{k + 1}{k} + \binom{k + 2}{k} + \cdots + \binom{n}{k} = \binom{n + 1}{k + 1}\]
    where $n$ and $k$ are positive integers $n \ge k$. This is called the 
    \emph{hockey stick identity}. Hint: Imagine arranging a group of people by age,
    and then think about the oldest person in a chosen subgroup.

    (b) Suppose that a large pack of Haribo gummi bears can have anywhere between $30$ and $50$
    gummi bears. There are $5$ delicious flavors: pineapple (clear), raspberry (red),
    orange (orange), strawberry (green, mysteriously), and lemon (yellow).
    There are $0$ non-delicious flavors. How many possibilities are there for the 
    composition of such a pack of gummi bears? You can leave your answer in terms of a 
    couple binomial coefficients, but not a sum of lots of binomial coefficients.
\end{tcolorbox}

\begin{tcolorbox}[title=Problem 22, breakable]
    The dutch mathematician R.J. Stroeker remakarked:

    \emph{Every beginning student of number theory surely must have marveled at the 
    miraculous fact that for each natural number $n$ the sum of the first $n$ positive 
    consecutive cubes is a perfect square.}

    Usually this identity is proven by induction, but that does not give much insight
    into why the result is true, nor does it help much if we wanted to compute the 
    left-hand side but didn't already know this result. In this problem, you will give a 
    sotry proof the identity.
    
    (a) Give a story proof of the identity
    \[1 + 2 + \cdots + n = \binom{n + 1}{2}.\]
    Hint: Consider a round-robin tournament (see Exercise $4$).

    (b) Give a story proof of the identity
    \[1^3 + 2^3 + \cdots + n^3 = 6\binom{n + 1}{4} + 6\binom{n + 1}{3} + \binom{n + 1}{2}.\]
    It is then just basic algebra (not required for this problem) to check that the square of 
    the right-hand side in (a) is the right-hand side in (b).

    Hint: Imagine choosing a number between $1$ and $n$ and then choosing $3$ numbers between $0$ and $n$
    smaller than the original number, with replacement. Then consider cases based on how many distinct 
    numbers were chosen.
\end{tcolorbox}

\subsection{Naive Definition of Probability}

\begin{tcolorbox}[title=Problem 23, breakable]
\end{tcolorbox}

\begin{tcolorbox}[title=Problem 26, breakable]
\end{tcolorbox}

\begin{tcolorbox}[title=Problem 31, breakable]
\end{tcolorbox}

\begin{tcolorbox}[title=Problem 40, breakable]
\end{tcolorbox}

\subsection{Axioms of Probability}

\begin{tcolorbox}[title=Problem 43, breakable]
\end{tcolorbox}

\begin{tcolorbox}[title=Problem 44, breakable]
\end{tcolorbox}

\begin{tcolorbox}[title=Problem 48, breakable]
\end{tcolorbox}

\subsection{Inclusion-Exclusion}

\begin{tcolorbox}[title=Problem 49, breakable]
\end{tcolorbox}

\subsection{Mixed Practice}

\begin{tcolorbox}[title=Problem 57, breakable]
\end{tcolorbox}

\begin{tcolorbox}[title=Problem 60, breakable]
\end{tcolorbox}

\begin{tcolorbox}[title=Problem 61, breakable]
\end{tcolorbox}

\begin{tcolorbox}[title=Problem 62, breakable]
\end{tcolorbox}