\subsection{The Archimedean Understanding}

\begin{tcolorbox}[title=Problem 6, breakable]
    Consider the series 
    \[1 + \frac{1}{2} + \frac{1}{4} + \frac{1}{8} + \cdots + \frac{1}{2^k} + \cdots.\]
    Find the target value, $T$, of the partial sums. 
    How do you know that for any $M$ greater than your target 
        value, all of the partial sums are strictly less than $M$?
    How many terms do you have to take in order to garuntee 
        that all of the partial sums from that point on will be 
        larger than $L = T - \frac{1}{10}$.
\end{tcolorbox}

\begin{proof}
    The target value $T$ is equal to $2$.
    Suppose $M$ is a real number greater than $2$.
    Then 
    \[
        2(1 - (1/2)^n) = M > 2 \implies 1 - (1/2)^n > 1 \implies -(1/2)^n > 0,
    \]
    which is impossible since $(1/2)^n > 0$ for all $n$. 

    Notice $1.9 < 2(1 - (1/2)^n) \iff \frac{1.9}{2} < 1 - (1/2)^n 
    \iff (1/2)^n < 1 - 0.95 = 0.05$.
    Taking logarithms shows $n > \log_{1/2}(0.05)$.
\end{proof}

\begin{tcolorbox}[title=Problem 9, breakable]
    What is the Archimedean understanding of the infinite series 
        $1 - 1 + 1 - 1 + \cdots$? Explain why this series cannot have a value under this understanding.
\end{tcolorbox}

\textbf{Solution: }
The Archimedean understanding relies on the assumption that 
a sum is approximated by partial sums.
But for even $S_n$ the sum seems to go to $0$,
while for odd $S_n$ the sum seems to go to $1$.
Since the partial sums do not approach a single number,
the series cannot have a value with this understanding.

\subsection{Geometric Series}

\begin{tcolorbox}[title=Problem 1, breakable]
    Find the target value of the series 
    \[1 + \frac{1}{3} + \frac{1}{9} + \cdots + \frac{1}{3^k} + \cdots\]
    Find a value of $n$ so that any partial sum with at least $n$ terms 
    is within $0.001$ of the target value. Justify your answer.
\end{tcolorbox}

\begin{tcolorbox}[title=Problem 5, breakable]
    It is tempting to differentiate each side of equation (2.11) with respect to $x$
        and to assert that 
    \[1 + 2x + 3x^2 + 4x^3 + \cdots = \frac{1}{(1 - x)^2}.\]
    Following Cauchy's advice, we know we need to be careful. Differentiate each side 
    of equation (2.10). What is the difference between $1 + 2x + 3x^2 + \cdots + n x^{n - 1}$
    and $(1 - x)^2$? For which values of $x$ will this difference approach $0$ as $n$ increases?
\end{tcolorbox}