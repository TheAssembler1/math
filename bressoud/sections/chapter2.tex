\subsection{The Archimedian Understanding}

\begin{definition}[Archimedian Understanding of an Infinite Series]
    The \textbf{Archimedian Understanding} of an infinite series is that it is shorthand 
    for the sequence of finite summations. The \textbf{value} of an infinite series,
    if it exists, is that number $T$ such that given any $L < T$ and any $M > T$, all of the finite 
    sums from some point on will be strictly contained in the interval between $L$ and $M$.
    More precisely, given $L < T < M$, there is an integer $n$, whose value depends on the choice 
    of $L$ and $M$, such that every partial sum with at least $n$ terms lies inside the interval $(L, M)$.
\end{definition}

\subsection{Geometric Series}

\begin{definition}[Convergence of an Infinite Series]
    An infinite series \textbf{converges} if there is a target value $T$
    such that for any $L < T$ and any $M > T$, all of the partial sums 
    from some point on are strictly between $L$ and $M$.
\end{definition}

\subsection{Calculating $\pi$}

\begin{theorem}[Newton's Binomial Series]
    For any real number $a$ and any $x$ such that $|x| < 1$,
    we have that 
    \[(1 + x)^a = 1 + ax + \frac{a(a - 1)}{2!} + \frac{a(a - 1)(a - 2)}{3!} x^3 + \cdots.\]
\end{theorem}

\subsection{Logarithms and Harmonic Series}

\begin{definition}[Divergence to Infinity]
    When we write that an infinite series equals $\infty$, we mean that no matter 
    what number we pick, we can find an $n$ so that the partial sums with at least $n$ 
    terms will exceed that number.
\end{definition}

\begin{definition}[Euler's constant, $\gamma$]
    Euler's constant is defined as the limit between the partial sum of the harmonic series 
    and the natural logarithm,
    \[\gamma = \lim_{n \rightarrow \infty} \left(1 + \frac{1}{2} + \frac{1}{3} + \cdots + \frac{1}{n - 1} - \ln n\right)\]
\end{definition}

\begin{definition}[Nested Interval Principle]
    Given an increasing sequence, $x_1 \le x_2 \le x_3 \le \cdots$, and a decreasing sequence,
    $y_1 \ge y_2 \ge y_3 \ge \cdots$, such that $y_n$ is always larger than $x_n$ but the difference
    between $y_n$ and $x_n$ can be made arbitrarily small by taking $n$ sufficiently large, there 
    is exactly one real number that is greater than or equal to every $x_n$ and less than or equal to 
    every $y_n$.
\end{definition}